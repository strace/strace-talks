% Copyright (C) 2012-2023 Dmitry V. Levin <ldv@strace.io>
% Permission is granted to copy, distribute and/or modify this document
% under the terms of the GNU Free Documentation License, Version 1.2
% or any later version published by the Free Software Foundation;
% with no Invariant Sections, no Front-Cover Texts, and no Back-Cover Texts.

\documentclass[unicode,aspectratio=169,xcolor={table,dvipsnames,usernames}]{beamer}

\usepackage[utf8]{inputenc}
\usepackage[T2A]{fontenc}
\usepackage[english]{babel}
\usepackage{alltt}

\colorlet{darkred}{BrickRed}
\colorlet{darkgreen}{OliveGreen}
\newcommand{\symlinebreak}{\textcolor{blue}{\(\hookleftarrow\)}}
\newcommand{\symlinecont}{\textcolor{blue}{\(\longrightarrow\)}}

\pgfdeclareimage[height=0.6cm]{left-corner-logo}{devconf-cz-icon.pdf}
\pgfdeclareimage[height=1.2cm]{right-corner-logo}{strace-straus.pdf}
\pgfdeclareimage[height=2cm]{title-logo}{devconf-cz.pdf}
\pgfdeclareimage[height=5.8cm]{wat}{wat.pdf}
\pgfdeclareimage[height=8cm]{strace-logo}{strace-straus.pdf}

\usetheme{Warsaw}
\setbeamertemplate{headline}{}
\setbeamertemplate{footline}{%
	\pgfuseimage{left-corner-logo}%
	\hfill%
	\pgfuseimage{right-corner-logo}%
}
\setbeamertemplate{navigation symbols}{}
\setbeamertemplate{logo}{}
\setbeamercovered{transparent}

\title{\Huge Modern strace}
\author{\LARGE Dmitry~Levin}
\date{\Large Brno, 2023}
\titlegraphic{\pgfuseimage{title-logo}}

\begin{document}

%%%%%%%
\begin{frame}[noframenumbering]
\titlepage
\end{frame}

%%%%%%%
\begin{frame}{New strace features implemented since DevConf.cz 2019 \hfill [\insertframenumber/\inserttotalframenumber]}
\begin{block}{Tracee startup}
\begin{description}
	\item[\textbf{-{}-argv0}=\textit{name}] set argv[0] of the command being executed
\end{description}
\end{block}

\begin{block}{Tracing control}
\begin{description}
	\item[\textbf{-{}-seccomp-bpf}] activate seccomp-assisted syscall filtering
	\item[\textbf{-{}-daemonize}=\{\textbf{pgroup}|\textbf{session}\}] daemonize strace further
	\item[\textbf{-{}-syscall-limit}=\textit{number}] detach after capturing the specified number of syscalls
\end{description}
\end{block}

\begin{block}{Tracing output}
\begin{description}
	\item[\textbf{-{}-decode-fds}=\textbf{signalfd}] decode signal masks associated with signalfd fds
	\item[\textbf{-{}-decode-pids}=\textbf{comm}] display of command names for PIDs
	\item[\textbf{-{}-decode-pids}=\textbf{pidns}] PID namespace translation
	\item[\textbf{-{}-secontext}[=\textit{format}{]}] display of SELinux contexts
	\item[\textbf{-{}-syscall-number}] display of syscall numbers
\end{description}
\end{block}
\end{frame}

%%%%%%%
\begin{frame}{New strace features implemented since DevConf.cz 2019 \hfill [\insertframenumber/\inserttotalframenumber]}
\begin{block}{Filtering}
\begin{description}
	\item[\textbf{-{}-status}=\textit{set}\hfill] syscall return status filtering
	\item[\textbf{-{}-trace-fds}=\textit{set}] filtering syscalls operating on the specified set of fds
	\item[\textbf{\%clock}, \textbf{\%creds}] new syscall filtering classes
\end{description}
\end{block}

\begin{block}{Tampering}
\begin{description}
	\item[\textbf{poke\_enter}=@\textit{argN}=\textit{dataN}] memory tampering on entering syscall
	\item[\textbf{poke\_exit}=@\textit{argN}=\textit{dataN}] memory tampering on exiting syscall
	\item[\textbf{when}=\textit{first}[..\textit{last}{]}[+[\textit{step}{]}{]}] interval specification syntax
\end{description}
\end{block}

\begin{block}{Statistics}
\begin{description}
	\item[\textbf{-{}-summary-columns}=\textit{set}] call summary configuration
\end{description}
\end{block}

\begin{block}{Miscellaneous}
\begin{description}
	\item[\textbf{-{}-tips}\hfill] display of strace tips
\end{description}
\end{block}
\end{frame}

%%%%%%%
\begin{frame}[fragile]{Seccomp-assisted syscall filtering: \textbf{-{}-seccomp-bpf} \hfill [\insertframenumber/\inserttotalframenumber]}
\begin{block}{Introduced in v5.3 (September 2019)}
\small
\begin{itemize}
\item Automatically generates and attaches a BPF program to filter syscalls
\item Makes execution of untraced syscalls two orders of magnitude faster
\end{itemize}
\end{block}

\begin{block}{The infamous dd example}
\scriptsize
\begin{alltt}
\$ dd if=/dev/zero of=/dev/null bs=1 count=1M 2>\&1 | grep -v records
1048576 bytes (1.0 MB, 1.0 MiB) copied, \textcolor{darkred}{0.668768} s, 1.6 MB/s

\$ strace \textcolor{darkgreen}{-{}-seccomp-bpf} -f -qq -e signal=none -e trace=fchdir \textbackslash
  dd if=/dev/zero of=/dev/null bs=1 count=1M 2>\&1 | grep -v records
1048576 bytes (1.0 MB, 1.0 MiB) copied, \textcolor{darkred}{0.736511} s, 1.4 MB/s

\$ strace -f -qq -e signal=none -e trace=fchdir \textbackslash
  dd if=/dev/zero of=/dev/null bs=1 count=1M 2>\&1 | grep -v records
1048576 bytes (1.0 MB, 1.0 MiB) copied, \textcolor{darkred}{28.0445} s, 37.4 kB/s
\end{alltt}
\end{block}

\begin{block}{Comparative slowdown}
$0.736511 / 0.668768 \approx \textcolor{darkred}{1.10}$,
$28.0445 / 0.736511 \approx \textcolor{darkred}{38}$,
$28.0445 / 0.668768 \approx \textcolor{darkred}{42}$
\end{block}
\end{frame}

%%%%%%%
\begin{frame}{Seccomp-assisted syscall filtering: \textbf{-{}-seccomp-bpf} \hfill [\insertframenumber/\inserttotalframenumber]}
\begin{block}{Limitations}
\begin{itemize}
	\item Has no effect unless \textbf{-{}-follow-forks} (\textbf{-f}) option is specified
	\item Incompatible with \textbf{-{}-syscall-limit}=\textit{number} option
	\item Incompatible with \textbf{-{}-detach-on}=\textbf{execve} (\textbf{-b execve}) option
	\item Not applicable to tracees attached using \textbf{-{}-attach}=\textit{pid} (\textbf{-p} \textit{pid}) option
	\item Problematic in case of premature strace termination
\end{itemize}
\end{block}

\begin{block}{Why?}
\begin{itemize}
	\item seccomp bpf installed by strace uses SECCOMP\_RET\_TRACE API \\ to trace syscalls
	\item Once installed, seccomp bpf cannot be removed
	\item If there is no tracer, SECCOMP\_RET\_TRACE is treated like SECCOMP\_RET\_ERRNO
		so all formerly traced syscalls would fail with ENOSYS
\end{itemize}
\end{block}

Being an optimization, \textbf{-{}-seccomp-bpf} is automatically turned off when not applicable.
\end{frame}

%%%%%%%
\begin{frame}[fragile]{Seccomp-assisted syscall filtering: \textbf{-{}-seccomp-bpf} \hfill [\insertframenumber/\inserttotalframenumber]}
\begin{block}{Premature strace termination, without \textbf{-{}-seccomp-bpf}}
\scriptsize
\begin{alltt}
\$ strace -f -e/exit sh -c 'trap "" TERM; sleep 2; echo $?' & sleep 1 && kill -TERM $!
[1] 12345
strace: Process 12350 attached
strace: Process 12349 detached
strace: Process 12350 detached
$ [1]+  Terminated              strace -f -e/exit sh -c 'trap "" TERM; sleep 2; echo $?'
0
\end{alltt}
\end{block}

\begin{block}{Premature strace termination, with \textbf{-{}-seccomp-bpf}}
\scriptsize
\begin{alltt}
\$ strace -f --secc -e/exit sh -c 'trap "" TERM; sleep 2; echo $?' & sleep 1 && kill -TERM $!
[1] 12345
strace: Process 12350 attached
strace: Process 12349 detached
strace: Process 12350 detached
$ [1]+  Terminated              strace -f --secc -e/exit sh -c 'trap "" TERM; sleep 2; echo $?'
sh: line 1: 12350 Segmentation fault      sleep 2
139
\end{alltt}
\end{block}
\end{frame}

%%%%%%%
\begin{frame}[fragile]{Daemonize strace further: \textbf{-DD}, \textbf{-DDD} \hfill [\insertframenumber/\inserttotalframenumber]}
\begin{block}{Traditional: run the tracee as a child of strace}
\small
\begin{alltt}
\$ timeout 1 strace -e/nanosleep sleep 2
clock_nanosleep(CLOCK_REALTIME, 0, \{tv_sec=2, tv_nsec=0\}, \symlinebreak
\symlinecont strace: Process 123459 detached
 <detached ...>
\end{alltt}
\end{block}

\begin{block}{\textbf{-D}, \textbf{-{}-daemonize}: run strace as a grandchild of the tracee \hfill v4.6 (March 2011)}
\small
\begin{alltt}
\$ timeout 1 strace \textcolor{darkred}{-D} -e/nanosleep sleep 2
strace: Process 123457 attached
clock_nanosleep(CLOCK_REALTIME, 0, \{tv_sec=2, tv_nsec=0\}, \symlinebreak
\symlinecont \{tv_sec=1, tv_nsec=12345678\}) \symlinebreak
\symlinecont = ? ERESTART_RESTARTBLOCK (Interrupted by signal)
--- SIGTERM \{si_signo=SIGTERM, si_code=SI_USER, si_pid=123456, si_uid=1000\} ---
+++ killed by SIGTERM +++
\end{alltt}
\end{block}
\end{frame}

%%%%%%%
\begin{frame}[fragile]{Daemonize strace further: \textbf{-DD}, \textbf{-DDD} \hfill [\insertframenumber/\inserttotalframenumber]}
\begin{block}{strace \textbf{-D}}
\small
\begin{alltt}
\$ timeout -s KILL 1 strace \textcolor{darkred}{-D} -e/nanosleep sleep 2
strace: Process 123457 attached
clock_nanosleep(CLOCK_REALTIME, 0, \{tv_sec=2, tv_nsec=0\}, Killed
\end{alltt}
\end{block}

\begin{block}{Introduced in v5.4 (November 2019)}
\begin{description}
	\item[\textbf{-DD}, \textbf{-{}-daemonize}=\textbf{pgroup}] Move daemonized strace into a separate process group
\end{description}
\end{block}

\begin{block}{strace \textbf{-DD}}
\small
\begin{alltt}
\$ timeout -s KILL 1 strace \textcolor{darkred}{-DD} -e/nanosleep sleep 2
strace: Process 123457 attached
clock_nanosleep(CLOCK_REALTIME, 0, \{tv_sec=2, tv_nsec=0\}, <unfinished ...>) = ?
Killed
+++ killed by SIGKILL +++
\end{alltt}
\end{block}
\end{frame}

%%%%%%%
\begin{frame}[fragile]{Detaching strace: \textbf{-{}-syscall-limit}=\textit{number} \hfill [\insertframenumber/\inserttotalframenumber]}
\begin{block}{Introduced in v6.3 (May 2023)}
\begin{itemize}
\item Detaches all tracees when the specified \textit{number} of syscalls have been captured
\item Syscalls filtered out via -{}-trace, -{}-trace-path or -{}-status options are not considered when keeping track of the number of captured syscalls
\item It could be useful in automatic tests scenarios to attach to some running
process, make some action, and detach after e.g. 1000 syscalls gets captured.
\end{itemize}
\end{block}

\begin{block}{\$ sh -c 'for i in \{0..99\}; do sleep 0.01; done' \& \\ \$ strace -f --trace=/sleep --syscall-limit=1 -p \$!}
\scriptsize
\begin{alltt}
[1] 78901
strace: Process 78901 attached
--- SIGCHLD \{si_signo=SIGCHLD, si_code=CLD_EXITED, si_pid=78903, si_uid=500, si_status=0, si_utime=0, si_stime=0\} ---
strace: Process 78906 attached
[pid 78906] clock_nanosleep(CLOCK_REALTIME, 0, \{tv_sec=0, tv_nsec=10000000\}, 0x7ffdadf29640) = 0
strace: System call limit has been reached, detaching tracees
strace: Process 78901 detached
strace: Process 78906 detached
\$ [1]+  Done                    sh -c 'for i in \{0..99\}; do sleep 0.01; done'
\end{alltt}
\end{block}
\end{frame}

%%%%%%%
\begin{frame}[fragile]{Set argv[0] of the command being executed: \textbf{-{}-argv0}=\textit{name} \hfill [\insertframenumber/\inserttotalframenumber]}
\begin{block}{Introduced in v6.3 (May 2023)}
\begin{itemize}
\item Sets argv[0] of the command being executed
\item Useful for tracing multi-call executables, such as busybox or kmod
\end{itemize}
\end{block}
\begin{block}{strace -{}-argv0=\textcolor{darkred}{insmod} -{}-trace=execve kmod -{}-help}
\small
\begin{alltt}
	execve("/bin/kmod", ["\textcolor{darkred}{insmod}", "--help"], 0x7fffeb80fa58 /* 42 vars */) = 0
Usage:
	\textcolor{darkred}{insmod} [options] filename [args]
Options:
	-V, --version     show version
	-h, --help        show this help
+++ exited with 0 +++
\end{alltt}
\end{block}
\end{frame}

%%%%%%%
\begin{frame}[fragile]{Display of command names for PIDs: \textbf{-{}-decode-pids}=\textbf{comm}, \textbf{-Y} \hfill [\insertframenumber/\inserttotalframenumber]}
\small
\begin{block}{strace -f \textbf{-Y} -e\%process timeout 1 sleep 2 \hfill since v5.15 (December 2021)}
\scriptsize
\begin{alltt}
execve("/usr/bin/timeout", ["timeout", "1", "sleep", "2"], 0x7ffcfb9bb348 /* 17 vars */) = 0
clone(child_stack=NULL, flags=CLONE_CHILD_CLEARTID|CLONE_CHILD_SETTID|SIGCHLD, \symlinebreak
\symlinecont child_tidptr=0x7f116942da10) = 123452\textcolor{darkgreen}{<timeout>}
strace: Process 123452 attached
[pid 123451\textcolor{darkgreen}{<timeout>}] wait4(123452\textcolor{darkgreen}{<timeout>}, 0x7ffc132b1a28, WNOHANG, NULL) = 0
[pid 123452\textcolor{darkgreen}{<timeout>}] execve("/bin/sleep", ["sleep", "2"], 0x7ffc132b1ce0 /* 17 vars */) = 0
[pid 123451\textcolor{darkgreen}{<timeout>}] --- SIGALRM \{si_signo=SIGALRM, si_code=SI_TIMER, \symlinebreak
\symlinecont si_timerid=0, si_overrun=0, si_int=0, si_ptr=NULL\} ---
[pid 123451\textcolor{darkgreen}{<timeout>}] kill(123452\textcolor{darkred}{<sleep>}, SIGTERM) = 0
[pid 123452\textcolor{darkred}{<sleep>}] --- SIGTERM \{si_signo=SIGTERM, si_code=SI_USER, \symlinebreak
\symlinecont si_pid=123451\textcolor{darkgreen}{<timeout>}, si_uid=1000\} ---
[pid 123451\textcolor{darkgreen}{<timeout>}] kill(0, SIGTERM) = 0
[pid 123451\textcolor{darkgreen}{<timeout>}] --- SIGTERM \{si_signo=SIGTERM, si_code=SI_USER, \symlinebreak
\symlinecont si_pid=123451\textcolor{darkgreen}{<timeout>}, si_uid=1000\} ---
[pid 123451\textcolor{darkgreen}{<timeout>}] kill(123452\textcolor{darkred}{<sleep>}, SIGCONT) = 0
[pid 123452\textcolor{darkred}{<sleep>}] +++ killed by SIGTERM +++
--- SIGCHLD \{si_signo=SIGCHLD, si_code=CLD_KILLED, si_pid=123452\textcolor{darkred}{<sleep>}, \symlinebreak
\symlinecont si_uid=1000, si_status=SIGTERM, si_utime=0, si_stime=0\} ---
kill(0, SIGCONT)                        = 0
--- SIGCONT \{si_signo=SIGCONT, si_code=SI_USER, si_pid=123451\textcolor{darkgreen}{<timeout>}, si_uid=1000\} ---
wait4(123452\textcolor{darkred}{<sleep>}, [{WIFSIGNALED(s) && WTERMSIG(s) == SIGTERM}], WNOHANG, NULL) = 123452
exit_group(124)                         = ?
+++ exited with 124 +++
\end{alltt}
\end{block}
\end{frame}

%%%%%%%
\begin{frame}[fragile]{PID namespace translation: \textbf{-{}-decode-pids}=\textbf{pidns} \hfill [\insertframenumber/\inserttotalframenumber]}
Introduced in v5.15 (December 2021)
\begin{block}{strace \textbf{-{}-decode-pids}=\textbf{pidns} -q -f -e clone,kill \textbackslash \\ unshare -Urpf sh -c 'sleep 2 \& sleep 1 \&\& kill \$!'}
\scriptsize
\begin{alltt}
clone(child_stack=NULL, flags=CLONE_CHILD_CLEARTID|CLONE_CHILD_SETTID|SIGCHLD, \symlinebreak
\symlinecont child_tidptr=0x7faa52812a10) = 123456
[pid 123456] clone(child_stack=NULL, flags=CLONE_CHILD_CLEARTID|CLONE_CHILD_SETTID|SIGCHLD, \symlinebreak
\symlinecont child_tidptr=0x7f277f7d1a10) = 2 \textcolor{darkred}{/* 123457 in strace's PID NS */}
[pid 123456] clone(child_stack=NULL, flags=CLONE_CHILD_CLEARTID|CLONE_CHILD_SETTID|SIGCHLD, \symlinebreak
\symlinecont child_tidptr=0x7f277f7d1a10) = 3 \textcolor{darkred}{/* 123458 in strace's PID NS */}
[pid 123458] +++ exited with 0 +++
[pid 123456] --- SIGCHLD \{si_signo=SIGCHLD, si_code=CLD_EXITED, \symlinebreak
\symlinecont si_pid=3, si_uid=0, si_status=0, si_utime=0, si_stime=0\} ---
[pid 123456] kill(2 \textcolor{darkred}{/* 123457 in strace's PID NS */}, SIGTERM) = 0
[pid 123457] --- SIGTERM \{si_signo=SIGTERM, si_code=SI_USER, \symlinebreak
\symlinecont si_pid=1 \textcolor{darkred}{/* 123456 in strace's PID NS */}, si_uid=0\} ---
[pid 123457] +++ killed by SIGTERM +++
[pid 123456] +++ exited with 0 +++
--- SIGCHLD \{si_signo=SIGCHLD, si_code=CLD_EXITED, si_pid=123456, \symlinebreak
\symlinecont si_uid=0, si_status=0, si_utime=0, si_stime=0\} ---
+++ exited with 0 +++
\end{alltt}
\end{block}
\end{frame}

%%%%%%%
\begin{frame}[fragile]{Display of command names and PID namespace translation \hfill [\insertframenumber/\inserttotalframenumber]}
\large
\textbf{-{}-decode-pids}=\textbf{comm},\textbf{pidns} == \textbf{-{}-decode-pids}=\textbf{all}

\begin{block}{strace \textbf{-{}-decode-pids=all} -q -f -e clone,kill \textbackslash \\ unshare -Urpf sh -c 'sleep 2 \& sleep 1 \&\& kill \$!'}
\scriptsize
\begin{alltt}
clone(child_stack=NULL, flags=CLONE_CHILD_CLEARTID|CLONE_CHILD_SETTID|SIGCHLD, \symlinebreak
\symlinecont child_tidptr=0x7ff30b283a10) = 123456\textcolor{darkgreen}{<unshare>}
[pid 123456\textcolor{darkgreen}{<sh>}] clone(child_stack=NULL, flags=CLONE_CHILD_CLEARTID|CLONE_CHILD_SETTID|SIGCHLD, \symlinebreak
\symlinecont child_tidptr=0x7f6dccd89a10) = 2\textcolor{darkgreen}{<sh>} \textcolor{darkred}{/* 123457 in strace's PID NS */}
[pid 123456\textcolor{darkgreen}{<sh>}] clone(child_stack=NULL, flags=CLONE_CHILD_CLEARTID|CLONE_CHILD_SETTID|SIGCHLD, \symlinebreak
\symlinecont child_tidptr=0x7f6dccd89a10) = 3\textcolor{darkgreen}{<sh>} \textcolor{darkred}{/* 123458 in strace's PID NS */}
[pid 123458\textcolor{darkgreen}{<sleep>}] +++ exited with 0 +++
[pid 123456\textcolor{darkgreen}{<sh>}] --- SIGCHLD \{si_signo=SIGCHLD, si_code=CLD_EXITED, \symlinebreak
\symlinecont si_pid=3, si_uid=0, si_status=0, si_utime=0, si_stime=0\} ---
[pid 123456\textcolor{darkgreen}{<sh>}] kill(2\textcolor{darkgreen}{<sleep>} \textcolor{darkred}{/* 123457 in strace's PID NS */}, SIGTERM) = 0
[pid 123457\textcolor{darkgreen}{<sleep>}] --- SIGTERM \{si_signo=SIGTERM, si_code=SI_USER, \symlinebreak
\symlinecont si_pid=1\textcolor{darkgreen}{<sh>} \textcolor{darkred}{/* 123456 in strace's PID NS */}, si_uid=0\} ---
[pid 123457\textcolor{darkgreen}{<sleep>}] +++ killed by SIGTERM +++
[pid 123456\textcolor{darkgreen}{<sh>}] +++ exited with 0 +++
--- SIGCHLD \{si_signo=SIGCHLD, si_code=CLD_EXITED, si_pid=123456, \symlinebreak
\symlinecont si_uid=0, si_status=0, si_utime=0, si_stime=0\} ---
+++ exited with 0 +++
\end{alltt}
\end{block}
\end{frame}

%%%%%%%
\begin{frame}[fragile]{Decoding signalfd signal masks: \textbf{-{}-decode-fds}=\textbf{signalfd} \hfill [\insertframenumber/\inserttotalframenumber]}
\large
\begin{block}{Introduced in v6.3 (May 2023)}
\begin{itemize}
\item Enables decoding of signal masks associated with signalfd file descriptors
\item \textbf{-{}-decode-fds}=\textbf{signalfd} is a subset of \textbf{-{}-decode-fds}=\textbf{all} == \textbf{-yy}
\end{itemize}
\end{block}

\begin{block}{strace -q -yy --trace=/signalfd,/epoll\_ctl --syscall-limit=3 \textbackslash \\ /usr/lib/systemd/systemd-portabled}
\begin{alltt}
\small
signalfd4(-1, [INT], 8, SFD_CLOEXEC|SFD_NONBLOCK) = 4<signalfd:[\textcolor{darkred}{INT}]>
epoll_ctl(3<anon_inode:[eventpoll]>, EPOLL_CTL_ADD, 4<signalfd:[\textcolor{darkred}{INT}]>, \symlinebreak
\symlinecont {events=EPOLLIN, data={u32=112224560, u64=94214514829616}}) = 0
signalfd4(4<signalfd:[\textcolor{darkred}{INT}]>, [INT TERM], 8, SFD_CLOEXEC|SFD_NONBLOCK) \symlinebreak
\symlinecont = 4<signalfd:[\textcolor{darkred}{INT TERM}]>
\end{alltt}
\end{block}
\end{frame}

%%%%%%%
\begin{frame}[fragile]{Display of SELinux contexts: \textbf{-{}-secontext} \hfill [\insertframenumber/\inserttotalframenumber]}
Introduced in v5.12 (April 2021)

\begin{block}{strace -P /dev/null cat /dev/null}
\scriptsize
\begin{alltt}
openat(AT_FDCWD, "/dev/null", O_RDONLY) = 3
newfstatat(3, "", \{st_mode=S_IFCHR|0666, st_rdev=makedev(0x1, 0x3), ...\}, AT_EMPTY_PATH) = 0
fadvise64(3, 0, 0, POSIX_FADV_SEQUENTIAL) = 0
read(3, "", 131072)                     = 0
close(3)                                = 0
+++ exited with 0 +++
\end{alltt}
\end{block}

\begin{block}{strace \textbf{-{}-secontext} -P /dev/null cat /dev/null}
\scriptsize
\begin{alltt}
[\textcolor{darkgreen}{unconfined_t}] openat(AT_FDCWD, "/dev/null" [\textcolor{darkred}{null_device_t}], O_RDONLY) = 3 [\textcolor{darkred}{null_device_t}]
[\textcolor{darkgreen}{unconfined_t}] newfstatat(3 [\textcolor{darkred}{null_device_t}], "", \{st_mode=S_IFCHR|0666, \symlinebreak
\symlinecont st_rdev=makedev(0x1, 0x3), ...\}, AT_EMPTY_PATH) = 0
[\textcolor{darkgreen}{unconfined_t}] fadvise64(3 [\textcolor{darkred}{null_device_t}], 0, 0, POSIX_FADV_SEQUENTIAL) = 0
[\textcolor{darkgreen}{unconfined_t}] read(3 [\textcolor{darkred}{null_device_t}], "", 131072) = 0
[\textcolor{darkgreen}{unconfined_t}] close(3 [\textcolor{darkred}{null_device_t}]) = 0
+++ exited with 0 +++
\end{alltt}
\end{block}
\end{frame}

%%%%%%%
\begin{frame}[fragile]{Display of SELinux contexts: \textbf{-{}-secontext}=\textbf{full} \hfill [\insertframenumber/\inserttotalframenumber]}
Introduced in v5.12 (April 2021)

\begin{block}{strace \textbf{-{}-secontext}=\textbf{full} -P /dev/null cat /dev/null}
\scriptsize
\begin{alltt}
[\textcolor{darkgreen}{unconfined_u:unconfined_r:unconfined_t:s0-s0:c0.c1023}] \symlinebreak
\symlinecont openat(AT_FDCWD, "/dev/null" [\textcolor{darkred}{system_u:object_r:null_device_t:s0}], O_RDONLY) \symlinebreak
\symlinecont = 3 [\textcolor{darkred}{system_u:object_r:null_device_t:s0}]
[\textcolor{darkgreen}{unconfined_u:unconfined_r:unconfined_t:s0-s0:c0.c1023}] \symlinebreak
\symlinecont newfstatat(3 [\textcolor{darkred}{system_u:object_r:null_device_t:s0}], "", \{st_mode=S_IFCHR|0666, \symlinebreak
\symlinecont st_rdev=makedev(0x1, 0x3), ...\}, AT_EMPTY_PATH) = 0
[\textcolor{darkgreen}{unconfined_u:unconfined_r:unconfined_t:s0-s0:c0.c1023}] \symlinebreak
\symlinecont fadvise64(3 [\textcolor{darkred}{system_u:object_r:null_device_t:s0}], 0, 0, POSIX_FADV_SEQUENTIAL) = 0
[\textcolor{darkgreen}{unconfined_u:unconfined_r:unconfined_t:s0-s0:c0.c1023}] \symlinebreak
\symlinecont read(3 [\textcolor{darkred}{system_u:object_r:null_device_t:s0}], "", 131072) = 0
[\textcolor{darkgreen}{unconfined_u:unconfined_r:unconfined_t:s0-s0:c0.c1023}] \symlinebreak
\symlinecont close(3 [\textcolor{darkred}{system_u:object_r:null_device_t:s0}]) = 0
+++ exited with 0 +++
\end{alltt}
\end{block}
\end{frame}

%%%%%%%
\begin{frame}[fragile]{Display of SELinux contexts: \textbf{-{}-secontext}=\textbf{mismatch} \hfill [\insertframenumber/\inserttotalframenumber]}
\large
\begin{block}{Sample SELinux context mismatch}
\scriptsize
\begin{alltt}
\$ matchpathcon $PWD/config.h
/home/me/strace/src/config.h	unconfined_u:object_r:user_home_t:s0
\$ ls -Z $PWD/config.h
system_u:object_r:user_home_t:s0 /home/me/strace/src/config.h
\end{alltt}
\end{block}

\begin{block}{strace \textbf{-{}-secontext}=\textbf{full},\textbf{mismatch} \hfill since v5.16 (January 2022)}
\scriptsize
\begin{alltt}
\$ strace --secontext=full,mismatch -e \%file stat config.h
\ldots
[unconfined_u:unconfined_r:unconfined_t:s0-s0:c0.c1023] statx(AT_FDCWD, "config.h" \symlinebreak
\symlinecont [\textcolor{darkred}{system_u:object_r:user_home_t:s0!!unconfined_u:object_r:user_home_t:s0}], \symlinebreak
\symlinecont AT_STATX_SYNC_AS_STAT|AT_SYMLINK_NOFOLLOW|AT_NO_AUTOMOUNT, \symlinebreak
\symlinecont STATX_ALL, {stx_mask=STATX_ALL|STATX_MNT_ID, stx_attributes=0, stx_mode=S_IFREG|0644, \symlinebreak
\symlinecont stx_size=50008, ...}) = 0
\end{alltt}
\end{block}

Hint: combine \textbf{-{}-secontext}=\textbf{full},\textbf{mismatch} with \textbf{-Z}
\end{frame}

%%%%%%%
\begin{frame}[fragile]{Display of syscall numbers: \textbf{-{}-syscall-number}, \textbf{-n} \hfill [\insertframenumber/\inserttotalframenumber]}
Introduced in v5.9 (September 2020)

\begin{block}{x86\_64: timeout 1 strace \textbf{-n} -e\%net -qDD nc -l 127.0.0.1 1234}
\scriptsize
\begin{alltt}
[  41] socket(AF_INET, SOCK_STREAM, IPPROTO_TCP) = 3
[  54] setsockopt(3, SOL_SOCKET, SO_REUSEADDR, [1], 4) = 0
[  49] bind(3, \{sa_family=AF_INET, sin_port=htons(1234), sin_addr=inet_addr("127.0.0.1")\}, 16) = 0
[  50] listen(3, 1)                     = 0
[  43] accept(3, 0x7fff307140e0, [128]) = ? ERESTARTSYS (To be restarted if SA_RESTART is set)
[  43] --- SIGTERM \{si_signo=SIGTERM, si_code=SI_USER, si_pid=1234567, si_uid=1000\} ---
[  43] +++ killed by SIGTERM +++
\end{alltt}
\end{block}

\begin{block}{x86: timeout 1 strace \textbf{-n} -e\%net -qDD nc -l 127.0.0.1 1234}
\scriptsize
\begin{alltt}
[ 102] socket(AF_INET, SOCK_STREAM, IPPROTO_TCP) = 3
[ 102] setsockopt(3, SOL_SOCKET, SO_REUSEADDR, [1], 4) = 0
[ 102] bind(3, \{sa_family=AF_INET, sin_port=htons(1234), sin_addr=inet_addr("127.0.0.1")\}, 16) = 0
[ 102] listen(3, 1)                     = 0
[ 102] accept(3, 0xfff1a52c, [128])     = ? ERESTARTSYS (To be restarted if SA_RESTART is set)
[ 102] --- SIGTERM \{si_signo=SIGTERM, si_code=SI_USER, si_pid=1234567, si_uid=1000\} ---
[ 102] +++ killed by SIGTERM +++
\end{alltt}
\end{block}
\end{frame}

%%%%%%%
\begin{frame}{System call return status filtering: \textbf{-e status}=\textit{set} \hfill [\insertframenumber/\inserttotalframenumber]}
\Large
\begin{block}{Introduced in v5.2 (July 2019)}
\large
Print only syscalls with the specified return status. \\
\textit{set} can include the following elements:
\begin{description}
	\item[successful]: returned without an error code, alias to \textbf{-z}
	\item[failed]: returned with an error code, alias to \textbf{-Z}
	\item[unfinished]: did not return
	\item[detached]: detached before return
	\item[unavailable]: returned but failed to fetch the error status
\end{description}
\end{block}
The default is \textbf{-e status}=\textbf{all}.
\end{frame}

%%%%%%%
\begin{frame}[fragile]{System call return status filtering: \textbf{-z}, \textbf{-Z} \hfill [\insertframenumber/\inserttotalframenumber]}
\scriptsize
\begin{block}{env -i LD\_LIBRARY\_PATH=/lib64 strace -z -e\%file /bin/cat < /dev/null}
\begin{alltt}
execve("/bin/cat", ["/bin/cat"], 0x7ffc2b781ca0 /* 1 var */) = 0
openat(AT_FDCWD, "/lib64/libc.so.6", O_RDONLY|O_CLOEXEC) = 3
+++ exited with 0 +++
\end{alltt}
\end{block}
\begin{block}{env -i LD\_LIBRARY\_PATH=/lib64 strace -Z -e\%file /bin/cat </dev/null}
\begin{alltt}
access("/etc/ld.so.preload", R_OK) = -1 ENOENT (No such file or directory)
openat(AT_FDCWD, "/lib64/tls/x86_64/x86_64/libc.so.6", O_RDONLY|O_CLOEXEC) = -1 ENOENT (No suc
stat("/lib64/tls/x86_64/x86_64", 0x7ffdcc6a0c20) = -1 ENOENT (No such file or directory)
openat(AT_FDCWD, "/lib64/tls/x86_64/libc.so.6", O_RDONLY|O_CLOEXEC) = -1 ENOENT (No such file
stat("/lib64/tls/x86_64", 0x7ffdcc6a0c20) = -1 ENOENT (No such file or directory)
openat(AT_FDCWD, "/lib64/tls/x86_64/libc.so.6", O_RDONLY|O_CLOEXEC) = -1 ENOENT (No such file
stat("/lib64/tls/x86_64", 0x7ffdcc6a0c20) = -1 ENOENT (No such file or directory)
openat(AT_FDCWD, "/lib64/tls/libc.so.6", O_RDONLY|O_CLOEXEC) = -1 ENOENT (No such file or dire
stat("/lib64/tls", 0x7ffdcc6a0c20) = -1 ENOENT (No such file or directory)
openat(AT_FDCWD, "/lib64/x86_64/x86_64/libc.so.6", O_RDONLY|O_CLOEXEC) = -1 ENOENT (No such fi
stat("/lib64/x86_64/x86_64", 0x7ffdcc6a0c20) = -1 ENOENT (No such file or directory)
openat(AT_FDCWD, "/lib64/x86_64/libc.so.6", O_RDONLY|O_CLOEXEC) = -1 ENOENT (No such file or d
stat("/lib64/x86_64", 0x7ffdcc6a0c20) = -1 ENOENT (No such file or directory)
openat(AT_FDCWD, "/lib64/x86_64/libc.so.6", O_RDONLY|O_CLOEXEC) = -1 ENOENT (No such file or d
stat("/lib64/x86_64", 0x7ffdcc6a0c20) = -1 ENOENT (No such file or directory)
+++ exited with 0 +++
\end{alltt}
\end{block}
\end{frame}

%%%%%%%
\begin{frame}[fragile]{System call return status filtering: \textbf{-z} aggregation \hfill [\insertframenumber/\inserttotalframenumber]}
\begin{block}{strace -o log -f -e signal=none -e trace=execve,nanosleep \textbackslash \\
	sh -c 'sleep 0.1 \& sleep 0.2 \& sleep 0.3' \&\& cat log}
\scriptsize
\begin{alltt}
13475 execve("/bin/sh", ["sh", "-c", "sleep 0.1 & sleep 0.2 & sleep 0."...],
      0x5631be4f87a8 /* 42 vars */) = 0
13476 execve("/bin/sleep", ["sleep", "0.1"], 0xe4c4f0 /* 33 vars */ <unfinished ...>
13477 execve("/bin/sleep", ["sleep", "0.2"], 0xe4c4f0 /* 33 vars */ <unfinished ...>
13478 execve("/bin/sleep", ["sleep", "0.3"], 0xe4c4f0 /* 33 vars */ <unfinished ...>
13476 <... execve resumed>)             = 0
13477 <... execve resumed>)             = 0
13478 <... execve resumed>)             = 0
13476 nanosleep({tv_sec=0, tv_nsec=100000000},  <unfinished ...>
13477 nanosleep({tv_sec=0, tv_nsec=200000000},  <unfinished ...>
13478 nanosleep({tv_sec=0, tv_nsec=300000000},  <unfinished ...>
13476 <... nanosleep resumed>NULL)      = 0
13476 +++ exited with 0 +++
13477 <... nanosleep resumed>NULL)      = 0
13477 +++ exited with 0 +++
13478 <... nanosleep resumed>NULL)      = 0
13478 +++ exited with 0 +++
13475 +++ exited with 0 +++
\end{alltt}
\end{block}
\end{frame}

%%%%%%%
\begin{frame}[fragile]{System call return status filtering: \textbf{-z} aggregation \hfill [\insertframenumber/\inserttotalframenumber]}
\begin{block}{strace -o log \textbf{-z} -f -e signal=none -e trace=execve,nanosleep \textbackslash \\
	sh -c 'sleep 0.1 \& sleep 0.2 \& sleep 0.3' \&\& cat log}
\small
\begin{alltt}
13475 execve("/bin/sh", ["sh", "-c", "sleep 0.1 & sleep 0.2 & sleep 0."...],
      0x5631be4f87a8 /* 42 vars */) = 0
13476 execve("/bin/sleep", ["sleep", "0.1"], 0xe4c4f0 /* 33 vars */) = 0
13477 execve("/bin/sleep", ["sleep", "0.2"], 0xe4c4f0 /* 33 vars */) = 0
13478 execve("/bin/sleep", ["sleep", "0.3"], 0xe4c4f0 /* 33 vars */) = 0
13476 nanosleep({tv_sec=0, tv_nsec=100000000}, NULL) = 0
13476 +++ exited with 0 +++
13477 nanosleep({tv_sec=0, tv_nsec=200000000}, NULL) = 0
13477 +++ exited with 0 +++
13478 nanosleep({tv_sec=0, tv_nsec=300000000}, NULL) = 0
13478 +++ exited with 0 +++
13475 +++ exited with 0 +++
\end{alltt}
\end{block}
\end{frame}

%%%%%%%
\begin{frame}[fragile]{Filtering syscalls operating on specific fds: \textbf{-{}-trace-fds}=\textit{set} \hfill [\insertframenumber/\inserttotalframenumber]}
\large
\begin{block}{Introduced in v6.3 (May 2023)}
Trace only those syscalls that operate on the specified set of file descriptors.
\end{block}
\begin{block}{strace \textbf{-{}-trace-fds}=0,2 cat < /dev/null }
\small
\begin{alltt}
fstat(0, \{st_mode=S_IFCHR|0666, st_rdev=makedev(0x1, 0x3), ...\}) = 0
fadvise64(0, 0, 0, POSIX_FADV_SEQUENTIAL) = 0
read(0, "", 131072)                     = 0
close(0)                                = 0
close(2)                                = 0
+++ exited with 0 +++
\end{alltt}
\end{block}
\end{frame}

%%%%%%%
\begin{frame}{Syscall filtering classes: \textbf{\%clock}, \textbf{\%creds} \hfill [\insertframenumber/\inserttotalframenumber]}
\begin{block}{Introduced in v5.5 (February 2020)}
\textbf{\%creds}: syscalls related to process credentials.
This includes
\begin{description}
	\item[uid] getuid, geteuid, getresuid, setuid, setreuid, setresuid, setfsuid
	\item[gid] getgid, getegid, getresgid, setgid, setregid, setresgid, setfsgid
	\item[groups] getgroups, setgroups
	\item[caps] capget, capset, prctl
\end{description}
\end{block}

\begin{block}{Introduced in v5.10 (December 2020)}
\textbf{\%clock}: syscalls reading or modifying system clocks.
This includes
\begin{description}
	\item[get] clock\_gettime, clock\_getres, gettimeofday, time
	\item[set] clock\_settime, settimeofday
	\item[tune] clock\_adjtime, adjtimex
\end{description}
\end{block}
\end{frame}

%%%%%%%
\begin{frame}[fragile]{Poke injection: \textbf{poke\_enter}=@\textit{argN}=\textit{dataN} \hfill [\insertframenumber/\inserttotalframenumber]}
\begin{block}{Introduced in v5.11 (February 2021)}
\large
Tracee's memory at the location pointed to by syscall argument \textit{argN} \\ is overwritten by \textit{dataN} on entering syscall
\end{block}

\begin{block}{strace -P /etc/shadow cat /etc/shadow}
\begin{alltt}
openat(AT_FDCWD, "/etc/shadow", O_RDONLY) = -1 EACCES (Permission denied)
cat: /etc/shadow: Permission denied
+++ exited with 1 +++
\end{alltt}
\end{block}

\begin{block}{hex=2F6465762F6E756C6C00 \\ strace -P /etc/shadow \textbackslash \\ --inject=openat:poke\_enter=@arg2=\$hex cat /etc/shadow}
\begin{alltt}
openat(AT_FDCWD, "/etc/shadow", O_RDONLY) = 3 (INJECTED: args)
+++ exited with 0 +++
\end{alltt}
\end{block}
\end{frame}

%%%%%%%
\begin{frame}[fragile]{Poke injection: \textbf{poke\_exit}=@\textit{argN}=\textit{dataN} \hfill [\insertframenumber/\inserttotalframenumber]}
\begin{block}{Introduced in v5.11 (February 2021)}
\large
Tracee's memory at the location pointed to by syscall argument \textit{argN} \\ is overwritten by \textit{dataN} on exiting syscall
\end{block}

\begin{block}{strace -s256 -e readlink readlink /etc/localtime}
\small
\begin{alltt}
readlink("/etc/localtime", "../usr/share/zoneinfo/Asia/Jerusalem", 64) = 36
../usr/share/zoneinfo/Asia/Jerusalem
+++ exited with 0 +++
\end{alltt}
\end{block}

\begin{block}{hex=2E2E2F7573722F73686172652F7A6F6E65696E666F2F4574632F555443 \\ strace -e readlink --inject=readlink:retval=29:poke\_exit=@arg2=\$hex \textbackslash \\ readlink /etc/localtime}
\scriptsize
\begin{alltt}
readlink("/etc/localtime", "../usr/share/zoneinfo/Etc/UTC", 64) = 29 (INJECTED: args, retval)
../usr/share/zoneinfo/Etc/UTC
+++ exited with 0 +++
\end{alltt}
\end{block}
\end{frame}

%%%%%%%
\begin{frame}[fragile]{Call summary configuration: \textbf{-{}-summary-columns}=\textit{set} \hfill [\insertframenumber/\inserttotalframenumber]}
\begin{block}{strace -c cat < /dev/null}
\small
\begin{alltt}
\% time     seconds  usecs/call     calls    errors syscall
------ ----------- ----------- --------- --------- ----------------
 31.54    0.000205           6        30        13 openat
 23.23    0.000151           6        22           mmap
 14.00    0.000091           4        19           newfstatat
 12.92    0.000084           4        20           close
  4.46    0.000029           9         3           mprotect
  4.15    0.000027          13         2           munmap
  2.46    0.000016           4         4           read
  1.54    0.000010           3         3           brk
  0.92    0.000006           6         1           getrandom
\ldots
------ ----------- ----------- --------- --------- ----------------
100.00    0.000650           5       116        15 total
\end{alltt}
\end{block}
\end{frame}

%%%%%%%
\begin{frame}[fragile]{Call summary configuration: \textbf{-{}-summary-columns}=\textit{set}, \textbf{-U} \textit{set} \hfill [\insertframenumber/\inserttotalframenumber]}
\begin{block}{Introduced in v5.6 (April 2020)}
\begin{itemize}
\item Selects the set of displayed columns in the call summary output
\end{itemize}
\end{block}

\begin{block}{strace -c -U min-time,max-time,avg-time,calls cat < /dev/null}
\small
\begin{alltt}
shortest  longest  usecs/call     calls syscall
-------- -------- ----------- --------- ----------------
0.000000 0.000016           5        30 openat
0.000000 0.000010           3        22 mmap
0.000000 0.000019           4        19 newfstatat
0.000000 0.000005           2        20 close
0.000009 0.000018          12         3 mprotect
0.000005 0.000015          10         2 munmap
0.000000 0.000006           3         4 read
0.000000 0.000006           3         3 brk
0.000005 0.000005           5         1 getrandom
\ldots
-------- -------- ----------- --------- ----------------
0.000000 0.000019           4       116 total
\end{alltt}
\end{block}
\end{frame}

%%%%%%%
\begin{frame}[fragile]{Show strace tips, tricks, and tweaks: \textbf{-{}-tips} \hfill [\insertframenumber/\inserttotalframenumber]}
\large
\begin{block}{Introduced in v5.18 (June 2022) and v6.3 (May 2023)}
Show strace tips, tricks, and tweaks before exit.
\end{block}
\begin{block}{strace \textbf{-{}-tips}[=[[\textbf{id}:]\textit{id}],[[\textbf{format}:]\textit{format}]]}
\begin{itemize}
\item \textit{id} can be one of the following:
\begin{description}
	\item[\textbf{random}] prints a random tip (the default)
	\item[\textit{integer}] prints the tip, trick, or tweak \\ corresponding to the specified number
\end{description}
\item \textit{format} can be one of the following:
\begin{description}
	\item[\textbf{compact}] prints the tip just big enough to contain all the text \\ (the default)
	\item[\textbf{full}] prints the tip in its full glory
	\item[\textbf{none}] prints no tip
\end{description}
\end{itemize}
\end{block}
\end{frame}

%%%%%%%
\begin{frame}[fragile]{Show strace tips, tricks, and tweaks: \textbf{-{}-tips} \hfill [\insertframenumber/\inserttotalframenumber]}
\large
\begin{block}{strace --tips=id:31}
\begin{alltt}
 ______________________________________________         ____
/                                              \textbackslash       /    \textbackslash
| Medicinal effects of strace can be achieved  |      |-. .-.|
| by invoking it with the following options:   \textbackslash      (_@)(_@)
|                                               \textbackslash     .---_  \textbackslash
|     strace -DDDqqq -enone --signal=none       _\textbackslash   /..   \textbackslash_/
|                                              /     |__.-^ /
|                                              |         \}  |
\textbackslash______________________________________________/        |   [
                                                        [  ]
\end{alltt}
\end{block}

\begin{block}{\url{https://github.com/strace/strace/issues/14}}
Document how to just get the 'medicinal' effects of strace with no overhead.
\end{block}
\end{frame}

%%%%%%%
\begin{frame}{More strace talks at DevConf.cz 2023 \hfill [\insertframenumber/\inserttotalframenumber]}
\Large
\begin{block}{Current state of netlink decoding in strace}
\begin{description}
	\item[who] Eugene Syromiatnikov
	\item[when] Saturday, June 17, 1:15pm -- 1:50pm
	\item[where] G202
\end{description}
\end{block}

\begin{block}{Using strace to troubleshoot issues}
\begin{description}
	\item[who] Renaud Metrich
	\item[when] Saturday, June 17, 2:45pm -- 3:20pm
	\item[where] G202
\end{description}
\end{block}
\end{frame}

%%%%%%%
{
\setbeamertemplate{footline}{\pgfuseimage{left-corner-logo}}
\begin{frame}[noframenumbering]{Questions?}
	\begin{columns}
		\column{9cm}
\begin{block}{\large homepage}
	\url{https://strace.io}
\end{block}
\begin{block}{\large strace.git, strace-talks.git}
	\url{https://github.com/strace/strace.git}

	\url{https://gitlab.com/strace/strace.git}

\smallskip

	\url{https://github.com/strace/strace-talks.git}

	\url{https://gitlab.com/strace/strace-talks.git}
\end{block}
\begin{block}{\large mailing list}
	strace-devel@lists.strace.io
\end{block}
\begin{block}{\large IRC channel}
	\#strace@oftc
\end{block}
		\column{3cm}
			\centerline{\pgfuseimage{strace-logo}}
	\end{columns}
\end{frame}
}

\end{document}
