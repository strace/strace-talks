% Copyright (C) 2012-2018 Dmitry V. Levin <ldv@altlinux.org>
% Permission is granted to copy, distribute and/or modify this document
% under the terms of the GNU Free Documentation License, Version 1.2
% or any later version published by the Free Software Foundation;
% with no Invariant Sections, no Front-Cover Texts, and no Back-Cover Texts.

\documentclass[unicode]{beamer}

\mode<presentation>
{
	\usetheme{Warsaw}
	\setbeamertemplate{headline}{}
	\setbeamertemplate{footline}{}
}

\usepackage[utf8]{inputenc}
\usepackage[T2A]{fontenc}
\usepackage[english]{babel}
\usepackage{alltt}

\title{\Huge Modern strace}
\author{\Huge Dmitry~Levin}
\institute[BaseALT]{\Large BaseALT}
\date{OSSDEVCONF 2018}

\logo{\includegraphics[height=1cm]{strace-straus.pdf}}

\begin{document}

%%%%%%%
\begin{frame}
\titlepage
\end{frame}

%%%%%%%
\begin{frame}{Modern strace \hfill [1/2]}
\begin{block}{\large Tracing output format}
\begin{itemize}
\item pathnames accessed by name or descriptor: \textbf{-y} option
\item network protocol associated with descriptors: \textbf{-yy} option
\item stack of function calls: \textbf{-k} option
\item open \textbf{-o} in append mode: \textbf{-A} option
\item format of named constants and flags: \textbf{-X} option
\end{itemize}
\end{block}

\begin{block}{\large System call filtering}
\begin{itemize}
\item pathnames accessed by name or descriptor: \textbf{-P} option
\item regular expressions: -e \textbf{trace}=/\textit{regexp}
\item optional specifications: -e \textbf{trace}=?\textit{spec}
\item new syscall classes: \%stat, \%lstat, \%fstat, \\ \%statfs, \%fstatfs, \%\%stat, \%\%statfs
\end{itemize}
\end{block}
\end{frame}

%%%%%%%
\begin{frame}{Modern strace \hfill [2/2]}
\begin{block}{\large System call statistics}
\begin{itemize}
\item wall clock time spent in syscalls: \textbf{-w} option
\item combine statistics with regular output: \textbf{-C} option
\end{itemize}
\end{block}

\begin{block}{\large Tracing control}
\begin{itemize}
\item attaching to multiple processes: \textbf{-p} option
%\item detaching on execve: \textbf{-b execve} option
\item running as a detached grandchild: \textbf{-D} option
\item interruptibility: \textbf{-I} option
\item postprocessing: \textbf{strace-log-merge}
\end{itemize}
\end{block}

\begin{block}{\large System call tampering}
\begin{itemize}
\item fault injection
\item return value injection
\item signal injection
\item delay injection
\end{itemize}
\end{block}

\end{frame}

%%%%%%%
\begin{frame}[fragile]{regular expressions: -e trace=/\textit{regexp} \hfill [1/2]}
\begin{block}{glibc: open or openat?}
\begin{alltt}
glibc-2.25$ strace -qq \textcolor{red}{-e open} cat /dev/null
open("/etc/ld.so.cache", O_RDONLY|O_CLOEXEC) = 3
open("/lib64/libc.so.6", O_RDONLY|O_CLOEXEC) = 3
open("/dev/null", O_RDONLY)             = 3
glibc-2.26$ strace -qq \textcolor{red}{-e open} cat /dev/null
glibc-2.26$ strace -qq \textcolor{red}{-e openat} cat /dev/null
openat(AT_FDCWD, "/etc/ld.so.cache", O_RDONLY|O_CLOEXEC) = 3
openat(AT_FDCWD, "/lib64/libc.so.6", O_RDONLY|O_CLOEXEC) = 3
openat(AT_FDCWD, "/dev/null", O_RDONLY) = 3
glibc-2.25$ strace -qq \textcolor{red}{-e openat} cat /dev/null
\end{alltt}
\end{block}

\begin{block}{traditional solution is not portable}
\begin{alltt}
riscv$ strace \textcolor{red}{-e open,openat}
strace: invalid system call 'open'
\end{alltt}
\end{block}
\end{frame}

%%%%%%%
\begin{frame}[fragile]{regular expressions: -e trace=/\textit{regexp} \hfill [2/2]}
\begin{block}{naive solution is inexact}
\small
\begin{alltt}
$ asinfo --set-arch x86_64,riscv --list-abi \textbackslash
  --nargs --get-sname \textcolor{red}{/open}
|   |                   | x86_64 | x86_64 | riscv | riscv |
| N |      Syscall name |  64bit |  32bit | 64bit | 32bit |
| 1 |           mq_open |      4 |      4 |     4 |     4 |
\textcolor{red}{| 2 |              open |      3 |      3 |     - |     - |}
| 3 | open_by_handle_at |      3 |      3 |     3 |     3 |
\textcolor{red}{| 4 |            openat |      4 |      4 |     4 |     4 |}
| 5 |   perf_event_open |      5 |      5 |     5 |     5 |
\end{alltt}
\end{block}

\begin{block}{accurate and portable solution}
\texttt{\$ strace -e '/{\textasciicircum}open(at)?\$'}
\end{block}
\end{frame}

%%%%%%%
\begin{frame}[fragile]{New system call classes \hfill [1/3]}
\begin{block}{\large traditional syscall classes}
\begin{description}
	\item[desc]: take or return a descriptor
	\item[file]: take a file name
	\item[memory]: memory mapping, memory policy
	\item[process]: process management
	\item[signal]: signal related
	\item[ipc]: SysV IPC related
	\item[network]: network related
\end{description}
\begin{alltt}
$ strace -e trace=\textit{class}
$ strace -e \textit{class}
\end{alltt}
\end{block}

\begin{block}{\large all syscall classes now have \% prefix}
\begin{alltt}
$ strace -e trace=%\textit{class}
$ strace -e %\textit{class}
\end{alltt}
\end{block}
\end{frame}

%%%%%%%
\begin{frame}[fragile]{New system call classes \hfill [2/3]}
\begin{block}{\large new syscall classes}
\begin{itemize}
	\item \%stat, \%lstat, \%fstat
	\item \%statfs, \%fstatfs
	\item \%\%stat = \%stat + \%lstat + \%fstat + statx
	\item \%\%statfs = \%statfs + \%fstatfs + ustat
\end{itemize}
\end{block}

\begin{block}{\large strace -y -e \%\%stat ls /var/empty}
\begin{alltt}
\textcolor{red}{fstat}(3</etc/ld.so.cache>, {st_mode=S_IFREG|0644, st_size=30341, ...}) = 0
...
\textcolor{red}{fstat}(3</proc/filesystems>, {st_mode=S_IFREG|0444, st_size=0, ...}) = 0
\textcolor{red}{stat}("/var/empty", {st_mode=S_IFDIR|0555, st_size=40, ...}) = 0
\textcolor{red}{fstat}(3</var/empty>, {st_mode=S_IFDIR|0555, st_size=40, ...}) = 0
+++ exited with 0 +++
\end{alltt}
\end{block}
\end{frame}

%%%%%%%
\begin{frame}[fragile]{New system call classes \hfill [3/3]}
\begin{block}{\large \%stat + \%lstat + \%fstat + statx = \%\%stat}
\begin{alltt}
$ asinfo --set-arch x86_64,riscv --list-abi \textbackslash
  --nargs --get-sname \%\%stat
|    |              | x86_64 | x86_64 | riscv | riscv |
|  N | Syscall name |  64bit |  32bit | 64bit | 32bit |
\textcolor{blue}{|  1 |        fstat |      2 |      2 |     2 |     - |}
\textcolor{blue}{|  2 |      fstat64 |      - |      2 |     - |     2 |}
\textcolor{blue}{|  3 |    fstatat64 |      - |      4 |     - |     4 |}
\textcolor{green}{|  4 |        lstat |      2 |      2 |     - |     - |}
\textcolor{green}{|  5 |      lstat64 |      - |      2 |     - |     - |}
\textcolor{blue}{|  6 |   newfstatat |      4 |      - |     4 |     - |}
\textcolor{blue}{|  7 |     oldfstat |      - |      2 |     - |     - |}
\textcolor{green}{|  8 |     oldlstat |      - |      2 |     - |     - |}
\textcolor{red}{|  9 |      oldstat |      - |      2 |     - |     - |}
\textcolor{red}{| 10 |         stat |      2 |      2 |     - |     - |}
\textcolor{red}{| 11 |       stat64 |      - |      2 |     - |     - |}
| 12 |        statx |      5 |      5 |     5 |     5 |
\end{alltt}
\end{block}
\end{frame}

%%%%%%%
\begin{frame}[fragile]{Network protocol associated with descriptors: \textbf{-yy} option}
\begin{block}{\large strace \textbf{-yy} -e \%desc,\%network netcat 127.0.0.1 22 </dev/null}
\scriptsize
\begin{alltt}
\ldots
socket(AF_INET, SOCK_STREAM, IPPROTO_TCP) = 3\textcolor{red}{<TCP:[518663]>}
connect(3\textcolor{red}{<TCP:[518663]>}, {sa_family=AF_INET, sin_port=htons(22),
        sin_addr=inet_addr("127.0.0.1")}, 16) = 0
poll([{fd=3\textcolor{red}{<TCP:[127.0.0.1:45678->127.0.0.1:22]>}, events=POLLIN},
     {fd=0\textcolor{red}{</dev/null<char 1:3>>}, events=POLLIN}], 2, -1)
     = 1 ([{fd=0, revents=POLLIN}])
read(0\textcolor{red}{</dev/null<char 1:3>>}, "", 2048)  = 0
shutdown(3\textcolor{red}{<TCP:[127.0.0.1:45678->127.0.0.1:22]>}, SHUT_WR) = 0
poll([{fd=3\textcolor{red}{<TCP:[127.0.0.1:45678->127.0.0.1:22]>}, events=POLLIN},
     {fd=-1}], 2, -1) = 1 ([{fd=3, revents=POLLIN}])
read(3\textcolor{red}{<TCP:[127.0.0.1:45678->127.0.0.1:22]>},
     "SSH-2.0-OpenSSH_7.2{\textbackslash}r{\textbackslash}n", 2048) = 21
write(1\textcolor{red}{</dev/pts/9<char 136:9>>},
      "SSH-2.0-OpenSSH_7.2{\textbackslash}r{\textbackslash}n", 21) = 21
poll([{fd=3\textcolor{red}{<TCP:[127.0.0.1:45678->127.0.0.1:22]>}, events=POLLIN},
     {fd=-1}], 2, -1) = 1 ([{fd=3, revents=POLLIN|POLLHUP}])
read(3\textcolor{red}{<TCP:[127.0.0.1:45678->127.0.0.1:22]>}, "", 2048) = 0
shutdown(3\textcolor{red}{<TCP:[127.0.0.1:45678->127.0.0.1:22]>}, SHUT_RD)
         = -1 ENOTCONN (Transport endpoint is not connected)
close(3\textcolor{red}{<TCP:[127.0.0.1:45678->127.0.0.1:22]>}) = 0
+++ exited with 0 +++
\end{alltt}
\end{block}
\end{frame}

%%%%%%%
\begin{frame}[fragile]{Stack of function calls: \textbf{-k} option}
\begin{block}{\large strace -qq -P /dev/full cat /dev/null > /dev/full}
\scriptsize
\begin{alltt}
fstat(1, {st_mode=S_IFCHR|0666, st_rdev=makedev(1, 7), ...}) = 0
close(1)                                = 0
\end{alltt}
\end{block}

\begin{block}{\large strace \textbf{-k} -qq -P /dev/full cat /dev/null > /dev/full}
\scriptsize
\begin{alltt}
fstat(1, {st_mode=S_IFCHR|0666, st_rdev=makedev(1, 7), ...}) = 0
 > /lib64/libc-2.27.so(__fxstat64+0x13) [0xe79c3]
 > /bin/cat(main+0x1b3) [0x4017e3]
 > /lib64/libc-2.27.so(__libc_start_main+0xe6) [0x21bd6]
 > /bin/cat(_start+0x29) [0x402179]
close(1)                                = 0
 > /lib64/libc-2.27.so(__close_nocancel+0x7) [0xe8b47]
 > /lib64/libc-2.27.so(_IO_file_close_it@@GLIBC_2.2.5+0x67) [0x79fd7]
 > /lib64/libc-2.27.so(fclose@@GLIBC_2.2.5+0x136) [0x6d376]
 > /bin/cat(close_stream+0x19) [0x404ce9]
 > /bin/cat(close_stdout+0x11) [0x402691]
 > /lib64/libc-2.27.so(__run_exit_handlers+0x170) [0x379c0]
 > /lib64/libc-2.27.so(exit+0x19) [0x37ab9]
 > /lib64/libc-2.27.so(__libc_start_main+0xed) [0x21bdd]
 > /bin/cat(_start+0x29) [0x402179]
\end{alltt}
\end{block}
\end{frame}

%%%%%%%
\begin{frame}[fragile]{Format of named constants and flags: \textbf{-X} option}
\begin{block}{\large strace -e /open cat /dev/null}
\scriptsize
\begin{alltt}
openat(\textcolor{red}{AT_FDCWD}, "/etc/ld.so.cache", \textcolor{red}{O_RDONLY|O_CLOEXEC}) = 3
openat(\textcolor{red}{AT_FDCWD}, "/lib64/libc.so.6", \textcolor{red}{O_RDONLY|O_CLOEXEC}) = 3
openat(\textcolor{red}{AT_FDCWD}, "/dev/null", \textcolor{red}{O_RDONLY}) = 3
+++ exited with 0 +++
\end{alltt}
\end{block}

\begin{block}{\large strace \textbf{-X verbose} -e /open cat /dev/null}
\scriptsize
\begin{alltt}
openat(\textcolor{red}{-100 /* AT_FDCWD */}, "/etc/ld.so.cache",
       \textcolor{red}{0x80000 /* O_RDONLY|O_CLOEXEC */}) = 3
openat(\textcolor{red}{-100 /* AT_FDCWD */}, "/lib64/libc.so.6",
       \textcolor{red}{0x80000 /* O_RDONLY|O_CLOEXEC */}) = 3
openat(\textcolor{red}{-100 /* AT_FDCWD */}, "/dev/null", \textcolor{red}{0 /* O_RDONLY */}) = 3
+++ exited with 0 +++
\end{alltt}
\end{block}

\begin{block}{\large strace \textbf{-X raw} -e /open cat /dev/null}
\scriptsize
\begin{alltt}
openat(\textcolor{red}{-100}, "/etc/ld.so.cache", \textcolor{red}{0x80000}) = 3
openat(\textcolor{red}{-100}, "/lib64/libc.so.6", \textcolor{red}{0x80000}) = 3
openat(\textcolor{red}{-100}, "/dev/null", \textcolor{red}{0})            = 3
+++ exited with 0 +++
\end{alltt}
\end{block}
\end{frame}

%%%%%%%
\begin{frame}[fragile]{System time spent in syscalls: \textbf{-c} option}
\begin{block}{\large strace \textbf{-c} sleep 1}
\begin{alltt}
% time  seconds usecs/call calls errors syscall
------ -------- ---------- ----- ------ ----------
 31.45 0.000078         19     4        close
 25.40 0.000063         15     4        mprotect
\textcolor{red}{ 12.90 0.000032         32     1        nanosleep}
 11.29 0.000028          5     5        mmap
  8.47 0.000021          5     4        brk
  8.06 0.000020         20     1        munmap
  2.42 0.000006          6     1        arch_prctl
  0.00 0.000000          0     1        read
  0.00 0.000000          0     2        fstat
  0.00 0.000000          0     1      1 access
  0.00 0.000000          0     1        execve
  0.00 0.000000          0     2        openat
------ -------- ---------- ----- ------ ----------
100.00 0.000248               27      1 total
\end{alltt}
\end{block}
\end{frame}

%%%%%%%
\begin{frame}[fragile]{Wall clock time spent in syscalls: \textbf{-c -w} option}
\begin{block}{\large strace \textbf{-c -w} sleep 1}
\begin{alltt}
% time  seconds usecs/call calls errors syscall
------ -------- ---------- ----- ------ ----------
\textcolor{red}{ 99.91 1.000184    1000183     1        nanosleep}
  0.04 0.000367        366     1        execve
  0.02 0.000216         53     4        close
  0.01 0.000087         17     5        mmap
  0.01 0.000075         18     4        mprotect
  0.01 0.000052         13     4        brk
  0.00 0.000037         18     2        openat
  0.00 0.000027         26     1        munmap
  0.00 0.000024         12     2        fstat
  0.00 0.000019         19     1      1 access
  0.00 0.000015         14     1        read
  0.00 0.000013         13     1        arch_prctl
------ -------- ---------- ----- ------ ----------
100.00 1.001116               27      1 total
\end{alltt}
\end{block}
\end{frame}

%%%%%%%
\begin{frame}[fragile]{Running as a detached grandchild: \textbf{-D} option}
\begin{block}{\large foreground strace}
\begin{alltt}
$ echo $$ && strace -e none sh -c 'echo $PPID'
1234
23456
+++ exited with 0 +++
$ echo $$ && strace -e none sh -c 'echo $PPID'
1234
23459
+++ exited with 0 +++
\end{alltt}
\end{block}

\begin{block}{\large background strace}
\begin{alltt}
$ echo $$ && strace \textcolor{red}{-D} -e none sh -c 'echo $PPID'
1234
1234
+++ exited with 0 +++
\end{alltt}
\end{block}
\end{frame}

%%%%%%%
\begin{frame}[fragile]{Postprocessing: \textbf{strace-log-merge}}
\scriptsize
\begin{alltt}
$ strace -o log -ff -tt -e trace=execve,nanosleep \textbackslash
      sh -c 'sleep 0.1 & sleep 0.2 & sleep 0.3'
$ strace-log-merge log
13475 21:13:52.040837 execve("/bin/sh", ["sh", "-c", "sleep 0.1 &
      sleep 0.2 & sleep 0."...], 0x7ffde54b2450 /* 33 vars */) = 0
\textcolor{blue}{13478 21:13:52.044050 execve("/bin/sleep", ["sleep", "0.3"],
      0x5631be4f87a8 /* 33 vars */) = 0}
\textcolor{red}{13476 21:13:52.044269 execve("/bin/sleep", ["sleep", "0.1"],
      0x5631be4f87a8 /* 33 vars */) = 0}
\textcolor{green}{13477 21:13:52.044389 execve("/bin/sleep", ["sleep", "0.2"],
      0x5631be4f87a8 /* 33 vars */) = 0}
\textcolor{blue}{13478 21:13:52.046207 nanosleep(\{tv_sec=0, tv_nsec=300000000\}, NULL) = 0}
\textcolor{red}{13476 21:13:52.046303 nanosleep(\{tv_sec=0, tv_nsec=100000000\}, NULL) = 0}
\textcolor{green}{13477 21:13:52.046318 nanosleep(\{tv_sec=0, tv_nsec=200000000\}, NULL) = 0}
\textcolor{red}{13476 21:13:52.146852 +++ exited with 0 +++}
13475 21:13:52.146942 --- SIGCHLD \{si_signo=SIGCHLD, si_code=CLD_EXITED,
      si_pid=13476, si_uid=1000, si_status=0, si_utime=0, si_stime=0\} ---
\textcolor{green}{13477 21:13:52.247782 +++ exited with 0 +++}
13475 21:13:52.247885 --- SIGCHLD \{si_signo=SIGCHLD, si_code=CLD_EXITED,
      si_pid=13477, si_uid=1000, si_status=0, si_utime=0, si_stime=0\} ---
\textcolor{blue}{13478 21:13:52.347680 +++ exited with 0 +++}
13475 21:13:52.347786 --- SIGCHLD \{si_signo=SIGCHLD, si_code=CLD_EXITED,
      si_pid=13478, si_uid=1000, si_status=0, si_utime=0, si_stime=0\} ---
13475 21:13:52.348069 +++ exited with 0 +++
\end{alltt}
\end{frame}

%%%%%%%
\begin{frame}[fragile]{System call tampering: fault injection \hfill [1/3]}
\begin{block}{\large -e \textbf{inject}=\textit{set}:\textbf{error}=\textit{errno}[:\textbf{when}=\textit{expr}][:\textbf{syscall}=\textit{syscall}]}
\textbf{inject}=\textit{set} -- fault injection for the specified set of syscalls \\
\textbf{error}=\textit{errno} -- the error code to fail syscalls with \\
\textbf{when}=\textit{expr} -- when to inject, in \textit{first[+[step]]} form \\
\textbf{syscall}=\textit{syscall} -- inject the specified syscall instead of -1
\end{block}

\begin{block}{\large strace -e /open -e \textbf{inject}=all:\textbf{error}=EACCES:\textbf{when}=3 \textbackslash\\
cat /dev/full /dev/null}
\begin{alltt}
openat(AT_FDCWD, "/etc/ld.so.cache", O_RDONLY|O_CLOEXEC) = 3
openat(AT_FDCWD, "/lib64/libc.so.6", O_RDONLY|O_CLOEXEC) = 3
\textcolor{red}{openat(AT_FDCWD, "/dev/full", O_RDONLY)
 = -1 EACCES (Permission denied) (INJECTED)}
cat: /dev/full: Permission denied
openat(AT_FDCWD, "/dev/null", O_RDONLY) = 3
+++ exited with 1 +++
\end{alltt}
\end{block}
\end{frame}

%%%%%%%
\begin{frame}[fragile]{System call tampering: fault injection \hfill [2/3]}
\begin{block}{\large python3.5 bug: error opening /dev/urandom}
\scriptsize
\begin{alltt}
$ strace -P /dev/urandom -e inject=\%file:error=ENOENT python3
\textcolor{red}{openat(AT_FDCWD, "/dev/urandom", O_RDONLY|O_CLOEXEC)
 = -1 ENOENT (No such file or directory) (INJECTED)}
Fatal Python error: Failed to open /dev/urandom
--- SIGSEGV \{si_signo=SIGSEGV, si_code=SEGV_MAPERR, si_addr=0x50\} ---
+++ killed by SIGSEGV +++
Segmentation fault
\end{alltt}
\end{block}

\begin{block}{\large python3.5 bug: error reading /dev/urandom}
\scriptsize
\begin{alltt}
$ strace -a0 -e read -P /dev/urandom -e inject=all:error=EIO python3
\textcolor{red}{read(3, 0x8db610, 24) = -1 EIO (Input/output error) (INJECTED)}
Fatal Python error: Failed to read bytes from /dev/urandom
--- SIGSEGV \{si_signo=SIGSEGV, si_code=SEGV_MAPERR, si_addr=0x50\} ---
+++ killed by SIGSEGV +++
Segmentation fault
\end{alltt}
\end{block}
\end{frame}

%%%%%%%
\begin{frame}[fragile]{System call tampering: fault injection \hfill [3/3]}
\begin{block}{\large glibc <= 2.25 dynamic linker bug}
\begin{alltt}
$ strace -e mprotect -efault=all:error=EPERM:when=1 pwd
\textcolor{red}{mprotect(0x7fabcd00f000, 2097152, PROT_NONE)
 = -1 EPERM (Operation not permitted) (INJECTED)}
mprotect(0x7fabcd20f000, 16384, PROT_READ) = 0
mprotect(0x606000, 4096, PROT_READ)     = 0
mprotect(0x7fabcd441000, 4096, PROT_READ) = 0
/
+++ exited with 0 +++
\end{alltt}
\end{block}

\begin{block}{\large glibc >= 2.26: with a proper check}
\begin{alltt}
$ strace -e mprotect -efault=all:error=EPERM:when=1 pwd
\textcolor{red}{mprotect(0x7fabcd00f000, 2097152, PROT_NONE)
 = -1 EPERM (Operation not permitted) (INJECTED)}
pwd: error while loading shared libraries: libc.so.6:
 cannot change memory protections
+++ exited with 127 +++
\end{alltt}
\end{block}
\end{frame}

%%%%%%%
\begin{frame}[fragile]{System call tampering: return value injection}
\begin{block}{\large -e \textbf{inject}=\textit{set}:\textbf{retval}=\textit{value}[:\textbf{when}=\textit{expr}][:\textbf{syscall}=\textit{syscall}]}
\textbf{inject}=\textit{set} -- fault injection for the specified set of syscalls \\
\textbf{retval}=\textit{value} -- the return value to return \\
\textbf{when}=\textit{expr} -- when to inject, in \textit{first[+[step]]} form \\
\textbf{syscall}=\textit{syscall} -- inject the specified syscall instead of -1
\end{block}

\begin{block}{example: recovery of temporary files}
\begin{alltt}
$ cat script.sh
t=`mktemp`; trap 'rm -f "$t"' 0; echo secret $$ > $t
$ strace -qq -f -e signal=none -e /unlink
  -e inject=all:retval=0 sh script.sh
[pid 347] \textcolor{red}{unlinkat(AT_FDCWD, "/tmp/tmp.l1AlwyCYH3", 0)
 = 0 (INJECTED)}
$ cat /tmp/tmp.l1AlwyCYH3
secret 345
\end{alltt}
\end{block}

\end{frame}

%%%%%%%
\begin{frame}[fragile]{System call tampering: delay injection}
\begin{block}{\large syscall delay injection}
strace -e \textbf{inject}=\textit{set}:delay\_enter=\textit{usecs} \\
strace -e \textbf{inject}=\textit{set}:delay\_exit=\textit{usecs}
\end{block}

\begin{block}{dd if=/dev/zero of=/dev/null bs=1M count=10}
\begin{alltt}
10+0 records in
10+0 records out
10485760 bytes (10 MB, 10 MiB) copied, \textcolor{red}{0.00211354} s,
\textcolor{red}{5.0 GB}/s
\end{alltt}
\end{block}

\begin{block}{strace -einject=write:delay\_exit=100000 -ewrite -o/dev/null \textbackslash\\
dd if=/dev/zero of=/dev/null bs=1M count=10}
\begin{alltt}
10+0 records in
10+0 records out
10485760 bytes (10 MB, 10 MiB) copied, \textcolor{red}{1.10658} s,
\textcolor{red}{9.5 MB}/s
\end{alltt}
\end{block}
\end{frame}

%%%%%%%
{
\setbeamertemplate{logo}{}
\begin{frame}{Questions?}
	\begin{columns}
		\column{7cm}
\begin{block}{\large homepage}
	https://strace.io
\end{block}
\begin{block}{\large strace.git}
	https://github.com/strace/strace.git

	https://gitlab.com/strace/strace.git
\end{block}
\begin{block}{\large mailing list}
	strace-devel@lists.strace.io
\end{block}
\begin{block}{\large IRC channel}
	\#strace@freenode
\end{block}
		\column{3cm}
			\centerline{\includegraphics[height=7.2cm]{strace-straus.pdf}}
	\end{columns}
\end{frame}
}

\end{document}
