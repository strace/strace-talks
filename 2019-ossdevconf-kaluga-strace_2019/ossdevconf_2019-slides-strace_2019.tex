% Copyright (C) 2012-2019 Dmitry V. Levin <ldv@altlinux.org>
% Permission is granted to copy, distribute and/or modify this document
% under the terms of the GNU Free Documentation License, Version 1.2
% or any later version published by the Free Software Foundation;
% with no Invariant Sections, no Front-Cover Texts, and no Back-Cover Texts.

\documentclass[unicode,aspectratio=169]{beamer}

\usepackage[utf8]{inputenc}
\usepackage[T2A]{fontenc}
\usepackage[english]{babel}
\usepackage{alltt}

\pgfdeclareimage[height=0.6cm]{basealt-corner-logo}{basealt.jpg}
\pgfdeclareimage[height=1.2cm]{strace-corner-logo}{strace-straus.pdf}
\pgfdeclareimage[height=1.2cm]{title-logo}{ossdevconf-logo.pdf}
\pgfdeclareimage[height=7cm]{strace-logo}{strace-straus.pdf}

\usetheme{Warsaw}
\setbeamertemplate{headline}{}
\setbeamertemplate{footline}{%
	\pgfuseimage{basealt-corner-logo}%
	\hfill%
	\pgfuseimage{strace-corner-logo}%
}
\setbeamertemplate{navigation symbols}{}
\setbeamertemplate{logo}{}
\setbeamercovered{transparent}

\title{\Huge strace 2019}
\author{\Huge Dmitry~Levin}
\date{\Large September 2019}
\titlegraphic{\pgfuseimage{title-logo}}

\begin{document}

%%%%%%%
\begin{frame}
\titlepage
\end{frame}

%%%%%%
\begin{frame}{strace 2019 \hfill [\insertframenumber/\inserttotalframenumber]}
\Large
\begin{block}{v4.20 \ldots v5.3}
\begin{itemize}
\item seccomp-assisted system call filtering: \textbf{--seccomp-bpf} option
\item system call return status filtering: \textbf{-e status}=\textit{set}, \textbf{-z}, \textbf{-Z} options
\item PTRACE\_GET\_SYSCALL\_INFO API support
\item support of new system calls (32)
\item elaborate syscall parsers
\item long options
\item copyleft license
\end{itemize}
\end{block}
\end{frame}

%%%%%%%
\begin{frame}[fragile]{seccomp-assisted system call filtering: \textbf{--seccomp-bpf} \hfill [\insertframenumber/\inserttotalframenumber]}
\large
\begin{block}{Introduced in v5.3 (September 2019)}
\begin{itemize}
\item Automatically generates and attaches a BPF program to filter system calls
\item Makes execution of untraced system calls two orders of magnitude faster
\end{itemize}
\end{block}

\begin{block}{tests/filter\_seccomp-perf.c}
\scriptsize
\begin{alltt}
static volatile sig_atomic_t stop;
static void handler(int signo) \{
    stop = true;
\}
int main(void) \{
    signal(SIGALRM, handler);
    alarm(1);
    unsigned int i;
    for (i = 0; !stop; i++)
        chdir(".");
    printf("%d{\textbackslash}n", i);
    return 0;
\}
\end{alltt}
\end{block}
\end{frame}

%%%%%%%
\begin{frame}[fragile]{seccomp-assisted system call filtering: \textbf{--seccomp-bpf} \hfill [\insertframenumber/\inserttotalframenumber]}
\begin{block}{time ./filter\_seccomp-perf}
\begin{alltt}
3480990
0.05user 0.94system 0:01.00elapsed 100%CPU (0avgtext+0avgdata 1324maxresident)k
0inputs+0outputs (0major+66minor)pagefaults 0swaps
\end{alltt}
\end{block}

\begin{block}{time strace \textbf{--seccomp-bpf} -f -qq -e signal=none -e trace=fchdir ./filter\_seccomp-perf}
\begin{alltt}
2962562
0.05user 0.94system 0:01.00elapsed 100%CPU (0avgtext+0avgdata 3280maxresident)k
0inputs+0outputs (0major+321minor)pagefaults 0swaps
\end{alltt}
\end{block}

\begin{block}{time strace -f -qq -e signal=none -e trace=fchdir ./filter\_seccomp-perf}
\begin{alltt}
81429
0.53user 0.73system 0:01.00elapsed 127%CPU (0avgtext+0avgdata 3156maxresident)k
0inputs+0outputs (0major+284minor)pagefaults 0swaps
\end{alltt}
\end{block}

$2962562 / 3480990 \approx 85.1\%$, $81429 / 3480990 \approx 2.3\%$, $81429 / 2962562 \approx 2.7\%$
\end{frame}

%%%%%%%
\begin{frame}{system call return status filtering: \textbf{-e status}=\textit{set} \hfill [\insertframenumber/\inserttotalframenumber]}
\Large
\begin{block}{Introduced in v5.2 (July 2019)}
Print only system calls with the specified return status. \\
\textit{set} can include the following elements:
\begin{description}
	\item[successful]: returned without an error code, alias to \textbf{-z}
	\item[failed]: returned with an error code, alias to \textbf{-Z}
	\item[unfinished]: did not return
	\item[detached]: detached before return
	\item[unavailable]: returned but failed to fetch the error status
\end{description}
\end{block}
The  default is \textbf{-e status}=\textbf{all}.
\end{frame}

%%%%%%%
\begin{frame}[fragile]{system call return status filtering: \textbf{-z} \hfill [\insertframenumber/\inserttotalframenumber]}
\Large
Trace only those system calls that returned without an error code: \\
\textbf{-z}, \textbf{-e status}=\textbf{successful}
\begin{block}{buildreq: rpm-utils commit 0.10.3-alt1$\sim$2}
\large
\begin{alltt}
--- a/rpm-utils/strace_files
+++ b/rpm-utils/strace_files
@@ -94,3 +94,3 @@ spp_output="$workdir"/spp_output
 rc=0
-strace -qq -f -e signal=none -e trace="$\{trace:-file\}" {\textbackslash}
+strace -qq -f -z -e signal=none -e trace="$\{trace:-file\}" {\textbackslash}
        -o "|/usr/share/buildreqs/spp >$spp_output" -- "$@"
\end{alltt}
\end{block}
\end{frame}

%%%%%%%
\begin{frame}[fragile]{system call return status filtering: \textbf{-z} aggregation \hfill [\insertframenumber/\inserttotalframenumber]}
\begin{block}{strace -o log -f -e signal=none -e trace=execve,nanosleep {\textbackslash} \\
	sh -c 'sleep 0.1 \& sleep 0.2 \& sleep 0.3' \&\& cat log}
\scriptsize
\begin{alltt}
13475 execve("/bin/sh", ["sh", "-c", "sleep 0.1 & sleep 0.2 & sleep 0."...],
      0x5631be4f87a8 /* 42 vars */) = 0
13476 execve("/bin/sleep", ["sleep", "0.1"], 0xe4c4f0 /* 33 vars */ <unfinished ...>
13477 execve("/bin/sleep", ["sleep", "0.2"], 0xe4c4f0 /* 33 vars */ <unfinished ...>
13478 execve("/bin/sleep", ["sleep", "0.3"], 0xe4c4f0 /* 33 vars */ <unfinished ...>
13476 <... execve resumed>)             = 0
13477 <... execve resumed>)             = 0
13478 <... execve resumed>)             = 0
13476 nanosleep({tv_sec=0, tv_nsec=100000000},  <unfinished ...>
13477 nanosleep({tv_sec=0, tv_nsec=200000000},  <unfinished ...>
13478 nanosleep({tv_sec=0, tv_nsec=300000000},  <unfinished ...>
13476 <... nanosleep resumed>NULL)      = 0
13476 +++ exited with 0 +++
13477 <... nanosleep resumed>NULL)      = 0
13477 +++ exited with 0 +++
13478 <... nanosleep resumed>NULL)      = 0
13478 +++ exited with 0 +++
13475 +++ exited with 0 +++
\end{alltt}
\end{block}
\end{frame}

%%%%%%%
\begin{frame}[fragile]{system call return status filtering: \textbf{-z} aggregation \hfill [\insertframenumber/\inserttotalframenumber]}
\begin{block}{strace -o log \textbf{-z} -f -e signal=none -e trace=execve,nanosleep {\textbackslash} \\
	sh -c 'sleep 0.1 \& sleep 0.2 \& sleep 0.3' \&\& cat log}
\small
\begin{alltt}
13475 execve("/bin/sh", ["sh", "-c", "sleep 0.1 & sleep 0.2 & sleep 0."...],
      0x5631be4f87a8 /* 42 vars */) = 0
13476 execve("/bin/sleep", ["sleep", "0.1"], 0xe4c4f0 /* 33 vars */) = 0
13477 execve("/bin/sleep", ["sleep", "0.2"], 0xe4c4f0 /* 33 vars */) = 0
13478 execve("/bin/sleep", ["sleep", "0.3"], 0xe4c4f0 /* 33 vars */) = 0
13476 nanosleep({tv_sec=0, tv_nsec=100000000}, NULL) = 0
13476 +++ exited with 0 +++
13477 nanosleep({tv_sec=0, tv_nsec=200000000}, NULL) = 0
13477 +++ exited with 0 +++
13478 nanosleep({tv_sec=0, tv_nsec=300000000}, NULL) = 0
13478 +++ exited with 0 +++
13475 +++ exited with 0 +++
\end{alltt}
\end{block}
\end{frame}

%%%%%%%
\begin{frame}[fragile]{system call return status filtering: \textbf{-Z} \hfill [\insertframenumber/\inserttotalframenumber]}
\Large
Trace only those system calls that returned with an error code: \\
\textbf{-Z}, \textbf{-e status}=\textbf{failed}

\begin{block}{eu-elflint no-such-file1 no-such-file2}
\small
\begin{alltt}
eu-elflint: cannot open input file: No such file or directory
eu-elflint: cannot open input file: No such file or directory
\end{alltt}
\end{block}

\begin{block}{strace -qq -Z -efile eu-elflint no-such-file1 no-such-file2}
\small
\begin{alltt}
access("/etc/ld.so.preload", R_OK) = -1 ENOENT (No such file or directory)
openat(AT_FDCWD, "no-such-file1", O_RDONLY) = -1 ENOENT (No such file or directory)
eu-elflint: cannot open input file: No such file or directory
openat(AT_FDCWD, "no-such-file2", O_RDONLY) = -1 ENOENT (No such file or directory)
eu-elflint: cannot open input file: No such file or directory
\end{alltt}
\end{block}

\end{frame}

%%%%%%%
\begin{frame}[fragile]{PTRACE\_GET\_SYSCALL\_INFO API support \hfill [\insertframenumber/\inserttotalframenumber]}
\large
\begin{block}{Example based on Debian bug report \#459820 submitted in 2008}
\begin{alltt}
#include <stdio.h>
#include <unistd.h>
int main() \{
    setlinebuf(stdout);
    puts("------------");
    \textcolor{red}{__asm__("movl $2, %eax; int $0x80");}
    printf("[I am %d]{\textbackslash}n", getpid());
    return 0;
\}
\end{alltt}
\end{block}

\begin{block}{Regular invocation: ./debbug459820}
\begin{alltt}
------------
[I am 23450]
[I am 23451]
\end{alltt}
\end{block}
\end{frame}

%%%%%%%
\begin{frame}[fragile]{PTRACE\_GET\_SYSCALL\_INFO API support \hfill [\insertframenumber/\inserttotalframenumber]}
\Large
\begin{block}{strace -f -z ./debbug459820 > /dev/null}
\begin{alltt}
\ldots
write(1, "------------{\textbackslash}n", 13)          = 13
strace: Process 23451 attached
\textcolor{red}{open(NULL, O_RDONLY|O_CREAT|O_TRUNC|O_DSYNC|O_DIRECT
|O_NOATIME|O_CLOEXEC|O_PATH|O_TMPFILE|0x1000020,
0134300) = 23451}
[pid 23450] getpid()                    = 23450
[pid 23451] getpid()                    = 23451
[pid 23450] write(1, "[I am 23450]{\textbackslash}n", 13) = 13
[pid 23451] write(1, "[I am 23451]{\textbackslash}n", 13) = 13
[pid 23450] +++ exited with 0 +++
+++ exited with 0 +++
\end{alltt}
\end{block}
\end{frame}

%%%%%%%
\begin{frame}[fragile]{PTRACE\_GET\_SYSCALL\_INFO API support \hfill [\insertframenumber/\inserttotalframenumber]}
\large
\begin{block}{PTRACE\_GET\_SYSCALL\_INFO: strace >= v4.26, linux >= v5.3-rc1}
\begin{alltt}
\ldots
write(1, "------------{\textbackslash}n", 13)          = 13
\textcolor{red}{strace: [ Process PID=23450 runs in 32 bit mode. ]}
strace: Process 23451 attached
fork()                                  = 23451
\textcolor{red}{strace: [ Process PID=23450 runs in 64 bit mode. ]}
[pid 23450] getpid()                    = 23450
\textcolor{red}{strace: [ Process PID=23451 runs in 64 bit mode. ]}
[pid 23451] getpid()                    = 23451
[pid 23450] write(1, "[I am 23450]{\textbackslash}n", 13) = 13
[pid 23451] write(1, "[I am 23451]{\textbackslash}n", 13) = 13
[pid 23450] +++ exited with 0 +++
+++ exited with 0 +++
\end{alltt}
\end{block}
\end{frame}

%%%%%%%
\begin{frame}{New system calls \hfill [\insertframenumber/\inserttotalframenumber]}
\begin{block}{\Large Introduced in v5.3 (September 2019)}
pidfd\_open, clone3
\end{block}

\begin{block}{\Large Introduced in v5.2 (July 2019)}
open\_tree, move\_mount, fsopen, fsconfig, fsmount, fspick
\end{block}

\begin{block}{\Large Introduced in v5.1 (May 2019)}
clock\_gettime64,
clock\_settime64,
clock\_adjtime64,
clock\_getres\_time64,
clock\_nanosleep\_time64,
timer\_gettime64,
timer\_settime64,
timerfd\_gettime64,
timerfd\_settime64,
utimensat\_time64,
pselect6\_time64,
ppoll\_time64,
io\_pgetevents\_time64,
recvmmsg\_time64,
mq\_timedsend\_time64,
mq\_timedreceive\_time64,
semtimedop\_time64,
rt\_sigtimedwait\_time64,
futex\_time64,
sched\_rr\_get\_interval\_time64,
pidfd\_send\_signal,
io\_uring\_setup,
io\_uring\_enter,
io\_uring\_register
\end{block}
\end{frame}

%%%%%%%
\begin{frame}{Long options \hfill [\insertframenumber/\inserttotalframenumber]}
\Large
\begin{block}{1991 \ldots 2019: short options only}
	a:Ab:cCdDe:E:fFhiI:ko:O:p:P:qrs:S:tTu:vVwxX:yzZ
\end{block}
\begin{block}{Introduced in v5.3 (September 2019)}
\begin{description}
	\item[--help]: alias to \textbf{-h}
	\item[--version]: alias to \textbf{-V}
	\item[--seccomp-bpf]: seccomp-assisted system call filtering
\end{description}
\end{block}
\end{frame}

%%%%%%%
\begin{frame}{Copyleft license \hfill [\insertframenumber/\inserttotalframenumber]}
\Large
\begin{block}{\Large 1991 \ldots 2018: permissive license}
strace was released under a Berkeley-style license \\ at the request of Paul Kranenburg.
\end{block}
\begin{block}{\Large Since v4.26$\sim$52 (December 2018): copyleft license}
\begin{itemize}
	\item test suite: GNU General Public License v2+
	\item the rest of strace: GNU Lesser General Public License v2.1+
\end{itemize}
\end{block}
The first major strace feature implemented after this change is PTRACE\_GET\_SYSCALL\_INFO API support.
\end{frame}

%%%%%%%
{
\setbeamertemplate{footline}{}
\begin{frame}{Questions?}
	\begin{columns}
		\column{7cm}
\begin{block}{\large homepage}
	https://strace.io
\end{block}
\begin{block}{\large strace.git}
	https://github.com/strace/strace.git

	https://gitlab.com/strace/strace.git
\end{block}
\begin{block}{\large mailing list}
	strace-devel@lists.strace.io
\end{block}
\begin{block}{\large IRC channel}
	\#strace@freenode
\end{block}
		\column{3cm}
			\centerline{\includegraphics[height=7cm]{strace-straus.pdf}}
	\end{columns}
\end{frame}
}

\end{document}
