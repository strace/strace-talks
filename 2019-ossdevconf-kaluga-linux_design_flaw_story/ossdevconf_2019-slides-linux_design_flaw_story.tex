% Copyright (C) 2012-2019 Dmitry V. Levin <ldv@altlinux.org>
% Permission is granted to copy, distribute and/or modify this document
% under the terms of the GNU Free Documentation License, Version 1.2
% or any later version published by the Free Software Foundation;
% with no Invariant Sections, no Front-Cover Texts, and no Back-Cover Texts.

\documentclass[unicode,aspectratio=169]{beamer}

\usepackage[utf8]{inputenc}
\usepackage[T2A]{fontenc}
\usepackage[english]{babel}
\usepackage{alltt}

\pgfdeclareimage[height=0.6cm]{basealt-corner-logo}{basealt.jpg}
\pgfdeclareimage[height=1.2cm]{strace-corner-logo}{strace-straus.pdf}
\pgfdeclareimage[height=1.2cm]{title-logo}{ossdevconf-logo.pdf}
\pgfdeclareimage[height=3cm]{lineprinter}{lineprinter.png}
\pgfdeclareimage[height=7cm]{strace-logo}{strace-straus.pdf}
\pgfdeclareimage[height=7cm]{linux-logo}{tux-flat.png}

\usetheme{Warsaw}
\setbeamertemplate{headline}{}
\setbeamertemplate{footline}{%
	\pgfuseimage{basealt-corner-logo}%
	\hfill%
	\pgfuseimage{strace-corner-logo}%
}
\setbeamertemplate{navigation symbols}{}
\setbeamertemplate{logo}{}
\setbeamercovered{transparent}

\title{\Huge linux kernel \\ the history of a design flaw}
\author{\Huge Dmitry~Levin}
\date{\Large September 2019}
\titlegraphic{\pgfuseimage{title-logo}}

\begin{document}

%%%%%%%
\begin{frame}
\titlepage
\end{frame}

%%%%%%%
\begin{frame}{x86-64 architecture in Linux kernel \hfill [\insertframenumber/\inserttotalframenumber]}
\large
\begin{block}{Operating modes}
\begin{description}
	\item[64-bit mode]: CS register value == 0x33
	\item[32-bit mode]: CS register value == 0x23
\end{description}
\end{block}

\begin{block}{Several methods of system call invocation}
\begin{description}
	\item[int 0x80]: Legacy 32-bit
	\item[sysenter]: Fast 32-bit
	\item[syscall]: 64-bit
\end{description}
\end{block}

64-bit processes can invoke both 64-bit and 32-bit system calls.

\begin{block}{Linux API provides}
\begin{itemize}
	\item The system call number
	\item The value of CS register
	\item The value of RIP register
\end{itemize}
\end{block}
\end{frame}

%%%%%%%
\begin{frame}{Linux system call debuggers and tracers on x86-64 architecture \hfill [\insertframenumber/\inserttotalframenumber]}
\large
\begin{block}{Legacy method of obtaining system call information}
\begin{itemize}
	\item Fetch the system call number (PTRACE\_PEEKUSER ORIG\_RAX)
	\item Fetch the value of CS register (PTRACE\_PEEKUSER CS)
	\item {\bf Guess the system call bitness by the value of CS register}
	\item Determine the system call by its number and bitness
	\item Fetch the system call arguments accordingly
\end{itemize}
\end{block}

\begin{block}{Traditional method of obtaining system call information}
\begin{itemize}
	\item Fetch the whole set of registers (PTRACE\_GETREGSET NT\_PRSTATUS), \\
		the return value is decided by the value of CS register
	\item {\bf Guess the system call bitness by the return value}
	\item Determine the system call by its number and bitness
	\item Interpret other registers as the system call arguments accordingly
\end{itemize}
\end{block}
\end{frame}

%%%%%%%
\begin{frame}[fragile]{Process bitness does not match syscall bitness: problem \hfill [\insertframenumber/\inserttotalframenumber]}
\large
\begin{block}{Example based on Debian bug report \#459820 submitted in 2008}
\begin{alltt}
#include <stdio.h>
#include <unistd.h>
int main() \{
    setlinebuf(stdout);
    puts("------------");
    \textcolor{red}{__asm__("movl $2, %eax; int $0x80");}
    printf("[I am %d]{\textbackslash}n", getpid());
    return 0;
\}
\end{alltt}
\end{block}

\begin{block}{Regular invocation: ./debbug459820}
\begin{alltt}
------------
[I am 23450]
[I am 23451]
\end{alltt}
\end{block}
\end{frame}

%%%%%%%
\begin{frame}[fragile]{Process bitness does not match syscall bitness: problem \hfill [\insertframenumber/\inserttotalframenumber]}
\Large
\begin{block}{Invocation under strace: strace -f -z ./debbug459820 > /dev/null}
\begin{alltt}
\ldots
write(1, "------------{\textbackslash}n", 13)          = 13
strace: Process 23451 attached
\textcolor{red}{open(NULL, O_RDONLY|O_CREAT|O_TRUNC|O_DSYNC|O_DIRECT
|O_NOATIME|O_CLOEXEC|O_PATH|O_TMPFILE|0x1000020,
0134300) = 23451}
[pid 23450] getpid()                    = 23450
[pid 23451] getpid()                    = 23451
[pid 23450] write(1, "[I am 23450]{\textbackslash}n", 13) = 13
[pid 23451] write(1, "[I am 23451]{\textbackslash}n", 13) = 13
[pid 23450] +++ exited with 0 +++
+++ exited with 0 +++
\end{alltt}
\end{block}
\end{frame}

%%%%%%%
\begin{frame}{Process bitness does not match syscall bitness: problem \hfill [\insertframenumber/\inserttotalframenumber]}
\Large
\begin{block}{Alternative method of obtaining system call information}
\begin{itemize}
	\item Fetch the instruction pointer (PTRACE\_PEEKUSER RIP)
	\item {\bf Fetch the instruction (PTRACE\_PEEKTEXT rip-2)
	\item Determine the system call bitness by the opcode}
	\item Determine the system call by its number and bitness
	\item Fetch the system call arguments accordingly
\end{itemize}
\end{block}

\begin{block}{Drawbacks of the alternative method}
\begin{itemize}
	\item Inherent race condition between the instruction executed \\ and the instruction fetched
	\item Extra ptrace system call invocation
\end{itemize}
\end{block}
\end{frame}

%%%%%%%
\begin{frame}{Process bitness does not match syscall bitness: early approaches \hfill [\insertframenumber/\inserttotalframenumber]}
\begin{block}{Problem denial}
\begin{itemize}
	\item Pretend the problem does not exist
	\item Argue the problem has no consequences
	\item Claim the race is not practical
	\item Shrug and let those who care come up with fixes
\end{itemize}
\end{block}
\begin{block}{Early approaches to the problem: lively discussion in 2012}
Indan Zupancic: Compat 32-bit syscall entry from 64-bit task!?
\scriptsize
\begin{itemize}
	\item Linus Torvalds: abuse bits \#32 and \#33 of eflags
	\item H. Peter Anvin: abuse bits [31:16] of CS
	\item Roland McGrath: add a new regset flavour with a pseudo-register
	\item Denys Vlasenko: extend NT\_PRSTATUS regset
	\item Oleg Nesterov: add PTRACE\_O\_TRACESYS\_VERY\_GOOD
	\item Pedro Alves: extend PTRACE\_GETEVENTMSG
	\item Andrew Lutomirski (2017): PTRACE\_GET\_SYSCALL\_INFO
\end{itemize}
\end{block}
\end{frame}

%%%%%%%
\begin{frame}[fragile]{Process bitness does not match syscall bitness: first patches \hfill [\insertframenumber/\inserttotalframenumber]}
\begin{block}{07.11.2018: First edition}
	\begin{columns}
		\column{3cm}
			\centerline{\pgfuseimage{lineprinter}}
		\column{7cm}
			\begin{itemize}
				\item RFC PATCH: ptrace: add PTRACE\_GET\_SYSCALL\_INFO request
			\end{itemize}
	\end{columns}
\begin{alltt}
include/linux/ptrace.h      |  8 +++++++
include/linux/sched.h       |  1 +
include/linux/tracehook.h   |  9 +++++---
include/uapi/linux/ptrace.h | 23 ++++++++++++++++++++
kernel/ptrace.c             | 53 +++++++++++++++++++++++++++++++++++++++++++++
kernel/signal.c             |  1 +
6 files changed, 92 insertions(+), 3 deletions(-)
\end{alltt}
\end{block}
\end{frame}

%%%%%%%
\begin{frame}{Process bitness does not match syscall bitness: first patches \hfill [\insertframenumber/\inserttotalframenumber]}
\begin{block}{20.11.2018 -- 21.11.2018: Second edition}
\begin{itemize}
	\item v2 00/15: Prepare for PTRACE\_GET\_SYSCALL\_INFO
	\begin{itemize}
		\tiny
		\item Move EM\_HEXAGON to uapi/linux/elf-em.h
		\item Move EM\_ARCOMPACT and EM\_ARCV2 to uapi/linux/elf-em.h
		\item Move EM\_UNICORE to uapi/linux/elf-em.h
		\item elf-em.h: add EM\_NDS32
		\item elf-em.h: add EM\_XTENSA
		\item m68k: define syscall\_get\_arch()
		\item arc: define syscall\_get\_arch()
		\item c6x: define syscall\_get\_arch()
		\item h8300: define syscall\_get\_arch()
		\item hexagon: define syscall\_get\_arch()
		\item nds32: define syscall\_get\_arch()
		\item nios2: define syscall\_get\_arch()
		\item riscv: define syscall\_get\_arch()
		\item unicore32: define syscall\_get\_arch()
		\item {\bf xtensa: define syscall\_get\_arch()}
		\item syscall\_get\_arch: add "struct task\_struct *" argument
	\end{itemize}
	\item RFC PATCH v2: ptrace: add PTRACE\_GET\_SYSCALL\_INFO request
	\item v3: powerpc/ptrace: replace ptrace\_report\_syscall() with a tracehook call
	\item {\bf x86/ptrace: Fix documentation for tracehook\_report\_syscall\_entry()}
	\item {\bf mips: fix mips\_get\_syscall\_arg o32 check}
\end{itemize}
\end{block}
\end{frame}

%%%%%%%
\begin{frame}{Process bitness does not match syscall bitness: first patches \hfill [\insertframenumber/\inserttotalframenumber]}
\Large
\begin{block}{25.11.2018 -- 28.11.2018: 3rd and 4th editions}
\begin{itemize}
	\item v3 0/3: ptrace: add PTRACE\_GET\_SYSCALL\_INFO request
	\begin{itemize}
		\item ptrace: pass type of a syscall-stop in ptrace\_message
		\item ptrace: add PTRACE\_GET\_SYSCALL\_INFO request
		\item ptrace: add PTRACE\_EVENT\_SECCOMP support to PTRACE\_GET\_SYSCALL\_INFO
	\end{itemize}
	\item v4 0/2: ptrace: add PTRACE\_GET\_SYSCALL\_INFO request
	\begin{itemize}
		\item ptrace: save the type of syscall-stop in ptrace\_message
		\item ptrace: add PTRACE\_GET\_SYSCALL\_INFO request
	\end{itemize}
\end{itemize}
\end{block}
\end{frame}

%%%%%%%
\begin{frame}{Process bitness does not match syscall bitness: first patches \hfill [\insertframenumber/\inserttotalframenumber]}
\scriptsize
\begin{block}{03.12.2018 -- 10.12.2018: 5th edition}
\begin{itemize}
	\setlength{\itemsep}{0pt}
	\item ia64: fix syscall\_get\_error()
	\item microblaze: fix syscall\_set\_return\_value()
	\item nios2: fix syscall\_get\_error()
	\item sh: fix syscall\_set\_return\_value()
	\item {\bf selftests: do not macro-expand failed assertion expressions}
	\item {\bf v5: powerpc/ptrace: replace ptrace\_report\_syscall() with a tracehook call}
	\item v5 00/25: ptrace: add PTRACE\_GET\_SYSCALL\_INFO request
	\setlength{\itemsep}{0pt}
	\begin{columns}
		\column{7cm}
			\begin{itemize}
				\tiny
				\setlength{\itemsep}{0pt}
				\item alpha: define remaining syscall\_get\_* functions
				\item Move EM\_ARCOMPACT and EM\_ARCV2 to \\ uapi/linux/elf-em.h
				\item arc: define syscall\_get\_arch()
				\item c6x: define syscall\_get\_arch()
				\item elf-em.h: add EM\_CSKY
				\item csky: define syscall\_get\_arch()
				\item h8300: define remaining syscall\_get\_* functions
				\item Move EM\_HEXAGON to uapi/linux/elf-em.h
				\item hexagon: define remaining syscall\_get\_* functions
				\item Move EM\_NDS32 to uapi/linux/elf-em.h
				\item nds32: define syscall\_get\_arch()
				\item nios2: define syscall\_get\_arch()
			\end{itemize}
		\column{7cm}
			\begin{itemize}
				\tiny
				\setlength{\itemsep}{0pt}
				\item m68k: add asm/syscall.h
				\item mips: define syscall\_get\_error()
				\item parisc: define syscall\_get\_error()
				\item powerpc: define syscall\_get\_error()
				\item riscv: define syscall\_get\_arch()
				\item {\bf Move EM\_XTENSA to uapi/linux/elf-em.h}
				\item xtensa: define syscall\_get\_* functions
				\item Move EM\_UNICORE to uapi/linux/elf-em.h
				\item unicore32: add asm/syscall.h
				\item syscall\_get\_arch: add "struct task\_struct *" argument
				\item ptrace: add PTRACE\_GET\_SYSCALL\_INFO request
				\item selftests/ptrace: add a test case for \\ PTRACE\_GET\_SYSCALL\_INFO
			\end{itemize}
	\end{columns}
\end{itemize}
\end{block}
\end{frame}

%%%%%%%
\begin{frame}{Process bitness does not match syscall bitness: first patches \hfill [\insertframenumber/\inserttotalframenumber]}
\large
\begin{block}{13.12.2018: 6th edition (API finalized)}
\begin{itemize}
	\item v6 00/27: ptrace: add PTRACE\_GET\_SYSCALL\_INFO request
	{\bf
	\begin{itemize}
		\item Move EM\_XTENSA to uapi/linux/elf-em.h
		\item elf-em.h: add EM\_CSKY
		\item csky: define syscall\_get\_arch()
	\end{itemize}
	\item powerpc/ptrace: Combine SYSCALL\_EMU \& SYSCALL\_TRACE handling
	}
\end{itemize}
\end{block}

\begin{block}{07.01.2019: 7th edition}
\begin{itemize}
	\item v7 00/22: ptrace: add PTRACE\_GET\_SYSCALL\_INFO request
	\item Fundamentally failed to build on Alpha architecture
	\item Changes two different subsystems: audit and ptrace
	\item Changes too many architectures
	\item Too big to be accepted
\end{itemize}
\end{block}
\end{frame}

%%%%%%%
\begin{frame}{Process bitness does not match syscall bitness: final patches \hfill [\insertframenumber/\inserttotalframenumber]}
\Large
\begin{itemize}
	\item \textcolor{red}{06.01.2019: v5.0-rc1 tagged, merge window closed}
	\item 09.01.2019: 00/14 Prepare syscall\_get\_arch for PTRACE\_GET\_SYSCALL\_INFO
	\begin{itemize}
		\scriptsize
		\item Move EM\_ARCOMPACT and EM\_ARCV2 to uapi/linux/elf-em.h
		\item arc: define syscall\_get\_arch()
		\item c6x: define syscall\_get\_arch()
		\item h8300: define syscall\_get\_arch()
		\item Move EM\_HEXAGON to uapi/linux/elf-em.h
		\item hexagon: define syscall\_get\_arch()
		\item m68k: define syscall\_get\_arch()
		\item Move EM\_NDS32 to uapi/linux/elf-em.h
		\item nds32: define syscall\_get\_arch()
		\item nios2: define syscall\_get\_arch()
		\item riscv: define syscall\_get\_arch()
		\item Move EM\_UNICORE to uapi/linux/elf-em.h
		\item unicore32: define syscall\_get\_arch()
		\item syscall\_get\_arch: add "struct task\_struct *" argument
	\end{itemize}
	\item Pinging and waiting for Acked-by
	\item \textcolor{green}{03.03.2019: v5.0 released, merge window opened}
\end{itemize}
\end{frame}

%%%%%%%
\begin{frame}{Process bitness does not match syscall bitness: final patches \hfill [\insertframenumber/\inserttotalframenumber]}
\large
\begin{itemize}
	\item \textcolor{red}{17.03.2019: v5.1-rc1 tagged, merge window closed}
	\item 18.03.2019: \\ v2 00/13: Prepare syscall\_get\_arch for PTRACE\_GET\_SYSCALL\_INFO
	\begin{itemize}
		\scriptsize
		\item Move EM\_ARCOMPACT and EM\_ARCV2 to uapi/linux/elf-em.h
		\item arc: define syscall\_get\_arch()
		\item c6x: define syscall\_get\_arch()
		\item h8300: define syscall\_get\_arch()
		\item Move EM\_HEXAGON to uapi/linux/elf-em.h
		\item hexagon: define syscall\_get\_arch()
		\item m68k: define syscall\_get\_arch()
		\item Move EM\_NDS32 to uapi/linux/elf-em.h
		\item nds32: define syscall\_get\_arch()
		\item nios2: define syscall\_get\_arch()
		\item Move EM\_UNICORE to uapi/linux/elf-em.h
		\item unicore32: define syscall\_get\_arch()
		\item syscall\_get\_arch: add "struct task\_struct *" argument
	\end{itemize}
	\item 20.03.2019: Merged into audit/next
	\item \textcolor{green}{05.05.2019: v5.1 released, merge window opened}
	\item 08.05.2019: Merged via audit tree by commit v5.2-rc1$\sim$144
\end{itemize}
\end{frame}

%%%%%%%
\begin{frame}{Process bitness does not match syscall bitness: final patches \hfill [\insertframenumber/\inserttotalframenumber]}
\begin{itemize}
	\item \textcolor{red}{17.03.2019: v5.1-rc1 tagged, merge window closed}
	\item 22.03.2019: v8 0/7: ptrace: add PTRACE\_GET\_SYSCALL\_INFO request
	\begin{itemize}
		\item nds32: fix asm/syscall.h
		\item hexagon: define syscall\_get\_error() and syscall\_get\_return\_value()
		\item mips: define syscall\_get\_error()
		\item parisc: define syscall\_get\_error()
		\item powerpc: define syscall\_get\_error()
		\item ptrace: add PTRACE\_GET\_SYSCALL\_INFO request
		\item selftests/ptrace: add a test case for PTRACE\_GET\_SYSCALL\_INFO
	\end{itemize}
	\item 08.04.2019: v9 0/7: ptrace: add PTRACE\_GET\_SYSCALL\_INFO request
	\item 16.04.2019: v10 0/7: ptrace: add PTRACE\_GET\_SYSCALL\_INFO request
	\item \textcolor{green}{05.05.2019: v5.1 released, merge window opened}
	\item 10.05.2019: v11 0/7: ptrace: add PTRACE\_GET\_SYSCALL\_INFO request
	\item \textcolor{red}{19.05.2019: v5.2-rc1 tagged, merge window closed}
	\item 22.05.2019: Added to -mm patch queue
	\item \textcolor{green}{07.07.2019: v5.2 released, merge window opened}
	\item 17.07.2019: Merged via -mm patch queue, last commit is v5.3-rc1$\sim$65{\textasciicircum}2$\sim$22
\end{itemize}
\end{frame}

%%%%%%%
\begin{frame}[fragile]{Process bitness does not match syscall bitness: result \hfill [\insertframenumber/\inserttotalframenumber]}
\large
\begin{block}{PTRACE\_GET\_SYSCALL\_INFO: strace >= v4.26, linux >= v5.3-rc1}
\begin{verbatim}
$ git log -i -E --author=altlinux.org \
  --grep='ptrace|syscall_a|elf-em|selftests' \
  v4.20-rc2..v5.3-rc1
\end{verbatim}
\begin{itemize}
	\item 29 commits, 47 files changed, 703 insertions, 125 deletions
	\item 2 authors
	\item 22 Acked-by/Reviewed-by/Signed-off-by
	\item 07.11.2018: first RFC patch submitted
	\item 12.11.2018: first patch committed
	\item 13.12.2018: API finalized
	\item 17.07.2019: last patch committed
	\item Implements PTRACE\_GET\_SYSCALL\_INFO on those 19 architectures \\
		that enable CONFIG\_HAVE\_ARCH\_TRACEHOOK
\end{itemize}
\end{block}
\end{frame}

%%%%%%%
\begin{frame}[fragile]{Process bitness does not match syscall bitness: API \hfill [\insertframenumber/\inserttotalframenumber]}
\begin{block}{Extends ptrace with PTRACE\_GET\_SYSCALL\_INFO request}
\scriptsize
\begin{alltt}
struct ptrace_syscall_info \{
  __u8 op;			\hfill /* Type of system call stop */
  __aligned_u32 \textcolor{red}{arch};	\hfill /* AUDIT_ARCH_* value; see seccomp(2) */
  __u64 instruction_pointer;	\hfill /* CPU instruction pointer */
  __u64 stack_pointer;		\hfill /* CPU stack pointer */
  union \{
    struct \{			\hfill /* op == PTRACE_SYSCALL_INFO_ENTRY */
      __u64 nr;			\hfill /* System call number */
      __u64 args[6];		\hfill /* System call arguments */
    \} entry;
    struct \{			\hfill /* op == PTRACE_SYSCALL_INFO_EXIT */
      __s64 rval;		\hfill /* System call return value */
      __u8 is_error;		\hfill /* System call error boolean: does rval contain an error value? */
    \} exit;
    struct \{			\hfill /* op == PTRACE_SYSCALL_INFO_SECCOMP */
      __u64 nr;			\hfill /* System call number */
      __u64 args[6];		\hfill /* System call arguments */
      __u32 ret_data;		\hfill /* SECCOMP_RET_DATA portion of SECCOMP_RET_TRACE return value */
    \} seccomp;
  \};
\};
\end{alltt}
\end{block}
\end{frame}

%%%%%%%
\begin{frame}[fragile]{Process bitness does not match syscall bitness: no longer a problem \hfill [\insertframenumber/\inserttotalframenumber]}
\large
\begin{block}{Example based on Debian bug report \#459820 submitted in 2008}
\begin{alltt}
#include <stdio.h>
#include <unistd.h>
int main() \{
    setlinebuf(stdout);
    puts("------------");
    \textcolor{red}{__asm__("movl $2, %eax; int $0x80");}
    printf("[I am %d]{\textbackslash}n", getpid());
    return 0;
\}
\end{alltt}
\end{block}

\begin{block}{Regular invocation: ./debbug459820}
\begin{alltt}
------------
[I am 23450]
[I am 23451]
\end{alltt}
\end{block}
\end{frame}

%%%%%%%
\begin{frame}[fragile]{PTRACE\_GET\_SYSCALL\_INFO: strace >= v4.26, linux >= v5.3-rc1}
\large
\begin{block}{Invocation under strace: strace -f -z ./debbug459820 > /dev/null}
\begin{alltt}
\ldots
write(1, "------------{\textbackslash}n", 13)          = 13
\textcolor{red}{strace: [ Process PID=23450 runs in 32 bit mode. ]}
strace: Process 23451 attached
fork()                                  = 23451
\textcolor{red}{strace: [ Process PID=23450 runs in 64 bit mode. ]}
[pid 23450] getpid()                    = 23450
\textcolor{red}{strace: [ Process PID=23451 runs in 64 bit mode. ]}
[pid 23451] getpid()                    = 23451
[pid 23450] write(1, "[I am 23450]{\textbackslash}n", 13) = 13
[pid 23451] write(1, "[I am 23451]{\textbackslash}n", 13) = 13
[pid 23450] +++ exited with 0 +++
+++ exited with 0 +++
\end{alltt}
\end{block}
\end{frame}

%%%%%%%
{
\setbeamertemplate{footline}{}
\begin{frame}{Questions?}
	\begin{columns}
		\column{4cm}
			\centerline{\pgfuseimage{linux-logo}}
		\column{3cm}
		\column{3cm}
			\centerline{\pgfuseimage{strace-logo}}
	\end{columns}
\end{frame}
}

\end{document}
