% Copyright (C) 2012-2018 Dmitry V. Levin <ldv@altlinux.org>
% Permission is granted to copy, distribute and/or modify this document
% under the terms of the GNU Free Documentation License, Version 1.2
% or any later version published by the Free Software Foundation;
% with no Invariant Sections, no Front-Cover Texts, and no Back-Cover Texts.

\documentclass[unicode,aspectratio=169]{beamer}

\usepackage[utf8]{inputenc}
\usepackage[T2A]{fontenc}
\usepackage[english]{babel}
\usepackage{alltt}
\usepackage{hyperref}

\pgfdeclareimage[height=1cm]{strace-corner-logo}{strace-straus.pdf}
\pgfdeclareimage[height=0.9cm]{lpc-corner-logo}{lpc_weblogo-2017.png}
\pgfdeclareimage[height=2cm]{title-logo}{basealt.jpg}

\usetheme{Warsaw}
\setbeamertemplate{headline}{}
\setbeamertemplate{footline}{%
	\pgfuseimage{lpc-corner-logo}%
	\hfill%
	\pgfuseimage{strace-corner-logo}%
}
\setbeamertemplate{navigation symbols}{}
\setbeamertemplate{logo}{}
\setbeamercovered{transparent}

\title{\Huge What could be done in the kernel \\ to make strace happy}
\author{\Large Dmitry~Levin}
%\institute[BaseALT]{\large BaseALT}
\date{\Large Linux Plumbers Conference 2018}
\titlegraphic{\pgfuseimage{title-logo}}

\begin{document}

%%%%%%%
\begin{frame}
\titlepage
\end{frame}

%%%%%%%
\begin{frame}{Problems\hfill [\insertframenumber/\inserttotalframenumber]}
\begin{enumerate}
\large
\item There is no kernel API to find out whether the tracee is entering or exiting syscall.
\item There is no reliable way to distinguish between x86\_64 and x86 syscalls.
\item There is no kernel API to invoke \textbf{wait4} syscall with a different signal mask (like \textbf{pselect6} and \textbf{ppoll}).
\item The time precision provided by \textbf{struct rusage} is too low for syscall statistics (\textbf{strace -c}).
\item There is no proper kernel API to translate between tracer and tracee views of pids.

\smallskip
\hrule
\smallskip

\item There is no way to obtain network protocol details for descriptors of tracees running in different namespaces (\textbf{strace -yy}).
\item There are no consistent detailed declarative syscall descriptions, this forces every user to reinvent its own wheel and catch up with the kernel.
\item strace is slow, perf can lose data
\end{enumerate}
\end{frame}

%%%%%%%
\begin{frame}{Problem 1: whether the tracee is entering or exiting syscall? \hfill [\insertframenumber/\inserttotalframenumber]}
\LARGE

\begin{block}{Problem}
Both \textbf{syscall-enter-stop} and \textbf{syscall-exit-stop} \\
look the same for the tracer, \\
there is no kernel API to distinguish them.
\end{block}

\begin{block}{Workaround}
strace does its best to keep track of the sequence of ptrace events.
When attaching to a tracee inside exec, however,
its first syscall stop is \textbf{\textit{very likely}} going to be
\textbf{syscall-exit-stop} instead of \textbf{syscall-enter-stop}, \\
the workaround is fragile.
\end{block}
\end{frame}

%%%%%%%
\begin{frame}[fragile]{Problem 2: whether the invoked syscall is x86\_64 or x86? \hfill [\insertframenumber/\inserttotalframenumber]}
\begin{block}{Problem}
There is no reliable way to distinguish between x86\_64 and x86 syscalls.
\end{block}

\begin{block}{Current practice}
\scriptsize
\begin{alltt}
union \{
    struct x86_64_user_regs_struct    x86_64_r;
    struct i386_user_regs_struct      i386_r;
\} x86_regs_union;
struct iovec x86_io = \{
    .iov_base = &x86_regs_union,
    .iov_len = sizeof(x86_regs_union)
\};
rc = ptrace(PTRACE_GETREGSET, pid, NT_PRSTATUS, &x86_io);
\ldots
if (x86_io.iov_len == sizeof(x86_regs_union.i386_r)) \{
    scno = x86_regs_union.i386_r_regs.orig_eax;
    currpers = 1;
\} else \{
    scno = x86_regs_union.x86_64_r.orig_rax;
    currpers = 0;
\}
\end{alltt}
\end{block}
\end{frame}

%%%%%%%
\begin{frame}[fragile]{Problem 2: whether the invoked syscall is x86\_64 or x86? \hfill [\insertframenumber/\inserttotalframenumber]}
\begin{block}{Problem}
\small
In infamous case of int 0x80 on x86\_64 the PTRACE\_GETREGSET approach does not work.
\end{block}

\begin{block}{Example}
\small
\begin{alltt}
$ cat int_0x80.c
#include <stdio.h>
int main(void) \{
    /* 200 is __NR_getgid32 on x86 and __NR_tkill on x86_64. */
    __asm__("movq $246, %rsi; movq $135, %rdi; movq $200, %rax; int $0x80");
    printf("getegid returns %d{\textbackslash}n", getegid());
    return 0;
\}
$ gcc -Wall -O2 int_0x80.c
$ strace -qq -etrace=tkill,/getegid ./a.out
\textcolor{red}{tkill(135, 246)                         = 500}
getegid()                               = 500
getegid returns 500
\end{alltt}
\end{block}
\end{frame}

%%%%%%%
\begin{frame}[fragile]{Proposed solution: PTRACE\_GET\_SYSCALL\_INFO \hfill [\insertframenumber/\inserttotalframenumber]}
\begin{block}{Extend the ptrace API with PTRACE\_GET\_SYSCALL\_INFO request, \\
use it \textbf{instead of} PTRACE\_GETREGSET et al}
\scriptsize
\begin{alltt}
struct ptrace_syscall_info \{
    __u8    op;  /* 0 for entry, 1 for exit */
    __u8    __pad0[7];
    union \{
        struct \{
            __s32   nr;
            __u32   arch;
            __u64   ip;
            __u64   args[6];
        \} entry_info;
        struct \{
            __s64   rval;
            __u8    is_error;
            __u8    __pad2[7];
        \} exit_info;
    \};
\};
\end{alltt}
\end{block}
\scriptsize
RFC patch and discussion: \url{https://lkml.org/lkml/2018/11/7/313}
\end{frame}

%%%%%%%
\begin{frame}[fragile]{Problem 3: invoking wait4 with a different signal mask \hfill [\insertframenumber/\inserttotalframenumber]}
\begin{block}{strace main loop in case of delay injection enabled}
\begin{alltt}
for (;;) \{
    \textcolor{red}{/* What if the timer has expired at this point?  */}
    pid = wait4(-1, &status, \_\_WALL, &rusage);
    handle\_tracee(pid, status, &rusage);
\}
\end{alltt}
\end{block}

\begin{block}{Problem}
There is no kernel API to invoke \textbf{wait4} syscall with a different signal mask, \\
similar to \textbf{pselect6} extension over \textbf{select} and \textbf{ppoll} over \textbf{poll}.
\end{block}

\begin{block}{Workaround}
strace does its best to implement a race-free workaround \\
by doing a lot of non-trivial work inside a signal handler.

This is way too complex and very fragile.
\end{block}
\end{frame}

%%%%%%%
\begin{frame}[fragile]{Proposed solution: \textbf{pwait6} syscall \hfill [\insertframenumber/\inserttotalframenumber]}
\Large
\begin{block}{Add \textbf{pwait6} syscall}
Similar to \textbf{pselect6} extension over \textbf{select}
and \textbf{ppoll} over \textbf{poll}, \\
add \textbf{pwait6} syscall
which is \textbf{wait4} with additional signal mask arguments:
\begin{alltt}
pid_t
wait4(pid_t pid, int *wstatus,
      int options, struct rusage *rusage);
pid_t
pwait6(pid_t pid, int *wstatus,
       int options, struct rusage *rusage,
       \textcolor{red}{const sigset_t *sigmask, size_t sigsetsize});
\end{alltt}
\end{block}
\end{frame}

%%%%%%
\begin{frame}[fragile]{Problem 4: time precision limitations of \textbf{struct rusage} \hfill [\insertframenumber/\inserttotalframenumber]}
\begin{block}{The time precision provided by \textbf{struct rusage} is too low for syscall statistics}
\scriptsize
\begin{alltt}
$ strace -c -e%file pwd > /dev/null
% time     seconds  usecs/call     calls    errors syscall
------ ----------- ----------- --------- --------- ----------------
 53.09    0.000043          43         1           execve
 30.86    0.000025          12         2           openat
 16.05    0.000013          13         1         1 access
  0.00    0.000000           0         1           getcwd
------ ----------- ----------- --------- --------- ----------------
100.00    0.000081                     5         1 total

$ strace -c -e%file pwd > /dev/null
% time     seconds  usecs/call     calls    errors syscall
------ ----------- ----------- --------- --------- ----------------
100.00    0.000009           9         1           getcwd
  0.00    0.000000           0         1         1 access
  0.00    0.000000           0         1           execve
  0.00    0.000000           0         2           openat
------ ----------- ----------- --------- --------- ----------------
100.00    0.000009                     5         1 total
\end{alltt}
\end{block}
\end{frame}

%%%%%%%
\begin{frame}[fragile]{Proposed solution: \textbf{pwait6} syscall \hfill [\insertframenumber/\inserttotalframenumber]}
\begin{block}{Use a better structure than \textbf{struct rusage} in the new \textbf{pwait6} syscall}
Replace \textbf{struct rusage} argument of the new \textbf{pwait6} syscall with \textbf{struct rusage\_ts64}:
\begin{alltt}
struct rusage \{
    struct \textcolor{red}{timeval} ru_utime; /* user CPU time used */
    struct \textcolor{red}{timeval} ru_stime; /* system CPU time used */
    \ldots
\}
\end{alltt}
\begin{alltt}
struct rusage_ts64 \{
    struct \textcolor{red}{timespec64} ru_utime; /* user CPU time used */
    struct \textcolor{red}{timespec64} ru_stime; /* system CPU time used */
    \ldots
\}
\end{alltt}
struct \textbf{timespec64} is chosen over struct \textbf{timespec} to avoid 32-bit time\_t overflow.
\end{block}
\end{frame}

%%%%%%%
\begin{frame}[fragile]{Problem 5: no translation of pids \hfill [\insertframenumber/\inserttotalframenumber]}
\begin{block}{Problem}
PID namespaces have been introduced without a proper kernel API
to translate between tracer and tracee views of pids.
\end{block}

\begin{block}{strace users are getting confused by PID namespaces: \\ \url{https://bugzilla.redhat.com/1035433}}
\scriptsize
\begin{alltt}
# strace -qq -ff -e clone -o s.log unshare --pid -- sh -c 'sh -c "sh -c true & wait" & wait'
# ls s.log.*
s.log.4567  s.log.4568  s.log.\textcolor{red}{4569}
# grep \^ s.log.*
s.log.4567:clone(child_stack=NULL, flags=CLONE_CHILD_CLEARTID|CLONE_CHILD_SETTID|SIGCHLD, \(\longrightarrow\)
           child_tidptr=0x7fa7f9adba10) = 4568
s.log.4567:--- SIGCHLD {si_signo=SIGCHLD, si_code=CLD_EXITED, \(\longrightarrow\)
           si_pid=4568, si_uid=0, si_status=0, si_utime=0, si_stime=0} ---
s.log.4568:clone(child_stack=NULL, flags=CLONE_CHILD_CLEARTID|CLONE_CHILD_SETTID|SIGCHLD, \(\longrightarrow\)
           child_tidptr=0x7fe0f586aa10) = \textcolor{red}{2}
s.log.4568:--- SIGCHLD {si_signo=SIGCHLD, si_code=CLD_EXITED, \(\longrightarrow\)
           si_pid=\textcolor{red}{2}, si_uid=0, si_status=0, si_utime=0, si_stime=0} ---
\end{alltt}
\end{block}
\end{frame}

%%%%%%%
\begin{frame}[fragile]{Problem 5: no translation of pids \hfill [\insertframenumber/\inserttotalframenumber]}
\begin{block}{Problem}
PID namespaces have been introduced without a proper kernel API
to translate between tracer and tracee views of pids.
\end{block}

\begin{block}{strace users are getting confused by PID namespaces: \\ \url{https://bugzilla.redhat.com/1035433}}
\scriptsize
\begin{alltt}
# strace -qq -ff -e clone -o s.log unshare --pid -- sh -c 'sh -c "sh -c true & wait" & wait'
# ls s.log.*
s.log.4567  s.log.4568  s.log.\textcolor{red}{4569}
# grep \^ s.log.*
s.log.4567:clone(child_stack=NULL, flags=CLONE_CHILD_CLEARTID|CLONE_CHILD_SETTID|SIGCHLD, \(\longrightarrow\)
           child_tidptr=0x7fa7f9adba10) = 4568
s.log.4567:--- SIGCHLD {si_signo=SIGCHLD, si_code=CLD_EXITED, \(\longrightarrow\)
           si_pid=4568, si_uid=0, si_status=0, si_utime=0, si_stime=0} ---
s.log.4568:clone(child_stack=NULL, flags=CLONE_CHILD_CLEARTID|CLONE_CHILD_SETTID|SIGCHLD, \(\longrightarrow\)
           child_tidptr=0x7fe0f586aa10) = \textcolor{red}{2}\textcolor{green}{<4569>}
s.log.4568:--- SIGCHLD {si_signo=SIGCHLD, si_code=CLD_EXITED, \(\longrightarrow\)
           si_pid=\textcolor{red}{2}\textcolor{green}{<4569>}, si_uid=0, si_status=0, si_utime=0, si_stime=0} ---
\end{alltt}
\end{block}
\end{frame}

%%%%%%%
\begin{frame}[fragile]{Proposed solution: \textbf{translate\_pid} syscall \hfill [\insertframenumber/\inserttotalframenumber]}
\begin{block}{Add \textbf{translate\_pid} syscall proposed by Konstantin Khlebnikov: \\ \url{https://lkml.org/lkml/2018/6/1/788}}
\begin{alltt}
pid_t translate_pid(pid_t pid, int source, int target);
\end{alltt}

pid namespaces are referred by file descriptors opened to proc files
\texttt{/proc/[pid]/ns/pid} or \texttt{/proc/[pid]/ns/pid\_for\_children}.

Negative argument points to the current pid namespace.

Return value: \\ pid in the target pid namespace or zero if the task has no pid there.

Error codes:
\begin{description}
\item[EBADF]: source or target is not a valid open file descriptor
\item[EINVAL]: file descriptor does not refer to a pid namespace
\item[ESRCH]: task not found in the source namespace
\end{description}

Translation can breach pid namespace isolation and return pids from outer pid
namespaces iff process already has file descriptor for these namespaces.
\end{block}
\end{frame}

%%%%%%%
\begin{frame}[fragile]{Proposed solution: \textbf{translate\_pid} syscall \hfill [\insertframenumber/\inserttotalframenumber]}
\begin{block}{\textbf{translate\_pid} examples provided by Konstantin Khlebnikov}
\begin{alltt}
translate_pid(pid, ns, -1)      - translate pid to our pid namespace
translate_pid(pid, -1, ns)      - translate pid to other pid namespace
translate_pid(pid, -1, ns) > 0  - is pid reachable from ns?
translate_pid(1, ns1, ns2) > 0  - is ns1 inside ns2?
translate_pid(1, ns1, ns2) == 0 - is ns1 outside ns2?
translate_pid(1, ns1, ns2) == 1 - is ns1 equal to ns2?
\end{alltt}
\end{block}

\begin{block}{revision history}
\scriptsize
\begin{description}
\item[v1]: \url{https://lkml.org/lkml/2015/9/15/411}
\item[v2]: \url{https://lkml.org/lkml/2015/9/24/278}
\item[v3]: \url{https://lkml.org/lkml/2015/9/25/290}
\item[v4]: \url{https://lkml.org/lkml/2017/10/13/177}
\item[v5]: \url{https://lkml.org/lkml/2018/4/4/677}
\item[v6]: \url{https://lkml.org/lkml/2018/6/1/788}
\end{description}
\end{block}
\end{frame}

%%%%%%%
\begin{frame}{Problem 6: strace and tracees running in different net namespaces \hfill [\insertframenumber/\inserttotalframenumber]}
\small
\begin{block}{Problem}
strace -yy chromium-browser doesn't show network protocol details
because NETLINK\_SOCK\_DIAG does not report sockets of tracees running in different network namespaces:

\url{https://lists.strace.io/pipermail/strace-devel/2018-September/008374.html}
\end{block}

\begin{block}{Example}
Connected sockets should ve reported this way becase socketpair always generates a pair of connected sockets:
\begin{alltt}
socketpair(AF\_UNIX, SOCK\_STREAM, 0, [27<UNIX:[\textcolor{red}{7162769->7162770}]>, 28<UNIX:[\textcolor{red}{7162770->7162769}]>]) = 0
\end{alltt}

If the tracee runs in a different network namespaces, the output generated by strace looks as if these sockets are unconnected:
\begin{alltt}
socketpair(AF\_UNIX, SOCK\_STREAM, 0, [223<UNIX:[\textcolor{red}{7162686}]>, 224<UNIX:[\textcolor{red}{7162687}]>]) = 0
\end{alltt}
\end{block}
\end{frame}

%%%%%%%
\begin{frame}{Problem 7: no detailed declarative syscall descriptions in kernel \hfill [\insertframenumber/\inserttotalframenumber]}

\begin{block}{Problem}
There are no consistent detailed declarative machine readable syscall descriptions,
this forces every user to reinvent its own wheel and catch up with the kernel.
\end{block}

\begin{block}{Current practice}
\begin{description}
\item[strace]: A lot of manual work has been done to implement parsers of all syscalls in C,
some of these parsers are quite complex, there is a test suite with 85\% coverage.
\item[libc]: Every libc has its own wrappers for some subset of syscalls,
some of these wrappers are machine generated.
\item[syzkaller]: Detailed declarative machine readable descriptions.
\item[others]: Sanitizers, valgrind.
\end{description}
\end{block}

\begin{block}{Proposed solution}
Provide detailed declarative machine readable descriptions for all syscalls in the kernel.
\end{block}
\end{frame}

%%%%%%%
\begin{frame}{Problem 7: no detailed declarative syscall descriptions in kernel \hfill [\insertframenumber/\inserttotalframenumber]}
\begin{block}{hsh-run --mount=/proc $--$ strace -e trace=sendto,recvmsg ip route list}
\tiny
\begin{alltt}
sendto(3, \{\{len=40, type=RTM\_GETROUTE, flags=NLM\_F\_REQUEST|NLM\_F\_DUMP, seq=1357924680, pid=0\},
 \textcolor{blue}{\{rtm\_family=AF\_UNSPEC, rtm\_dst\_len=0, rtm\_src\_len=0, rtm\_tos=0, rtm\_table=RT\_TABLE\_UNSPEC, rtm\_protocol=RTPROT\_UNSPEC, rtm\_scope=RT\_SCOPE\_UNIVERSE, rtm\_type=RTN\_UNSPEC, rtm\_flags=0\}},
  \textcolor{red}{\{nla\_len=0, nla\_type=RTA\_UNSPEC\}}\}, 40, 0, NULL, 0) = 40

\smallskip
recvmsg(3, \{msg\_name=\{sa\_family=AF\_NETLINK, nl\_pid=0, nl\_groups=00000000\}, msg\_namelen=12, msg\_iov=[\{iov\_base=[
\textcolor{brown}{\{\{len=60, type=RTM\_NEWROUTE, flags=NLM\_F\_MULTI, seq=1357924680, pid=12345\},
 \textcolor{blue}{\{rtm\_family=AF\_INET, rtm\_dst\_len=32, rtm\_src\_len=0, rtm\_tos=0, rtm\_table=RT\_TABLE\_LOCAL, rtm\_protocol=RTPROT\_KERNEL, rtm\_scope=RT\_SCOPE\_LINK, rtm\_type=RTN\_BROADCAST, rtm\_flags=0\}},
  \textcolor{red}{[\{\{nla\_len=8, nla\_type=RTA\_TABLE\}, RT\_TABLE\_LOCAL\}, \{\{nla\_len=8, nla\_type=RTA\_DST\}, inet\_addr("127.0.0.0")\}, \{\{nla\_len=8, nla\_type=RTA\_PREFSRC\}, inet\_addr("127.0.0.1")\}, \{\{nla\_len=8, nla\_type=RTA\_OIF\}, if\_nametoindex("lo")\}]}\}},
\textcolor{brown}{\{\{len=60, type=RTM\_NEWROUTE, flags=NLM\_F\_MULTI, seq=1357924680, pid=12345\},
 \textcolor{blue}{\{rtm\_family=AF\_INET, rtm\_dst\_len=8, rtm\_src\_len=0, rtm\_tos=0, rtm\_table=RT\_TABLE\_LOCAL, rtm\_protocol=RTPROT\_KERNEL, rtm\_scope=RT\_SCOPE\_HOST, rtm\_type=RTN\_LOCAL, rtm\_flags=0\}},
  \textcolor{red}{[\{\{nla\_len=8, nla\_type=RTA\_TABLE\}, RT\_TABLE\_LOCAL\}, \{\{nla\_len=8, nla\_type=RTA\_DST\}, inet\_addr("127.0.0.0")\}, \{\{nla\_len=8, nla\_type=RTA\_PREFSRC\}, inet\_addr("127.0.0.1")\}, \{\{nla\_len=8, nla\_type=RTA\_OIF\}, if\_nametoindex("lo")\}]}\}},
\textcolor{brown}{\{\{len=60, type=RTM\_NEWROUTE, flags=NLM\_F\_MULTI, seq=1357924680, pid=12345\},
 \textcolor{blue}{\{rtm\_family=AF\_INET, rtm\_dst\_len=32, rtm\_src\_len=0, rtm\_tos=0, rtm\_table=RT\_TABLE\_LOCAL, rtm\_protocol=RTPROT\_KERNEL, rtm\_scope=RT\_SCOPE\_HOST, rtm\_type=RTN\_LOCAL, rtm\_flags=0\}},
  \textcolor{red}{[\{\{nla\_len=8, nla\_type=RTA\_TABLE\}, RT\_TABLE\_LOCAL\}, \{\{nla\_len=8, nla\_type=RTA\_DST\}, inet\_addr("127.0.0.1")\}, \{\{nla\_len=8, nla\_type=RTA\_PREFSRC\}, inet\_addr("127.0.0.1")\}, \{\{nla\_len=8, nla\_type=RTA\_OIF\}, if\_nametoindex("lo")\}]}\}},
\textcolor{brown}{\{\{len=60, type=RTM\_NEWROUTE, flags=NLM\_F\_MULTI, seq=1357924680, pid=12345\},
 \textcolor{blue}{\{rtm\_family=AF\_INET, rtm\_dst\_len=32, rtm\_src\_len=0, rtm\_tos=0, rtm\_table=RT\_TABLE\_LOCAL, rtm\_protocol=RTPROT\_KERNEL, rtm\_scope=RT\_SCOPE\_LINK, rtm\_type=RTN\_BROADCAST, rtm\_flags=0\}},
  \textcolor{red}{[\{\{nla\_len=8, nla\_type=RTA\_TABLE\}, RT\_TABLE\_LOCAL\}, \{\{nla\_len=8, nla\_type=RTA\_DST\}, inet\_addr("127.255.255.255")\}, \{\{nla\_len=8, nla\_type=RTA\_PREFSRC\}, inet\_addr("127.0.0.1")\}, \{\{nla\_len=8, nla\_type=RTA\_OIF\}, if\_nametoindex("lo")\}]}\}}
], iov\_len=32768\}], msg\_iovlen=1, msg\_controllen=0, msg\_flags=0\}, 0) = 240 \\
\ldots
\end{alltt}
\end{block}
\end{frame}

%%%%%%%
\begin{frame}[fragile]{Problem 7: no detailed declarative syscall descriptions in kernel \hfill [\insertframenumber/\inserttotalframenumber]}
\begin{block}{strace/msghdr.c}
\scriptsize
\begin{alltt}
SYS\_FUNC(recvmsg) \{
    int msg\_namelen;
    if (entering(tcp)) \{
        \textcolor{red}{printfd(tcp, tcp->u\_arg[0]);}
        tprints(", ");
        if (\textcolor{red}{fetch\_msghdr\_namelen(tcp, tcp->u\_arg[1], &msg\_namelen)}) \{
            set\_tcb\_priv\_ulong(tcp, msg\_namelen);
            return 0;
        \}
        printaddr(tcp->u\_arg[1]);
    \} else \{
        msg\_namelen = get\_tcb\_priv\_ulong(tcp);
        if (syserror(tcp))
            tprintf("{msg\_namelen=%d}", msg\_namelen);
        else
            \textcolor{red}{decode\_msghdr(tcp, &msg\_namelen, tcp->u\_arg[1], tcp->u\_rval);}
    \}
    tprints(", ");
    \textcolor{red}{printflags(msg\_flags, tcp->u\_arg[2], "MSG\_???");}
    return RVAL\_DECODED;
\}
\end{alltt}
\end{block}
\end{frame}

%%%%%%%
\begin{frame}{Problem 7: no detailed declarative syscall descriptions in kernel \hfill [\insertframenumber/\inserttotalframenumber]}
\begin{block}{syzkaller/sys/linux/socket.txt}
\scriptsize
recvmsg(fd sock, msg ptr[in, \textcolor{red}{recv\_msghdr}], f flags[\textcolor{red}{recv\_flags}])

\ldots

\textcolor{red}{recv\_flags} = MSG\_CMSG\_CLOEXEC, MSG\_DONTWAIT, MSG\_ERRQUEUE, MSG\_OOB, MSG\_PEEK, MSG\_TRUNC, MSG\_WAITALL, MSG\_WAITFORONE

\ldots

\textcolor{red}{recv\_msghdr} \{

\ \ \ \ msg\_name	ptr[out, sockaddr\_storage, opt]

\ \ \ \ msg\_namelen	len[msg\_name, int32]

\ \ \ \ msg\_iov		ptr[in, array[iovec\_out]]

\ \ \ \ msg\_iovlen	len[msg\_iov, intptr]

\ \ \ \ msg\_control	ptr[out, array[int8], opt]

\ \ \ \ msg\_controllen	bytesize[msg\_control, intptr]

\ \ \ \ msg\_flags	int32

\}
\end{block}

\begin{block}{net/socket.c}
\scriptsize
SYSCALL\_DEFINE3(recvmsg, \textcolor{red}{int, fd}, struct user\_msghdr \_\_user *, msg, \textcolor{red}{unsigned int, flags})

\{

\ \ \ \ return \_\_sys\_recvmsg(fd, msg, flags, true);

\}
\end{block}
\end{frame}

%%%%%%%
\begin{frame}{Problem 8: strace is slow, perf can lose data \hfill [\insertframenumber/\inserttotalframenumber]}

\begin{block}{ptrace API is slow}
There are two syscall stops per syscall: \textbf{syscall-enter-stop} and \textbf{syscall-exit-stop}.

There are two context switches per syscall stop: from tracee to tracer and back.

strace invokes at least three syscalls per syscall stop:

wait4, PTRACE\_GETREGSET, and PTRACE\_SYSCALL.
\end{block}

\begin{block}{kernel tracing can lose data}
The data is written to a ring buffer and could be lost if the reader is not fast enough.
\end{block}

\begin{block}{Ideas}
\begin{itemize}
\item Add a flag to struct perf\_event\_attr that new perf events should block on overflow
\item Implement a perf backend for strace
\item Compile strace decoders into eBPF
\end{itemize}
\end{block}
\end{frame}

%%%%%%%
{
\setbeamertemplate{footline}{}
\setbeamertemplate{logo}{}
\begin{frame}{Questions?}
	\begin{columns}
		\column{7cm}
\begin{block}{\large homepage}
	https://strace.io
\end{block}
\begin{block}{\large strace.git}
	https://github.com/strace/strace.git

	https://gitlab.com/strace/strace.git
\end{block}
\begin{block}{\large mailing list}
	strace-devel@lists.strace.io
\end{block}
\begin{block}{\large IRC channel}
	\#strace@freenode
\end{block}
		\column{3cm}
			\centerline{\includegraphics[height=7cm]{strace-straus.pdf}}
	\end{columns}
\end{frame}
}

\end{document}
