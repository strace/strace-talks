% Copyright (C) 2012-2023 Dmitry V. Levin <ldv@strace.io>
% Copyright (C) 2023-2024 Eugene Syromiatnikov <evgsyr@gmail.com>
% Permission is granted to copy, distribute and/or modify this document
% under the terms of the GNU Free Documentation License, Version 1.2
% or any later version published by the Free Software Foundation;
% with no Invariant Sections, no Front-Cover Texts, and no Back-Cover Texts.

\documentclass[unicode,aspectratio=169,xcolor={table,dvipsnames}]{beamer}

\mode<presentation>
{
	\usetheme{Warsaw}
	\setbeamertemplate{headline}{}
	\setbeamertemplate{footline}{\hfill \insertframenumber/\inserttotalframenumber}
	\setbeamertemplate{navigation symbols}{}
}

\usepackage[utf8]{inputenc}
\usepackage[T1,T2A]{fontenc}
\usepackage[english,czech]{babel}
\usepackage{alltt}
\usepackage{verbatim}
\usepackage{hyperref}
\usepackage{listings}
\lstset{
basicstyle=\scriptsize\ttfamily,
columns=flexible,
breaklines=true
}

\colorlet{darkred}{BrickRed}
\colorlet{darkgreen}{OliveGreen}

\newcommand{\symlinebreak}{\textcolor{blue}{\(\hookleftarrow\)}}
\newcommand{\symlinecont}{\textcolor{blue}{\(\longrightarrow\)}}

\pgfdeclareimage[height=1.2cm]{right-corner-logo}{strace-straus.pdf}
\pgfdeclareimage[height=8cm]{strace-logo}{strace-straus.pdf}

\title{\Huge Současný strace}
\author{\Huge Eugene Syromiatnikov}
\date{\Large Brno, 2024}

\logo{\pgfuseimage{right-corner-logo}}

\begin{document}

%%%%%%%
\begin{frame}
\titlepage
\end{frame}

%%%%%%%
\begin{frame}[fragile]{Úvod}
\begin{block}{Co to je \texttt{strace}?}
\begin{itemize}
\item \texttt{strace} je ladící program pro Linux, umožňující sledovaní (a případně
i pozměňování) interakci komunikaci mezi procesy a linuxovým jádrem, včetně
systémových volání, signálů a změn procesového stavu.
\item \texttt{strace} funguje na základě systémového volání \texttt{ptrace}.
\end{itemize}
\end{block}
\pause
\begin{block}{Příklad použití}
% Spuštění nového programu
\scriptsize
\begin{alltt}
\$ strace echo "Hello world"
execve("/usr/bin/echo", ["echo", "Hello world"], 0x7ffd5f11c288 /* 12 vars */) = 0
brk(NULL)                               = 0x56514d3ff000
mmap(NULL, 8192, PROT_READ|PROT_WRITE, MAP_PRIVATE|MAP_ANONYMOUS, -1, 0) = 0x7fccc36c0000
access("/etc/ld.so.preload", R_OK)      = -1 ENOENT (No such file or directory)
openat(AT_FDCWD, "/etc/ld.so.cache", O_RDONLY|O_CLOEXEC) = 3
fstat(3, \{st_mode=S_IFREG|0644, st_size=99278, ...\}) = 0
mmap(NULL, 99278, PROT_READ, MAP_PRIVATE, 3, 0) = 0x7fccc36a7000
close(3)                                = 0
...
\end{alltt}
% Už běžící program
\end{block}
\end{frame}

%%%%%%%
\begin{frame}{Tradiční funkce strace \hfill [1/3]}
\large
\begin{block}{Řízení trasování}
\begin{itemize}
\item Připojení k již běžícímu procesu: \textbf{-p \textit{pid}}
\item Trasovaní potomků procesů: \textbf{-f}
\end{itemize}
\end{block}

\begin{block}{Filtrace systémových volání}
\begin{itemize}
\item Trasovaní jen uvedené sady systémových volání: \textbf{-e trace=\textit{set}}
\begin{itemize}
\item Podporuje použití několika speciálních jmen pro některé vymezené skupiny voláni:
      \texttt{process}, \texttt{network}, \texttt{signal}, \texttt{ipc}, \texttt{desc}, \texttt{memory}
\end{itemize}
\end{itemize}
\end{block}

\begin{block}{Statistiky volání}
\begin{itemize}
\item Počítání času, volání, a chyb pro každé volání: \textbf{-c}
\item Seřazení tabulky volání: \textbf{-S \textit{sortby}}
\end{itemize}
\end{block}
\end{frame}

%%%%%%%
\begin{frame}{Tradiční funkce strace \hfill [2/3]}
\large
\begin{block}{Výpis instrukčního ukazatele and časového razítka}
\begin{itemize}
\item Výpis instrukčního ukazatele: \textbf{-i}
\item Výpis časového razítka: \textbf{-r}, \textbf{-t}, \textbf{-tt}, \textbf{-ttt} a \textbf{-T}
\end{itemize}
\end{block}

\begin{block}{Maximální velikost a formát řetězců}
\begin{itemize}
\item Velikost řetězce: \textbf{-s}
\item Formát řetězce: \textbf{-x}, \textbf{-xx}
\end{itemize}
\end{block}

\begin{block}{Podrobnost výpisu systémových volání}
\begin{itemize}
\item Zkrácení výpisu: \textbf{-e abbrev=\textit{set}}, \textbf{-v}
\item Dereference struktur: \textbf{-e verbose=\textit{set}}
\item Nezpracovaný výpis volání: \textbf{-e raw=\textit{set}}
\end{itemize}
\end{block}
\end{frame}

%%%%%%%
\begin{frame}{Tradiční funkce strace \hfill [3/3]}
\large
\begin{block}{Výpis signálů}
\begin{itemize}
\item Výpis signálů: \textbf{-e signal=\textit{set}}
\end{itemize}
\end{block}

\begin{block}{Výpis I/O}
\begin{itemize}
\item Výpis dat přečtených z uvedených deskriptorů: \textbf{-e read=\textit{set}}
\item Výpis dat zapsaných do uvedených deskriptorů: \textbf{-e write=\textit{set}}
\end{itemize}
\end{block}

\begin{block}{Přesměrování výstupů do souboru nebo roury}
\begin{itemize}
\item Vystupování trasy do souboru: \textbf{-o \textit{filename}}
\item Vystupování trasy do roury: \textbf{-o |\textit{command}}
\item Výpis tras procesů do samostatných souborů: \textbf{-ff -o \textit{prefix}}
\end{itemize}
\end{block}
\end{frame}

%%%%%%%
\begin{frame}[fragile]{Příklad použití tradičních funkci strace}
\begin{block}{strace -fttTv -s 1024 echo "Hello world"}
\scriptsize
\begin{alltt}
10:59:14.044687 execve("/usr/bin/echo", ["echo", "Hello world"], ["SHELL=/bin/bash", \symlinebreak
\symlinecont "LANGUAGE=en_US:en", "PWD=/esur", "LOGNAME=esyr", "HOME=/esyr", "LANG=en_US.UTF-8", \symlinebreak
\symlinecont "TERM=rxvt-unicode", "USER=esyr", "SHLVL=1", \symlinebreak
\symlinecont "PATH=/usr/local/sbin:/usr/local/bin:/usr/bin:/bin", "MAIL=/var/mail/esyr", \symlinebreak
\symlinecont "_=/usr/bin/strace"]) = 0 <0.000449>
10:59:14.045522 brk(NULL)               = 0x559a55a90000 <0.000137>
10:59:14.045917 mmap(NULL, 8192, PROT_READ|PROT_WRITE, MAP_PRIVATE|MAP_ANONYMOUS, -1, \symlinebreak
\symlinecont 0) = 0x7f40113d0000 <0.000166>
10:59:14.046322 access("/etc/ld.so.preload", R_OK) = -1 ENOENT (No such file or directory) \symlinebreak
\symlinecont <0.000136>
10:59:14.046869 openat(AT_FDCWD, "/etc/ld.so.cache", O_RDONLY|O_CLOEXEC) = 3 <0.000099>
10:59:14.047090 fstat(3, \{st_dev=makedev(0xfe, 0x1), st_ino=2095255, st_mode=S_IFREG|0644, \symlinebreak
\symlinecont st_nlink=1, st_uid=0, st_gid=0, st_blksize=4096, st_blocks=200, st_size=99278, \symlinebreak
\symlinecont st_atime=1730518958 /* 2024-11-02T04:42:38.145457664+0100 */, st_atime_nsec=145457664, \symlinebreak
\symlinecont st_mtime=1730518958 /* 2024-11-02T04:42:38.137457664+0100 */, st_mtime_nsec=137457664, \symlinebreak
\symlinecont st_ctime=1730518958 /* 2024-11-02T04:42:38.137457664+0100 */, st_ctime_nsec=137457664\}) \symlinebreak
\symlinecont = 0 <0.000033>
10:59:14.047301 mmap(NULL, 99278, PROT_READ, MAP_PRIVATE, 3, 0) = 0x7f40113b7000 <0.000025>
10:59:14.047377 close(3)                = 0 <0.000017>
...
\end{alltt}
\end{block}
\end{frame}

%%%%%%%
\begin{frame}[fragile]{Odkazy na přednášky o použití strace}
\begin{block}{Michael Kerrisk, "System Call Tracing with strace"}
\begin{itemize}
\item NDC TechTown 2018; 29. srpna 2018, Kongsberg
\item PDF: \href{https://www.man7.org/conf/ndctechtown2018/system_call_tracing_with_strace-NDC-TechTown-Kerrisk.pdf}f{https://www.man7.org/conf/ndctechtown2018/system\_call\_tracing\_with\_strace-NDC-TechTown-Kerrisk.pdf}
\item Youtube: \href{https://www.youtube.com/watch?v=oFt6V56BOlo}{https://www.youtube.com/watch?v=oFt6V56BOlo}
\end{itemize}
\end{block}
\begin{block}{Renaud Metrich, "Using strace to troubleshoot issues"}
\begin{itemize}
\item Devconf.cz 2023; 17. června 2023, Brno
\item PDF: \href{https://static.sched.com/hosted_files/devconfcz2023/f3/Devconf%20CZ%202023%20-%20Using%20STRACE%20to%20troubleshoot%20issues.pdf}{https://static.sched.com/hosted\_files/devconfcz2023/f3/Devconf\%20CZ\%202023\%20-\%20Using\%20STRACE\%20to\%20troubleshoot\%20issues.pdf}
\item Youtube: \href{https://www.youtube.com/watch?v=H4BiBn-7GT8}{https://www.youtube.com/watch?v=H4BiBn-7GT8}
\end{itemize}
\end{block}
\end{frame}

%%%%%%
\begin{frame}{Současní strace: přehled \hfill [1/7]}
\large
\begin{block}{\large Podpora dlouhých voleb přikazovacího řádku \hfill (od v5.6, duben 2020)}
\begin{description}
\item[\textbf{-e trace}]: \textbf{-{}-trace}
\item[\textbf{-c}]: \textbf{-{}-summary-only}
\item[\textbf{-f}]: \textbf{-{}-follow-forks}
\item[\textbf{-ff}]: \textbf{-{}-follow-forks -{}-output-separately}
\item[\textbf{-r}]: \textbf{-{}-relative-timestamps[=\textit{precision}]}
\item[\textbf{-t}]: \textbf{-{}-absolute-timestamps}
\item[\textbf{-tt}]: \textbf{-{}-absolute-timestamps=precision:us}
\item[\textbf{-ttt}]: \textbf{-{}-absolute-timestamps=format:unix,precision:us}
\item[\textbf{-T}]: \textbf{-{}-syscall-times}
\item[\textbf{-v}]: \textbf{-{}-no-abbrev}
\item[\textbf{-x}]: \textbf{-{}-strings-in-hex=non-ascii}
\item[\textbf{-xx}]: \textbf{-{}-strings-in-hex[=all]}
\end{description}
\end{block}
\end{frame}

%%%%%%
\begin{frame}{Současní strace: přehled \hfill [2/7]}
\large
\begin{block}{Výstup trasy \hfill [1/2]}
\begin{itemize}
\item Rozvoj rozboru systémových volání, příkazů ioctl, a protokolů netlink
\item Výpis cest k souborům u souborových deskriptorů: \textbf{-y}/\textbf{-{}-decode-fds} (v4.7, květen 2012)
\item Výpis doplňkové informace o souborových deskriptorech: \textbf{-{}-decode-fds=[\textit{set}]}
\begin{itemize}
\item síťový protokol a adresy (\textbf{socket}, v4.12, květen 2016)
\item major/minor čísla zařízení (\textbf{dev}, v4.22, duben 2018)
\item číslo procesu u pidfd (\textbf{pidfd}, v5.6, duben 2020)
\item sada signálů u signalfd (\textbf{signalfd}, v6.3, květen 2023)
\item detaily události u eventfd (\textbf{eventfd}, v6.10, červenec 2024)
\end{itemize}
\item Výpis doplňkové informace o PID: \textbf{-{}-decode-pids=[\textit{set}]}
\begin{itemize}
\item Výpis PID procesu v jmenném prostoru PID strace: \textbf{-{}-pidns-translation}/\textbf{-{}-decode-pids=pidns} (v5.9, září 2019)
\item Výpis \texttt{comm} procesů, spojených s PID: \textbf{-Y}/\textbf{-{}-decode-pids=comm} (v5.15, prosinec 2021)
\end{itemize}
\end{itemize}
\end{block}
\end{frame}

%%%%%%
\begin{frame}{Současní strace: přehled \hfill [3/7]}
\large
\begin{block}{Výstup trasy \hfill [2/2]}
\begin{itemize}
\item Výpis zásobníku volání funkci: \textbf{-k}/\textbf{-{}-stack-trace} (v4.9, srpen 2014)
\begin{itemize}
\item Výpis informace o souboru a řádku v zdrojovém kódu: \textbf{-kk}/\textbf{-{}-stack-trace=source} (v6.7, leden 2024)
\end{itemize}
\item Výpis kontextů SELinux: \textbf{-{}-secontext} (v5.12, duben 2021)
\begin{itemize}
\item Výpis nesouladů mezi aktuálním kontextem a databází SELinux: \textbf{-{}-secontext=mismatch} (v5.16, leden 2022)
\end{itemize}
\item Výpis čísla systémového volání: \textbf{-n}/\textbf{-{}-syscall-number} (v5.9, duben 2020)
\item Nucení výpisu PID procesu v trase: \textbf{-{}-always-show-pid} (v6.9, květen 2024)
\item Formát výpisu jmenovaných konstant a příznaku: \textbf{-X}/\textbf{-{}-const-print-style} (v4.23, červen 2018)
\item Otevírání výstupních souboru na doplnění: \textbf{-A}/\textbf{-{}-output-append-mode} (v4.22, duben 2018)
\end{itemize}
\end{block}
\end{frame}

%%%%%%
\begin{frame}{Současní strace: přehled \hfill [4/7]}
\large
\begin{block}{\large Filtrace systémových volání}
\begin{itemize}
\item Souborové cesty, přístup k nímž se realizuje přes jméno nebo deskriptor: \textbf{-P} (v4.7, květen 2012)
\item Návratový stav: \textbf{-{}-status}=\textit{set} (v5.2, červenec 2019)
\item Používané deskriptory: \textbf{-{}-trace-fds}=\textit{set} (v6.3, květen 2023)
\item Zadání systémových volání regulárními výrazy: \textbf{-e trace=/\textit{regexp}} (v4.17, květen 2017)
\item Nepovinné specifikace systémových volání: \textbf{-e trace=?\textit{spec}} (v4.17, květen 2017) % v4.17~48
\item Nové skupiny systémových volání: \textbf{\%clock} (v5.7, červen 2020), \textbf{\%creds} (v5.5, únor 2020), \textbf{\%pure} (v4.21, únor 2018), \textbf{\%stat}, \textbf{\%lstat}, \textbf{\%fstat}, \textbf{\%statfs}, \textbf{\%fstatfs}, \textbf{\%\%stat}, \textbf{\%\%statfs} (v4.17, květen 2017)
\end{itemize}
\end{block}
\end{frame}

%%%%%%%
\begin{frame}{Současní strace: přehled \hfill [5/7]}
\large
\begin{block}{Spuštění programu}
\begin{itemize}
\item Zadaní \texttt{argv[0]} prováděného programu: \textbf{-{}-argv0=\textit{name}} (v6.3, květen 2023)
\end{itemize}
\end{block}

\begin{block}{\large Řízení trasování}
\begin{itemize}
\item Přípojení k několika procesům: \textbf{-p \textit{pid\_set}} (v4.7, květen 2012)
\item Spuštění jako oddělené vnouče: \textbf{-D}/\textbf{-{}-daemonize} (v4.5.19, září 2009) % v4.5.19~111
\begin{itemize}
\item Další možnosti oddělení: \textbf{-{}-daemonize=\{pgroup|session\}} (v5.4, listopad 2019)
\end{itemize}
\item Odpojení po \texttt{execve}: \textbf{-b execve}/\textbf{-{}-detach-on=execve} (v4.7, květen 2012) % v4.7~97
\item Odpojení po dosahu specifikovaného počtů vypsaných systémových volání: \textbf{-{}-syscall-limit=\textit{number}} (v6.3, květen 2023)
\item Nastavení signálů, jež strace ignoruje: \textbf{-I}/\textbf{-{}-interruptible} (v4.7, květen 2012)
\item Ukončení tracee s ukončením strace: \textbf{-{}-kill-on-exit} (v6.6, říjen 2023)
\item Filtrace systémových volání přes \texttt{seccomp}: \textbf{-{}-seccomp-bpf} (v5.3, září 2019)
\end{itemize}
\end{block}

\end{frame}

%%%%%%%
\begin{frame}{Současní strace: přehled \hfill [6/7]}
\large
\begin{block}{\large Manipulovaní systémovými voláními}
\begin{itemize}
\item Injekce chyb (v4.15, prosinec 2016): \\
-e \textbf{inject}=\textit{set}:\textbf{error}=\textit{errno}[:\textbf{when}=\textit{expr}][:\textbf{syscall}=\textit{syscall}]
\item Injekce návratných hodnot (v4.16, únor 2017): \\
-e \textbf{inject}=\textit{set}:\textbf{retval}=\textit{value}[:\textbf{when}=\textit{expr}][:\textbf{syscall}=\textit{syscall}]
\item Injekce signálů (v4.16, únor 2017): \\
-e \textbf{inject}=\textit{set}:\textbf{signal}=\textit{set}
\item Injekce zdržení (v4.22, duben 2018): \\
-e \textbf{inject}=\textit{set}:\textbf{delay\_enter}=\textit{usecs} \\
-e \textbf{inject}=\textit{set}:\textbf{delay\_exit}=\textit{usecs}
\item Změna obsahu paměti (v5.11, únor 2021): \\
-e \textbf{inject}=\textit{set}:\textbf{poke\_enter}=@\textit{argN}=\textit{dataN}] \\
-e \textbf{inject}=\textit{set}:\textbf{poke\_exit}=@\textit{argN}=\textit{dataN}]
\end{itemize}
\end{block}
\end{frame}

%%%%%%%
\begin{frame}{Současní strace: přehled \hfill [7/7]}
\large
\begin{block}{\large Statistiky volání}
\begin{itemize}
\item Reálný čas ("nástěnných hodin"), strávený v systémovém voláním: \textbf{-w}/\textbf{-{}-summary-wall-clock} (v4.9, srpen 2014)
\item Kombinovaní regulárního výstupu a výpisu souhrnu volání: \textbf{-C}/\textbf{-{}-summary} (v4.5.20, duben 2010)
\item Nastavení sady sloupců tabulky souhrnu volání: \textbf{-U}/\textbf{-{}-summary-columns=\textit{set}} (v5.6, duben 2020)
\end{itemize}
\end{block}

\begin{block}{Různé}
\begin{itemize}
\item Výpis rad a vychytávek: \textbf{-{}-tips} (v5.18, červen 2022)
\end{itemize}
\end{block}
\end{frame}

%%%%%%%
\begin{frame}[fragile]{Rozbor operaci \texttt{ioctl} \hfill [1/2]}
\begin{block}{\large Některé skupiny operaci \texttt{ioctl}, jejichž rozbor podporuje strace}
\begin{columns}
	\column{6cm}
		\begin{itemize}
			\item BLK*
			\item BTRFS\_*
			\item DM\_*
			\item EV* (evdev)
			\item GPIO\_*
			\item HDIO\_*
			\item KD*
			\item LIRC\_*
			\item LOOP\_*
			\item MEM* (MTD)
			\item NBD\_*
			\item NS\_*
		\end{itemize}
	\column{6cm}
		\begin{itemize}
			\item PERF\_EVENT\_IOC\_*
			\item PTP\_*
			\item RANDOM\_*
			\item RTC\_*
			\item SECCOMP\_*
			\item SIOC\_*
			\item SG\_*
			\item TEE\_*
			\item UBI\_*
			\item UFFDIO\_*
			\item VIDIOC\_*
			\item WDIOC\_*
		\end{itemize}
\end{columns}
\end{block}
\end{frame}

%%%%%%%
\begin{frame}[fragile]{Rozbor operaci \texttt{ioctl} \hfill [2/2]}
\begin{block}{\scriptsize strace -v -{}-trace=ioctl dmsetup ls}
\scriptsize
\begin{alltt}
ioctl(3, DM_VERSION, \{version=4.0.0, data_size=16384, flags=DM_EXISTS_FLAG\}
\symlinecont => \{version=4.43.0, data_size=16384, flags=DM_EXISTS_FLAG\}) = 0
ioctl(3, DM_LIST_DEVICES, \{version=4.0.0, data_size=16384, data_start=312,
\symlinecont flags=DM_EXISTS_FLAG\} => \{version=4.43.0, data_size=472, data_start=312,
\symlinecont flags=DM_EXISTS_FLAG, \{dev=makedev(0xfe, 0), name="nvme0n1p3_crypt", event_nr=0\},
\symlinecont \{dev=makedev(0xfe, 0x3), name="nurgle--vg-home", event_nr=0\},
\symlinecont \{dev=makedev(0xfe, 0x1), name="nurgle--vg-root", event_nr=0\},
\symlinecont \{dev=makedev(0xfe, 0x2), name="nurgle--vg-swap_1"\}\}) = 0
\end{alltt}
\end{block}
\begin{block}{\scriptsize\url{https://git.kernel.org/pub/scm/linux/kernel/git/legion/kbd.git/commit/?id=1002d3b56b72}}
\scriptsize
\begin{alltt}
Author: Alexey Gladkov <gladkov.alexey@gmail.com>
Date:   Tue Apr 25 17:32:11 2023 +0200
\end{alltt}
\vspace{-3ex}
\begin{alltt}
    tests: Use strace to track syscalls
\end{alltt}
\vspace{-3ex}
\begin{alltt}
    Now strace is powerful enough to show ioctls specific to console
    configuration. Additional benefit of using strace is that we can
    enable end-to-end tests for statically linked utilities.
\end{alltt}
\vspace{-3ex}
\begin{alltt}
    Signed-off-by: Alexey Gladkov <gladkov.alexey@gmail.com>
\end{alltt}
\vspace{-3ex}
\begin{alltt}
 37 files changed, 15955 insertions(+), 105164 deletions(-)
\end{alltt}
\end{block}
\end{frame}

%%%%%%%
\begin{frame}[fragile]{Rozbor protokolů netlink}
\scriptsize
\begin{block}{\large Podporované v současnosti protokoly netlink}voleb\begin{itemize}
\item NETLINK\_AUDIT
\item NETLINK\_CRYPTO
\item NETLINK\_KOBJECT\_UEVENT
\item NETLINK\_NETFILTER
\item NETLINK\_ROUTE
\item NETLINK\_SELINUX
\item NETLINK\_SOCK\_DIAG
\item NETLINK\_XFRM
\item NETLINK\_GENERIC
\end{itemize}
\end{block}
\begin{block}{\large NETLINK\_ROUTE: ip route list table all}
\begin{alltt}
local 127.0.0.0/8 dev lo table local proto kernel scope host src 127.0.0.1
local 127.0.0.1 dev lo table local proto kernel scope host src 127.0.0.1
broadcast 127.255.255.255 dev lo table local proto kernel scope link src 127.0.0.1
local ::1 dev lo table local proto kernel metric 0 pref medium
\end{alltt}
\end{block}
\end{frame}

%%%%%%%
\begin{frame}[fragile]{Rozbor protokolů netlink: NETLINK\_ROUTE}
\tiny
\begin{block}{\large strace -{}-trace=sendto,recvmsg ip route list}
\begin{alltt}
sendto(3, [[\textcolor{darkgreen}{\{nlmsg_len=28, nlmsg_type=RTM_GETROUTE, nlmsg_flags=NLM_F_REQUEST|NLM_F_DUMP, nlmsg_seq=135792468, nlmsg_pid=0\}}, \symlinebreak
\symlinecont \textcolor{blue}{\{rtm_family=AF_UNSPEC, rtm_dst_len=0, rtm_src_len=0, rtm_tos=0, rtm_table=RT_TABLE_UNSPEC, \symlinebreak
\symlinecont rtm_protocol=RTPROT_UNSPEC, rtm_scope=RT_SCOPE_UNIVERSE, rtm_type=RTN_UNSPEC, rtm_flags=0\}}], \symlinebreak
\symlinecont \textcolor{darkgreen}{\{nlmsg_len=0, nlmsg_type=0, nlmsg_flags=0, nlmsg_seq=0, nlmsg_pid=0\}}], 156, 0, NULL, 0) = 156

recvmsg(3, \{msg_name=\{sa_family=AF_NETLINK, nl_pid=0, nl_groups=00000000\}, msg_namelen=12, \symlinebreak
\symlinecont msg_iov=[\{iov_base=[[\textcolor{darkgreen}{\{nlmsg_len=60, nlmsg_type=RTM_NEWROUTE, \symlinebreak
\symlinecont nlmsg_flags=NLM_F_MULTI|NLM_F_DUMP_FILTERED, nlmsg_seq=135792468, nlmsg_pid=1234567\}}, \symlinebreak
\symlinecont \textcolor{blue}{\{rtm_family=AF_INET, rtm_dst_len=8, rtm_src_len=0, rtm_tos=0, rtm_table=RT_TABLE_LOCAL, \symlinebreak
\symlinecont rtm_protocol=RTPROT_KERNEL, rtm_scope=RT_SCOPE_HOST, rtm_type=RTN_LOCAL, rtm_flags=0\}}, \symlinebreak
\symlinecont \textcolor{darkred}{[[\{nla_len=8, nla_type=RTA_TABLE\}, RT_TABLE_LOCAL], [\{nla_len=8, nla_type=RTA_DST\}, inet_addr("127.0.0.0")], \symlinebreak
\symlinecont [\{nla_len=8, nla_type=RTA_PREFSRC\}, inet_addr("127.0.0.1")], [\{nla_len=8, nla_type=RTA_OIF\}, if_nametoindex("lo")]]}], \symlinebreak
\symlinecont [\textcolor{darkgreen}{\{nlmsg_len=60, nlmsg_type=RTM_NEWROUTE, nlmsg_flags=NLM_F_MULTI|NLM_F_DUMP_FILTERED, \symlinebreak
\symlinecont nlmsg_seq=135792468, nlmsg_pid=1234567\}}, \symlinebreak
\symlinecont \textcolor{blue}{\{rtm_family=AF_INET, rtm_dst_len=32, rtm_src_len=0, rtm_tos=0, rtm_table=RT_TABLE_LOCAL, \symlinebreak
\symlinecont rtm_protocol=RTPROT_KERNEL, rtm_scope=RT_SCOPE_HOST, rtm_type=RTN_LOCAL, rtm_flags=0\}}, \symlinebreak
\symlinecont \textcolor{darkred}{[[\{nla_len=8, nla_type=RTA_TABLE\}, RT_TABLE_LOCAL], [\{nla_len=8, nla_type=RTA_DST\}, inet_addr("127.0.0.1")], \symlinebreak
\symlinecont [\{nla_len=8, nla_type=RTA_PREFSRC\}, inet_addr("127.0.0.1")], [\{nla_len=8, nla_type=RTA_OIF\}, if_nametoindex("lo")]]}], \symlinebreak
\symlinecont [\textcolor{darkgreen}{\{nlmsg_len=60, nlmsg_type=RTM_NEWROUTE, nlmsg_flags=NLM_F_MULTI|NLM_F_DUMP_FILTERED, \symlinebreak
\symlinecont nlmsg_seq=135792468, nlmsg_pid=1234567\}}, \symlinebreak
\symlinecont \textcolor{blue}{\{rtm_family=AF_INET, rtm_dst_len=32, rtm_src_len=0, rtm_tos=0, rtm_table=RT_TABLE_LOCAL, \symlinebreak
\symlinecont rtm_protocol=RTPROT_KERNEL, rtm_scope=RT_SCOPE_LINK, rtm_type=RTN_BROADCAST, rtm_flags=0\}}, \symlinebreak
\symlinecont \textcolor{darkred}{[[\{nla_len=8, nla_type=RTA_TABLE\}, RT_TABLE_LOCAL], [\{nla_len=8, nla_type=RTA_DST\}, inet_addr("127.255.255.255")], \symlinebreak
\symlinecont [\{nla_len=8, nla_type=RTA_PREFSRC\}, inet_addr("127.0.0.1")], [\{nla_len=8, nla_type=RTA_OIF\}, if_nametoindex("lo")]]}]], \symlinebreak
\symlinecont iov_len=32768\}], msg_iovlen=1, msg_controllen=0, msg_flags=0\}, 0) = 180

\ldots
\end{alltt}
\end{block}
\end{frame}

%%%%%%%
\begin{frame}{Výpis informace o souborových deskriptorech}
\large
\begin{block}{\large \textbf{-{}-decode-fds=\textit{set}}}
\begin{description}
	\item[path]: výpis souborové cesty spojené s deskriptorem a \texttt{AT\_FDCWD}
	\item[socket]: výpis informace o protokolu a adresách spojené s sokety
	\item[dev]: výpis čísel znakových/blokových zařízení
	\item[pidfd]: výpis PID spojených s deskriptorem pidfd
	\item[signalfd]: výpis signálové masky spojené s deskriptorem signalfd
	\item[eventfd]: výpis informace o události spojené s deskriptorem eventfd
\end{description}
\end{block}
\begin{block}{\large Implicitní hodnoty a pseudonymy}
\begin{itemize}
\item Standardně je \textbf{-{}-decode-fds}=\textbf{none}
\item \textbf{-y} je pseudonym pro \textbf{-{}-decode-fds}=\textbf{path}
\item \textbf{-yy} je pseudonym pro \textbf{-{}-decode-fds}=\textbf{all}
\end{itemize}
\end{block}
\end{frame}

%%%%%%%
\begin{frame}[fragile]{Výpis informace spojené s zařízeními and sokety}
\begin{block}{\large strace \textbf{-yy} -e \%desc,\%network nc 127.0.0.1 22 < /dev/null}
\scriptsize
\begin{alltt}
\ldots
socket(AF_INET, SOCK_STREAM, IPPROTO_TCP) = 3\textcolor{darkred}{<TCP:[518663]>}
connect(3\textcolor{darkred}{<TCP:[518663]>}, {sa_family=AF_INET, sin_port=htons(22), \symlinebreak
\symlinecont sin_addr=inet_addr("127.0.0.1")}, 16) = 0
poll([{fd=3\textcolor{darkred}{<TCP:[127.0.0.1:45678->127.0.0.1:22]>}, events=POLLIN}, \symlinebreak
\symlinecont {fd=0\textcolor{darkgreen}{</dev/null<char 1:3>>}, events=POLLIN}], 2, -1) = 1 ([{fd=0, revents=POLLIN}])
read(0\textcolor{darkgreen}{</dev/null<char 1:3>>}, "", 2048)  = 0
shutdown(3\textcolor{darkred}{<TCP:[127.0.0.1:45678->127.0.0.1:22]>}, SHUT_WR) = 0
poll([{fd=3\textcolor{darkred}{<TCP:[127.0.0.1:45678->127.0.0.1:22]>}, events=POLLIN}, {fd=-1}], 2, -1) \symlinebreak
\symlinecont = 1 ([{fd=3, revents=POLLIN}])
read(3\textcolor{darkred}{<TCP:[127.0.0.1:45678->127.0.0.1:22]>}, "SSH-2.0-OpenSSH_9.4{\textbackslash}r{\textbackslash}n", 2048) = 21
write(1\textcolor{darkgreen}{</dev/pts/1<char 136:1>>}, "SSH-2.0-OpenSSH_9.4{\textbackslash}r{\textbackslash}n", 21) = 21
poll([{fd=3\textcolor{darkred}{<TCP:[127.0.0.1:45678->127.0.0.1:22]>}, events=POLLIN}, {fd=-1}], 2, -1) \symlinebreak
\symlinecont = 1 ([{fd=3, revents=POLLIN|POLLHUP}])
read(3\textcolor{darkred}{<TCP:[127.0.0.1:45678->127.0.0.1:22]>}, "", 2048) = 0
shutdown(3\textcolor{darkred}{<TCP:[127.0.0.1:45678->127.0.0.1:22]>}, SHUT_RD) \symlinebreak
\symlinecont = -1 ENOTCONN (Transport endpoint is not connected)
close(3\textcolor{darkred}{<TCP:[127.0.0.1:45678->127.0.0.1:22]>}) = 0
+++ exited with 0 +++
\end{alltt}
\end{block}
\end{frame}

%%%%%%%
\begin{frame}[fragile]{Rozbor signálových masek deskriptorů signalfd: \textbf{-{}-decode-fds}=\textbf{signalfd}}
\begin{block}{strace -q -yy --trace=/signalfd,/epoll\_ctl --syscall-limit=3 \textbackslash \\ /usr/lib/systemd/systemd-portabled}
\begin{alltt}
\small
signalfd4(-1, [INT], 8, SFD_CLOEXEC|SFD_NONBLOCK) = 4<signalfd:[\textcolor{darkred}{INT}]>
epoll_ctl(3<anon_inode:[eventpoll]>, EPOLL_CTL_ADD, 4<signalfd:[\textcolor{darkred}{INT}]>, \symlinebreak
\symlinecont {events=EPOLLIN, data={u32=112224560, u64=94214514829616}}) = 0
signalfd4(4<signalfd:[\textcolor{darkred}{INT}]>, [INT TERM], 8, SFD_CLOEXEC|SFD_NONBLOCK) \symlinebreak
\symlinecont = 4<signalfd:[\textcolor{darkred}{INT TERM}]>
\end{alltt}
\end{block}
\end{frame}

%%%%%%%
\begin{frame}{Rozbor různé informace spojené s ID procesů}
\large
\begin{block}{\textbf{-{}-decode-pids}=\textit{set}}
\begin{description}
	\item[comm]: výpis jmena příkazu (právě \texttt{comm}) spojeného s ID vlákna nebo procesa
	\item[pidns]: výpis ID vlákna, procesa, procesaové skupiny, nebo seance v jmenném prostoru strace,
                      pokud sledovaný proces je v jiném jmenném prostoru PID
\end{description}
\end{block}

\begin{block}{\large Implicitní hodnoty a pseudonymy}
\begin{itemize}
\item Standardně je \textbf{-{}-decode-pids}=\textbf{none}
\item \textbf{-Y} je pseudonym pro \textbf{-{}-decode-pids}=\textbf{comm}
\item \textbf{-{}-decode-pids}=\textbf{all} je pseudonym pro \textbf{-{}-decode-pids}=\textbf{comm},\textbf{pidns}
\end{itemize}
\end{block}
\end{frame}

%%%%%%%
\begin{frame}[fragile]{Výpis jmen příkazů u PID: \textbf{-{}-decode-pids}=\textbf{comm}, \textbf{-Y} \hfill }
\small
\begin{block}{strace -f \textbf{-Y} -e\%process timeout 1 sleep 2}
\scriptsize
\begin{alltt}
execve("/usr/bin/timeout", ["timeout", "1", "sleep", "2"], 0x7ffcfb9bb348 /* 17 vars */) = 0
clone(child_stack=NULL, flags=CLONE_CHILD_CLEARTID|CLONE_CHILD_SETTID|SIGCHLD, \symlinebreak
\symlinecont child_tidptr=0x7f116942da10) = 123452\textcolor{darkgreen}{<timeout>}
strace: Process 123452 attached
[pid 123451\textcolor{darkgreen}{<timeout>}] wait4(123452\textcolor{darkgreen}{<timeout>}, 0x7ffc132b1a28, WNOHANG, NULL) = 0
[pid 123452\textcolor{darkgreen}{<timeout>}] execve("/bin/sleep", ["sleep", "2"], 0x7ffc132b1ce0 /* 17 vars */) = 0
[pid 123451\textcolor{darkgreen}{<timeout>}] --- SIGALRM \{si_signo=SIGALRM, si_code=SI_TIMER, \symlinebreak
\symlinecont si_timerid=0, si_overrun=0, si_int=0, si_ptr=NULL\} ---
[pid 123451\textcolor{darkgreen}{<timeout>}] kill(123452\textcolor{darkred}{<sleep>}, SIGTERM) = 0
[pid 123452\textcolor{darkred}{<sleep>}] --- SIGTERM \{si_signo=SIGTERM, si_code=SI_USER, \symlinebreak
\symlinecont si_pid=123451\textcolor{darkgreen}{<timeout>}, si_uid=1000\} ---
[pid 123451\textcolor{darkgreen}{<timeout>}] kill(0, SIGTERM) = 0
[pid 123451\textcolor{darkgreen}{<timeout>}] --- SIGTERM \{si_signo=SIGTERM, si_code=SI_USER, \symlinebreak
\symlinecont si_pid=123451\textcolor{darkgreen}{<timeout>}, si_uid=1000\} ---
[pid 123451\textcolor{darkgreen}{<timeout>}] kill(123452\textcolor{darkred}{<sleep>}, SIGCONT) = 0
[pid 123452\textcolor{darkred}{<sleep>}] +++ killed by SIGTERM +++
--- SIGCHLD \{si_signo=SIGCHLD, si_code=CLD_KILLED, si_pid=123452\textcolor{darkred}{<sleep>}, \symlinebreak
\symlinecont si_uid=1000, si_status=SIGTERM, si_utime=0, si_stime=0\} ---
kill(0, SIGCONT)                        = 0
--- SIGCONT \{si_signo=SIGCONT, si_code=SI_USER, si_pid=123451\textcolor{darkgreen}{<timeout>}, si_uid=1000\} ---
wait4(123452\textcolor{darkred}{<sleep>}, [{WIFSIGNALED(s) && WTERMSIG(s) == SIGTERM}], WNOHANG, NULL) = 123452
exit_group(124)                         = ?
+++ exited with 124 +++
\end{alltt}
\end{block}
\end{frame}

%%%%%%%
\begin{frame}[fragile]{Výpis jmen příkazů a překlad PID mezi jimennými prostory}
\large
\begin{block}{strace \textbf{-{}-decode-pids=all} -q -f -{}-trace=clone,kill \textbackslash \\ unshare -Urpf sh -c 'sleep 2 \& sleep 1 \&\& kill \$!'}
\scriptsize
\begin{alltt}
clone(child_stack=NULL, flags=CLONE_CHILD_CLEARTID|CLONE_CHILD_SETTID|SIGCHLD, \symlinebreak
\symlinecont child_tidptr=0x7ff30b283a10) = 123456\textcolor{darkgreen}{<unshare>}
[pid 123456\textcolor{darkgreen}{<sh>}] clone(child_stack=NULL, flags=CLONE_CHILD_CLEARTID|CLONE_CHILD_SETTID|SIGCHLD, \symlinebreak
\symlinecont child_tidptr=0x7f6dccd89a10) = 2\textcolor{darkgreen}{<sh>} \textcolor{darkred}{/* 123457 in strace's PID NS */}
[pid 123456\textcolor{darkgreen}{<sh>}] clone(child_stack=NULL, flags=CLONE_CHILD_CLEARTID|CLONE_CHILD_SETTID|SIGCHLD, \symlinebreak
\symlinecont child_tidptr=0x7f6dccd89a10) = 3\textcolor{darkgreen}{<sh>} \textcolor{darkred}{/* 123458 in strace's PID NS */}
[pid 123458\textcolor{darkgreen}{<sleep>}] +++ exited with 0 +++
[pid 123456\textcolor{darkgreen}{<sh>}] --- SIGCHLD \{si_signo=SIGCHLD, si_code=CLD_EXITED, \symlinebreak
\symlinecont si_pid=3, si_uid=0, si_status=0, si_utime=0, si_stime=0\} ---
[pid 123456\textcolor{darkgreen}{<sh>}] kill(2\textcolor{darkgreen}{<sleep>} \textcolor{darkred}{/* 123457 in strace's PID NS */}, SIGTERM) = 0
[pid 123457\textcolor{darkgreen}{<sleep>}] --- SIGTERM \{si_signo=SIGTERM, si_code=SI_USER, \symlinebreak
\symlinecont si_pid=1\textcolor{darkgreen}{<sh>} \textcolor{darkred}{/* 123456 in strace's PID NS */}, si_uid=0\} ---
[pid 123457\textcolor{darkgreen}{<sleep>}] +++ killed by SIGTERM +++
[pid 123456\textcolor{darkgreen}{<sh>}] +++ exited with 0 +++
--- SIGCHLD \{si_signo=SIGCHLD, si_code=CLD_EXITED, si_pid=123456, \symlinebreak
\symlinecont si_uid=0, si_status=0, si_utime=0, si_stime=0\} ---
+++ exited with 0 +++
\end{alltt}
\end{block}
\end{frame}

%%%%%%%
\begin{frame}[fragile]{Výpis zásobníku volání funkci: \textbf{-k}}
\begin{block}{\large strace -qq -P /dev/full cat /dev/null > /dev/full}
\scriptsize
\begin{alltt}
fstat(1, {st_mode=S_IFCHR|0666, st_rdev=makedev(1, 7), ...}) = 0
close(1)                                = 0
\end{alltt}
\end{block}

\begin{block}{\large strace \textbf{-k} -qq -P /dev/full cat /dev/null > /dev/full}
\tiny
\begin{alltt}
fstat(1, {st_mode=S_IFCHR|0666, st_rdev=makedev(0x1, 0x7), ...}) = 0
 > /usr/lib/x86_64-linux-gnu/libc.so.6(__fxstat+0x13) [0x107653]
 > /usr/bin/cat(main+0x1a8) [0x2548]
 > /usr/lib/x86_64-linux-gnu/libc.so.6(__libc_start_call_main+0x79) [0x29c89]
 > /usr/lib/x86_64-linux-gnu/libc.so.6(__libc_start_main@@GLIBC_2.34+0x84) [0x29d44]
 > /usr/bin/cat(_start+0x29) [0x2f09]
close(1)                                = 0
 > /usr/lib/x86_64-linux-gnu/libc.so.6(__close_nocancel+0x7) [0xfac57]
 > /usr/lib/x86_64-linux-gnu/libc.so.6(_IO_file_close_it@@GLIBC_2.2.5+0x62) [0x83492]
 > /usr/lib/x86_64-linux-gnu/libc.so.6(fclose@@GLIBC_2.2.5+0x102) [0x77932]
 > /usr/bin/cat(close_stream+0x1c) [0x631c]
 > /usr/bin/cat(close_stdout+0x12) [0x3492]
 > /usr/lib/x86_64-linux-gnu/libc.so.6(__run_exit_handlers+0x155) [0x41a35]
 > /usr/lib/x86_64-linux-gnu/libc.so.6(exit+0x19) [0x41b69]
 > /usr/lib/x86_64-linux-gnu/libc.so.6(__libc_start_call_main+0x80) [0x29c90]
 > /usr/lib/x86_64-linux-gnu/libc.so.6(__libc_start_main@@GLIBC_2.34+0x84) [0x29d44]
 > /usr/bin/cat(_start+0x29) [0x2f09]
\end{alltt}
\end{block}
\end{frame}

%%%%%%%
\begin{frame}[fragile]{Výpis místa volání v zdrojovém kódu: \textbf{-kk}}
\begin{block}{\large strace -qq -P /dev/full cat /dev/null > /dev/full}
\scriptsize
\begin{alltt}
fstat(1, {st_mode=S_IFCHR|0666, st_rdev=makedev(1, 7), ...}) = 0
close(1)                                = 0
\end{alltt}
\end{block}

\begin{block}{\large strace \textbf{-kk} -qq -P /dev/full cat /dev/null > /dev/full}
\tiny
\begin{alltt}
fstat(1, {st_mode=S_IFCHR|0666, st_rdev=makedev(0x1, 0x7), ...}) = 0
 > /usr/lib/x86_64-linux-gnu/libc.so.6(__fxstat+0x13) [0x107653] ../sysdeps/unix/sysv/linux/fxstat64.c:51
 > /usr/bin/cat(main+0x1a8) [0x2548] /usr/include/x86_64-linux-gnu/sys/stat.h:469
 > /usr/lib/x86_64-linux-gnu/libc.so.6(__libc_start_call_main+0x79) [0x29c89] ../sysdeps/nptl/libc_start_call_main.h:58
 > /usr/lib/x86_64-linux-gnu/libc.so.6(__libc_start_main@@GLIBC_2.34+0x84) [0x29d44] ../csu/libc-start.c:360
 > /usr/bin/cat(_start+0x29) [0x2f09]
close(1)                                = 0
 > /usr/lib/x86_64-linux-gnu/libc.so.6(__close_nocancel+0x7) [0xfac57] ../sysdeps/unix/sysv/linux/close_nocancel.c:26
 > /usr/lib/x86_64-linux-gnu/libc.so.6(_IO_file_close_it@@GLIBC_2.2.5+0x62) [0x83492] ./libio/fileops.c:142
 > /usr/lib/x86_64-linux-gnu/libc.so.6(fclose@@GLIBC_2.2.5+0x102) [0x77932] ./libio/iofclose.c:53
 > /usr/bin/cat(close_stream+0x1c) [0x631c] lib/close-stream.c:60
 > /usr/bin/cat(close_stdout+0x12) [0x3492] lib/closeout.c:119
 > /usr/lib/x86_64-linux-gnu/libc.so.6(__run_exit_handlers+0x155) [0x41a35] ./stdlib/exit.c:108
 > /usr/lib/x86_64-linux-gnu/libc.so.6(exit+0x19) [0x41b69] ./stdlib/exit.c:138
 > /usr/lib/x86_64-linux-gnu/libc.so.6(__libc_start_call_main+0x80) [0x29c90] ../sysdeps/nptl/libc_start_call_main.h:74
 > /usr/lib/x86_64-linux-gnu/libc.so.6(__libc_start_main@@GLIBC_2.34+0x84) [0x29d44] ../csu/libc-start.c:360
 > /usr/bin/cat(_start+0x29) [0x2f09]
\end{alltt}
\end{block}
\end{frame}

%%%%%%%
\begin{frame}[fragile]{Výpis kontextů SELinux: \textbf{-{}-secontext}}
\large
\begin{block}{strace -P /dev/null cat /dev/null}
\scriptsize
\begin{alltt}
openat(AT_FDCWD, "/dev/null", O_RDONLY) = 3
newfstatat(3, "", \{st_mode=S_IFCHR|0666, st_rdev=makedev(0x1, 0x3), ...\}, AT_EMPTY_PATH) = 0
fadvise64(3, 0, 0, POSIX_FADV_SEQUENTIAL) = 0
read(3, "", 131072)                     = 0
close(3)                                = 0
+++ exited with 0 +++
\end{alltt}
\end{block}
\large
\begin{block}{strace \textbf{-{}-secontext} -P /dev/null cat /dev/null}
\scriptsize
\begin{alltt}
[\textcolor{darkgreen}{unconfined_t}] openat(AT_FDCWD, "/dev/null" [\textcolor{darkred}{null_device_t}], O_RDONLY) = 3 [\textcolor{darkred}{null_device_t}]
[\textcolor{darkgreen}{unconfined_t}] newfstatat(3 [\textcolor{darkred}{null_device_t}], "", \{st_mode=S_IFCHR|0666, \symlinebreak
\symlinecont st_rdev=makedev(0x1, 0x3), ...\}, AT_EMPTY_PATH) = 0
[\textcolor{darkgreen}{unconfined_t}] fadvise64(3 [\textcolor{darkred}{null_device_t}], 0, 0, POSIX_FADV_SEQUENTIAL) = 0
[\textcolor{darkgreen}{unconfined_t}] read(3 [\textcolor{darkred}{null_device_t}], "", 131072) = 0
[\textcolor{darkgreen}{unconfined_t}] close(3 [\textcolor{darkred}{null_device_t}]) = 0
+++ exited with 0 +++
\end{alltt}
\end{block}
\end{frame}

%%%%%%%
\begin{frame}[fragile]{Výpis kontextů SELinux: \textbf{-{}-secontext}=\textbf{mismatch}}
\large
\begin{block}{Příklad nesouladu kontextů SELinux}
\scriptsize
\begin{alltt}
\$ matchpathcon $PWD/config.h
/home/me/strace/src/config.h	unconfined_u:object_r:user_home_t:s0
\$ ls -Z $PWD/config.h
system_u:object_r:user_home_t:s0 /home/me/strace/src/config.h
\end{alltt}
\end{block}
\large
\begin{block}{strace \textbf{-{}-secontext}=\textbf{full},\textbf{mismatch}}
\scriptsize
\begin{alltt}
\$ strace --secontext=full,mismatch --trace=\%file stat config.h
\ldots
[unconfined_u:unconfined_r:unconfined_t:s0-s0:c0.c1023] statx(AT_FDCWD, "config.h" \symlinebreak
\symlinecont [\textcolor{darkred}{system_u:object_r:user_home_t:s0!!unconfined_u:object_r:user_home_t:s0}], \symlinebreak
\symlinecont AT_STATX_SYNC_AS_STAT|AT_SYMLINK_NOFOLLOW|AT_NO_AUTOMOUNT, \symlinebreak
\symlinecont STATX_ALL, {stx_mask=STATX_ALL|STATX_MNT_ID, stx_attributes=0, stx_mode=S_IFREG|0644, \symlinebreak
\symlinecont stx_size=50008, ...}) = 0
\end{alltt}
\end{block}
\end{frame}

%%%%%%%
\begin{frame}[fragile]{Formát jmenných konstant a příznaků: \textbf{-X}/\textbf{-{}-const-print-style}}
\begin{block}{\large strace -e /open cat /dev/null}
\scriptsize
\begin{alltt}
openat(\textcolor{red}{AT_FDCWD}, "/etc/ld.so.cache", \textcolor{red}{O_RDONLY|O_CLOEXEC}) = 3
openat(\textcolor{red}{AT_FDCWD}, "/lib64/libc.so.6", \textcolor{red}{O_RDONLY|O_CLOEXEC}) = 3
openat(\textcolor{red}{AT_FDCWD}, "/dev/null", \textcolor{red}{O_RDONLY}) = 3
+++ exited with 0 +++
\end{alltt}
\end{block}

\begin{block}{\large strace \textbf{-X verbose} -e /open cat /dev/null}
\scriptsize
\begin{alltt}
openat(\textcolor{red}{-100 /* AT_FDCWD */}, "/etc/ld.so.cache",
       \textcolor{red}{0x80000 /* O_RDONLY|O_CLOEXEC */}) = 3
openat(\textcolor{red}{-100 /* AT_FDCWD */}, "/lib64/libc.so.6",
       \textcolor{red}{0x80000 /* O_RDONLY|O_CLOEXEC */}) = 3
openat(\textcolor{red}{-100 /* AT_FDCWD */}, "/dev/null", \textcolor{red}{0 /* O_RDONLY */}) = 3
+++ exited with 0 +++
\end{alltt}
\end{block}

\begin{block}{\large strace \textbf{-X raw} -e /open cat /dev/null}
\scriptsize
\begin{alltt}
openat(\textcolor{red}{-100}, "/etc/ld.so.cache", \textcolor{red}{0x80000}) = 3
openat(\textcolor{red}{-100}, "/lib64/libc.so.6", \textcolor{red}{0x80000}) = 3
openat(\textcolor{red}{-100}, "/dev/null", \textcolor{red}{0})            = 3
+++ exited with 0 +++
\end{alltt}
\end{block}
\end{frame}

%%%%%%%
\begin{frame}{Filtrovaní podle návratného stavu systémového volání}
\begin{block}{\textbf{-e status=\textit{set}}}
\begin{description}
	\item[successful]: systémové volání nevrátilo chybu, pseudonym pro \textbf{-z}
	\item[failed]: systémové volání vrátilo chybu, pseudonym pro \textbf{-Z}
	\item[unfinished]: systémové volání nevrátilo
	\item[detached]: proces byl odpojen do návratu systémového volání
	\item[unavailable]: systémové volání se vrátilo, ale se nepodařilo vyzvednout návratnou hodnotu
\end{description}
Standardně je \textbf{-e status}=\textbf{all}.
\end{block}
\end{frame}

%%%%%%%
\begin{frame}[fragile]{Filtrovaní podle návratného stavu systémového volání: \textbf{-z}, \textbf{-Z}}
\scriptsize
\begin{block}{env -i LD\_LIBRARY\_PATH=/lib64 strace -z -e\%file /bin/cat < /dev/null}
\begin{alltt}
execve("/bin/cat", ["/bin/cat"], 0x7ffc2b781ca0 /* 1 var */) = 0
openat(AT_FDCWD, "/lib64/libc.so.6", O_RDONLY|O_CLOEXEC) = 3
+++ exited with 0 +++
\end{alltt}
\end{block}
\begin{block}{env -i LD\_LIBRARY\_PATH=/lib64 strace -Z -e\%file /bin/cat </dev/null}
\begin{alltt}
access("/etc/ld.so.preload", R_OK) = -1 ENOENT (No such file or directory)
openat(AT_FDCWD, "/lib64/tls/x86_64/x86_64/libc.so.6", O_RDONLY|O_CLOEXEC) = -1 ENOENT (No suc
stat("/lib64/tls/x86_64/x86_64", 0x7ffdcc6a0c20) = -1 ENOENT (No such file or directory)
openat(AT_FDCWD, "/lib64/tls/x86_64/libc.so.6", O_RDONLY|O_CLOEXEC) = -1 ENOENT (No such file
stat("/lib64/tls/x86_64", 0x7ffdcc6a0c20) = -1 ENOENT (No such file or directory)
openat(AT_FDCWD, "/lib64/tls/x86_64/libc.so.6", O_RDONLY|O_CLOEXEC) = -1 ENOENT (No such file
stat("/lib64/tls/x86_64", 0x7ffdcc6a0c20) = -1 ENOENT (No such file or directory)
openat(AT_FDCWD, "/lib64/tls/libc.so.6", O_RDONLY|O_CLOEXEC) = -1 ENOENT (No such file or dire
stat("/lib64/tls", 0x7ffdcc6a0c20) = -1 ENOENT (No such file or directory)
openat(AT_FDCWD, "/lib64/x86_64/x86_64/libc.so.6", O_RDONLY|O_CLOEXEC) = -1 ENOENT (No such fi
stat("/lib64/x86_64/x86_64", 0x7ffdcc6a0c20) = -1 ENOENT (No such file or directory)
openat(AT_FDCWD, "/lib64/x86_64/libc.so.6", O_RDONLY|O_CLOEXEC) = -1 ENOENT (No such file or d
stat("/lib64/x86_64", 0x7ffdcc6a0c20) = -1 ENOENT (No such file or directory)
openat(AT_FDCWD, "/lib64/x86_64/libc.so.6", O_RDONLY|O_CLOEXEC) = -1 ENOENT (No such file or d
stat("/lib64/x86_64", 0x7ffdcc6a0c20) = -1 ENOENT (No such file or directory)
+++ exited with 0 +++
\end{alltt}
\end{block}
\end{frame}

%%%%%%%
\begin{frame}[fragile]{Filtrovaní podle návratného stavu systémového volání: agregace u \textbf{-z} \hfill [1/2]}
\begin{block}{strace -o log -f -e signal=none -e trace=execve,nanosleep \textbackslash \\
	sh -c 'sleep 0.1 \& sleep 0.2 \& sleep 0.3' \&\& cat log}
\scriptsize
\begin{alltt}
13475 execve("/bin/sh", ["sh", "-c", "sleep 0.1 & sleep 0.2 & sleep 0."...],
      0x5631be4f87a8 /* 42 vars */) = 0
13476 execve("/bin/sleep", ["sleep", "0.1"], 0xe4c4f0 /* 33 vars */ <unfinished ...>
13477 execve("/bin/sleep", ["sleep", "0.2"], 0xe4c4f0 /* 33 vars */ <unfinished ...>
13478 execve("/bin/sleep", ["sleep", "0.3"], 0xe4c4f0 /* 33 vars */ <unfinished ...>
13476 <... execve resumed>)             = 0
13477 <... execve resumed>)             = 0
13478 <... execve resumed>)             = 0
13476 nanosleep({tv_sec=0, tv_nsec=100000000},  <unfinished ...>
13477 nanosleep({tv_sec=0, tv_nsec=200000000},  <unfinished ...>
13478 nanosleep({tv_sec=0, tv_nsec=300000000},  <unfinished ...>
13476 <... nanosleep resumed>NULL)      = 0
13476 +++ exited with 0 +++
13477 <... nanosleep resumed>NULL)      = 0
13477 +++ exited with 0 +++
13478 <... nanosleep resumed>NULL)      = 0
13478 +++ exited with 0 +++
13475 +++ exited with 0 +++
\end{alltt}
\end{block}
\end{frame}

%%%%%%%
\begin{frame}[fragile]{Filtrovaní podle návratného stavu systémového volání: agregace u \textbf{-z} \hfill [2/2]}
\begin{block}{strace -o log \textbf{-z} -f -e signal=none -e trace=execve,nanosleep \textbackslash \\
	sh -c 'sleep 0.1 \& sleep 0.2 \& sleep 0.3' \&\& cat log}
\small
\begin{alltt}
13475 execve("/bin/sh", ["sh", "-c", "sleep 0.1 & sleep 0.2 & sleep 0."...],
      0x5631be4f87a8 /* 42 vars */) = 0
13476 execve("/bin/sleep", ["sleep", "0.1"], 0xe4c4f0 /* 33 vars */) = 0
13477 execve("/bin/sleep", ["sleep", "0.2"], 0xe4c4f0 /* 33 vars */) = 0
13478 execve("/bin/sleep", ["sleep", "0.3"], 0xe4c4f0 /* 33 vars */) = 0
13476 nanosleep({tv_sec=0, tv_nsec=100000000}, NULL) = 0
13476 +++ exited with 0 +++
13477 nanosleep({tv_sec=0, tv_nsec=200000000}, NULL) = 0
13477 +++ exited with 0 +++
13478 nanosleep({tv_sec=0, tv_nsec=300000000}, NULL) = 0
13478 +++ exited with 0 +++
13475 +++ exited with 0 +++
\end{alltt}
\end{block}
\end{frame}

%%%%%%%
\begin{frame}[fragile]{Problémy s filtrací systémových volání}
\begin{block}{\large glibc: open nebo openat?}
\begin{alltt}
glibc-2.25\$ strace -qq \textcolor{red}{-e open} cat /dev/null
open("/etc/ld.so.cache", O_RDONLY|O_CLOEXEC) = 3
open("/lib64/libc.so.6", O_RDONLY|O_CLOEXEC) = 3
open("/dev/null", O_RDONLY)             = 3 \pause
glibc-2.26\$ strace -qq \textcolor{red}{-e open} cat /dev/null \pause
glibc-2.26\$ strace -qq \textcolor{red}{-e openat} cat /dev/null
openat(AT_FDCWD, "/etc/ld.so.cache", O_RDONLY|O_CLOEXEC) = 3
openat(AT_FDCWD, "/lib64/libc.so.6", O_RDONLY|O_CLOEXEC) = 3
openat(AT_FDCWD, "/dev/null", O_RDONLY) = 3 \pause
glibc-2.25\$ strace -qq \textcolor{red}{-e openat} cat /dev/null
\end{alltt}
\end{block}
\pause
\begin{block}{\large Tradiční přístup nefunguje na všech architekturách}
\begin{alltt}
riscv\$ strace \textcolor{red}{-e open,openat}
strace: invalid system call 'open'
\end{alltt}
\end{block}
\end{frame}

%%%%%%%
\begin{frame}[fragile]{Nové možností zadaní filtru systémových volání}
\begin{block}{\large Rychlé řešení pomocí regulárních výrazů}
\texttt{\$ strace -e /open}
\end{block}
\pause
\begin{block}{\large Skutečný výsledek rychlého řešení}
Regulárnímu výrazu \texttt{.*open.*} odpovídají taky následující systémové volání:
\begin{itemize}
\item \texttt{mq\_open}
\item \texttt{open\_by\_handle\_at}
\item \texttt{perf\_event\_open}
\end{itemize}
\end{block}
\pause
\begin{block}{\large Přesný regulární výraz}
\texttt{\$ strace -e '/{\textasciicircum}open(at)?\$'}
\end{block}
\pause
\begin{block}{\large Řešení pomocí volitelné specifikaci systémových volání}
\texttt{\$ strace -e '?open,?openat'}
\end{block}
\end{frame}

%%%%%%%
\begin{frame}[fragile]{Zadání \texttt{argv[0]} prováděného příkazu: \textbf{-{}-argv0=\textit{name}}}
\begin{block}{strace -{}-argv0=\textcolor{darkred}{insmod} -{}-trace=execve kmod -{}-help}
\small
\begin{alltt}
	execve("/bin/kmod", ["\textcolor{darkred}{insmod}", "--help"], 0x7fffeb80fa58 /* 42 vars */) = 0
Usage:
	\textcolor{darkred}{insmod} [options] filename [args]
Options:
	-V, --version     show version
	-h, --help        show this help
+++ exited with 0 +++
\end{alltt}
\end{block}
\end{frame}

%%%%%%%
\begin{frame}[fragile]{Spuštění jako oddělené vnouče: \textbf{-D}/\textbf{-{}-daemonize}}
\begin{block}{\large Regulární volání strace}
\begin{alltt}
$ echo $$ && strace -e none sh -c 'echo $PPID'
1234
23456
+++ exited with 0 +++
$ echo $$ && strace -e none sh -c 'echo $PPID'
1234
23459
+++ exited with 0 +++
\end{alltt}
\end{block}

\begin{block}{\large Volání strace jako démona}
\begin{alltt}
$ echo $$ && strace \textcolor{red}{-D} -e none sh -c 'echo $PPID'
1234
1234
+++ exited with 0 +++
\end{alltt}
\end{block}
\end{frame}

%%%%%%%
\begin{frame}[fragile]{Další oddělení strace: \textbf{-DD}, \textbf{-DDD}}
\begin{block}{strace \textbf{-D}}
\small
\begin{alltt}
\$ timeout -s KILL 1 strace \textcolor{darkred}{-D} -e/nanosleep sleep 2
strace: Process 123457 attached
clock_nanosleep(CLOCK_REALTIME, 0, \{tv_sec=2, tv_nsec=0\}, Killed
\end{alltt}
\end{block}

\begin{block}{strace \textbf{-DD}}
\small
\begin{alltt}
\$ timeout -s KILL 1 strace \textcolor{darkred}{-DD} -e/nanosleep sleep 2
strace: Process 123457 attached
clock_nanosleep(CLOCK_REALTIME, 0, \{tv_sec=2, tv_nsec=0\}, <unfinished ...>) = ?
Killed
+++ killed by SIGKILL +++
\end{alltt}
\end{block}
\end{frame}

%%%%%%%
\begin{frame}[fragile]{Filtrovaní systémových volání pomocí seccomp \hfill [1/2]}
\begin{block}{\textbf{-{}-seccomp-bpf}}
\small
\begin{itemize}
\item Automaticky vytvaruje a připojuje BPF program na filtraci systémových volání
\item Zrychluje zpracování vyfiltrovaných systémových volání o přibližně stokrát
\end{itemize}
\end{block}

\begin{block}{Hanebný příklad s \texttt{dd}}
\scriptsize
\begin{alltt}
\$ dd if=/dev/zero of=/dev/null bs=1 count=1M 2>\&1 | grep -v records
1048576 bytes (1.0 MB, 1.0 MiB) copied, \textcolor{darkred}{0.281794} s, 3.7 MB/s

\$ strace -f -qq -{}-signal=none -{}-trace=fchdir \textbackslash
  dd if=/dev/zero of=/dev/null bs=1 count=1M 2>\&1 | grep -v records
1048576 bytes (1.0 MB, 1.0 MiB) copied, \textcolor{darkred}{13.9573} s, 75.1 kB/s

\$ strace -f -qq -{}-signal=none -{}-trace=fchdir \textcolor{darkgreen}{-{}-seccomp-bpf} \textbackslash
  dd if=/dev/zero of=/dev/null bs=1 count=1M 2>\&1 | grep -v records
1048576 bytes (1.0 MB, 1.0 MiB) copied, \textcolor{darkred}{0.307412} s, 3.4 MB/s
\end{alltt}
\end{block}

\begin{block}{Porovnání zpomalení}
$0.307412 / 0.281794 \approx \textcolor{darkred}{1.09}$ \hfill
$13.9573 / 0.307412 \approx \textcolor{darkred}{45.4}$ \hfill
$13.9573 / 0.281794 \approx \textcolor{darkred}{49.5}$
\end{block}
\end{frame}

%%%%%%%
\begin{frame}[fragile]{Filtrovaní systémových volání pomocí seccomp \hfill [2/2]}
\begin{block}{Omezení}
\begin{itemize}
	\item Nemá efekt pokud není specifikováno \textbf{-{}-follow-forks} (\textbf{-f})
	\item Nekompatibilní s \textbf{-{}-syscall-limit=\textit{number}}
	\item Nekompatibilní s \textbf{-{}-detach-on=\textbf{execve}} (\textbf{-b execve})
	\item Neplatí pro přípojené pomoci \textbf{-{}-attach=\textit{pid}} (\textbf{-p} \textit{pid}) programy
\end{itemize}
\end{block}

\begin{block}{Proč?}
\begin{itemize}
	\item Program seccomp BPF, jež nainstalován stracem, používá rozhraní \texttt{SECCOMP\_RET\_TRACE}
	\item Až nainstalován, seccomp BPF program nemůže být odstraněn
	\item Když není žádný tracer, \texttt{SECCOMP\_RET\_TRACE} se interpretuje jako \texttt{SECCOMP\_RET\_ERRNO},
              takže všechny dříve sledované systémové volání selhávají s chybou \texttt{ENOSYS}
\end{itemize}
\end{block}

Vzhledem k tomu, že \textbf{-{}-seccomp-bpf} je optimalizace, vypíná se v případech, když nemůže být použit.
\end{frame}

%%%%%%%
\begin{frame}[fragile]{Manipulovaní systémovými voláními: injekce chyb \hfill [1/3]}
\begin{block}{\large -e \textbf{inject}=\textit{set}:\textbf{error}=\textit{errno}[:\textbf{when}=\textit{expr}][:\textbf{syscall}=\textit{syscall}]}
\textbf{inject}=\textit{set} -- injekce chyb pro specifikovanou sadu systémových volání \\
\textbf{error}=\textit{errno} -- kód chyby pro navracení \\
\textbf{when}=\textit{expr} -- kdy dělat injekci, v podobě \textit{first[..last][+[step]]} \\
\textbf{syscall}=\textit{syscall} -- injekce specifikovaného systémového volání (ze skupiny \textbf{\%pure}) namísto -1
\end{block}

\begin{block}{\large strace -e /open -e \textbf{inject}=all:\textbf{error}=EACCES:\textbf{when}=3 \textbackslash\\
cat /dev/full /dev/null}
\begin{alltt}
openat(AT_FDCWD, "/etc/ld.so.cache", O_RDONLY|O_CLOEXEC) = 3
openat(AT_FDCWD, "/lib64/libc.so.6", O_RDONLY|O_CLOEXEC) = 3
\textcolor{red}{openat(AT_FDCWD, "/dev/full", O_RDONLY)
 = -1 EACCES (Permission denied) (INJECTED)}
cat: /dev/full: Permission denied
openat(AT_FDCWD, "/dev/null", O_RDONLY) = 3
+++ exited with 1 +++
\end{alltt}
\end{block}
\end{frame}

%%%%%%%
\begin{frame}[fragile]{Manipulovaní systémovými voláními: injekce chyb \hfill [2/3]}
\begin{block}{\large Chyba v python3.5: nesprávné zpracovaní chyb otevíraní \texttt{/dev/urandom}}
\small
\begin{alltt}
\$ strace -P /dev/urandom -e inject=\%file:error=ENOENT python3
\textcolor{red}{openat(AT_FDCWD, "/dev/urandom", O_RDONLY|O_CLOEXEC)
 = -1 ENOENT (No such file or directory) (INJECTED)}
Fatal Python error: Failed to open /dev/urandom
--- SIGSEGV \{si_signo=SIGSEGV, si_code=SEGV_MAPERR, si_addr=0x50\} ---
+++ killed by SIGSEGV +++
Segmentation fault
\end{alltt}
\end{block}

\begin{block}{\large Chyba v python3.5: nesprávné zpracovaní chyb čtení z \texttt{/dev/urandom}}
\small
\begin{alltt}
\$ strace -a0 -e read -P /dev/urandom -e inject=all:error=EIO python3
\textcolor{red}{read(3, 0x8db610, 24) = -1 EIO (Input/output error) (INJECTED)}
Fatal Python error: Failed to read bytes from /dev/urandom
--- SIGSEGV \{si_signo=SIGSEGV, si_code=SEGV_MAPERR, si_addr=0x50\} ---
+++ killed by SIGSEGV +++
Segmentation fault
\end{alltt}
\end{block}
\end{frame}

%%%%%%%
\begin{frame}[fragile]{Manipulovaní systémovými voláními: injekce chyb \hfill [3/3]}
\begin{block}{\large glibc <= 2.25 chyba v dynamickém linkeru}
\small
\begin{alltt}
\$ strace -e mprotect -efault=all:error=EPERM:when=1 pwd
\textcolor{red}{mprotect(0x7fabcd00f000, 2097152, PROT_NONE)
 = -1 EPERM (Operation not permitted) (INJECTED)}
mprotect(0x7fabcd20f000, 16384, PROT_READ) = 0
mprotect(0x606000, 4096, PROT_READ)     = 0
mprotect(0x7fabcd441000, 4096, PROT_READ) = 0
/
+++ exited with 0 +++
\end{alltt}
\end{block}

\begin{block}{\large glibc >= 2.26 za správnou kontrolou}
\small
\begin{alltt}
\$ strace -e mprotect -efault=all:error=EPERM:when=1 pwd
\textcolor{red}{mprotect(0x7fabcd00f000, 2097152, PROT_NONE)
 = -1 EPERM (Operation not permitted) (INJECTED)}
pwd: error while loading shared libraries: libc.so.6:
 cannot change memory protections
+++ exited with 127 +++
\end{alltt}
\end{block}
\end{frame}

%%%%%%%
\begin{frame}[fragile]{Manipulovaní systémovými voláními: injekce návratnou hodnoty}
\begin{block}{\large -e \textbf{inject}=\textit{set}:\textbf{retval}=\textit{value}[:\textbf{when}=\textit{expr}][:\textbf{syscall}=\textit{syscall}]}
\textbf{inject}=\textit{set} -- injekce pro specifikovanou sadu systémových volání \\
\textbf{retval}=\textit{value} -- návratná hodnota pro navracení \\
\textbf{when}=\textit{expr} -- kdy dělat injekci, v podobě \textit{first[..last][+[step]]} \\
\textbf{syscall}=\textit{syscall} -- injekce specifikovaného systémového volání (ze skupiny \textbf{\%pure}) namísto -1
\end{block}

\begin{block}{\large příklad: uchovaní dočasných souborů}
\begin{alltt}
\$ cat script.sh
t=`mktemp`; trap 'rm -f "$t"' 0; echo secret $$ > $t
\$ strace -qq -f -e signal=none -e /unlink
  -e inject=all:retval=0 sh script.sh
[pid 347] \textcolor{red}{unlinkat(AT_FDCWD, "/tmp/tmp.l1AlwyCYH3", 0) = 0 (INJECTED)}
\$ cat /tmp/tmp.l1AlwyCYH3
secret 345
\end{alltt}
\end{block}

\end{frame}

%%%%%%%
\begin{frame}[fragile]{Manipulovaní systémovými voláními: injekce zdržení}
\begin{block}{\large injekce zdržení systémového volání}
strace -e \textbf{inject}=\textit{set}:\textbf{delay\_enter}=\textit{usecs} \\
strace -e \textbf{inject}=\textit{set}:\textbf{delay\_exit}=\textit{usecs}
\end{block}

\begin{block}{\large dd if=/dev/zero of=/dev/null bs=1M count=10}
\begin{alltt}
10+0 records in
10+0 records out
10485760 bytes (10 MB, 10 MiB) copied, \textcolor{red}{0.00211354} s, \textcolor{red}{5.0 GB}/s
\end{alltt}
\end{block}

\begin{block}{\large strace -einject=write:delay\_exit=100000 -ewrite -o/dev/null \textbackslash\\
dd if=/dev/zero of=/dev/null bs=1M count=10}
\begin{alltt}
10+0 records in
10+0 records out
10485760 bytes (10 MB, 10 MiB) copied, \textcolor{red}{1.10658} s, \textcolor{red}{9.5 MB}/s
\end{alltt}
\end{block}
\end{frame}

%%%%%%%
\begin{frame}[fragile]{Manipulovaní systémovými voláními: pozměnění pamětí \hfill [1/2]}
\begin{block}{\large -e \textbf{inject}=\textit{set}:\textbf{poke\_enter}=\textit{@argN=DATAN,@argM=DATAM...}[:\textbf{when}=\textit{expr}]}
\textbf{inject}=\textit{set} -- injekce pro specifikovanou sadu systémových volání \\
\textbf{poke\_enter}=\textit{@argN=DATAN,@argM=DATAM,...} -- pozměnění pamětí po ukazateli \texttt{arg\textit{N}} na \texttt{DATAN}, po ukazateli \texttt{arg\textit{M}} na \texttt{DATAM}, a t.d. \\
\textbf{when}=\textit{expr} -- kdy dělat injekci, v podobě \textit{first[..last][+[step]]} \\
\end{block}

\begin{block}{strace -P /etc/shadow cat /etc/shadow}
\begin{alltt}
openat(AT_FDCWD, "/etc/shadow", O_RDONLY) = -1 EACCES (Permission denied)
cat: /etc/shadow: Permission denied
+++ exited with 1 +++
\end{alltt}
\end{block}

\begin{block}{hex=2F6465762F6E756C6C00 \\ strace -P /etc/shadow --inject=openat:poke\_enter=@arg2=\$hex cat /etc/shadow}
\begin{alltt}
openat(AT_FDCWD, "/etc/shadow", O_RDONLY) = 3 (INJECTED: args)
+++ exited with 0 +++
\end{alltt}
\end{block}
\end{frame}

%%%%%%%
\begin{frame}[fragile]{Manipulovaní systémovými voláními: pozměnění pamětí \hfill [2/2]}
\begin{block}{\large -e \textbf{inject}=\textit{set}:\textbf{poke\_exit}=\textit{@argN=DATAN,@argM=DATAM...}[:\textbf{when}=\textit{expr}]}
\textbf{inject}=\textit{set} -- injekce pro specifikovanou sadu systémových volání \\
\textbf{poke\_exit}=\textit{@argN=DATAN,@argM=DATAM,...} -- pozměnění pamětí po ukazateli \texttt{arg\textit{N}} na \texttt{DATAN}, po ukazateli \texttt{arg\textit{M}} na \texttt{DATAM}, a t.d. \\
\textbf{when}=\textit{expr} -- kdy dělat injekci, v podobě \textit{first[..last][+[step]]} \\
\end{block}


\begin{block}{strace -s256 -e readlink readlink /etc/localtime}
\small
\begin{alltt}
readlink("/etc/localtime", "../usr/share/zoneinfo/Asia/Jerusalem", 64) = 36
../usr/share/zoneinfo/Asia/Jerusalem
+++ exited with 0 +++
\end{alltt}
\end{block}

\begin{block}{hex=2E2E2F7573722F73686172652F7A6F6E65696E666F2F4574632F555443 \\ strace -e readlink --inject=readlink:retval=29:poke\_exit=@arg2=\$hex \textbackslash \\ readlink /etc/localtime}
\scriptsize
\begin{alltt}
readlink("/etc/localtime", "../usr/share/zoneinfo/Etc/UTC", 64) = 29 (INJECTED: args, retval)
../usr/share/zoneinfo/Etc/UTC
+++ exited with 0 +++
\end{alltt}
\end{block}
\end{frame}

%%%%%%%
\begin{frame}[fragile]{Systémový čas, strávený v systémových voláních: \textbf{-c}}
\begin{block}{\large strace \textbf{-c} sleep 1}
\small
\begin{alltt}
% time  seconds usecs/call calls errors syscall
------ -------- ---------- ----- ------ ----------
 31.45 0.000078         19     4        close
 25.40 0.000063         15     4        mprotect
\textcolor{red}{ 12.90 0.000032         32     1        nanosleep}
 11.29 0.000028          5     5        mmap
  8.47 0.000021          5     4        brk
  8.06 0.000020         20     1        munmap
  2.42 0.000006          6     1        arch_prctl
  0.00 0.000000          0     1        read
  0.00 0.000000          0     2        fstat
  0.00 0.000000          0     1      1 access
  0.00 0.000000          0     1        execve
  0.00 0.000000          0     2        openat
------ -------- ---------- ----- ------ ----------
100.00 0.000248               27      1 total
\end{alltt}
\end{block}
\end{frame}

%%%%%%%
\begin{frame}[fragile]{Reální čas, strávený v systémových voláních: \textbf{-c -w}}
\begin{block}{\large strace \textbf{-c -w} sleep 1}
\small
\begin{alltt}
% time  seconds usecs/call calls errors syscall
------ -------- ---------- ----- ------ ----------
\textcolor{red}{ 99.91 1.000184    1000183     1        nanosleep}
  0.04 0.000367        366     1        execve
  0.02 0.000216         53     4        close
  0.01 0.000087         17     5        mmap
  0.01 0.000075         18     4        mprotect
  0.01 0.000052         13     4        brk
  0.00 0.000037         18     2        openat
  0.00 0.000027         26     1        munmap
  0.00 0.000024         12     2        fstat
  0.00 0.000019         19     1      1 access
  0.00 0.000015         14     1        read
  0.00 0.000013         13     1        arch_prctl
------ -------- ---------- ----- ------ ----------
100.00 1.001116               27      1 total
\end{alltt}
\end{block}
\end{frame}

%%%%%%%
\begin{frame}[fragile]{Nastavení souhrnu volání: \textbf{-{}-summary-columns=\textit{set}}, \textbf{-U \textit{set}} \hfill [1/2]}
\begin{block}{strace -c cat < /dev/null}
\small
\begin{alltt}
\% time     seconds  usecs/call     calls    errors syscall
------ ----------- ----------- --------- --------- ----------------
 31.54    0.000205           6        30        13 openat
 23.23    0.000151           6        22           mmap
 14.00    0.000091           4        19           newfstatat
 12.92    0.000084           4        20           close
  4.46    0.000029           9         3           mprotect
  4.15    0.000027          13         2           munmap
  2.46    0.000016           4         4           read
  1.54    0.000010           3         3           brk
  0.92    0.000006           6         1           getrandom
\ldots
------ ----------- ----------- --------- --------- ----------------
100.00    0.000650           5       116        15 total
\end{alltt}
\end{block}
\end{frame}

%%%%%%%
\begin{frame}[fragile]{Nastavení souhrnu volání: \textbf{-{}-summary-columns=\textit{set}}, \textbf{-U \textit{set}} \hfill [2/2]}
\begin{block}{strace -c -U min-time,max-time,avg-time,calls cat < /dev/null}
\small
\begin{alltt}
shortest  longest  usecs/call     calls syscall
-------- -------- ----------- --------- ----------------
0.000000 0.000016           5        30 openat
0.000000 0.000010           3        22 mmap
0.000000 0.000019           4        19 newfstatat
0.000000 0.000005           2        20 close
0.000009 0.000018          12         3 mprotect
0.000005 0.000015          10         2 munmap
0.000000 0.000006           3         4 read
0.000000 0.000006           3         3 brk
0.000005 0.000005           5         1 getrandom
\ldots
-------- -------- ----------- --------- ----------------
0.000000 0.000019           4       116 total
\end{alltt}
\end{block}
\end{frame}

%%%%%%%
\begin{frame}[fragile]{Výpis rad a vychytávek: \textbf{-{}-tips} \hfill [1/2]}
\large
\begin{block}{strace \textbf{-{}-tips}[=[[\textbf{id}:]\textit{id}],[[\textbf{format}:]\textit{format}]]}
\begin{itemize}
\item \textit{id} může být jedním s následujícího:
\begin{description}
	\item[\textbf{random}] vypisuje náhodnou radu (implicitní hodnota)
	\item[\textit{číslo}] vypisuje určenou radu podle specifikovaného čísla
\end{description}
\item \textit{format} může být jedním s následujícího:
\begin{description}
	\item[\textbf{compact}] vypisuje radu jen dost velkou pro obdržení všeho textu
	\item[\textbf{full}] vypisuje radu ve své plné kráse
	\item[\textbf{none}] nevypisuje žádnou radu
\end{description}
\end{itemize}
\end{block}
\end{frame}

%%%%%%%
\begin{frame}[fragile]{Výpis rad a vychytávek: \textbf{-{}-tips} \hfill [2/2]}
\large
\begin{block}{strace -{}-tips=id:31}
\begin{alltt}
 ______________________________________________         ____
/                                              \textbackslash       /    \textbackslash
| Medicinal effects of strace can be achieved  |      |-. .-.|
| by invoking it with the following options:   \textbackslash      (_@)(_@)
|                                               \textbackslash     .-{}-{}-_  \textbackslash
|     strace -DDDqqq -enone -{}-signal=none       _\textbackslash   /..   \textbackslash_/
|                                              /     |__.-^ /
|                                              |         \}  |
\textbackslash______________________________________________/        |   [
                                                        [  ]
\end{alltt}
\end{block}
\pause
\begin{block}{\url{https://github.com/strace/strace/issues/14}}
Document how to just get the 'medicinal' effects of strace with no overhead.
\end{block}
\end{frame}

%%%%%%%
{
\setbeamertemplate{footline}{}
\begin{frame}{Otázky?}
	\begin{columns}
		\column{7cm}
\begin{block}{\large webové stránky}
	https://strace.io/
\end{block}
\begin{block}{\large strace.git}
	https://github.com/strace/strace.git

	https://gitlab.com/strace/strace.git
\end{block}
\begin{block}{\large mailing list}
	strace-devel@lists.strace.io
\end{block}
\begin{block}{\large IRC kanál}
	\#strace@OFTC
\end{block}
		\column{3cm}
			\centerline{\includegraphics[height=7cm]{strace-straus.pdf}}
	\end{columns}
\end{frame}

%%%%%%%
\begin{frame}[noframenumbering]{Why is \texttt{strace} slow? \hfill [1/2]}
\begin{block}{\large \texttt{ptrace(2)}}
\begin{itemize}
  \item \texttt{strace} utilizes \texttt{ptrace} infrastructure for tracing
  \item \texttt{ptrace(2)} is a generic debugging interface that provides
        a set of commands (\emph{requests}) that enable various operations:
	reading and writing tracee's memory, obtaining tracee's registers,
	and so on
  \item Almost all \texttt{ptrace} operations are performed on a stopped process
  \item The \texttt{ptrace} API (ab)uses the standard UNIX parent/child
        signaling over \texttt{waitpid(2)} in order to deliver notifications
	about changes in tracees' state (including \texttt{ptrace}-induced
	stops)
  \item This mechanism is used to notify tracer about all kinds of events:
        syscall stops (after resume with \texttt{PTRACE\_SYSCALL} request),
	signal deliveries, group stops, forks, execs, etc.
\end{itemize}
\end{block}
\end{frame}

%%%%%%%
\begin{frame}[noframenumbering]{Why is \texttt{strace} slow? \hfill [2/2]}
\begin{block}{\large How \texttt{strace} traces processes}
\begin{itemize}
  \item \texttt{strace} is waiting for events in \texttt{wait4(2)}
  \item Upon receiving a \texttt{ptrace} event, \texttt{strace} tries to figure
        out what happened (syscall stop, signal received by tracee, etc.)
  \begin{itemize}
    \item The only information it has at this point is the status returned
          by the \texttt{wait4(2)} syscall; as a result, additional
	  \texttt{PTRACE\_GETEVENTMSG} request is required to distinguish
	  some of the events
  \end{itemize}
  \item For syscall stops, the additional information (syscall number and
        arguments on entering, return code on exiting) is retrieved
  \item Syscall-specific decoder function is called, which, in turn, may perform
        additional reads from tracee's memory for elaborate argument printing
	(structures, arrays, linked lists...)
  \item When the decoding is finished, tracee is resumed
  \item For each syscall, syscall stop is happened twice: on syscall entering
        and exiting
\end{itemize}
\end{block}
\end{frame}

%%%%%%%
\begin{frame}[noframenumbering]{seccomp-assisted system call filtering}
\begin{block}{\large seccomp-bpf usage}
\begin{itemize}
  \item \texttt{seccomp} (for Secure Computing) is a Linux mechanism
        that provides an ability (in its \texttt{SECCOMP\_SET\_MODE\_FILTER}
        mode) to attach a BPF program to a process
  \item Since Linux 3.5, a \texttt{seccomp} program has an ability
        to return \texttt{SECCOMP\_RET\_TRACE} as a result of its execution,
        which, in turn, notifies \texttt{ptrace}-based tracer with
        \texttt{PTRACE\_EVENT\_SECCOMP}
  \item When \texttt{-{}-seccomp-bpf} command-line option is passed
        to \texttt{strace}, BPF bytecode is generated and attached to tracees,
        if possible
  \item The feature has been implemented as part of GSoC 2018 and 2019 projects
        by Chen Jingpiao and Paul Chaignon, and included
        in the 5.3 release of \texttt{strace}
\end{itemize}
\end{block}
\end{frame}
}

\end{document}
