% Copyright (C) 2012-2024 Dmitry V. Levin <ldv@strace.io>
% Permission is granted to copy, distribute and/or modify this document
% under the terms of the GNU Free Documentation License, Version 1.2
% or any later version published by the Free Software Foundation;
% with no Invariant Sections, no Front-Cover Texts, and no Back-Cover Texts.

\documentclass[unicode,aspectratio=169,xcolor={table,dvipsnames,usernames}]{beamer}

\usepackage[utf8]{inputenc}
\usepackage[T2A]{fontenc}
\usepackage[english]{babel}
\usepackage{alltt}

\colorlet{darkred}{BrickRed}
\colorlet{darkgreen}{OliveGreen}
\newcommand{\symlinebreak}{\textcolor{blue}{\(\hookleftarrow\)}}
\newcommand{\symlinecont}{\textcolor{blue}{\(\longrightarrow\)}}

\pgfdeclareimage[height=0.6cm]{left-corner-logo}{devconf-cz-icon.pdf}
\pgfdeclareimage[height=1.2cm]{right-corner-logo}{strace-straus.pdf}
\pgfdeclareimage[height=2cm]{title-logo}{devconf-cz.pdf}
\pgfdeclareimage[height=8cm]{strace-logo}{strace-straus.pdf}
\pgfdeclareimage[height=5.8cm]{wat}{wat.pdf}

\usetheme{Warsaw}
\setbeamertemplate{headline}{}
\setbeamertemplate{footline}{\pgfuseimage{left-corner-logo} \hfill \pgfuseimage{right-corner-logo}}
\setbeamertemplate{navigation symbols}{}
\setbeamertemplate{logo}{}
\setbeamercovered{transparent}

\title{\Huge Evolution of ptrace \\ from strace's perspective}
\author{\LARGE Dmitry~Levin}
\date{\Large Brno, 2024}
\titlegraphic{\pgfuseimage{title-logo}}

\begin{document}

%%%%%%%
\begin{frame}[noframenumbering]
\titlepage
\end{frame}

%%%%%%%
{
\setbeamertemplate{footline}{\pgfuseimage{left-corner-logo}}
\begin{frame}{\Large strace's perspective\hfill [\insertframenumber/\inserttotalframenumber]}
	\begin{columns}
		\column{10cm}
\begin{block}{What is strace?}
\begin{itemize}
	\item diagnostic, debugging and instructional utility
	\item monitors and tampers with syscalls and signals
	\item unprivileged, userspace
	\item uses ptrace API
	\item the only general purpose tracer of its kind
\end{itemize}
\end{block}

\begin{block}{The best syscall decoder in town}
\begin{itemize}
	\item can access tracee memory, filesystem, etc.
	\item can perform complex manipulations with the data
	\item can obtain a lot of associated information
	\item can tamper with syscall numbers and their return values
	\item can inject data both on entering and on exiting syscalls
\end{itemize}
\end{block}
		\column{2cm}
			\centerline{\pgfuseimage{strace-logo}}
	\end{columns}
\end{frame}
}

%%%%%%%
\begin{frame}{\Large strace's perspective\hfill [\insertframenumber/\inserttotalframenumber]}
\Large
\begin{block}{Scope of this talk}
\begin{itemize}
	\item architecture of a ptrace based tracer
	\item evolution of the ptrace API
	\item what is still broken in the kernel
	\item what is still missing in the kernel
	\item what could be done to address this
\end{itemize}
\end{block}

\begin{block}{You'll see why}
\begin{itemize}
	\item writing a basic system call tracer is relatively easy
	\item a decent syscall tracer is necessarily complicated
\end{itemize}
\end{block}
\end{frame}

%%%%%%%
\begin{frame}{\Large Architecture of a ptrace based tracer: overview\hfill [\insertframenumber/\inserttotalframenumber]}
\Large
\begin{block}{What is the ptrace API for?}
A means by which one process may
\begin{itemize}
	\item observe and control the execution of another process
	\item examine and change the tracee's registers and memory
\end{itemize}
\end{block}

\begin{block}{tracer architecture summary}
\begin{itemize}
	\item a tracee needs to be attached to the tracer
	\item the tracer enters event loop
	\item when the tracer is finished tracing, it can cause the tracee
		to continue executing in a regular, non-traced mode
\end{itemize}
\end{block}

So far everything looks straightforward, doesn't it?
\end{frame}

%%%%%%%
\begin{frame}{\Large Architecture of a ptrace based tracer: event loop\hfill [\insertframenumber/\inserttotalframenumber]}
\Large
\begin{block}{Tracer event loop}
\begin{itemize}
	\item the tracee stops each time a traceable event happens, \\
		these events are called ptrace stops
	\item the tracer is notified using a "wait" family syscall, \\
		e.g. \textbf{waitpid(-1, \&status, \_\_WALL)}
	\item that call returns a status value containing information \\
		that indicates the cause of the stop
	\item while the tracee is stopped, the tracer can use \\
		various ptrace commands to inspect and modify the tracee
	\item the tracer then causes the tracee to continue
\end{itemize}
\end{block}

There is nothing scary so far, isn't it?
\end{frame}

%%%%%%%
\begin{frame}[fragile]{\Large Attaching tracees\hfill [\insertframenumber/\inserttotalframenumber]}
\large
\begin{block}{Traditional approach}
\begin{itemize}
	\item attaching a process that already exists: \\
		\textbf{ptrace(PTRACE\_ATTACH, pid, 0, 0)}
	\item successful \textbf{PTRACE\_ATTACH} sends \textbf{SIGSTOP} to the tracee
	\item attaching a process that has to be created first:
	\begin{alltt}\bf
		pid = fork();
		if (pid == 0) \{
		    ptrace(PTRACE_TRACEME, 0, 0, 0);
		    raise(SIGSTOP);
		    \ldots
		\}
	\end{alltt}
	\item set options for the new tracee during the first stop: \\
		\textbf{ptrace(PTRACE\_SETOPTIONS, pid, 0, \textit{PTRACE\_O\_flags})}
	\item the first \textbf{SIGSTOP} is suppressed by the tracer
\end{itemize}
\end{block}
\end{frame}

%%%%%%%
\begin{frame}[fragile]{\Large Attaching tracees\hfill [\insertframenumber/\inserttotalframenumber]}
\large
\begin{block}{Modern approach (since Linux 3.4 released $\approx$ 12 years ago)}
\begin{itemize}
	\item attaching a process that already exists: \\
		\textbf{ptrace(PTRACE\_SEIZE, pid, 0, PTRACE\_O\_flags)}
	\item PTRACE\_SEIZE does not stop the new tracee, there is a separate \\
		\textbf{ptrace(PTRACE\_INTERRUPT, pid, 0, 0)} \\
		command to stop it without interfering with signals
	\item attaching a process that has to be created first:
	\begin{alltt}\bf
		pid = fork();
		if (pid == 0) \{
		    raise(SIGSTOP);
		    \ldots
		\}
		waitpid(pid, &status, WSTOPPED);
		ptrace(PTRACE_SEIZE, pid, 0, PTRACE_O_flags);
	\end{alltt}
\end{itemize}
\end{block}
\end{frame}

%%%%%%%
\begin{frame}{\Large Types of ptrace stops\hfill [\insertframenumber/\inserttotalframenumber]}
\begin{itemize}
	\item any stopped state in which the tracee is ready to accept ptrace commands
	\item every time the tracee enters a ptrace stop state, the tracer is notified
\end{itemize}
\Large
\begin{block}{Signal delivery stops}
	Just prior to signal delivery
\end{block}
\begin{block}{Syscall stops}
	Just prior to entering and exiting a system call
\end{block}
\begin{block}{Group stops}
	When a tracee receives a stopping signal
\end{block}
\begin{block}{PTRACE\_EVENT\_* stops}
	Depending on PTRACE\_O\_* options set
\end{block}
\end{frame}

%%%%%%%
\begin{frame}{\Large Signal delivery stops\hfill [\insertframenumber/\inserttotalframenumber]}
\Large
\begin{block}{Conditions}
\begin{itemize}
	\item each time a signal is about to be delivered
	\item even if the signal is being ignored
	\item except \textbf{SIGKILL} which causes no signal delivery stop
	\item the signal is not yet delivered to the tracee
	\item reported by "wait" family syscalls with \\
		$WIFSTOPPED(status)\ \&\&\ WSTOPSIG(status) == sig$
\end{itemize}
\end{block}
\end{frame}

%%%%%%%
\begin{frame}{\Large Signal delivery stops\hfill [\insertframenumber/\inserttotalframenumber]}
\Large
\begin{block}{Actions}
\begin{itemize}
	\item the corresponding siginfo structure can be retrieved using \\
		\textbf{ptrace(PTRACE\_GETSIGINFO, pid, 0, \&siginfo)} \\
		and modified using \\
		\textbf{ptrace(PTRACE\_SETSIGINFO, pid, 0, \&siginfo)}
	\item the tracer restarts the tracee with \\
		\textbf{ptrace(\textit{PTRACE\_restart}, pid, 0, sig)} \\
	where \textit{PTRACE\_restart} is one of ptrace restart operations
	\item if $sig == 0$, then the signal is not delivered
	\item otherwise, signal $sig$ is injected
	\item injected signal $sig$ can differ from $WSTOPSIG(status)$
\end{itemize}
\end{block}
\end{frame}

%%%%%%%
\begin{frame}{\Large Syscall stops\hfill [\insertframenumber/\inserttotalframenumber]}
\Large
\begin{block}{Conditions}
\begin{itemize}
	\item when the tracee is restarted from a ptrace stop using \\
		\textbf{ptrace(PTRACE\_SYSCALL, pid, 0, sig)}
	\item it will stop just prior to entering the next system call
	\item and just prior to exiting the current system call
	\item reported by "wait" family syscalls with either \\
		$WSTOPSIG(status) == SIGTRAP$, or, \\
		if \textbf{PTRACE\_O\_TRACESYSGOOD} option was set, \\
		$WSTOPSIG(status) == (SIGTRAP | 0x80)$
	\item makes it easy to distinguish signal and syscall stops
	\item PTRACE\_O\_TRACESYSGOOD is available since Linux 2.4.6
\end{itemize}
\end{block}
\end{frame}

%%%%%%%
\begin{frame}{\Large Syscall stops\hfill [\insertframenumber/\inserttotalframenumber]}
\Large
\begin{block}{Syscall and signal stops}
\begin{itemize}
	\item signal stop never happens between syscall enter and exit stops
	\item when syscall is interrupted by signal, the signal delivery stop happens after the syscall exit stop
\end{itemize}
\end{block}

\begin{block}{What can happen with a tracee between syscall enter and exit stops?}
\begin{itemize}
	\item stop in a \textbf{PTRACE\_EVENT\_*} stop
	\item exit, if it entered \_exit or exit\_group syscall
	\item be killed by \textbf{SIGKILL}
	\item die silently, if it is a thread group leader, the execve happened
		in another thread, and that thread is not traced by the same tracer
\end{itemize}
\end{block}
\end{frame}

%%%%%%%
\begin{frame}[fragile]{\Large Syscall enter and exit stops\hfill [\insertframenumber/\inserttotalframenumber]}
\Large
How to distinguish syscall enter and exit stops?
\begin{block}{Traditional approach}
	Keep track of the sequence of ptrace stops
\end{block}

\begin{block}{Modern approach (since Linux 5.3 released $\approx$ 5 years ago)}
	Use \textbf{PTRACE\_GET\_SYSCALL\_INFO}:
	\begin{alltt}
		struct ptrace_syscall_info psi;
		ptrace(PTRACE_GET_SYSCALL_INFO, pid, sizeof(psi), &psi)
	\end{alltt}
\end{block}
\end{frame}

%%%%%%%
\begin{frame}[fragile]{\Large Syscall enter and exit stops\hfill [\insertframenumber/\inserttotalframenumber]}
\Large
\begin{block}{PTRACE\_GET\_SYSCALL\_INFO API}
\begin{itemize}
	\item \begin{alltt}
	struct ptrace_syscall_info \{
	  __u8 op;	\hfill /* Type of system call stop */
	  \ldots
	\}
	\end{alltt}
	\item $op == PTRACE\_SYSCALL\_INFO\_ENTRY$ means \\
		it is a syscall enter stop
	\item $op == PTRACE\_SYSCALL\_INFO\_EXIT$ means \\
		it is a syscall exit stop
\end{itemize}
\end{block}
\end{frame}

%%%%%%%
\begin{frame}{\Large Syscall stops: obtaining syscall information\hfill [\insertframenumber/\inserttotalframenumber]}
How the tracer can inspect syscall number, arguments, and exit status?

\begin{block}{The ancient method}
\begin{itemize}
	\item to inspect registers: \\
		\textbf{ptrace(PTRACE\_PEEKUSER, pid, offset, 0)}
	\item painfully slow: one word at a time
\end{itemize}
\end{block}

\begin{block}{The very old method}
\begin{itemize}
	\item to fetch all registers in one go \\
		\textbf{ptrace(PTRACE\_GETREGS, pid, 0, \&user\_regs\_struct)}
	\item not available on modern architectures
\end{itemize}
\end{block}

\begin{block}{The old method (since Linux 2.6.34 released $\approx$ 14 years ago)}
\begin{itemize}
	\item to fetch all registers in one go \\
		\textbf{ptrace(PTRACE\_GETREGSET, pid, NT\_PRSTATUS, \&iovec)}
\end{itemize}
\end{block}
\end{frame}

%%%%%%%
\begin{frame}{\Large Syscall stops: obtaining syscall information\hfill [\insertframenumber/\inserttotalframenumber]}
\Large
\begin{block}{Problems with the register fetching approach}
\begin{itemize}
	\item very architecture-dependent
	\item unreliable at least on one popular 64-bit architecture
\end{itemize}
\end{block}

\begin{block}{What's wrong there?}
\begin{itemize}
	\item process bitness does not have to match syscall bitness
	\item 64-bit processes can invoke both 64-bit and 32-bit system calls
	\item no reliable way to tell them apart by inspecting registers
\end{itemize}
\end{block}
\end{frame}

%%%%%%%
\begin{frame}[fragile]{\Large Syscall stops: obtaining syscall information\hfill [\insertframenumber/\inserttotalframenumber]}
\large
Process bitness does not match syscall bitness
\begin{block}{Example based on Debian bug report \#459820 submitted in 2008}
\begin{alltt}
#include <stdio.h>
#include <unistd.h>
int main() \{
    setlinebuf(stdout);
    puts("------------");
    \textcolor{darkred}{__asm__("movl $2, %eax; int $0x80");}
    printf("[I am %d]{\textbackslash}n", getpid());
    return 0;
\}
\end{alltt}
\end{block}

\begin{block}{Regular invocation: ./debbug459820}
\small
\begin{alltt}
------------
[I am 23450]
[I am 23451]
\end{alltt}
\end{block}
\end{frame}

%%%%%%%
\begin{frame}[fragile]{\Large Syscall stops: obtaining syscall information\hfill [\insertframenumber/\inserttotalframenumber]}
Process bitness does not match syscall bitness
\begin{columns}
	\column{10cm}
		\Large
		\begin{block}{Invocation under strace, Linux < 5.3}
\begin{alltt}
$ strace -f ./debbug459820 > /dev/null
\ldots
write(1, "------------{\textbackslash}n", 13) = 13
strace: Process 23451 attached
\textcolor{darkred}{open(0x1, O_RDONLY|O_CREAT|O_TRUNC
|O_DSYNC|O_DIRECT|O_NOATIME|O_CLOEXEC
|O_PATH|O_TMPFILE|0x1000020, 0134300)
= 23451}
\ldots
\end{alltt}
		\end{block}
	\column{3cm}
		\begin{figure}
			\centering
			\pgfuseimage{wat}
		\end{figure}
\end{columns}
\end{frame}

%%%%%%%
\begin{frame}[fragile]{\Large Syscall stops: obtaining syscall information\hfill [\insertframenumber/\inserttotalframenumber]}
\begin{block}{PTRACE\_GET\_SYSCALL\_INFO (since Linux 5.3 released $\approx$ 5 years ago)}
\scriptsize
\begin{alltt}
struct ptrace_syscall_info \{
  __u8 op;			\hfill /* Type of system call stop */
  __aligned_u32 arch;	\hfill /* AUDIT_ARCH_* value; see seccomp(2) */
  __u64 instruction_pointer;	\hfill /* CPU instruction pointer */
  __u64 stack_pointer;		\hfill /* CPU stack pointer */
  union \{
    struct \{			\hfill /* op == PTRACE_SYSCALL_INFO_ENTRY */
      __u64 nr;			\hfill /* Syscall number */
      __u64 args[6];		\hfill /* Syscall arguments */
    \} entry;
    struct \{			\hfill /* op == PTRACE_SYSCALL_INFO_EXIT */
      __s64 rval;		\hfill /* Syscall return value */
      __u8 is_error;		\hfill /* Does rval contain an error value? */
    \} exit;
    struct \{			\hfill /* op == PTRACE_SYSCALL_INFO_SECCOMP */
      __u64 nr;			\hfill /* Syscall number */
      __u64 args[6];		\hfill /* Syscall arguments */
      __u32 ret_data;		\hfill /* SECCOMP_RET_DATA portion of SECCOMP_RET_TRACE */
    \} seccomp;
  \};
\};
\end{alltt}
\end{block}
\end{frame}

%%%%%%%
\begin{frame}{\Large Syscall stops: accessing tracee memory\hfill [\insertframenumber/\inserttotalframenumber]}
\large
Inspecting and modifying the tracee memory
\begin{block}{The old method}
\begin{itemize}
	\item to inspect memory: \\
		\textbf{ptrace(PTRACE\_PEEKDATA, pid, addr, 0)}
	\item to modify memory: \\
		\textbf{ptrace(PTRACE\_POKEDATA, pid, addr, word)}
	\item painfully slow: one word at a time
\end{itemize}
\end{block}

\begin{block}{The modern method (since Linux 3.2 released $\approx$ 12 years ago)}
\begin{itemize}
	\item to inspect memory: \textbf{process\_vm\_readv}()
	\item to modify memory: \textbf{process\_vm\_writev}()
	\item allows tracers to read/write pages from/to tracee's memory
\end{itemize}
\end{block}
\end{frame}

%%%%%%%
\begin{frame}{\Large Group stop\hfill [\insertframenumber/\inserttotalframenumber]}
\Large
\begin{block}{Conditions}
\begin{itemize}
	\item a stopping signal causes a signal delivery stop
	\item after it is injected by the tracer
	\item the tracee enters a group stop state
	\item reported by "wait" family syscalls with \\
		$WSTOPSIG(status) == stopsig$
	\item \textbf{ptrace(PTRACE\_GETSIGINFO, pid, 0, \&siginfo)} \\
		can be used to tell it from a signal stop
	\item when the tracee is attached using \textbf{PTRACE\_SEIZE}, then \\
		$(status >> 16) == PTRACE\_EVENT\_STOP$, \\
		and \textbf{PTRACE\_GETSIGINFO} is not needed
\end{itemize}
\end{block}
\end{frame}
%%%%%%%

\begin{frame}{\Large Group stop\hfill [\insertframenumber/\inserttotalframenumber]}
\Large
\begin{block}{Problem of transparent handling of stopping signals}
\begin{itemize}
	\item the tracee will not run until restarted by the tracer
	\item the tracee will not send notifications, besides \textbf{SIGKILL} death
	\item if restarted, the stopping signal is effectively ignored
	\item if not restarted, future \textbf{SIGCONT} signals will not be reported \\
		and would not have effect on the tracee
\end{itemize}
\end{block}
\end{frame}

%%%%%%%
\begin{frame}{\Large Group stop\hfill [\insertframenumber/\inserttotalframenumber]}
\Large
\begin{block}{Solution (since Linux 3.4 released $\approx$ 12 years ago)}
\begin{itemize}
	\item instead of \\
		\textbf{ptrace(PTRACE\_SYSCALL, pid, 0, 0)}
	\item restart the tracee in a way where it does not execute, \\
		but waits for new events: \\
		\textbf{ptrace(PTRACE\_LISTEN, pid, 0, 0)}
	\item this method is available only if the tracee was attached \\
		using \textbf{PTRACE\_SEIZE}
\end{itemize}
\end{block}
\end{frame}

%%%%%%%
\begin{frame}[fragile]{\Large Seccomp-assisted tracing and syscall filtering\hfill [\insertframenumber/\inserttotalframenumber]}
\begin{block}{\textbf{strace -{}-seccomp-bpf}}
\small
\begin{itemize}
	\item automatically generates and attaches a BPF program to filter syscalls
	\item makes execution of untraced syscalls two orders of magnitude faster
\end{itemize}
\end{block}

\begin{block}{The infamous dd example}
\scriptsize
\begin{alltt}
\$ dd if=/dev/zero of=/dev/null bs=1 count=1M 2>\&1 | grep -v records
1048576 bytes (1.0 MB, 1.0 MiB) copied, \textcolor{darkred}{0.281794} s, 3.7 MB/s

\$ strace -f -qq --signal=none --trace=fchdir \textbackslash
  dd if=/dev/zero of=/dev/null bs=1 count=1M 2>\&1 | grep -v records
1048576 bytes (1.0 MB, 1.0 MiB) copied, \textcolor{darkred}{13.9573} s, 75.1 kB/s

\$ strace -f -qq --signal=none --trace=fchdir \textcolor{darkgreen}{--seccomp-bpf} \textbackslash
  dd if=/dev/zero of=/dev/null bs=1 count=1M 2>\&1 | grep -v records
1048576 bytes (1.0 MB, 1.0 MiB) copied, \textcolor{darkred}{0.307412} s, 3.4 MB/s
\end{alltt}
\end{block}

\begin{block}{Comparative slowdown}
$0.307412 / 0.281794 \approx \textcolor{darkred}{1.09}$ \hfill
$13.9573 / 0.307412 \approx \textcolor{darkred}{45.4}$ \hfill
$13.9573 / 0.281794 \approx \textcolor{darkred}{49.5}$
\end{block}
\end{frame}

%%%%%%%
\begin{frame}{\Large PTRACE\_EVENT\_SECCOMP\hfill [\insertframenumber/\inserttotalframenumber]}
\large
\begin{block}{Conditions}
\begin{itemize}
	\item \textbf{PTRACE\_O\_TRACESECCOMP} option is set
	\item \textbf{PTRACE\_EVENT\_SECCOMP} stop occurs whenever \\
		a SECCOMP\_RET\_TRACE rule of a seccomp filter is triggered
	\item independent of the method that was used to restart the tracee
	\item available since Linux 3.5 released $\approx$ 12 years ago, but
	\item starting with Linux 4.8 released $\approx$ 8 years ago, \\
	       this event occurs between syscall enter and exit stops
\end{itemize}
\end{block}

\begin{block}{Actions}
\begin{itemize}
	\item similar to syscall enter stops
	\item \textbf{PTRACE\_CONT} is used instead of \textbf{PTRACE\_SYSCALL} \\
		on exiting syscall
\end{itemize}
\end{block}
\end{frame}

%%%%%%%
\begin{frame}{\Large Following forks\hfill [\insertframenumber/\inserttotalframenumber]}
\large
\begin{block}{The stone age (before Linux 2.5.46)}
\begin{itemize}
	\item quite a bit of work to follow clones, forks, and vforks
	\item new process had to be created using clone() syscall \\
		with \textbf{CLONE\_PTRACE} flag set
	\item on entering clone(), fork(), and vfork() syscalls, \\
		the tracer had to change them to clone() with appropriate flags
	\item \textbf{strace} had to turn vfork() into fork() by stripping \textbf{CLONE\_VFORK} flag
\end{itemize}
\end{block}
\end{frame}

%%%%%%%
\begin{frame}{\Large Following forks\hfill [\insertframenumber/\inserttotalframenumber]}
\large
\begin{block}{The modern era (since Linux 2.5.46)}
\begin{itemize}
	\item introduced new options:
		\textbf{PTRACE\_O\_TRACECLONE}, \textbf{PTRACE\_O\_TRACEFORK},
		and \textbf{PTRACE\_O\_TRACEVFORK}
	\item newly created threads are traced automatically
	\item the tracee will enter the corresponding \textbf{PTRACE\_EVENT\_*} stops
	\item the tracer can find out the new pid in the tracer namespace using
		\textbf{ptrace(PTRACE\_GETEVENTMSG, pid, 0, \&msg)} \\
\end{itemize}
\end{block}

\begin{block}{Related PTRACE\_EVENT\_* stops}
\begin{itemize}
	\item \textbf{PTRACE\_EVENT\_CLONE} - stop before return from clone
	\item \textbf{PTRACE\_EVENT\_FORK} - stop before return from fork
	\item \textbf{PTRACE\_EVENT\_VFORK} - stop before return from vfork
\end{itemize}
\end{block}
\end{frame}

%%%%%%%
\begin{frame}{\Large Following execs\hfill [\insertframenumber/\inserttotalframenumber]}
\Large
\begin{block}{What happens when a thread in a thread group calls execve()}
\begin{itemize}
	\item all other threads in the process are destroyed
	\item the thread ID of the thread that invoked execve() \\
		is set to the thread group ID
	\item at completion of the system call, it appears as though \\
		the execve() occurred in the thread group leader, \\
		regardless of which thread did the execve()
	\item this resetting of the thread ID looks very confusing to tracers
\end{itemize}
\end{block}
\end{frame}

%%%%%%%
\begin{frame}{\Large Following execs\hfill [\insertframenumber/\inserttotalframenumber]}
\begin{block}{Resetting of the thread ID may look very confusing to tracers}
\begin{itemize}
	\item all other threads stop in \textbf{PTRACE\_EVENT\_EXIT} stop, \\
		if \textbf{PTRACE\_O\_TRACEEXIT} option was turned on
	\item all other threads except the thread group leader report death \\
		as if they exited via \_exit() with exit code 0.
	\item the tracee's thread ID is reset to the thread group leader's thread ID
	\item then a \textbf{PTRACE\_EVENT\_EXEC} stop happens, \\
		if the \textbf{PTRACE\_O\_TRACEEXEC} option was turned on
	\item if the thread group leader has reported its \textbf{PTRACE\_EVENT\_EXIT} by this time,
		it appears to the tracer that the dead thread leader reappears from nowhere
	\item if the thread group leader was still alive, it appears to the tracer that \\
		the thread group leader returns from a different system call than it entered, \\
		or even returned from a system call even though it was not in any system call
\end{itemize}
\end{block}

All of the above effects are the artifacts of the thread ID change in the tracee
\end{frame}

%%%%%%%
\begin{frame}{\Large Following execs\hfill [\insertframenumber/\inserttotalframenumber]}
\Large
\begin{block}{Solution (since Linux 2.5.46)}
\begin{itemize}
	\item set \textbf{PTRACE\_O\_TRACEEXEC} option
	\item it enables \textbf{PTRACE\_EVENT\_EXEC} stop, \\
		which occurs before execve() returns
	\item in this stop (since Linux 3.0 released $\approx$ 13 years ago), \\
		the tracer can use \\
		\textbf{ptrace(PTRACE\_GETEVENTMSG, pid, 0, \&msg)} \\
		to retrieve the tracee's former thread ID.
	\item when the tracer receives \textbf{PTRACE\_EVENT\_EXEC} stop notification,
		it is guaranteed that except this tracee and the thread group leader,
		no other threads from the process are alive
\end{itemize}
\end{block}
\end{frame}

%%%%%%%
\begin{frame}{\Large Detaching tracees\hfill [\insertframenumber/\inserttotalframenumber]}
\Large
\begin{block}{PTRACE\_DETACH}
\begin{itemize}
	\item detaching the tracee is performed using \\
		\textbf{ptrace(PTRACE\_DETACH, pid, 0, sig)}
	\item \textbf{PTRACE\_DETACH} is a restarting operation
	\item it requires the tracee to be in a ptrace stop
	\item if the tracee is in a signal delivery stop,
		a signal can be injected.
\end{itemize}
\end{block}

How do you detach the tracee which is not in a ptrace stop?
\end{frame}

%%%%%%%
\begin{frame}{\Large Detaching tracees\hfill [\insertframenumber/\inserttotalframenumber]}
\Large
How do you detach the tracee which is not in a ptrace stop?

\begin{block}{Traditional approach}
\begin{itemize}
	\item the tracer sends a \textbf{SIGSTOP} signal to the running tracee,
	\item then it waits for the tracee to stop in a signal delivery stop state \\
		for \textbf{SIGSTOP},
	\item and, finally, detaches the tracee, suppressing the \textbf{SIGSTOP}
\end{itemize}
\end{block}

\begin{block}{What's the problem with this approach?}
\begin{itemize}
	\item this can race with concurrent \textbf{SIGSTOP}s
	\item the tracee may enter other ptrace stops and needs
		to be restarted and waited for again, until \textbf{SIGSTOP} is seen
\end{itemize}
\end{block}
\end{frame}

%%%%%%%
\begin{frame}{\Large Detaching tracees\hfill [\insertframenumber/\inserttotalframenumber]}
\Large
\begin{block}{Modern approach (since Linux 3.4 released $\approx$ 12 years ago)}
\begin{itemize}
	\item the tracer forces the tracee to enter a ptrace stop using
		\textbf{ptrace(PTRACE\_INTERRUPT, pid, 0, 0)}
	\item depending on the circumstances, it's either a syscall exit stop, \\
		signal delivery stop, or a \textbf{PTRACE\_EVENT\_STOP}
	\item at this point it's safe to detach the tracee using \\
		\textbf{ptrace(PTRACE\_DETACH, pid, 0, sig)}
	\item this method is available only when the tracee was attached \\
		using \textbf{PTRACE\_SEIZE}
\end{itemize}
\end{block}
\end{frame}

%%%%%%%
\begin{frame}{\Large Seccomp-assisted syscall filtering limitations\hfill [\insertframenumber/\inserttotalframenumber]}
\Large
\begin{block}{Design of seccomp filters evaluation}
\begin{itemize}
	\item process can have several seccomp filters
	\item if multiple filters exist, they are all executed
	\item the most recently installed filter is executed first
	\item the result of the evaluation for a given system call \\
		is the \textit{SECCOMP\_RET\_*} action value of highest precedence
\end{itemize}
\end{block}
\end{frame}

%%%%%%%
\begin{frame}{\Large Seccomp-assisted syscall filtering limitations\hfill [\insertframenumber/\inserttotalframenumber]}
\Large
\begin{block}{List of seccomp action values in decreasing order of precedence}
\begin{itemize}
	\item SECCOMP\_RET\_KILL\_PROCESS
	\item SECCOMP\_RET\_KILL\_THREAD
	\item SECCOMP\_RET\_TRAP
	\item SECCOMP\_RET\_ERRNO
	\item SECCOMP\_RET\_USER\_NOTIF
	\item \textbf{SECCOMP\_RET\_TRACE}
	\item SECCOMP\_RET\_LOG
	\item SECCOMP\_RET\_ALLOW
\end{itemize}
\end{block}
\end{frame}

%%%%%%%
\begin{frame}{\Large SECCOMP\_RET\_TRACE Linux kernel misdesign\hfill [\insertframenumber/\inserttotalframenumber]}
\Large
\begin{block}{Problem}
\begin{itemize}
	\item if the tracee has another seccomp filter that
	\item returns an action value with a precedence greater than \textbf{SECCOMP\_RET\_TRACE},
	\item then the tracee will not enter \\
		\textbf{PTRACE\_EVENT\_SECCOMP} stop
	\item and the tracer will not be notified
	\item if another seccomp filter e.g. makes the syscall fail, or \\
		kills the tracee, \textbf{strace -{}-seccomp-bpf} will not be aware \\
		of that syscall invocation \textbf{\textit{at all}}
	\item this defeats the very purpose of syscall tracing
\end{itemize}
\end{block}
\end{frame}

%%%%%%%
\begin{frame}{\Large SECCOMP\_RET\_TRACE Linux kernel misdesign\hfill [\insertframenumber/\inserttotalframenumber]}
\Large
\begin{block}{Current SECCOMP\_RET\_TRACE implementation}
\begin{itemize}
	\item if \textbf{SECCOMP\_RET\_TRACE} has the highest precedence action value of all evaluated seccomp filters,
	\item the tracee enters \textbf{PTRACE\_EVENT\_SECCOMP} stop
	\item when the tracee is restarted, seccomp filters are evaluated again
	\item if \textbf{SECCOMP\_RET\_TRACE} again has the highest precedence action value
	\item then the syscall is allowed to proceed, similar to \textbf{SECCOMP\_RET\_ALLOW}
	\item otherwise the highest precedence action value will take effect
\end{itemize}
\end{block}
\end{frame}

%%%%%%%
\begin{frame}{\Large SECCOMP\_RET\_TRACE Linux kernel misdesign\hfill [\insertframenumber/\inserttotalframenumber]}
\Large
\begin{block}{Suggested SECCOMP\_RET\_TRACE implementation changes}
\begin{itemize}
	\item during the first evaluation of seccomp filters,
		\textbf{SECCOMP\_RET\_TRACE} should have the highest
		precedence, allowing the tracer to inspect and interfere
	\item security-wise, it's safe given that the tracer can interfere \\
		during syscall enter stop anyway
	\item during the second evaluation of seccomp filters,
		\textbf{SECCOMP\_RET\_TRACE} should have the lowest
		precedence, allowing other action values e.g.
		\textbf{SECCOMP\_RET\_LOG} to take effect
\end{itemize}
\end{block}
\end{frame}

%%%%%%%
\begin{frame}{\Large Seccomp-assisted syscall filtering limitations\hfill [\insertframenumber/\inserttotalframenumber]}
\Large
A seccomp program can be added, but cannot be removed

\begin{block}{Problem: SECCOMP\_RET\_TRACE rule without a tracer}
\begin{itemize}
	\item when a \textbf{SECCOMP\_RET\_TRACE} rule is triggered, \\
		the syscall is not executed
	\item instead, it returns a failure status with errno set to \textbf{ENOSYS}, \\
		similar to a \textbf{SECCOMP\_RET\_ERRNO(ENOSYS)} rule
	\item former tracee is most likely not equipped to handle \\
		situations like this
	\item consequently, all sorts of disasters may follow
\end{itemize}
\end{block}
\end{frame}

%%%%%%%
\begin{frame}{\Large Why seccomp-assisted syscall filtering is not the default?\hfill [\insertframenumber/\inserttotalframenumber]}
\large
\begin{block}{\textbf{strace -{}-seccomp-bpf} workarounds and limitations}
\begin{itemize}
	\item implies \textbf{-{}-kill-on-exit} option that sets \textbf{PTRACE\_O\_EXITKILL} option
	\item has no effect unless \textbf{-{}-follow-forks} (\textbf{-f}) option is specified
	\item incompatible with \textbf{-{}-syscall-limit}=\textit{number} option
	\item incompatible with \textbf{-{}-detach-on}=\textbf{execve} (\textbf{-b execve}) option
	\item not applicable to tracees attached using \textbf{-{}-attach}=\textit{pid} (\textbf{-p} \textit{pid}) option
	\item problematic if the tracee creates threads using \textbf{CLONE\_UNTRACED} flag
\end{itemize}
\end{block}

\begin{block}{Why?}
\begin{itemize}
	\item \textbf{strace} tries hard to ensure that it will not leave behind \\
		any unattended processes with the seccomp program it installed \\
		for syscall filtering purposes
	\item but we are not there yet
\end{itemize}
\end{block}
\end{frame}

%%%%%%%
\begin{frame}{\Large PTRACE\_ADD\_SECCOMP\_FILTER?\hfill [\insertframenumber/\inserttotalframenumber]}
\large
\begin{block}{What if?}
\begin{itemize}
	\item the tracer could install on its tracee a seccomp filter \\
		that would be tied to the tracer
	\item when a process ceases to be the tracer of its tracee, all seccomp filters \\
		associated with that tracer would be automatically removed \\
		(or permanently deactivated, if removal is not practical)
\end{itemize}
\end{block}

\begin{block}{Idea}
\begin{itemize}
	\item this could be a new ptrace command issued for tracees in a ptrace stop state:
		\textbf{ptrace(PTRACE\_ADD\_SECCOMP\_FILTER, pid, flags, args)}
	\item this would solve all problems that arise currently when a process with \\
		a seccomp filter containing a \textbf{SECCOMP\_RET\_TRACE} rule \\
		is left without its tracer.
\end{itemize}
\end{block}
\end{frame}

%%%%%%%
\begin{frame}{\Large PTRACE\_ADD\_SECCOMP\_FILTER?\hfill [\insertframenumber/\inserttotalframenumber]}
\Large
\begin{block}{Open API questions}
\begin{itemize}
	\item whether such seccomp filters should be returned by \textbf{PTRACE\_SECCOMP\_GET\_FILTER}
	\item how to return information about the tracer associated with
		the seccomp filter given that the array of \textbf{struct sock\_filter} objects
		returned by this ptrace command have no room for that
	\item how to inject such seccomp filters
\end{itemize}
\end{block}
\end{frame}

%%%%%%%
\begin{frame}{\Large PTRACE\_SET\_SYSCALL\_INFO?\hfill [\insertframenumber/\inserttotalframenumber]}
\Large
\begin{block}{Evolution of ptrace commands for obtaining syscall information}
\begin{itemize}
	\item PTRACE\_PEEKUSER -- ancient
	\item PTRACE\_GETREGS -- old
	\item PTRACE\_GETREGSET -- traditional
	\item PTRACE\_GET\_SYSCALL\_INFO -- modern
\end{itemize}
\end{block}

\begin{block}{Evolution of ptrace commands for changing syscall information}
\begin{itemize}
	\item PTRACE\_POKEUSER -- ancient
	\item PTRACE\_SETREGS -- old
	\item PTRACE\_SETREGSET -- traditional
	\item \textbf{PTRACE\_SET\_SYSCALL\_INFO} -- not implemented yet
\end{itemize}
\end{block}
\end{frame}

%%%%%%%
{
\setbeamertemplate{footline}{\pgfuseimage{left-corner-logo}}
\begin{frame}[noframenumbering]{Questions?}
	\begin{columns}
		\column{9cm}
\begin{block}{\large homepage}
	\url{https://strace.io}
\end{block}
\begin{block}{\large strace.git, strace-talks.git}
	\url{https://github.com/strace/strace.git}

	\url{https://gitlab.com/strace/strace.git}

\smallskip

	\url{https://github.com/strace/strace-talks.git}

	\url{https://gitlab.com/strace/strace-talks.git}
\end{block}
\begin{block}{\large mailing list}
	strace-devel@lists.strace.io
\end{block}
\begin{block}{\large IRC channel}
	\#strace@oftc
\end{block}
		\column{3cm}
			\centerline{\pgfuseimage{strace-logo}}
	\end{columns}
\end{frame}
}

\end{document}
