% Copyright (C) 2012-2020 Dmitry V. Levin <ldv@altlinux.org>
% Permission is granted to copy, distribute and/or modify this document
% under the terms of the GNU Free Documentation License, Version 1.2
% or any later version published by the Free Software Foundation;
% with no Invariant Sections, no Front-Cover Texts, and no Back-Cover Texts.

\documentclass[unicode,aspectratio=169,xcolor={table,dvipsnames,usernames}]{beamer}

\usepackage[utf8]{inputenc}
\usepackage[T2A]{fontenc}
\usepackage[english]{babel}
\usepackage{alltt}

\colorlet{darkred}{BrickRed}
\colorlet{darkgreen}{OliveGreen}

\pgfdeclareimage[height=2cm]{title-logo}{fosdem.pdf}
\pgfdeclareimage[height=0.6cm]{left-corner-logo}{basealt.jpg}
\pgfdeclareimage[height=1.2cm]{right-corner-logo}{strace-straus.pdf}
\pgfdeclareimage[height=5.8cm]{wat}{wat.pdf}
\pgfdeclareimage[height=7cm]{kaleidoscope}{kaleidoscope.jpg}
\pgfdeclareimage[height=8cm]{strace-logo}{strace-straus.pdf}

\usetheme{Warsaw}
\setbeamertemplate{headline}{}
\setbeamertemplate{footline}{%
	\pgfuseimage{left-corner-logo}%
	\hfill%
	\pgfuseimage{right-corner-logo}%
}
\setbeamertemplate{navigation symbols}{}
\setbeamertemplate{logo}{}
\setbeamercovered{transparent}

\title{\Huge Postmodern strace}
\author{\Huge Dmitry~Levin}
\date{\Large Brussels, 2020}
\titlegraphic{\pgfuseimage{title-logo}}

\begin{document}

%%%%%%%
\begin{frame}[noframenumbering]
\titlepage
\end{frame}

%%%%%%%
\begin{frame}{Traditional strace \hfill [\insertframenumber/\inserttotalframenumber]}
\large
\begin{block}{Printing instruction pointer and timestamps}
\begin{itemize}
\item print instruction pointer: \textbf{-i} option
\item print timestamps: \textbf{-r}, \textbf{-t}, \textbf{-tt}, \textbf{-ttt}, and \textbf{-T} options
\end{itemize}
\end{block}

\begin{block}{Size and format of strings}
\begin{itemize}
\item string size: \textbf{-s} option
\item string format: \textbf{-x} and \textbf{-xx} options
\end{itemize}
\end{block}

\begin{block}{Verbosity of syscall decoding}
\begin{itemize}
\item abbreviate output: \textbf{-e abbrev=\textit{set}}, \textbf{-v} option
\item dereference structures: \textbf{-e verbose=\textit{set}}
\item print raw undecoded syscalls: \textbf{-e raw=\textit{set}}
\end{itemize}
\end{block}
\end{frame}

%%%%%%%
\begin{frame}{Traditional strace \hfill [\insertframenumber/\inserttotalframenumber]}
\large
\begin{block}{Printing signals}
\begin{itemize}
\item print signals: \textbf{-e signal=\textit{set}}
\end{itemize}
\end{block}

\begin{block}{Dumping}
\begin{itemize}
\item dump the data read from the specified descriptors: \textbf{-e read=\textit{set}}
\item dump the data written to the specified descriptors: \textbf{-e write=\textit{set}}
\end{itemize}
\end{block}

\begin{block}{Redirecting output to files or pipelines}
\begin{itemize}
\item write the trace to a file or pipeline: \textbf{-o \textit{filename}} option
\item write traces of processes to separate files: \textbf{-ff -o \textit{filename}}
\end{itemize}
\end{block}
\end{frame}

%%%%%%%
\begin{frame}{Traditional strace \hfill [\insertframenumber/\inserttotalframenumber]}
\large
\begin{block}{System call filtering}
\begin{itemize}
\item trace only the specified set of system calls: \textbf{-e trace=\textit{set}}
\end{itemize}
\end{block}

\begin{block}{System call statistics}
\begin{itemize}
\item count time, calls, and errors for each system call: \textbf{-c} option
\item sort the histogram printed by the \textbf{-c} option: \textbf{-S \textit{sortby}} option
\end{itemize}
\end{block}

\begin{block}{Tracing control}
\begin{itemize}
\item attach to existing processes: \textbf{-p \textit{pid}} option
\item trace child processes: \textbf{-f} option
\end{itemize}
\end{block}
\end{frame}

%%%%%%
\begin{frame}{Modern strace \hfill [\insertframenumber/\inserttotalframenumber]}
\Large
\begin{block}{Tracing output format}
\begin{itemize}
\item pathnames accessed by name or descriptor: \textbf{-y} option
\item network protocol associated with descriptors: \textbf{-yy} option
\item stack of function calls: \textbf{-k} option
\end{itemize}
\end{block}

\begin{block}{System call filtering}
\begin{itemize}
\item pathnames accessed by name or descriptor: \textbf{-P} option
\item regular expressions: -e \textbf{trace}=/\textit{regexp}
\item optional specifications: -e \textbf{trace}=?\textit{spec}
\item new syscall classes: \%stat, \%lstat, \%fstat, \%statfs, \%fstatfs, \%\%stat, \%\%statfs
\end{itemize}
\end{block}
\end{frame}

%%%%%%%
\begin{frame}{Modern strace \hfill [\insertframenumber/\inserttotalframenumber]}
\Large
\begin{block}{System call statistics}
\begin{itemize}
\item wall clock time spent in syscalls: \textbf{-w} option
\item combine statistics with regular output: \textbf{-C} option
\end{itemize}
\end{block}

\begin{block}{Tracing control}
\begin{itemize}
\item attach to multiple processes: \textbf{-p \textit{pid\_set}} option
\item detach on execve: \textbf{-b execve} option
\item run as a detached grandchild: \textbf{-D} option
\item interruptibility: \textbf{-I} option
\item postprocessing: \textbf{strace-log-merge}
\end{itemize}
\end{block}
\end{frame}

%%%%%%%
\begin{frame}{Modern strace \hfill [\insertframenumber/\inserttotalframenumber]}
\Large
\begin{block}{System call tampering}
\begin{itemize}
\item fault injection: \\
-e \textbf{inject}=\textit{set}:\textbf{error}=\textit{errno}[:\textbf{when}=\textit{expr}][:\textbf{syscall}=\textit{syscall}]
\item return value injection: \\
-e \textbf{inject}=\textit{set}:\textbf{retval}=\textit{value}[:\textbf{when}=\textit{expr}][:\textbf{syscall}=\textit{syscall}]
\item signal injection: \\
-e \textbf{inject}=\textit{set}:\textbf{signal}=\textit{set}
\item delay injection: \\
-e \textbf{inject}=\textit{set}:\textbf{delay\_enter}=\textit{usecs} \\
-e \textbf{inject}=\textit{set}:\textbf{delay\_exit}=\textit{usecs}
\end{itemize}
\end{block}
\end{frame}

%%%%%%%
\begin{frame}{Postmodern strace\hfill [\insertframenumber/\inserttotalframenumber]}
\Large
\begin{block}{New features since FOSDEM 2018}
\begin{itemize}
\item PTRACE\_GET\_SYSCALL\_INFO API support
\item system call return status filtering: \\ \textbf{-e status}=\textit{set}, \textbf{-z}, \textbf{-Z} options
\item seccomp-assisted system call filtering: \textbf{-{}-seccomp-bpf} option
\item format of named constants and flags: \textbf{-X} option
\item support of new system calls ($ \approx 35$)
\item elaborate syscall parsers
\item long options
\item copyleft license
\end{itemize}
\end{block}
\end{frame}

%%%%%%%
\begin{frame}{x86-64 architecture in Linux kernel \hfill [\insertframenumber/\inserttotalframenumber]}
\large
\begin{block}{Operating modes}
\begin{description}
	\item[64-bit mode]: CS register value == 0x33
	\item[32-bit mode]: CS register value == 0x23
\end{description}
\end{block}

\begin{block}{Several methods of system call invocation}
\begin{description}
	\item[int 0x80]: Legacy 32-bit
	\item[sysenter]: Fast 32-bit
	\item[syscall]: 64-bit
\end{description}
\end{block}

Surprise: 64-bit processes can invoke both 64-bit and 32-bit system calls.

\begin{block}{Linux API provides}
\begin{itemize}
	\item The system call number
	\item The value of CS register
	\item The value of RIP register
\end{itemize}
\end{block}
\end{frame}

%%%%%%%
\begin{frame}{Linux system call tracers on x86-64 architecture \hfill [\insertframenumber/\inserttotalframenumber]}
\begin{block}{Legacy method of obtaining system call information}
\begin{itemize}
	\item Fetch the system call number (PTRACE\_PEEKUSER ORIG\_RAX)
	\item Fetch the value of CS register (PTRACE\_PEEKUSER CS)
	\item {\bf Guess the system call bitness by the value of CS register}
	\item Determine the system call by its number and bitness
	\item Fetch the system call arguments accordingly
\end{itemize}
\end{block}

\begin{block}{Traditional method of obtaining system call information}
\begin{itemize}
	\item Fetch the whole set of registers (PTRACE\_GETREGSET NT\_PRSTATUS), \\
		the return value is decided by the value of CS register
	\item {\bf Guess the system call bitness by the return value}
	\item Determine the system call by its number and bitness
	\item Interpret other registers as the system call arguments accordingly
\end{itemize}
\end{block}
\end{frame}

%%%%%%%
\begin{frame}[fragile]{Process bitness does not match syscall bitness \hfill [\insertframenumber/\inserttotalframenumber]}
\large
\begin{block}{Example based on Debian bug report \#459820 submitted in 2008}
\begin{alltt}
#include <stdio.h>
#include <unistd.h>
int main() \{
    setlinebuf(stdout);
    puts("------------");
    \textcolor{darkred}{__asm__("movl $2, %eax; int $0x80");}
    printf("[I am %d]{\textbackslash}n", getpid());
    return 0;
\}
\end{alltt}
\end{block}

\begin{block}{Regular invocation: ./debbug459820}
\begin{alltt}
------------
[I am 23450]
[I am 23451]
\end{alltt}
\end{block}
\end{frame}

%%%%%%%
\begin{frame}[fragile]{Process bitness does not match syscall bitness \hfill [\insertframenumber/\inserttotalframenumber]}
\begin{columns}
	\column{10cm}
		\Large
		\begin{block}{Invocation under strace}
\begin{alltt}
$ strace -f ./debbug459820 > /dev/null
\ldots
write(1, "------------{\textbackslash}n", 13) = 13
strace: Process 23451 attached
\textcolor{darkred}{open(0x1, O_RDONLY|O_CREAT|O_TRUNC
|O_DSYNC|O_DIRECT|O_NOATIME|O_CLOEXEC
|O_PATH|O_TMPFILE|0x1000020, 0134300)
= 23451}
\ldots
\end{alltt}
		\end{block}
	\column{3cm}
\end{columns}
\end{frame}

%%%%%%%
{
\begin{frame}[fragile,noframenumbering]{Process bitness does not match syscall bitness \hfill [\insertframenumber/\inserttotalframenumber]}
\begin{columns}
	\column{10cm}
		\Large
		\begin{block}{Invocation under strace}
\begin{alltt}
$ strace -f ./debbug459820 > /dev/null
\ldots
write(1, "------------{\textbackslash}n", 13) = 13
strace: Process 23451 attached
\textcolor{darkred}{open(0x1, O_RDONLY|O_CREAT|O_TRUNC
|O_DSYNC|O_DIRECT|O_NOATIME|O_CLOEXEC
|O_PATH|O_TMPFILE|0x1000020, 0134300)
= 23451}
\ldots
\end{alltt}
		\end{block}
	\column{3cm}
		\begin{figure}
			\centering
			\pgfuseimage{wat}
		\end{figure}
\end{columns}
\end{frame}
}

%%%%%%%
\begin{frame}[fragile]{Process bitness does not match syscall bitness \hfill [\insertframenumber/\inserttotalframenumber]}
\Large
\begin{block}{\$ strace -f -z ./debbug459820 > /dev/null}
\begin{alltt}
write(1, "------------{\textbackslash}n", 13)          = 13
strace: Process 23451 attached
\textcolor{darkred}{open(0x1, O_RDONLY|O_CREAT|O_TRUNC|O_DSYNC|O_DIRECT
|O_NOATIME|O_CLOEXEC|O_PATH|O_TMPFILE|0x1000020,
0134300) = 23451}
[pid 23450] getpid()                    = 23450
[pid 23451] getpid()                    = 23451
[pid 23450] write(1, "[I am 23450]{\textbackslash}n", 13) = 13
[pid 23451] write(1, "[I am 23451]{\textbackslash}n", 13) = 13
[pid 23450] +++ exited with 0 +++
+++ exited with 0 +++
\end{alltt}
\end{block}
\end{frame}

%%%%%%%
\begin{frame}[fragile]{Process bitness does not match syscall bitness: problem \hfill [\insertframenumber/\inserttotalframenumber]}
\scriptsize
\begin{columns}
	\column{11cm}
		\begin{block}{for i in `seq 0 9`; do strace -qq -esignal=none -eopen ./debbug459820 >/dev/null; done}
		%\textcolor{blue}{\(\longrightarrow\)}
\begin{alltt}
open(0x1, O_RDONLY|O_CREAT|O_EXCL|O_TRUNC|O_SYNC|O_DIRECT|O_LARGEFILE|
          O_NOFOLLOW|O_CLOEXEC|0x4f000008, 0151330) = 15565
open(0x1, O_RDONLY|O_EXCL|O_NOCTTY|O_APPEND|O_NONBLOCK|O_DSYNC|O_TMPFILE|
          FASYNC|0x57800008, 036630) = 15570
open(0x1, O_RDONLY|O_EXCL|O_NOCTTY|O_APPEND|O_SYNC|O_LARGEFILE|O_NOATIME|
          O_PATH|O_DIRECTORY|FASYNC|0x8e800038) = 15575
open(0x1, O_RDONLY|O_CREAT|O_EXCL|O_APPEND|O_DSYNC|O_DIRECT|O_NOFOLLOW|
          O_PATH|O_DIRECTORY|FASYNC|0xe2800018, 072350) = 15580
open(0x1, O_RDONLY|O_CREAT|O_NOCTTY|O_SYNC|O_NOFOLLOW|O_CLOEXEC|FASYNC|
          0xcf800038, 030610) = 15585
open(0x1, O_RDONLY|O_TRUNC|O_NOFOLLOW|O_CLOEXEC|O_DIRECTORY|FASYNC|
          0x11800008) = 15590
open(0x1, O_RDONLY|O_CREAT|O_EXCL|O_NOCTTY|__O_SYNC|O_LARGEFILE|O_NOATIME|
          O_CLOEXEC|O_PATH|O_TMPFILE|FASYNC|0x43000038, 0121010) = 15595
open(0x1, O_RDONLY|O_EXCL|O_NONBLOCK|__O_SYNC|O_DIRECT|O_CLOEXEC|O_PATH|
          __O_TMPFILE|FASYNC|0x3a800038, 064310) = 15600
open(0x1, O_RDONLY|O_CREAT|O_EXCL|O_NOCTTY|O_NONBLOCK|O_DSYNC|O_DIRECT|
          O_LARGEFILE|O_DIRECTORY|0x47800028, 0154770) = 15610
open(0x1, O_RDONLY|O_NOFOLLOW|O_CLOEXEC|O_PATH|FASYNC|0x5e000008) = 15605
\end{alltt}
		\end{block}
	\column{3cm}
		\begin{figure}
			\centering
			\pgfuseimage{kaleidoscope}
		\end{figure}
\end{columns}
\end{frame}

%%%%%%%
\begin{frame}[fragile]{PTRACE\_GET\_SYSCALL\_INFO support \hfill [\insertframenumber/\inserttotalframenumber]}
\begin{block}{\Large Linux >= v5.3-rc1}
\begin{verbatim}
$ git log -i -E --author=altlinux.org \
  --grep='ptrace|syscall_a|elf-em|selftests' \
  v4.20-rc2..v5.3-rc1
\end{verbatim}
\begin{itemize}
	\item 29 commits, 47 files changed, 703 insertions, 125 deletions
	\item 2 authors: Elvira Khabirova, Dmitry Levin
	\item 22 persons added their Acked-by/Reviewed-by/Signed-off-by
	\item 07.11.2018: first RFC patch submitted
	\item 12.11.2018: first patch committed
	\item 13.12.2018: API finalized
	\item 17.07.2019: last patch committed
	\item Implements PTRACE\_GET\_SYSCALL\_INFO on those 19 architectures
		that enable CONFIG\_HAVE\_ARCH\_TRACEHOOK
\end{itemize}
\end{block}
\end{frame}

%%%%%%%
\begin{frame}[fragile]{PTRACE\_GET\_SYSCALL\_INFO support \hfill [\insertframenumber/\inserttotalframenumber]}
\begin{block}{Linux PTRACE\_GET\_SYSCALL\_INFO API}
\scriptsize
\begin{alltt}
struct ptrace_syscall_info \{
  __u8 op;			\hfill /* Type of system call stop */
  \textcolor{darkgreen}{__aligned_u32 arch;	\hfill /* AUDIT_ARCH_* value; see seccomp(2) */}
  __u64 instruction_pointer;	\hfill /* CPU instruction pointer */
  __u64 stack_pointer;		\hfill /* CPU stack pointer */
  union \{
    struct \{			\hfill /* op == PTRACE_SYSCALL_INFO_ENTRY */
      __u64 nr;			\hfill /* Syscall number */
      __u64 args[6];		\hfill /* Syscall arguments */
    \} entry;
    struct \{			\hfill /* op == PTRACE_SYSCALL_INFO_EXIT */
      __s64 rval;		\hfill /* Syscall return value */
      __u8 is_error;		\hfill /* Does rval contain an error value? */
    \} exit;
    struct \{			\hfill /* op == PTRACE_SYSCALL_INFO_SECCOMP */
      __u64 nr;			\hfill /* Syscall number */
      __u64 args[6];		\hfill /* Syscall arguments */
      __u32 ret_data;		\hfill /* SECCOMP_RET_DATA portion of SECCOMP_RET_TRACE */
    \} seccomp;
  \};
\};
\end{alltt}
\end{block}
\end{frame}

%%%%%%%
\begin{frame}[fragile]{PTRACE\_GET\_SYSCALL\_INFO support \hfill [\insertframenumber/\inserttotalframenumber]}

{\Large strace >= v4.26, linux >= v5.3-rc1}
\begin{block}{Invocation under strace: strace -f -z ./debbug459820 > /dev/null}
\begin{alltt}
\ldots
write(1, "------------{\textbackslash}n", 13)          = 13
\textcolor{darkred}{strace: [ Process PID=23450 runs in 32 bit mode. ]}
strace: Process 23451 attached
fork()                                  = 23451
\textcolor{darkred}{strace: [ Process PID=23450 runs in 64 bit mode. ]}
[pid 23450] getpid()                    = 23450
\textcolor{darkred}{strace: [ Process PID=23451 runs in 64 bit mode. ]}
[pid 23451] getpid()                    = 23451
[pid 23450] write(1, "[I am 23450]{\textbackslash}n", 13) = 13
[pid 23451] write(1, "[I am 23451]{\textbackslash}n", 13) = 13
[pid 23450] +++ exited with 0 +++
+++ exited with 0 +++
\end{alltt}
\end{block}
\end{frame}
%%%%%%%
\begin{frame}{System call return status filtering: \textbf{-e status}=\textit{set} \hfill [\insertframenumber/\inserttotalframenumber]}
\Large
\begin{block}{Introduced in v5.2 (July 2019)}
\large
Print only system calls with the specified return status. \\
\textit{set} can include the following elements:
\begin{description}
	\item[successful]: returned without an error code, alias to \textbf{-z}
	\item[failed]: returned with an error code, alias to \textbf{-Z}
	\item[unfinished]: did not return
	\item[detached]: detached before return
	\item[unavailable]: returned but failed to fetch the error status
\end{description}
\end{block}
The default is \textbf{-e status}=\textbf{all}.
\end{frame}

%%%%%%%
\begin{frame}[fragile]{System call return status filtering: \textbf{-z}, \textbf{-Z} \hfill [\insertframenumber/\inserttotalframenumber]}
\scriptsize
\begin{block}{env -i LD\_LIBRARY\_PATH=/lib64 strace -z -e\%file /bin/cat < /dev/null}
\begin{alltt}
execve("/bin/cat", ["/bin/cat"], 0x7ffc2b781ca0 /* 1 var */) = 0
openat(AT_FDCWD, "/lib64/libc.so.6", O_RDONLY|O_CLOEXEC) = 3
+++ exited with 0 +++
\end{alltt}
\end{block}
\begin{block}{env -i LD\_LIBRARY\_PATH=/lib64 strace -Z -e\%file /bin/cat </dev/null}
\begin{alltt}
access("/etc/ld.so.preload", R_OK) = -1 ENOENT (No such file or directory)
openat(AT_FDCWD, "/lib64/tls/x86_64/x86_64/libc.so.6", O_RDONLY|O_CLOEXEC) = -1 ENOENT (No suc
stat("/lib64/tls/x86_64/x86_64", 0x7ffdcc6a0c20) = -1 ENOENT (No such file or directory)
openat(AT_FDCWD, "/lib64/tls/x86_64/libc.so.6", O_RDONLY|O_CLOEXEC) = -1 ENOENT (No such file
stat("/lib64/tls/x86_64", 0x7ffdcc6a0c20) = -1 ENOENT (No such file or directory)
openat(AT_FDCWD, "/lib64/tls/x86_64/libc.so.6", O_RDONLY|O_CLOEXEC) = -1 ENOENT (No such file
stat("/lib64/tls/x86_64", 0x7ffdcc6a0c20) = -1 ENOENT (No such file or directory)
openat(AT_FDCWD, "/lib64/tls/libc.so.6", O_RDONLY|O_CLOEXEC) = -1 ENOENT (No such file or dire
stat("/lib64/tls", 0x7ffdcc6a0c20) = -1 ENOENT (No such file or directory)
openat(AT_FDCWD, "/lib64/x86_64/x86_64/libc.so.6", O_RDONLY|O_CLOEXEC) = -1 ENOENT (No such fi
stat("/lib64/x86_64/x86_64", 0x7ffdcc6a0c20) = -1 ENOENT (No such file or directory)
openat(AT_FDCWD, "/lib64/x86_64/libc.so.6", O_RDONLY|O_CLOEXEC) = -1 ENOENT (No such file or d
stat("/lib64/x86_64", 0x7ffdcc6a0c20) = -1 ENOENT (No such file or directory)
openat(AT_FDCWD, "/lib64/x86_64/libc.so.6", O_RDONLY|O_CLOEXEC) = -1 ENOENT (No such file or d
stat("/lib64/x86_64", 0x7ffdcc6a0c20) = -1 ENOENT (No such file or directory)
+++ exited with 0 +++
\end{alltt}
\end{block}
\end{frame}

%%%%%%%
\begin{frame}[fragile]{System call return status filtering: \textbf{-z} aggregation \hfill [\insertframenumber/\inserttotalframenumber]}
\begin{block}{strace -o log -f -e signal=none -e trace=execve,nanosleep {\textbackslash} \\
	sh -c 'sleep 0.1 \& sleep 0.2 \& sleep 0.3' \&\& cat log}
\scriptsize
\begin{alltt}
13475 execve("/bin/sh", ["sh", "-c", "sleep 0.1 & sleep 0.2 & sleep 0."...],
      0x5631be4f87a8 /* 42 vars */) = 0
13476 execve("/bin/sleep", ["sleep", "0.1"], 0xe4c4f0 /* 33 vars */ <unfinished ...>
13477 execve("/bin/sleep", ["sleep", "0.2"], 0xe4c4f0 /* 33 vars */ <unfinished ...>
13478 execve("/bin/sleep", ["sleep", "0.3"], 0xe4c4f0 /* 33 vars */ <unfinished ...>
13476 <... execve resumed>)             = 0
13477 <... execve resumed>)             = 0
13478 <... execve resumed>)             = 0
13476 nanosleep({tv_sec=0, tv_nsec=100000000},  <unfinished ...>
13477 nanosleep({tv_sec=0, tv_nsec=200000000},  <unfinished ...>
13478 nanosleep({tv_sec=0, tv_nsec=300000000},  <unfinished ...>
13476 <... nanosleep resumed>NULL)      = 0
13476 +++ exited with 0 +++
13477 <... nanosleep resumed>NULL)      = 0
13477 +++ exited with 0 +++
13478 <... nanosleep resumed>NULL)      = 0
13478 +++ exited with 0 +++
13475 +++ exited with 0 +++
\end{alltt}
\end{block}
\end{frame}

%%%%%%%
\begin{frame}[fragile]{System call return status filtering: \textbf{-z} aggregation \hfill [\insertframenumber/\inserttotalframenumber]}
\begin{block}{strace -o log \textbf{-z} -f -e signal=none -e trace=execve,nanosleep {\textbackslash} \\
	sh -c 'sleep 0.1 \& sleep 0.2 \& sleep 0.3' \&\& cat log}
\small
\begin{alltt}
13475 execve("/bin/sh", ["sh", "-c", "sleep 0.1 & sleep 0.2 & sleep 0."...],
      0x5631be4f87a8 /* 42 vars */) = 0
13476 execve("/bin/sleep", ["sleep", "0.1"], 0xe4c4f0 /* 33 vars */) = 0
13477 execve("/bin/sleep", ["sleep", "0.2"], 0xe4c4f0 /* 33 vars */) = 0
13478 execve("/bin/sleep", ["sleep", "0.3"], 0xe4c4f0 /* 33 vars */) = 0
13476 nanosleep({tv_sec=0, tv_nsec=100000000}, NULL) = 0
13476 +++ exited with 0 +++
13477 nanosleep({tv_sec=0, tv_nsec=200000000}, NULL) = 0
13477 +++ exited with 0 +++
13478 nanosleep({tv_sec=0, tv_nsec=300000000}, NULL) = 0
13478 +++ exited with 0 +++
13475 +++ exited with 0 +++
\end{alltt}
\end{block}
\end{frame}

%%%%%%%
\begin{frame}[fragile]{System call return status filtering: \textbf{-Z} \hfill [\insertframenumber/\inserttotalframenumber]}
\large
\begin{block}{Introduced in v5.2 (July 2019)}
Trace only those system calls that returned with an error code: \\
\textbf{-Z}, \textbf{-e status}=\textbf{failed}
\end{block}

\begin{block}{eu-elflint no-such-file1 no-such-file2}
\scriptsize
\begin{alltt}
eu-elflint: cannot open input file: No such file or directory
eu-elflint: cannot open input file: No such file or directory
\end{alltt}
\end{block}

\begin{block}{strace -qq -Z -efile eu-elflint no-such-file1 no-such-file2}
\scriptsize
\begin{alltt}
access("/etc/ld.so.preload", R_OK) = -1 ENOENT (No such file or directory)
openat(AT_FDCWD, "no-such-file1", O_RDONLY) = -1 ENOENT (No such file or directory)
eu-elflint: cannot open input file: No such file or directory
openat(AT_FDCWD, "no-such-file2", O_RDONLY) = -1 ENOENT (No such file or directory)
eu-elflint: cannot open input file: No such file or directory
\end{alltt}
\end{block}
Fixed in elfutils-0.178.
\end{frame}

%%%%%%%
\begin{frame}[fragile]{Seccomp-assisted system call filtering \hfill [\insertframenumber/\inserttotalframenumber]}
\begin{block}{Introduced in v5.3 (September 2019)}
\scriptsize
\begin{itemize}
\item Automatically generates and attaches a BPF program to filter system calls
\item Makes execution of untraced system calls two orders of magnitude faster
\end{itemize}
\end{block}

\begin{block}{The infamous dd example}
\scriptsize
\begin{alltt}
\$ dd if=/dev/zero of=/dev/null bs=1 count=1M 2>\&1 | grep -v records
1048576 bytes (1.0 MB, 1.0 MiB) copied, \textcolor{darkred}{0.668768} s, 1.6 MB/s

\$ strace \textcolor{darkgreen}{-{}-seccomp-bpf} -f -qq -e signal=none -e trace=fchdir {\textbackslash}
  dd if=/dev/zero of=/dev/null bs=1 count=1M 2>\&1 | grep -v records
1048576 bytes (1.0 MB, 1.0 MiB) copied, \textcolor{darkred}{0.736511} s, 1.4 MB/s

\$ strace -f -qq -e signal=none -e trace=fchdir {\textbackslash}
  dd if=/dev/zero of=/dev/null bs=1 count=1M 2>\&1 | grep -v records
1048576 bytes (1.0 MB, 1.0 MiB) copied, \textcolor{darkred}{28.0445} s, 37.4 kB/s
\end{alltt}
\end{block}

\begin{block}{Comparative slowdown}
$0.736511 / 0.668768 \approx \textcolor{darkred}{1.10}$,
$28.0445 / 0.736511 \approx \textcolor{darkred}{38}$,
$28.0445 / 0.668768 \approx \textcolor{darkred}{42}$
\end{block}
\end{frame}

%%%%%%%
\begin{frame}{Long options \hfill [\insertframenumber/\inserttotalframenumber]}
\Large
\begin{block}{1991 \ldots 2019: short options only}
a:Ab:cCdDe:E:fFhiI:ko:O:p:P:qrs:S:tTu:vVwxX:yzZ
\end{block}
\begin{block}{Introduced in v5.3 (September 2019)}
\begin{itemize}
	\item \texttt{-{}-help}: alias to \texttt{-h}
	\item \texttt{-{}-version}: alias to \texttt{-V}
	\item \texttt{-{}-seccomp-bpf}: seccomp-assisted system call filtering
\end{itemize}
\end{block}
\begin{block}{Introduced in v5.5 (February 2020)}
\begin{itemize}
	\item \texttt{-{}-debug}: alias to \texttt{-d}
	\item \ldots
\end{itemize}
\end{block}
\end{frame}

%%%%%%%
\begin{frame}[fragile]{Stack of function calls: \textbf{-k} option \hfill [\insertframenumber/\inserttotalframenumber]}
\begin{block}{\large strace -qq -P /dev/full cat /dev/null > /dev/full}
\scriptsize
\begin{alltt}
fstat(1, {st_mode=S_IFCHR|0666, st_rdev=makedev(1, 7), ...}) = 0
close(1)                                = 0
\end{alltt}
\end{block}

\begin{block}{\large strace \textbf{-k} -qq -P /dev/full cat /dev/null > /dev/full}
\scriptsize
\begin{alltt}
fstat(1, {st_mode=S_IFCHR|0666, st_rdev=makedev(1, 7), ...}) = 0
 > /lib64/libc-2.27.so(__fxstat64+0x13) [0xe79c3]
 > /bin/cat(main+0x1b3) [0x4017e3]
 > /lib64/libc-2.27.so(__libc_start_main+0xe6) [0x21bd6]
 > /bin/cat(_start+0x29) [0x402179]
close(1)                                = 0
 > /lib64/libc-2.27.so(__close_nocancel+0x7) [0xe8b47]
 > /lib64/libc-2.27.so(_IO_file_close_it@@GLIBC_2.2.5+0x67) [0x79fd7]
 > /lib64/libc-2.27.so(fclose@@GLIBC_2.2.5+0x136) [0x6d376]
 > /bin/cat(close_stream+0x19) [0x404ce9]
 > /bin/cat(close_stdout+0x11) [0x402691]
 > /lib64/libc-2.27.so(__run_exit_handlers+0x170) [0x379c0]
 > /lib64/libc-2.27.so(exit+0x19) [0x37ab9]
 > /lib64/libc-2.27.so(__libc_start_main+0xed) [0x21bdd]
 > /bin/cat(_start+0x29) [0x402179]
\end{alltt}
\end{block}
\end{frame}

%%%%%%%
\begin{frame}[fragile]{Running as a detached grandchild: \textbf{-D} option \hfill [\insertframenumber/\inserttotalframenumber]}
\begin{block}{\large foreground strace}
\begin{alltt}
$ echo $$ && strace -e none sh -c 'echo $PPID'
1234
23456
+++ exited with 0 +++
$ echo $$ && strace -e none sh -c 'echo $PPID'
1234
23459
+++ exited with 0 +++
\end{alltt}
\end{block}

\begin{block}{\large background strace}
\begin{alltt}
$ echo $$ && strace \textcolor{darkred}{-D} -e none sh -c 'echo $PPID'
1234
1234
+++ exited with 0 +++
\end{alltt}
\end{block}
\end{frame}

%%%%%%%
\begin{frame}[fragile]{Format of named constants and flags: \textbf{-X} option \hfill [\insertframenumber/\inserttotalframenumber]}
\begin{block}{\large strace -e /open cat /dev/null}
\scriptsize
\begin{alltt}
openat(\textcolor{darkred}{AT_FDCWD}, "/etc/ld.so.cache", \textcolor{darkred}{O_RDONLY|O_CLOEXEC}) = 3
openat(\textcolor{darkred}{AT_FDCWD}, "/lib64/libc.so.6", \textcolor{darkred}{O_RDONLY|O_CLOEXEC}) = 3
openat(\textcolor{darkred}{AT_FDCWD}, "/dev/null", \textcolor{darkred}{O_RDONLY}) = 3
+++ exited with 0 +++
\end{alltt}
\end{block}

\begin{block}{\large strace \textbf{-X verbose} -e /open cat /dev/null}
\scriptsize
\begin{alltt}
openat(\textcolor{darkred}{-100 /* AT_FDCWD */}, "/etc/ld.so.cache",
       \textcolor{darkred}{0x80000 /* O_RDONLY|O_CLOEXEC */}) = 3
openat(\textcolor{darkred}{-100 /* AT_FDCWD */}, "/lib64/libc.so.6",
       \textcolor{darkred}{0x80000 /* O_RDONLY|O_CLOEXEC */}) = 3
openat(\textcolor{darkred}{-100 /* AT_FDCWD */}, "/dev/null", \textcolor{darkred}{0 /* O_RDONLY */}) = 3
+++ exited with 0 +++
\end{alltt}
\end{block}

\begin{block}{\large strace \textbf{-X raw} -e /open cat /dev/null}
\scriptsize
\begin{alltt}
openat(\textcolor{darkred}{-100}, "/etc/ld.so.cache", \textcolor{darkred}{0x80000}) = 3
openat(\textcolor{darkred}{-100}, "/lib64/libc.so.6", \textcolor{darkred}{0x80000}) = 3
openat(\textcolor{darkred}{-100}, "/dev/null", \textcolor{darkred}{0})            = 3
+++ exited with 0 +++
\end{alltt}
\end{block}
\end{frame}

%%%%%%%
\begin{frame}{New system calls \hfill [\insertframenumber/\inserttotalframenumber]}
\begin{block}{\Large Introduced in v5.3 (September 2019)}
pidfd\_open, clone3
\end{block}

\begin{block}{\Large Introduced in v5.2 (July 2019)}
open\_tree, move\_mount, fsopen, fsconfig, fsmount, fspick
\end{block}

\begin{block}{\Large Introduced in v5.1 (May 2019)}
clock\_gettime64,
clock\_settime64,
clock\_adjtime64,
clock\_getres\_time64,
clock\_nanosleep\_time64,
timer\_gettime64,
timer\_settime64,
timerfd\_gettime64,
timerfd\_settime64,
utimensat\_time64,
pselect6\_time64,
ppoll\_time64,
io\_pgetevents\_time64,
recvmmsg\_time64,
mq\_timedsend\_time64,
mq\_timedreceive\_time64,
semtimedop\_time64,
rt\_sigtimedwait\_time64,
futex\_time64,
sched\_rr\_get\_interval\_time64,
pidfd\_send\_signal,
io\_uring\_setup,
io\_uring\_enter,
io\_uring\_register
\end{block}
\end{frame}

%%%%%%%
\begin{frame}{Netlink socket parsers \hfill [\insertframenumber/\inserttotalframenumber]}
\tiny
\begin{block}{\large Currently supported netlink protocols}
\begin{itemize}
\item NETLINK\_AUDIT
\item NETLINK\_CRYPTO
\item NETLINK\_KOBJECT\_UEVENT
\item NETLINK\_NETFILTER
\item NETLINK\_ROUTE
\item NETLINK\_SELINUX
\item NETLINK\_SOCK\_DIAG
\item NETLINK\_XFRM
\item NETLINK\_GENERIC
\end{itemize}
\end{block}

\begin{block}{\large NETLINK\_ROUTE: ip route list table all}
\begin{alltt}
broadcast 127.0.0.0 dev lo table local proto kernel scope link src 127.0.0.1

local 127.0.0.0/8 dev lo table local proto kernel scope host src 127.0.0.1

local 127.0.0.1 dev lo table local proto kernel scope host src 127.0.0.1

broadcast 127.255.255.255 dev lo table local proto kernel scope link src 127.0.0.1
\end{alltt}
\end{block}
\end{frame}

%%%%%%%
\begin{frame}{Netlink socket parsers: NETLINK\_ROUTE \hfill [\insertframenumber/\inserttotalframenumber]}
\tiny
\begin{block}{\large strace -e trace=sendto,recvmsg ip route list}
\begin{alltt}
sendto(3, \{\{len=40, type=RTM\_GETROUTE, flags=NLM\_F\_REQUEST|NLM\_F\_DUMP, seq=1357924680, pid=0\},
 \textcolor{blue}{\{rtm\_family=AF\_UNSPEC, rtm\_dst\_len=0, rtm\_src\_len=0, rtm\_tos=0, rtm\_table=RT\_TABLE\_UNSPEC, rtm\_protocol=RTPROT\_UNSPEC, rtm\_scope=RT\_SCOPE\_UNIVERSE, rtm\_type=RTN\_UNSPEC, rtm\_flags=0\}},
  \textcolor{darkred}{\{nla\_len=0, nla\_type=RTA\_UNSPEC\}}\}, 40, 0, NULL, 0) = 40

\smallskip
recvmsg(3, \{msg\_name=\{sa\_family=AF\_NETLINK, nl\_pid=0, nl\_groups=00000000\}, msg\_namelen=12, msg\_iov=[\{iov\_base=[
\textcolor{darkgreen}{\{\{len=60, type=RTM\_NEWROUTE, flags=NLM\_F\_MULTI, seq=1357924680, pid=12345\},
 \textcolor{blue}{\{rtm\_family=AF\_INET, rtm\_dst\_len=32, rtm\_src\_len=0, rtm\_tos=0, rtm\_table=RT\_TABLE\_LOCAL, rtm\_protocol=RTPROT\_KERNEL, rtm\_scope=RT\_SCOPE\_LINK, rtm\_type=RTN\_BROADCAST, rtm\_flags=0\}},
  \textcolor{darkred}{[\{\{nla\_len=8, nla\_type=RTA\_TABLE\}, RT\_TABLE\_LOCAL\}, \{\{nla\_len=8, nla\_type=RTA\_DST\}, inet\_addr("127.0.0.0")\}, \{\{nla\_len=8, nla\_type=RTA\_PREFSRC\}, inet\_addr("127.0.0.1")\}, \{\{nla\_len=8, nla\_type=RTA\_OIF\}, if\_nametoindex("lo")\}]}\}},
\textcolor{darkgreen}{\{\{len=60, type=RTM\_NEWROUTE, flags=NLM\_F\_MULTI, seq=1357924680, pid=12345\},
 \textcolor{blue}{\{rtm\_family=AF\_INET, rtm\_dst\_len=8, rtm\_src\_len=0, rtm\_tos=0, rtm\_table=RT\_TABLE\_LOCAL, rtm\_protocol=RTPROT\_KERNEL, rtm\_scope=RT\_SCOPE\_HOST, rtm\_type=RTN\_LOCAL, rtm\_flags=0\}},
  \textcolor{darkred}{[\{\{nla\_len=8, nla\_type=RTA\_TABLE\}, RT\_TABLE\_LOCAL\}, \{\{nla\_len=8, nla\_type=RTA\_DST\}, inet\_addr("127.0.0.0")\}, \{\{nla\_len=8, nla\_type=RTA\_PREFSRC\}, inet\_addr("127.0.0.1")\}, \{\{nla\_len=8, nla\_type=RTA\_OIF\}, if\_nametoindex("lo")\}]}\}},
\textcolor{darkgreen}{\{\{len=60, type=RTM\_NEWROUTE, flags=NLM\_F\_MULTI, seq=1357924680, pid=12345\},
 \textcolor{blue}{\{rtm\_family=AF\_INET, rtm\_dst\_len=32, rtm\_src\_len=0, rtm\_tos=0, rtm\_table=RT\_TABLE\_LOCAL, rtm\_protocol=RTPROT\_KERNEL, rtm\_scope=RT\_SCOPE\_HOST, rtm\_type=RTN\_LOCAL, rtm\_flags=0\}},
  \textcolor{darkred}{[\{\{nla\_len=8, nla\_type=RTA\_TABLE\}, RT\_TABLE\_LOCAL\}, \{\{nla\_len=8, nla\_type=RTA\_DST\}, inet\_addr("127.0.0.1")\}, \{\{nla\_len=8, nla\_type=RTA\_PREFSRC\}, inet\_addr("127.0.0.1")\}, \{\{nla\_len=8, nla\_type=RTA\_OIF\}, if\_nametoindex("lo")\}]}\}},
\textcolor{darkgreen}{\{\{len=60, type=RTM\_NEWROUTE, flags=NLM\_F\_MULTI, seq=1357924680, pid=12345\},
 \textcolor{blue}{\{rtm\_family=AF\_INET, rtm\_dst\_len=32, rtm\_src\_len=0, rtm\_tos=0, rtm\_table=RT\_TABLE\_LOCAL, rtm\_protocol=RTPROT\_KERNEL, rtm\_scope=RT\_SCOPE\_LINK, rtm\_type=RTN\_BROADCAST, rtm\_flags=0\}},
  \textcolor{darkred}{[\{\{nla\_len=8, nla\_type=RTA\_TABLE\}, RT\_TABLE\_LOCAL\}, \{\{nla\_len=8, nla\_type=RTA\_DST\}, inet\_addr("127.255.255.255")\}, \{\{nla\_len=8, nla\_type=RTA\_PREFSRC\}, inet\_addr("127.0.0.1")\}, \{\{nla\_len=8, nla\_type=RTA\_OIF\}, if\_nametoindex("lo")\}]}\}}
], iov\_len=32768\}], msg\_iovlen=1, msg\_controllen=0, msg\_flags=0\}, 0) = 240 \\
\ldots
\end{alltt}
\end{block}
\end{frame}
%%%%%%%
\begin{frame}{Copyleft license \hfill [\insertframenumber/\inserttotalframenumber]}
\Large
\begin{block}{\Large 1991 \ldots 2018: permissive license}
strace was released under a Berkeley-style license \\ at the request of Paul Kranenburg.
\end{block}
\begin{block}{\Large Since v4.26$\sim$52 (December 2018): copyleft license}
\begin{itemize}
	\item test suite: GNU General Public License v2+
	\item the rest of strace: GNU Lesser General Public License v2.1+
\end{itemize}
\end{block}
The first major strace feature implemented after this change is PTRACE\_GET\_SYSCALL\_INFO API support.
\end{frame}

%%%%%%%
{
\setbeamertemplate{footline}{}
\begin{frame}[noframenumbering]{Questions?}
	\begin{columns}
		\column{7cm}
\begin{block}{\large homepage}
	https://strace.io
\end{block}
\begin{block}{\large strace.git, strace-talks.git}
	https://github.com/strace/strace.git

	https://gitlab.com/strace/strace.git

\smallskip

	https://github.com/strace/strace-talks.git

	https://gitlab.com/strace/strace-talks.git
\end{block}
\begin{block}{\large mailing list}
	strace-devel@lists.strace.io
\end{block}
\begin{block}{\large IRC channel}
	\#strace@freenode
\end{block}
		\column{3cm}
			\centerline{\pgfuseimage{strace-logo}}
	\end{columns}
\end{frame}
}

\end{document}
