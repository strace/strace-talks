% Copyright (C) 2012-2018 Dmitry V. Levin <ldv@altlinux.org>
% Permission is granted to copy, distribute and/or modify this document
% under the terms of the GNU Free Documentation License, Version 1.2
% or any later version published by the Free Software Foundation;
% with no Invariant Sections, no Front-Cover Texts, and no Back-Cover Texts.

\documentclass[unicode]{beamer}

\mode<presentation>
{
	\usetheme{Warsaw}
	\setbeamertemplate{headline}{}
	\setbeamertemplate{footline}{}
}

\usepackage[utf8]{inputenc}
\usepackage[T2A]{fontenc}
\usepackage[english]{babel}
\usepackage{alltt}

\title{\Huge strace: new features}
\author{\Huge Dmitry~Levin}
\institute[BaseALT]{\Large BaseALT}
\date{DevConf.cz 2018}

%\logo{\includegraphics[height=1cm]{strace-straus.pdf}}
\logo{\includegraphics[height=0.5cm]{basealt.jpg}}

\begin{document}

%%%%%%%
\begin{frame}
\titlepage
\end{frame}

%%%%%%%
\begin{frame}{New features since DevConf.cz 2017}
\begin{block}{\large Released}
\begin{description}
	\item[4.16]: Syscall return value injection
	\item[4.16]: Syscall signal injection
	\item[4.17]: Syscall specification improvements
	\item[4.18, 4.19]: Netlink socket parsers
\end{description}
\end{block}
\begin{block}{\large Being merged}
\begin{itemize}
	\item Syscall delay injection
	\item Advanced syscall filtering syntax
	\item Advanced syscall manipulations with Lua
	\item Advanced syscall information tool
\end{itemize}
\end{block}
\end{frame}

%%%%%%%
\begin{frame}[fragile]{System call tampering and fault injection \hfill [1/5]}
\begin{block}{\large traditional syscall fault injection}
-e fault=\textit{set}[:error=\textit{errno}][:when=\textit{expr}]
\end{block}
\begin{block}{\large strace -a28 -e trace=open \\ {\bf -e fault=}{\it open}{\bf :when=}{\it 3}{\bf :error=}{\sc EACCES} cat /dev/null}
\begin{alltt}
open("/etc/ld.so.cache", O_RDONLY|O_CLOEXEC) = 3
open("/lib64/libc.so.6", O_RDONLY|O_CLOEXEC) = 3
\textcolor{red}{open("/dev/null", O_RDONLY) = -1 EACCES (Permission denied) (INJECTED)}
cat: /dev/null: Permission denied
+++ exited with 1 +++
\end{alltt}
\end{block}
\end{frame}

%%%%%%%
\begin{frame}[fragile]{System call tampering and fault injection \hfill [2/5]}
\begin{block}{\large syscall tampering improvements}
\begin{itemize}
	\item new -e \textbf{inject=} option
	\item return value injection
	\item signal injection
	\item delay injection
\end{itemize}
\end{block}

\begin{block}{\large strace -e \textbf{inject}=\textit{set}[:paramN]...}
Valid inject parameters:
\begin{itemize}
\item error=\textit{errno} or retval=\textit{value}
\item signal=\textit{sig}
\item delay\_enter=\textit{usecs}
\item delay\_exit=\textit{usecs}
\item when=\textit{expr}
\end{itemize}
\end{block}
\end{frame}

%%%%%%%
\begin{frame}[fragile]{System call tampering and fault injection \hfill [3/5]}
\begin{block}{\large syscall return value injection}
strace -e \textbf{inject}=\textit{set}:retval=\textit{value}
\end{block}

\begin{block}{\large strace -e trace=open \\ {\bf -e inject=}{\it open}{\bf :when=}{\it 3}{\bf :retval=}{\sc 42}
cat /dev/null}
\begin{alltt}
open("/etc/ld.so.cache", O_RDONLY|O_CLOEXEC) = 3
open("/lib64/libc.so.6", O_RDONLY|O_CLOEXEC) = 3
\textcolor{red}{open("/dev/null", O_RDONLY)             = 42 (INJECTED)}
cat: /dev/null: Bad file descriptor
cat: /dev/null: Bad file descriptor
+++ exited with 1 +++
\end{alltt}
\end{block}
\end{frame}

%%%%%%%
\begin{frame}[fragile]{System call tampering and fault injection \hfill [4/5]}
\begin{block}{\large syscall signal injection}
strace -e \textbf{inject}=\textit{set}:signal=\textit{sig}
\end{block}

\begin{block}{\large strace -a20 -P precious.txt \\ -e inject=unlink:error=EACCES:signal=ABRT \\ unlink precious.txt}
\begin{alltt}
\textcolor{red}{unlink("precious.txt") = -1 EACCES (Permission denied)
 (INJECTED)}
--- SIGABRT {si_signo=SIGABRT, si_code=SI_KERNEL} ---
+++ killed by SIGABRT (core dumped) +++
\end{alltt}
\end{block}
\end{frame}

%%%%%%%
\begin{frame}[fragile]{System call tampering and fault injection \hfill [5/5]}
\begin{block}{\large syscall delay injection}
strace -e \textbf{inject}=\textit{set}:delay\_enter=\textit{usecs} \\
strace -e \textbf{inject}=\textit{set}:delay\_exit=\textit{usecs}
\end{block}

\begin{block}{dd if=/dev/zero of=/dev/null bs=1M count=10}
\begin{alltt}
10+0 records in
10+0 records out
10485760 bytes (10 MB, 10 MiB) copied, \textcolor{red}{0.00211354} s,
\textcolor{red}{5.0 GB}/s
\end{alltt}
\end{block}

\begin{block}{strace -einject=write:delay\_exit=100000 -ewrite -o/dev/null \\ dd if=/dev/zero of=/dev/null bs=1M count=10}
\begin{alltt}
10+0 records in
10+0 records out
10485760 bytes (10 MB, 10 MiB) copied, \textcolor{red}{1.10658} s,
\textcolor{red}{9.5 MB}/s
\end{alltt}
\end{block}
\end{frame}

%%%%%%%
\begin{frame}{Netlink socket parsers \hfill [1/3]}
\begin{block}{Result of joined efforts}
\begin{description}
\item[GSoC 2016]: Fabien Siron, mentored by Gabriel Laskar
\item[GSoC 2017]: JingPiao Chen
\end{description}
\end{block}
\begin{block}{Currently supported netlink protocols}
\begin{itemize}
\item NETLINK\_AUDIT
\item NETLINK\_CRYPTO
\item NETLINK\_KOBJECT\_UEVENT
\item NETLINK\_NETFILTER
\item NETLINK\_ROUTE
\item NETLINK\_SELINUX
\item NETLINK\_SOCK\_DIAG
\item NETLINK\_XFRM
\item NETLINK\_GENERIC
\end{itemize}
\end{block}
\end{frame}

%%%%%%%
\begin{frame}{Netlink socket parsers: NETLINK\_ROUTE \hfill [2/3]}
\begin{block}{hsh-run $--$ ip route list table all}
\begin{alltt}
broadcast 127.0.0.0 dev lo table local proto kernel scope link src 127.0.0.1

local 127.0.0.0/8 dev lo table local proto kernel scope host src 127.0.0.1

local 127.0.0.1 dev lo table local proto kernel scope host src 127.0.0.1

broadcast 127.255.255.255 dev lo table local proto kernel scope link src 127.0.0.1
\end{alltt}
\end{block}
\end{frame}

%%%%%%%
\begin{frame}[fragile]{\small hsh-run --mount=/proc $--$ strace -e trace=sendto,recvmsg ip route list}

{\texttt\tiny
\fontsize{4pt}{4pt}\selectfont
sendto(3, \{\{len=40, type=RTM\_GETROUTE, flags=NLM\_F\_REQUEST|NLM\_F\_DUMP, seq=1357924680, pid=0\},
 \textcolor{blue}{\{rtm\_family=AF\_UNSPEC, rtm\_dst\_len=0, rtm\_src\_len=0, rtm\_tos=0, rtm\_table=RT\_TABLE\_UNSPEC, rtm\_protocol=RTPROT\_UNSPEC, rtm\_scope=RT\_SCOPE\_UNIVERSE, rtm\_type=RTN\_UNSPEC, rtm\_flags=0\}},
  \textcolor{red}{\{nla\_len=0, nla\_type=RTA\_UNSPEC\}}\}, 40, 0, NULL, 0) = 40

recvmsg(3, \{msg\_name=\{sa\_family=AF\_NETLINK, nl\_pid=0, nl\_groups=00000000\}, msg\_namelen=12, msg\_iov=[\{iov\_base=[
\textcolor{brown}{\{\{len=60, type=RTM\_NEWROUTE, flags=NLM\_F\_MULTI, seq=1357924680, pid=12345\},
 \textcolor{blue}{\{rtm\_family=AF\_INET, rtm\_dst\_len=32, rtm\_src\_len=0, rtm\_tos=0, rtm\_table=RT\_TABLE\_LOCAL, rtm\_protocol=RTPROT\_KERNEL, rtm\_scope=RT\_SCOPE\_LINK, rtm\_type=RTN\_BROADCAST, rtm\_flags=0\}},
  \textcolor{red}{[\{\{nla\_len=8, nla\_type=RTA\_TABLE\}, RT\_TABLE\_LOCAL\}, \{\{nla\_len=8, nla\_type=RTA\_DST\}, 127.0.0.0\}, \{\{nla\_len=8, nla\_type=RTA\_PREFSRC\}, 127.0.0.1\}, \{\{nla\_len=8, nla\_type=RTA\_OIF\}, if\_nametoindex("lo")\}]}\}},
\textcolor{brown}{\{\{len=60, type=RTM\_NEWROUTE, flags=NLM\_F\_MULTI, seq=1357924680, pid=12345\},
 \textcolor{blue}{\{rtm\_family=AF\_INET, rtm\_dst\_len=8, rtm\_src\_len=0, rtm\_tos=0, rtm\_table=RT\_TABLE\_LOCAL, rtm\_protocol=RTPROT\_KERNEL, rtm\_scope=RT\_SCOPE\_HOST, rtm\_type=RTN\_LOCAL, rtm\_flags=0\}},
  \textcolor{red}{[\{\{nla\_len=8, nla\_type=RTA\_TABLE\}, RT\_TABLE\_LOCAL\}, \{\{nla\_len=8, nla\_type=RTA\_DST\}, 127.0.0.0\}, \{\{nla\_len=8, nla\_type=RTA\_PREFSRC\}, 127.0.0.1\}, \{\{nla\_len=8, nla\_type=RTA\_OIF\}, if\_nametoindex("lo")\}]}\}},
\textcolor{brown}{\{\{len=60, type=RTM\_NEWROUTE, flags=NLM\_F\_MULTI, seq=1357924680, pid=12345\},
 \textcolor{blue}{\{rtm\_family=AF\_INET, rtm\_dst\_len=32, rtm\_src\_len=0, rtm\_tos=0, rtm\_table=RT\_TABLE\_LOCAL, rtm\_protocol=RTPROT\_KERNEL, rtm\_scope=RT\_SCOPE\_HOST, rtm\_type=RTN\_LOCAL, rtm\_flags=0\}},
  \textcolor{red}{[\{\{nla\_len=8, nla\_type=RTA\_TABLE\}, RT\_TABLE\_LOCAL\}, \{\{nla\_len=8, nla\_type=RTA\_DST\}, 127.0.0.1\}, \{\{nla\_len=8, nla\_type=RTA\_PREFSRC\}, 127.0.0.1\}, \{\{nla\_len=8, nla\_type=RTA\_OIF\}, if\_nametoindex("lo")\}]}\}},
\textcolor{brown}{\{\{len=60, type=RTM\_NEWROUTE, flags=NLM\_F\_MULTI, seq=1357924680, pid=12345\},
 \textcolor{blue}{\{rtm\_family=AF\_INET, rtm\_dst\_len=32, rtm\_src\_len=0, rtm\_tos=0, rtm\_table=RT\_TABLE\_LOCAL, rtm\_protocol=RTPROT\_KERNEL, rtm\_scope=RT\_SCOPE\_LINK, rtm\_type=RTN\_BROADCAST, rtm\_flags=0\}},
  \textcolor{red}{[\{\{nla\_len=8, nla\_type=RTA\_TABLE\}, RT\_TABLE\_LOCAL\}, \{\{nla\_len=8, nla\_type=RTA\_DST\}, 127.255.255.255\}, \{\{nla\_len=8, nla\_type=RTA\_PREFSRC\}, 127.0.0.1\}, \{\{nla\_len=8, nla\_type=RTA\_OIF\}, if\_nametoindex("lo")\}]}\}}
], iov\_len=32768\}], msg\_iovlen=1, msg\_controllen=0, msg\_flags=0\}, 0) = 240 \\
...
%broadcast 127.0.0.0 dev lo table local proto kernel scope link src 127.0.0.1  \\
%local 127.0.0.0/8 dev lo table local proto kernel scope host src 127.0.0.1 \\
%local 127.0.0.1 dev lo table local proto kernel scope host src 127.0.0.1 \\
%broadcast 127.255.255.255 dev lo table local proto kernel scope link src 127.0.0.1 \\
%recvmsg(3, \{msg\_name=\{sa\_family=AF\_NETLINK, nl\_pid=0, nl\_groups=00000000\}, msg\_namelen=12, msg\_iov=[\{iov\_base=\{\{len=20, type=NLMSG\_DONE, flags=NLM\_F\_MULTI, seq=1505349241, pid=12345\}, 0\}, iov\_len=32768\}], msg\_iovlen=1, msg\_controllen=0, msg\_flags=0\}, 0) = 20
}
\end{frame}

%%%%%%%
\begin{frame}[fragile]{System call specification improvements \hfill [1/6]}
\begin{block}{\large syscall classes now have \% prefix}
strace -e trace=\%\textit{class}
\end{block}
\begin{block}{\large traditional syscall classes}
\begin{description}
	\item[desc]: take or return a descriptor
	\item[file]: take a file name
	\item[memory]: memory mapping, memory policy
	\item[process]: process management
	\item[signal]: signal related
	\item[ipc]: SysV IPC related
	\item[network]: network related
\end{description}
\end{block}
\end{frame}

%%%%%%%
\begin{frame}[fragile]{System call specification improvements \hfill [2/6]}
\begin{block}{\large new syscall classes}
\begin{itemize}
	\item \%stat, \%lstat, \%fstat
	\item \%statfs, \%fstatfs
	\item \%\%stat, \%\%statfs
\end{itemize}
\end{block}
\begin{block}{\large strace -y -e \%\%stat ls /var/empty}
\begin{alltt}
\textcolor{red}{fstat}(3</etc/ld.so.cache>, {st_mode=S_IFREG|0644, st_size=30341, ...}) = 0
...
\textcolor{red}{fstat}(3</proc/filesystems>, {st_mode=S_IFREG|0444, st_size=0, ...}) = 0
\textcolor{red}{stat}("/var/empty", {st_mode=S_IFDIR|0555, st_size=40, ...}) = 0
\textcolor{red}{fstat}(3</var/empty>, {st_mode=S_IFDIR|0555, st_size=40, ...}) = 0
+++ exited with 0 +++
\end{alltt}
\end{block}
\end{frame}

%%%%%%%
\begin{frame}[fragile]{System call specification improvements \hfill [3/6]}
\begin{block}{\large \%fstat syscall class}
\begin{itemize}
	\item fstat, fstat64
	\item fstatat64, newfstatat
	\item oldfstat
\end{itemize}
\end{block}
\begin{block}{\large strace -y -e \%fstat find /var/empty -mindepth 1}
\begin{alltt}
\textcolor{red}{fstat}(3</etc/ld.so.cache>, {st_mode=S_IFREG|0644, ...}) = 0
...
\textcolor{red}{fstat}(3</lib64/libdl-2.26.so>, {st_mode=S_IFREG|0644, ...}) = 0
\textcolor{red}{fstat}(3</proc/filesystems>, {st_mode=S_IFREG|0444, ...}) = 0
\textcolor{red}{newfstatat}(AT_FDCWD, "/var/empty", {st_mode=S_IFDIR|0555, ...}, AT_SYMLINK_NOFOLLOW) = 0
\textcolor{red}{fstat}(4</var/empty>, {st_mode=S_IFDIR|0555, ...}) = 0
\textcolor{red}{newfstatat}(AT_FDCWD, "/var/empty", {st_mode=S_IFDIR|0555, ...}, AT_SYMLINK_NOFOLLOW) = 0
\end{alltt}
\end{block}
\end{frame}

%%%%%%%
\begin{frame}[fragile]{System call specification improvements \hfill [4/6]}
\begin{block}{\large added support of regular expressions}
strace -e trace=/\textit{regexp}
\end{block}

\textit{regexp} is an extended regular expression

\begin{block}{\large strace -e '/fstat(at)?(64)?\$' find /var/empty -mindepth 1}
\begin{alltt}
\textcolor{red}{fstat}(3</etc/ld.so.cache>, {st_mode=S_IFREG|0644, ...}) = 0
...
\textcolor{red}{fstat}(3</lib64/libdl-2.26.so>, {st_mode=S_IFREG|0644, ...}) = 0
\textcolor{red}{fstat}(3</proc/filesystems>, {st_mode=S_IFREG|0444, ...}) = 0
\textcolor{red}{newfstatat}(AT_FDCWD, "/var/empty", {st_mode=S_IFDIR|0555, ...}, AT_SYMLINK_NOFOLLOW) = 0
\textcolor{red}{fstat}(4</var/empty>, {st_mode=S_IFDIR|0555, ...}) = 0
\textcolor{red}{newfstatat}(AT_FDCWD, "/var/empty", {st_mode=S_IFDIR|0555, ...}, AT_SYMLINK_NOFOLLOW) = 0
\end{alltt}
\end{block}
\end{frame}

%%%%%%%
\begin{frame}[fragile]{System call specification improvements \hfill [5/6]}
\begin{block}{\large added support of conditional descriptions}
strace -e trace=?\textit{set}
\end{block}
\begin{block}{\large strace -e trace=?statx ./tests/statx}
\begin{alltt}
statx(AT_FDCWD, "/dev/full", AT_STATX_SYNC_AS_STAT,
STATX_ALL, {stx_mask=STATX_BASIC_STATS, stx_attributes=0,
stx_mode=S_IFCHR|0666, stx_size=0, ...}) = 0
\end{alltt}
\end{block}
\begin{block}{\large strace -e trace=/statx might not work}
\begin{alltt}
strace: invalid system call '/statx'
\end{alltt}
\end{block}
\end{frame}

%%%%%%%
\begin{frame}[fragile]{System call specification improvements \hfill [6/6]}
\begin{block}{glibc: open vs openat}
\begin{alltt}
glibc-2.25$ strace -qq -e open cat /dev/null
open("/etc/ld.so.cache", O_RDONLY|O_CLOEXEC) = 3
open("/lib64/libc.so.6", O_RDONLY|O_CLOEXEC) = 3
open("/dev/null", O_RDONLY)             = 3
\textcolor{red}{glibc-2.26$ strace -qq -e open cat /dev/null}
glibc-2.26$ strace -qq -e openat cat /dev/null
openat(AT_FDCWD, "/etc/ld.so.cache", O_RDONLY|O_CLOEXEC) = 3
openat(AT_FDCWD, "/lib64/libc.so.6", O_RDONLY|O_CLOEXEC) = 3
openat(AT_FDCWD, "/dev/null", O_RDONLY) = 3
\end{alltt}
\end{block}

strace -e \textbf{open},\textbf{openat} is not portable

\begin{block}{portable syscall specifications}
\begin{description}
	\item[regexp]: /{\textasciicircum}open(at)?\$
	\item[condition]: ?open,openat
\end{description}
\end{block}
\end{frame}

%%%%%%%
\begin{frame}[fragile]{Advanced syscall filtering syntax \hfill [1/4]}
\begin{block}{new syntax}
[\textbf{\textit{action(}}]\textbf{\textit{filter\_expression}}[;\textbf{\textit{arg1}}[;\textbf{\textit{arg2}}...]][)]
\begin{description}
\item[\textit{action}] is one of \textbf{trace}, \textbf{abbrev}, \textbf{verbose},
\textbf{raw}, \textbf{read}, \textbf{write}, \textbf{fault},
\textbf{inject}, or \textbf{stacktrace};
\item[\textit{argN}] are arguments of \textbf{\textit{action}};
\item[\textit{filter\_expression}] is a combination of filters.
\end{description}
\end{block}
\begin{block}{supported filters}
\begin{description}
\item[syscall \textit{set}]: set of syscalls described by \textbf{\textit{set}};
\item[fd \textit{fd1}\dots]: set of syscalls operating with descriptor numbers
described by \textbf{\textit{fd1}}\dots;
\item[path \textit{path}]: set of syscalls operating with paths
described by \textbf{\textit{path}}.
\end{description}
\end{block}
\end{frame}

%%%%%%%
\begin{frame}[fragile]{Advanced syscall filtering syntax \hfill [2/4]}
\begin{block}{\large echo -n foo | strace -e 'trace(fd 1)' cat >/dev/null}
\begin{alltt}
fstat(1, {st_mode=S_IFCHR|0666, st_rdev=makedev(1, 3), ...}) = 0
write(1, "foo", 3)                      = 3
close(1)                                = 0
+++ exited with 0 +++
\end{alltt}
\end{block}
\begin{block}{\large strace -y -s4 -e 'trace(syscall read)' -e 'read(path /dev/zero)' \\ head -c5 /dev/zero}
\begin{verbatim}
read(3</lib64/libc-2.26.so>, "\177ELF"..., 832) = 832
read(3</dev/zero>, "\0\0\0\0"..., 5)    = 5
 | 00000  00 00 00 00 00            .....            |
+++ exited with 0 +++
\end{verbatim}
\end{block}
\end{frame}

%%%%%%%
\begin{frame}[fragile]{Advanced syscall filtering syntax \hfill [3/4]}
\begin{block}{\large strace -ve 'syscall \%file and not syscall \%desc' cat /dev/null}
\begin{alltt}
execve("/usr/bin/cat", ["/usr/bin/cat", "/dev/null"], []) = 0
access("/etc/ld.so.preload", R_OK) = -1 ENOENT (No such file or directory)
+++ exited with 0 +++
\end{alltt}
\end{block}

\begin{block}{\large strace -ve 'syscall \%file and \\ !(syscall \%desc || path /usr/bin/cat)' \\ /usr/bin/cat /dev/null}
\begin{alltt}
access("/etc/ld.so.preload", R_OK) = -1 ENOENT (No such file or directory)
+++ exited with 0 +++
\end{alltt}
\end{block}
\end{frame}

%%%%%%%
\begin{frame}[fragile]{Advanced syscall filtering syntax \hfill [4/4]}
\begin{block}{\large strace -e 'fd 1' -e 'stacktrace(syscall close)' cat /dev/null}
\begin{alltt}
fstat(1, {st_mode=S_IFCHR|0620, st_rdev=makedev(136, 1), ...}) = 0
close(1)                                = 0
 > /lib64/libc-2.26.so(_IO_file_close+0xb) [0x77a2b]
 > /lib64/libc-2.26.so(_IO_file_close_it+0x13c) [0x791fc]
 > /lib64/libc-2.26.so(fclose+0x1bf) [0x6c4df]
 > /bin/cat() [0x4cea]
 > /bin/cat() [0x2692]
 > /lib64/libc-2.26.so(__locale_getenv+0x140) [0x37680]
 > /lib64/libc-2.26.so(exit+0x1a) [0x376da]
 > /lib64/libc-2.26.so(__libc_start_main+0xf8) [0x21128]
 > /bin/cat() [0x217a]
+++ exited with 0 +++
\end{alltt}
\end{block}
\end{frame}

%%%%%%%
{
\setbeamertemplate{logo}{}
\begin{frame}{Questions?}
	\begin{columns}
		\column{7cm}
\begin{block}{\large homepage}
	https://strace.io
\end{block}
\begin{block}{\large strace.git}
	https://github.com/strace/strace.git

	https://gitlab.com/strace/strace.git

	git://git.code.sf.net/p/strace/code.git
\end{block}
\begin{block}{\large mailing list}
	strace-devel@lists.sourceforge.net
\end{block}
\begin{block}{\large IRC channel}
	\#strace@freenode
\end{block}
		\column{3cm}
			\centerline{\includegraphics[height=7.2cm]{strace-straus.pdf}}
	\end{columns}
\end{frame}
}

\end{document}
