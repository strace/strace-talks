% Copyright (C) 2012-2016  Dmitry V. Levin <ldv@altlinux.org>
% Permission is granted to copy, distribute and/or modify this document
% under the terms of the GNU Free Documentation License, Version 1.2
% or any later version published by the Free Software Foundation;
% with no Invariant Sections, no Front-Cover Texts, and no Back-Cover Texts.

\documentclass[unicode,aspectratio=169]{beamer}

\mode<presentation>
{
	\usetheme{Warsaw}
	\setbeamertemplate{headline}{}
	\setbeamertemplate{footline}{}
}

\usepackage[utf8]{inputenc}
\usepackage[T2A]{fontenc}
\usepackage[english]{babel}

\title{\Huge Can strace make you fail?}
\subtitle{\huge strace syscall fault injection}
\author{\Huge Dmitry~Levin}
\date{OSSDEVCONF 2016}

\logo{\includegraphics[height=0.5cm]{basealt.jpg}}

\begin{document}

\begin{frame}
\titlepage
\end{frame}

\begin{frame}[fragile]{What is strace?}
\begin{itemize}
	\item A diagnostic, debugging and instructional userspace
	utility for Linux.

	\item It is used to monitor interactions between processes and
	the Linux kernel, which include system calls, signal
	deliveries, and changes of process state.

	\item CLI and multiple filtering capabilities make it a powerful
	yet easy to use tracing tool.

	\item This year strace has been extended to tamper with tracees
	using syscall fault injection.
\end{itemize}
\scriptsize
\begin{block}{\large sample output}
\begin{verbatim}
$ strace -s 0 -P /dev/urandom python3 < /dev/null
open("/dev/urandom", O_RDONLY|O_CLOEXEC) = 3
fcntl(3, F_GETFD)                       = 0x1 (flags FD_CLOEXEC)
read(3, ""..., 24)                      = 24
close(3)                                = 0
+++ exited with 0 +++
\end{verbatim}
\end{block}
\end{frame}

\begin{frame}{strace workflow: entering syscall}
\begin{block}{\large tracee}
	invokes a syscall
\end{block}
\begin{block}{\large kernel}
	puts the tracee into {\sc syscall-enter-stop} state
\end{block}
\begin{block}{\large strace}
\begin{itemize}
	\item awakens
	\item fetches the syscall number and arguments
	\item applies filters
	\item may print something
	\item tells the kernel to let the tracee go on
\end{itemize}
\end{block}
\end{frame}

\begin{frame}{strace workflow: exiting syscall}
\begin{block}{\large kernel}
\begin{itemize}
	\item executes the syscall
	\item puts the tracee into {\sc syscall-exit-stop} state
\end{itemize}
\end{block}
\begin{block}{\large strace}
\begin{itemize}
	\item awakens
	\item may fetch the syscall return code and some arguments
	\item may print something
	\item tells the kernel to let the tracee go on
\end{itemize}
\end{block}
\end{frame}

\begin{frame}{strace workflow: tampering with syscalls}
\begin{itemize}
	\item awakens
	\item fetches the syscall number and arguments
	\item applies filters
	\item \textcolor{red}{\bf may tamper with the syscall number and arguments}
	\item may print something
	\item tells the kernel to let the tracee go on
	\item {\it the kernel executes the syscall}
	\item awakens
	\item may fetch the syscall return code and some arguments
	\item \textcolor{red}{\bf may tamper with the syscall return code and arguments}
	\item may print something
	\item tells the kernel to let the tracee go on
\end{itemize}
\end{frame}

\begin{frame}{strace as a fault injection tool}
\begin{block}{\large What is fault injection?}
	\begin{itemize}
	\item A software testing technique used for improving test coverage
	\item of error handling code paths that might otherwise rarely be followed
	\item by introducing faults
	\end{itemize}
\end{block}
\begin{block}{\large Where to place strace among other fault injection tools?}
	\begin{itemize}
		\item software implemented fault injection
		\item runtime fault injection
		\item syscall interposition
		\item unprivileged userspace
		\item command-line interface
	\end{itemize}
	Unprivileged userspace command-line runtime syscall fault injection tool
\end{block}
\end{frame}

\begin{frame}[fragile]{strace syscall fault injection syntax}
\scriptsize
\begin{block}{\large strace -e trace=open cat /dev/null}
\begin{verbatim}
open("/etc/ld.so.cache", O_RDONLY|O_CLOEXEC) = 3
open("/lib64/libc.so.6", O_RDONLY|O_CLOEXEC) = 3
open("/dev/null", O_RDONLY)             = 3
+++ exited with 0 +++
\end{verbatim}
\end{block}
\begin{block}{\large strace -e trace=open {\bf -e fault=}{\it open} cat}
\begin{verbatim}
open("/etc/ld.so.cache", O_RDONLY|O_CLOEXEC) = -1 ENOSYS (Function not implemented) (INJECTED)
open("/lib64/tls/x86_64/libc.so.6", O_RDONLY|O_CLOEXEC) = -1 ENOSYS (Function not implemented) (INJECTED)
open("/lib64/tls/libc.so.6", O_RDONLY|O_CLOEXEC) = -1 ENOSYS (Function not implemented) (INJECTED)
open("/lib64/x86_64/libc.so.6", O_RDONLY|O_CLOEXEC) = -1 ENOSYS (Function not implemented) (INJECTED)
open("/lib64/libc.so.6", O_RDONLY|O_CLOEXEC) = -1 ENOSYS (Function not implemented) (INJECTED)
cat: error while loading shared libraries: libc.so.6: cannot open shared object file: Error 38
+++ exited with 127 +++
\end{verbatim}
\end{block}
\end{frame}

\begin{frame}[fragile]{strace syscall fault injection syntax}
\scriptsize
\begin{block}{\large strace -e trace=open cat /dev/null}
\begin{verbatim}
open("/etc/ld.so.cache", O_RDONLY|O_CLOEXEC) = 3
open("/lib64/libc.so.6", O_RDONLY|O_CLOEXEC) = 3
open("/dev/null", O_RDONLY)             = 3
+++ exited with 0 +++
\end{verbatim}
\end{block}
\begin{block}{\large strace -e trace=open {\bf -e fault=}{\it open}{\bf :error=}{\sc ENOENT} cat /dev/null}
\begin{verbatim}
open("/etc/ld.so.cache", O_RDONLY|O_CLOEXEC) = -1 ENOENT (No such file or directory) (INJECTED)
open("/lib64/tls/x86_64/libc.so.6", O_RDONLY|O_CLOEXEC) = -1 ENOENT (No such file or directory) (INJECTED)
open("/lib64/tls/libc.so.6", O_RDONLY|O_CLOEXEC) = -1 ENOENT (No such file or directory) (INJECTED)
open("/lib64/x86_64/libc.so.6", O_RDONLY|O_CLOEXEC) = -1 ENOENT (No such file or directory) (INJECTED)
open("/lib64/libc.so.6", O_RDONLY|O_CLOEXEC) = -1 ENOENT (No such file or directory) (INJECTED)
open("/usr/lib64/tls/x86_64/libc.so.6", O_RDONLY|O_CLOEXEC) = -1 ENOENT (No such file or directory) (INJECTED)
open("/usr/lib64/tls/libc.so.6", O_RDONLY|O_CLOEXEC) = -1 ENOENT (No such file or directory) (INJECTED)
open("/usr/lib64/x86_64/libc.so.6", O_RDONLY|O_CLOEXEC) = -1 ENOENT (No such file or directory) (INJECTED)
open("/usr/lib64/libc.so.6", O_RDONLY|O_CLOEXEC) = -1 ENOENT (No such file or directory) (INJECTED)
cat: error while loading shared libraries: libc.so.6: cannot open shared object file: No such file or directory
+++ exited with 127 +++
\end{verbatim}
\end{block}
\end{frame}

\begin{frame}[fragile]{strace syscall fault injection syntax}
\scriptsize
\begin{block}{\large strace -e trace=open cat /dev/null}
\begin{verbatim}
open("/etc/ld.so.cache", O_RDONLY|O_CLOEXEC) = 3
open("/lib64/libc.so.6", O_RDONLY|O_CLOEXEC) = 3
open("/dev/null", O_RDONLY)             = 3
+++ exited with 0 +++
\end{verbatim}
\end{block}
\begin{block}{\large strace -e trace=open {\bf -e fault=}{\it open}{\bf :when=}{\it 1}{\bf :error=}{\sc EACCES} cat /dev/null}
\begin{verbatim}
open("/etc/ld.so.cache", O_RDONLY|O_CLOEXEC) = -1 EACCES (Permission denied) (INJECTED)
open("/lib64/tls/x86_64/libc.so.6", O_RDONLY|O_CLOEXEC) = -1 ENOENT (No such file or directory)
open("/lib64/tls/libc.so.6", O_RDONLY|O_CLOEXEC) = -1 ENOENT (No such file or directory)
open("/lib64/x86_64/libc.so.6", O_RDONLY|O_CLOEXEC) = -1 ENOENT (No such file or directory)
open("/lib64/libc.so.6", O_RDONLY|O_CLOEXEC) = 3
open("/dev/null", O_RDONLY)             = 3
+++ exited with 0 +++
\end{verbatim}
\end{block}
\end{frame}

\begin{frame}[fragile]{strace syscall fault injection syntax}
\scriptsize
\begin{block}{\large strace -e trace=open cat /dev/null}
\begin{verbatim}
open("/etc/ld.so.cache", O_RDONLY|O_CLOEXEC) = 3
open("/lib64/libc.so.6", O_RDONLY|O_CLOEXEC) = 3
open("/dev/null", O_RDONLY)             = 3
+++ exited with 0 +++
\end{verbatim}
\end{block}
\begin{block}{\large strace -e trace=open {\bf -e fault=}{\it open}{\bf :when=}{\it 2}{\bf +:error=}{\sc EPERM} cat /dev/null}
\begin{verbatim}
open("/etc/ld.so.cache", O_RDONLY|O_CLOEXEC) = 3
open("/lib64/libc.so.6", O_RDONLY|O_CLOEXEC) = -1 EPERM (Operation not permitted) (INJECTED)
open("/lib64/tls/x86_64/libc.so.6", O_RDONLY|O_CLOEXEC) = -1 EPERM (Operation not permitted) (INJECTED)
open("/lib64/tls/libc.so.6", O_RDONLY|O_CLOEXEC) = -1 EPERM (Operation not permitted) (INJECTED)
open("/lib64/x86_64/libc.so.6", O_RDONLY|O_CLOEXEC) = -1 EPERM (Operation not permitted) (INJECTED)
open("/lib64/libc.so.6", O_RDONLY|O_CLOEXEC) = -1 EPERM (Operation not permitted) (INJECTED)
cat: error while loading shared libraries: libc.so.6: cannot open shared object file: Operation not permitted
+++ exited with 127 +++
\end{verbatim}
\end{block}
\end{frame}

\begin{frame}[fragile]{strace syscall fault injection syntax}
\scriptsize
\begin{block}{\large strace -e trace=open cat /dev/null}
\begin{verbatim}
open("/etc/ld.so.cache", O_RDONLY|O_CLOEXEC) = 3
open("/lib64/libc.so.6", O_RDONLY|O_CLOEXEC) = 3
open("/dev/null", O_RDONLY)             = 3
+++ exited with 0 +++
\end{verbatim}
\end{block}
\begin{block}{\large strace -e trace=open {\bf -e fault=}{\it open}{\bf :when=}{\it 3}{\bf :error=}{\sc EACCES} cat /dev/null}
\begin{verbatim}
open("/etc/ld.so.cache", O_RDONLY|O_CLOEXEC) = 3
open("/lib64/libc.so.6", O_RDONLY|O_CLOEXEC) = 3
open("/dev/null", O_RDONLY)             = -1 EACCES (Permission denied) (INJECTED)
cat: /dev/null: Permission denied
+++ exited with 1 +++
\end{verbatim}
\end{block}
\end{frame}

\begin{frame}[fragile]{strace syscall fault injection syntax}
\scriptsize
\begin{block}{strace -e trace=open cat /dev/null\{,\}\{,\}}
\begin{verbatim}
$ C strace -eopen cat /dev/null{,}{,}
open("/etc/ld.so.cache", O_RDONLY|O_CLOEXEC) = 3
open("/lib64/libc.so.6", O_RDONLY|O_CLOEXEC) = 3
open("/dev/null", O_RDONLY)             = 3
open("/dev/null", O_RDONLY)             = 3
open("/dev/null", O_RDONLY)             = 3
open("/dev/null", O_RDONLY)             = 3
+++ exited with 0 +++
\end{verbatim}
\end{block}
\begin{block}{strace -e trace=open {\bf -e fault=}{\it open}{\bf :when=}{\it 1}{\bf +}{\it 3}{\bf :error=}{\sc EACCES} cat /dev/null\{,\}\{,\}}
\begin{verbatim}
open("/etc/ld.so.cache", O_RDONLY|O_CLOEXEC) = 3
open("/lib64/libc.so.6", O_RDONLY|O_CLOEXEC) = 3
open("/dev/null", O_RDONLY)             = -1 EACCES (Permission denied) (INJECTED)
cat: /dev/null: Permission denied
open("/dev/null", O_RDONLY)             = 3
open("/dev/null", O_RDONLY)             = 3
open("/dev/null", O_RDONLY)             = -1 EACCES (Permission denied) (INJECTED)
cat: /dev/null: Permission denied
+++ exited with 1 +++
\end{verbatim}
\end{block}
\end{frame}

\begin{frame}[fragile]{strace syscall fault injection syntax}
\scriptsize
\begin{block}{\large strace -P /dev/null cat /dev/null}
\begin{verbatim}
$ strace -P /dev/null cat /dev/null
open("/dev/null", O_RDONLY)             = 3
fstat(3, {st_mode=S_IFCHR|0666, st_rdev=makedev(1, 3), ...}) = 0
fadvise64(3, 0, 0, POSIX_FADV_SEQUENTIAL) = 0
read(3, "", 131072)                     = 0
close(3)                                = 0
+++ exited with 0 +++
\end{verbatim}
\end{block}
\begin{block}{\large strace -P /dev/null {\bf -e fault=}{\it all} cat /dev/null}
\begin{verbatim}
open("/dev/null", O_RDONLY)             = -1 ENOSYS (Function not implemented) (INJECTED)
cat: /dev/null: Function not implemented
+++ exited with 1 +++
\end{verbatim}
\end{block}
\end{frame}

\begin{frame}[fragile]{strace syscall fault injection syntax}
\scriptsize
\begin{block}{\large strace -P /dev/null cat /dev/null}
\begin{verbatim}
$ strace -P /dev/null cat /dev/null
open("/dev/null", O_RDONLY)             = 3
fstat(3, {st_mode=S_IFCHR|0666, st_rdev=makedev(1, 3), ...}) = 0
fadvise64(3, 0, 0, POSIX_FADV_SEQUENTIAL) = 0
read(3, "", 131072)                     = 0
close(3)                                = 0
+++ exited with 0 +++
\end{verbatim}
\end{block}
\begin{block}{\large strace -P /dev/null {\bf -e fault=}{\it fstat}{\bf :error=}{\sc ENOMEM} cat /dev/null}
\begin{verbatim}
open("/dev/null", O_RDONLY)             = 3
fstat(3, 0x7ffd970bb080)                = -1 ENOMEM (Cannot allocate memory) (INJECTED)
cat: /dev/null: Cannot allocate memory
close(3)                                = 0
+++ exited with 1 +++
\end{verbatim}
\end{block}
\end{frame}

\begin{frame}[fragile]{strace syscall fault injection syntax}
\scriptsize
\begin{block}{\large strace -P /dev/null cat /dev/null}
\begin{verbatim}
$ strace -P /dev/null cat /dev/null
open("/dev/null", O_RDONLY)             = 3
fstat(3, {st_mode=S_IFCHR|0666, st_rdev=makedev(1, 3), ...}) = 0
fadvise64(3, 0, 0, POSIX_FADV_SEQUENTIAL) = 0
read(3, "", 131072)                     = 0
close(3)                                = 0
+++ exited with 0 +++
\end{verbatim}
\end{block}
\begin{block}{\large strace -P /dev/null {\bf -e fault=}{\it fadvise64} cat /dev/null}
\begin{verbatim}
open("/dev/null", O_RDONLY)             = 3
fstat(3, {st_mode=S_IFCHR|0666, st_rdev=makedev(1, 3), ...}) = 0
fadvise64(3, 0, 0, POSIX_FADV_SEQUENTIAL) = -1 ENOSYS (Function not implemented) (INJECTED)
read(3, "", 131072)                     = 0
close(3)                                = 0
+++ exited with 0 +++
\end{verbatim}
\end{block}
\end{frame}

\begin{frame}[fragile]{strace syscall fault injection syntax}
\scriptsize
\begin{block}{\large strace -P /dev/null cat /dev/null}
\begin{verbatim}
$ strace -P /dev/null cat /dev/null
open("/dev/null", O_RDONLY)             = 3
fstat(3, {st_mode=S_IFCHR|0666, st_rdev=makedev(1, 3), ...}) = 0
fadvise64(3, 0, 0, POSIX_FADV_SEQUENTIAL) = 0
read(3, "", 131072)                     = 0
close(3)                                = 0
+++ exited with 0 +++
\end{verbatim}
\end{block}
\begin{block}{\large strace -P /dev/null {\bf -e fault=}{\it read}{\bf :when=}{\it 1}{\bf :error=}{\sc EINTR} cat /dev/null}
\begin{verbatim}
open("/dev/null", O_RDONLY)             = 3
fstat(3, {st_mode=S_IFCHR|0666, st_rdev=makedev(1, 3), ...}) = 0
fadvise64(3, 0, 0, POSIX_FADV_SEQUENTIAL) = 0
read(3, 0x7f057432c000, 131072)         = -1 EINTR (Interrupted system call) (INJECTED)
read(3, "", 131072)                     = 0
close(3)                                = 0
+++ exited with 0 +++
\end{verbatim}
\end{block}
\end{frame}

\begin{frame}[fragile]{Examples: strace syscall fault injection: python3}
\begin{block}{\large Error opening /dev/urandom}
\scriptsize
\begin{verbatim}
$ strace -P /dev/urandom -e fault=open:error=ENOENT python3 < /dev/null
open("/dev/urandom", O_RDONLY|O_CLOEXEC) = -1 ENOENT (No such file or directory) (INJECTED)
Fatal Python error: Failed to open /dev/urandom
--- SIGSEGV {si_signo=SIGSEGV, si_code=SEGV_MAPERR, si_addr=0x50} ---
+++ killed by SIGSEGV +++
Segmentation fault
\end{verbatim}
\end{block}
\begin{block}{\large Error reading /dev/urandom}
\scriptsize
\begin{verbatim}
$ strace -P /dev/urandom -e fault=read:error=EIO python3 < /dev/null
open("/dev/urandom", O_RDONLY|O_CLOEXEC) = 3
fcntl(3, F_GETFD)                       = 0x1 (flags FD_CLOEXEC)
read(3, 0x8db610, 24)                   = -1 EIO (Input/output error) (INJECTED)
Fatal Python error: Failed to read bytes from /dev/urandom
--- SIGSEGV {si_signo=SIGSEGV, si_code=SEGV_MAPERR, si_addr=0x50} ---
+++ killed by SIGSEGV +++
Segmentation fault
\end{verbatim}
\end{block}
\end{frame}

\begin{frame}[fragile]{Examples: strace syscall fault injection: dynamic linker}
\begin{block}{\large First mprotect invocation: without a proper check}
\scriptsize
\begin{verbatim}
$ strace -e mprotect -e fault=mprotect:when=1:error=EPERM pwd > /dev/null
mprotect(0x7fabcd00f000, 2097152, PROT_NONE) = -1 EPERM (Operation not permitted) (INJECTED)
mprotect(0x7fabcd20f000, 16384, PROT_READ) = 0
mprotect(0x606000, 4096, PROT_READ)     = 0
mprotect(0x7fabcd441000, 4096, PROT_READ) = 0
+++ exited with 0 +++
\end{verbatim}
\end{block}
\begin{block}{\large Second mprotect invocation: with a proper check}
\scriptsize
\begin{verbatim}
$ strace -e mprotect -e fault=mprotect:when=2+:error=EPERM pwd > /dev/null
mprotect(0x7fabcd00f000, 2097152, PROT_NONE) = 0
mprotect(0x7fabcd20f000, 16384, PROT_READ) = -1 EPERM (Operation not permitted) (INJECTED)
pwd: error while loading shared libraries: /lib64/libc.so.6: cannot apply additional memory
protection after relocation: Operation not permitted
+++ exited with 127 +++
\end{verbatim}
\end{block}
\end{frame}

\begin{frame}[fragile]{Examples: strace access control: access by path argument}
\scriptsize
\begin{block}{\large strace -P /etc/passwd -e fault=desc:error=EPERM cat /etc/passwd}
\begin{verbatim}
open("/etc/passwd", O_RDONLY)           = -1 EPERM (Operation not permitted) (INJECTED)
cat: /etc/passwd: Operation not permitted
+++ exited with 1 +++
\end{verbatim}
\end{block}
\begin{block}{\large strace -P /etc/passwd -e fault=all:error=EACCES,desc:error=EPERM stat /etc/passwd}
\begin{verbatim}
lstat("/etc/passwd", 0x7ffd282659d0)    = -1 EACCES (Permission denied) (INJECTED)
stat: cannot stat '/etc/passwd': Permission denied
+++ exited with 1 +++
\end{verbatim}
\end{block}
\end{frame}

\begin{frame}[fragile]{Examples: strace access control: access by descriptor argument}
\scriptsize
\begin{block}{\large strace -P /etc/passwd -e fault=desc:error=EPERM cat < /etc/passwd}
\begin{verbatim}
fstat(0, 0x7ffd6c14daa0)                = -1 EPERM (Operation not permitted) (INJECTED)
cat: -: Operation not permitted
close(0)                                = -1 EPERM (Operation not permitted) (INJECTED)
cat: closing standard input: Operation not permitted
+++ exited with 1 +++
\end{verbatim}
\end{block}
\begin{block}{\large strace -P /etc/passwd -e fault=all:error=EPERM,desc:error=EACCES stat - </etc/passwd}
\begin{verbatim}
fstat(0, 0x7ffdb068e650)                = -1 EACCES (Permission denied) (INJECTED)
stat: cannot stat standard input: Permission denied
+++ exited with 1 +++
\end{verbatim}
\end{block}
\end{frame}

\begin{frame}{Questions?}
\begin{block}{\large homepage}
	https://sourceforge.net/projects/strace/

	http://strace.io
\end{block}
\begin{block}{\large strace.git}
	git://git.code.sf.net/p/strace/code.git

	https://github.com/strace/strace.git
\end{block}
\begin{block}{\large mailing list}
	strace-devel@lists.sourceforge.net
\end{block}
\end{frame}

\end{document}
