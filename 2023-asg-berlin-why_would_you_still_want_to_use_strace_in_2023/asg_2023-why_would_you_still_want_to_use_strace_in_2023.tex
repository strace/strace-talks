% Copyright (C) 2012-2023 Dmitry V. Levin <ldv@strace.io>
% Permission is granted to copy, distribute and/or modify this document
% under the terms of the GNU Free Documentation License, Version 1.2
% or any later version published by the Free Software Foundation;
% with no Invariant Sections, no Front-Cover Texts, and no Back-Cover Texts.

\documentclass[unicode,aspectratio=169,xcolor={table,dvipsnames,usernames}]{beamer}

\usepackage[utf8]{inputenc}
\usepackage[T2A]{fontenc}
\usepackage[english]{babel}
\usepackage{alltt}

\colorlet{darkred}{BrickRed}
\colorlet{darkgreen}{OliveGreen}
\newcommand{\symlinebreak}{\textcolor{blue}{\(\hookleftarrow\)}}
\newcommand{\symlinecont}{\textcolor{blue}{\(\longrightarrow\)}}

\pgfdeclareimage[height=0.6cm]{left-corner-logo}{asg.jpg}
\pgfdeclareimage[height=1.2cm]{right-corner-logo}{strace-straus.pdf}
\pgfdeclareimage[height=1.2cm]{title-logo}{asg.jpg}
\pgfdeclareimage[height=8cm]{strace-logo}{strace-straus.pdf}

\usetheme{Warsaw}
\setbeamertemplate{headline}{}
\setbeamertemplate{footline}{%
	\pgfuseimage{left-corner-logo}%
	\hfill%
	\pgfuseimage{right-corner-logo}%
}
\setbeamertemplate{navigation symbols}{}
\setbeamertemplate{logo}{}
\setbeamercovered{transparent}

\title{\Huge Why would you still want \\ to use strace in 2023?}

\author{\LARGE Dmitry~Levin \\ Eugene Syromiatnikov}
\date{\Large Berlin, 2023}
%\titlegraphic{\pgfuseimage{title-logo}}

\begin{document}

%%%%%%%
{
\setbeamertemplate{footline}{%
	\pgfuseimage{title-logo}%
	\hfill%
	\pgfuseimage{right-corner-logo}%
}
\begin{frame}[noframenumbering]
\titlepage
\end{frame}
}

%%%%%%%
\begin{frame}{\Large What is special about strace?\hfill [\insertframenumber/\inserttotalframenumber]}
\begin{block}{Unprivileged}
\begin{itemize}
	\item Uses ptrace API
	\item The only general purpose tracer of its kind
\end{itemize}
\end{block}

\begin{block}{Userspace}
\begin{itemize}
	\item Can access tracee memory, filesystem, etc.
	\item Can perform complex manipulations with the data
\end{itemize}
\end{block}

\begin{block}{Decoding}
\begin{itemize}
	\item The best syscall decoder in town
	\item Can print a lot of information associated with file descriptors and pids
\end{itemize}
\end{block}

\begin{block}{Tampering}
\begin{itemize}
	\item Can tamper with syscall numbers and their return values
	\item Can inject data and delays both on entering and on exiting syscalls
\end{itemize}
\end{block}
\end{frame}

%%%%%%%
\begin{frame}[fragile]{Seccomp-assisted syscall filtering: \textbf{-{}-seccomp-bpf} \hfill [\insertframenumber/\inserttotalframenumber]}
\begin{block}{Introduced in v5.3 (September 2019)}
\small
\begin{itemize}
\item Automatically generates and attaches a BPF program to filter syscalls
\item Makes execution of untraced syscalls two orders of magnitude faster
\end{itemize}
\end{block}

\begin{block}{The infamous dd example}
\scriptsize
\begin{alltt}
\$ dd if=/dev/zero of=/dev/null bs=1 count=1M 2>\&1 | grep -v records
1048576 bytes (1.0 MB, 1.0 MiB) copied, \textcolor{darkred}{0.281794} s, 3.7 MB/s

\$ strace -f -qq --signal=none --trace=fchdir \textbackslash
  dd if=/dev/zero of=/dev/null bs=1 count=1M 2>\&1 | grep -v records
1048576 bytes (1.0 MB, 1.0 MiB) copied, \textcolor{darkred}{13.9573} s, 75.1 kB/s

\$ strace -f -qq --signal=none --trace=fchdir \textcolor{darkgreen}{--seccomp-bpf} \textbackslash
  dd if=/dev/zero of=/dev/null bs=1 count=1M 2>\&1 | grep -v records
1048576 bytes (1.0 MB, 1.0 MiB) copied, \textcolor{darkred}{0.307412} s, 3.4 MB/s
\end{alltt}
\end{block}

\begin{block}{Comparative slowdown}
$0.307412 / 0.281794 \approx \textcolor{darkred}{1.09}$ \hfill
$13.9573 / 0.307412 \approx \textcolor{darkred}{45.4}$ \hfill
$13.9573 / 0.281794 \approx \textcolor{darkred}{49.5}$
\end{block}
\end{frame}

%%%%%%%
\begin{frame}{Seccomp-assisted syscall filtering: \textbf{-{}-seccomp-bpf} \hfill [\insertframenumber/\inserttotalframenumber]}
\begin{block}{Limitations}
\begin{itemize}
	\item Has no effect unless \textbf{-{}-follow-forks} (\textbf{-f}) option is specified
	\item Incompatible with \textbf{-{}-syscall-limit}=\textit{number} option
	\item Incompatible with \textbf{-{}-detach-on}=\textbf{execve} (\textbf{-b execve}) option
	\item Not applicable to tracees attached using \textbf{-{}-attach}=\textit{pid} (\textbf{-p} \textit{pid}) option
	\item Problematic in case of premature strace termination
\end{itemize}
\end{block}

\begin{block}{Why?}
\begin{itemize}
	\item Seccomp bpf installed by strace uses SECCOMP\_RET\_TRACE API \\ to trace syscalls
	\item Once installed, seccomp bpf cannot be removed
	\item If there is no tracer, SECCOMP\_RET\_TRACE is treated like SECCOMP\_RET\_ERRNO
		so all formerly traced syscalls would fail with ENOSYS
\end{itemize}
\end{block}

Being an optimization, \textbf{-{}-seccomp-bpf} is automatically turned off when not applicable.
\end{frame}

%%%%%%%
\begin{frame}{Syscall filtering: trace only the specified set of syscalls \hfill [\insertframenumber/\inserttotalframenumber]}
\textbf{-{}-trace}=\textit{set}, where \textit{set} can include the following elements
\begin{description}
	\item[syscall]: specific syscall specified by its name
	\item[/regex]: syscalls that match the \textit{regex}
	\item[\%class]: syscalls that belong to the \textit{class}
\end{description}
\scriptsize
\begin{block}{\large Syscall classes}
\begin{itemize}
\item \textbf{\%file}: take a file name
\item \textbf{\%desc}: take or return a descriptor
\item \textbf{\%memory}: memory mapping, memory policy
\item \textbf{\%process}: process management
\item \textbf{\%signal}: signal related
\item \textbf{\%ipc}: SysV IPC related
\item \textbf{\%network}: network related
\item \textbf{\%creds}: read or modify user and group identifiers or capabilities
\item \textbf{\%clock}: read or modify system clocks
\item \textbf{\%stat}, \textbf{\%lstat}, \textbf{\%fstat}, \textbf{\%\%stat}: stat variants
\item \textbf{\%statfs}, \textbf{\%fstatfs}, \textbf{\%\%statfs}: statfs variants
\end{itemize}
\end{block}
\end{frame}

%%%%%%%
\begin{frame}[fragile]{\texttt{ioctl} parsers \hfill [\insertframenumber/\inserttotalframenumber]}
\begin{block}{\large Examples of currently supported \texttt{ioctl(2)} command classes}
\begin{columns}
	\column{6cm}
		\begin{itemize}
			\item BLK*
			\item BTRFS\_*
			\item DM\_*
			\item EV* (evdev)
			\item GPIO\_*
			\item HDIO\_*
			\item KD*
			\item LIRC\_*
			\item LOOP\_*
			\item MEM* (MTD)
			\item NBD\_*
			\item NS\_*
		\end{itemize}
	\column{6cm}
		\begin{itemize}
			\item PERF\_EVENT\_IOC\_*
			\item PTP\_*
			\item RANDOM\_*
			\item RTC\_*
			\item SECCOMP\_*
			\item SIOC\_*
			\item SG\_*
			\item TEE\_*
			\item UBI\_*
			\item UFFDIO\_*
			\item VIDIOC\_*
			\item WDIOC\_*
		\end{itemize}
\end{columns}
\end{block}
\end{frame}

%%%%%%%
\begin{frame}[fragile]{\texttt{ioctl} parsers \hfill [\insertframenumber/\inserttotalframenumber]}
\begin{block}{\scriptsize strace -v -{}-trace=ioctl dmsetup ls}
\scriptsize
\begin{alltt}
ioctl(3, DM_VERSION, \{version=4.0.0, data_size=16384, flags=DM_EXISTS_FLAG\}
\symlinecont => \{version=4.43.0, data_size=16384, flags=DM_EXISTS_FLAG\}) = 0
ioctl(3, DM_LIST_DEVICES, \{version=4.0.0, data_size=16384, data_start=312,
\symlinecont flags=DM_EXISTS_FLAG\} => \{version=4.43.0, data_size=472, data_start=312,
\symlinecont flags=DM_EXISTS_FLAG, \{dev=makedev(0xfe, 0), name="nvme0n1p3_crypt", event_nr=0\},
\symlinecont \{dev=makedev(0xfe, 0x3), name="nurgle--vg-home", event_nr=0\},
\symlinecont \{dev=makedev(0xfe, 0x1), name="nurgle--vg-root", event_nr=0\},
\symlinecont \{dev=makedev(0xfe, 0x2), name="nurgle--vg-swap_1"\}\}) = 0
\end{alltt}
\end{block}
\begin{block}{\scriptsize\url{https://git.kernel.org/pub/scm/linux/kernel/git/legion/kbd.git/commit/?id=1002d3b56b72}}
\scriptsize
\begin{alltt}
Author: Alexey Gladkov <gladkov.alexey@gmail.com>
Date:   Tue Apr 25 17:32:11 2023 +0200
\end{alltt}
\vspace{-3ex}
\begin{alltt}
    tests: Use strace to track syscalls
\end{alltt}
\vspace{-3ex}
\begin{alltt}
    Now strace is powerful enough to show ioctls specific to console
    configuration. Additional benefit of using strace is that we can
    enable end-to-end tests for statically linked utilities.
\end{alltt}
\vspace{-3ex}
\begin{alltt}
    Signed-off-by: Alexey Gladkov <gladkov.alexey@gmail.com>
\end{alltt}
\vspace{-3ex}
\begin{alltt}
 37 files changed, 15955 insertions(+), 105164 deletions(-)
\end{alltt}
\end{block}
\end{frame}

%%%%%%%
\begin{frame}[fragile]{Netlink socket parsers \hfill [\insertframenumber/\inserttotalframenumber]}
\scriptsize
\begin{block}{\large Currently supported netlink protocols}
\begin{itemize}
\item NETLINK\_AUDIT
\item NETLINK\_CRYPTO
\item NETLINK\_KOBJECT\_UEVENT
\item NETLINK\_NETFILTER
\item NETLINK\_ROUTE
\item NETLINK\_SELINUX
\item NETLINK\_SOCK\_DIAG
\item NETLINK\_XFRM
\item NETLINK\_GENERIC
\end{itemize}
\end{block}
\begin{block}{\large NETLINK\_ROUTE: ip route list table all}
\begin{alltt}
local 127.0.0.0/8 dev lo table local proto kernel scope host src 127.0.0.1
local 127.0.0.1 dev lo table local proto kernel scope host src 127.0.0.1
broadcast 127.255.255.255 dev lo table local proto kernel scope link src 127.0.0.1
local ::1 dev lo table local proto kernel metric 0 pref medium
\end{alltt}
\end{block}
\end{frame}

%%%%%%%
\begin{frame}[fragile]{Netlink socket parsers: NETLINK\_ROUTE \hfill [\insertframenumber/\inserttotalframenumber]}
\tiny
\begin{block}{\large strace -{}-trace=sendto,recvmsg ip route list}
\begin{alltt}
sendto(3, [[\textcolor{darkgreen}{\{nlmsg_len=28, nlmsg_type=RTM_GETROUTE, nlmsg_flags=NLM_F_REQUEST|NLM_F_DUMP, nlmsg_seq=135792468, nlmsg_pid=0\}}, \symlinebreak
\symlinecont \textcolor{blue}{\{rtm_family=AF_UNSPEC, rtm_dst_len=0, rtm_src_len=0, rtm_tos=0, rtm_table=RT_TABLE_UNSPEC, \symlinebreak
\symlinecont rtm_protocol=RTPROT_UNSPEC, rtm_scope=RT_SCOPE_UNIVERSE, rtm_type=RTN_UNSPEC, rtm_flags=0\}}], \symlinebreak
\symlinecont \textcolor{darkgreen}{\{nlmsg_len=0, nlmsg_type=0, nlmsg_flags=0, nlmsg_seq=0, nlmsg_pid=0\}}], 156, 0, NULL, 0) = 156

recvmsg(3, \{msg_name=\{sa_family=AF_NETLINK, nl_pid=0, nl_groups=00000000\}, msg_namelen=12, \symlinebreak
\symlinecont msg_iov=[\{iov_base=[[\textcolor{darkgreen}{\{nlmsg_len=60, nlmsg_type=RTM_NEWROUTE, \symlinebreak
\symlinecont nlmsg_flags=NLM_F_MULTI|NLM_F_DUMP_FILTERED, nlmsg_seq=135792468, nlmsg_pid=1234567\}}, \symlinebreak
\symlinecont \textcolor{blue}{\{rtm_family=AF_INET, rtm_dst_len=8, rtm_src_len=0, rtm_tos=0, rtm_table=RT_TABLE_LOCAL, \symlinebreak
\symlinecont rtm_protocol=RTPROT_KERNEL, rtm_scope=RT_SCOPE_HOST, rtm_type=RTN_LOCAL, rtm_flags=0\}}, \symlinebreak
\symlinecont \textcolor{darkred}{[[\{nla_len=8, nla_type=RTA_TABLE\}, RT_TABLE_LOCAL], [\{nla_len=8, nla_type=RTA_DST\}, inet_addr("127.0.0.0")], \symlinebreak
\symlinecont [\{nla_len=8, nla_type=RTA_PREFSRC\}, inet_addr("127.0.0.1")], [\{nla_len=8, nla_type=RTA_OIF\}, if_nametoindex("lo")]]}], \symlinebreak
\symlinecont [\textcolor{darkgreen}{\{nlmsg_len=60, nlmsg_type=RTM_NEWROUTE, nlmsg_flags=NLM_F_MULTI|NLM_F_DUMP_FILTERED, \symlinebreak
\symlinecont nlmsg_seq=135792468, nlmsg_pid=1234567\}}, \symlinebreak
\symlinecont \textcolor{blue}{\{rtm_family=AF_INET, rtm_dst_len=32, rtm_src_len=0, rtm_tos=0, rtm_table=RT_TABLE_LOCAL, \symlinebreak
\symlinecont rtm_protocol=RTPROT_KERNEL, rtm_scope=RT_SCOPE_HOST, rtm_type=RTN_LOCAL, rtm_flags=0\}}, \symlinebreak
\symlinecont \textcolor{darkred}{[[\{nla_len=8, nla_type=RTA_TABLE\}, RT_TABLE_LOCAL], [\{nla_len=8, nla_type=RTA_DST\}, inet_addr("127.0.0.1")], \symlinebreak
\symlinecont [\{nla_len=8, nla_type=RTA_PREFSRC\}, inet_addr("127.0.0.1")], [\{nla_len=8, nla_type=RTA_OIF\}, if_nametoindex("lo")]]}], \symlinebreak
\symlinecont [\textcolor{darkgreen}{\{nlmsg_len=60, nlmsg_type=RTM_NEWROUTE, nlmsg_flags=NLM_F_MULTI|NLM_F_DUMP_FILTERED, \symlinebreak
\symlinecont nlmsg_seq=135792468, nlmsg_pid=1234567\}}, \symlinebreak
\symlinecont \textcolor{blue}{\{rtm_family=AF_INET, rtm_dst_len=32, rtm_src_len=0, rtm_tos=0, rtm_table=RT_TABLE_LOCAL, \symlinebreak
\symlinecont rtm_protocol=RTPROT_KERNEL, rtm_scope=RT_SCOPE_LINK, rtm_type=RTN_BROADCAST, rtm_flags=0\}}, \symlinebreak
\symlinecont \textcolor{darkred}{[[\{nla_len=8, nla_type=RTA_TABLE\}, RT_TABLE_LOCAL], [\{nla_len=8, nla_type=RTA_DST\}, inet_addr("127.255.255.255")], \symlinebreak
\symlinecont [\{nla_len=8, nla_type=RTA_PREFSRC\}, inet_addr("127.0.0.1")], [\{nla_len=8, nla_type=RTA_OIF\}, if_nametoindex("lo")]]}]], \symlinebreak
\symlinecont iov_len=32768\}], msg_iovlen=1, msg_controllen=0, msg_flags=0\}, 0) = 180

\ldots
\end{alltt}
\end{block}
\end{frame}

%%%%%%%
\begin{frame}[fragile]{Stack of function calls: \textbf{-k} option \hfill [\insertframenumber/\inserttotalframenumber]}
\begin{block}{\large \textbf{-k} is an alias for \textbf{-{}-stack-traces}}
Print the execution stack trace of the traced processes after each traced syscall or signal
\end{block}
\begin{block}{\large strace \textbf{-k} -qq -P /dev/full cat < /dev/null > /dev/full}
\scriptsize
\begin{alltt}
newfstatat(1, "", \{st_mode=S_IFCHR|0666, st_rdev=makedev(0x1, 0x7), ...\}, AT_EMPTY_PATH) = 0
 > /lib64/libc.so.6(fstatat64+0xa) [0xfbdda]
 > /bin/cat(main+0x217) [0x2597]
 > /lib64/libc.so.6(__libc_start_call_main+0x7b) [0x27c8b]
 > /lib64/libc.so.6(__libc_start_main@@GLIBC_2.34+0x84) [0x27d44]
 > /bin/cat(_start+0x20) [0x3620]
close(1)                                = 0
 > /lib64/libc.so.6(__close_nocancel+0x7) [0x100647]
 > /lib64/libc.so.6(_IO_file_close_it@@GLIBC_2.2.5+0x62) [0x81842]
 > /lib64/libc.so.6(fclose@@GLIBC_2.2.5+0x102) [0x75b92]
 > /bin/cat(close_stream+0x1b) [0x3b4b]
 > /bin/cat(close_stdout+0xf) [0x5c1f]
 > /lib64/libc.so.6(__run_exit_handlers+0x155) [0x3fec5]
 > /lib64/libc.so.6(exit+0x1b) [0x3fffb]
 > /lib64/libc.so.6(__libc_start_call_main+0x82) [0x27c92]
 > /lib64/libc.so.6(__libc_start_main@@GLIBC_2.34+0x84) [0x27d44]
 > /bin/cat(_start+0x20) [0x3620]
\end{alltt}
\end{block}
\end{frame}

%%%%%%%
\begin{frame}[fragile]{Stack of function calls: \textbf{-k} option \hfill [\insertframenumber/\inserttotalframenumber]}
\begin{block}{strace -qq -{}-trace=none timeout 1 sleep 2}
\scriptsize
\begin{alltt}
--- SIGALRM \{si_signo=SIGALRM, si_code=SI_TIMER, si_timerid=0, si_overrun=0, si_int=0, si_ptr=NULL\} ---
--- SIGTERM \{si_signo=SIGTERM, si_code=SI_USER, si_pid=1234567, si_uid=1000\} ---
--- SIGCHLD \{si_signo=SIGCHLD, si_code=CLD_KILLED, si_pid=1234568, si_uid=1000, si_status=SIGTERM, si_utime=0, si_stime=0\} ---
--- SIGCONT \{si_signo=SIGCONT, si_code=SI_USER, si_pid=1234567, si_uid=1000\} ---
\end{alltt}
\end{block}
\begin{block}{strace \textbf{-k} -qq -{}-trace=none -{}-signal=alrm,term timeout 1 sleep 2}
\scriptsize
\begin{alltt}
--- SIGALRM \{si_signo=SIGALRM, si_code=SI_TIMER, si_timerid=0, si_overrun=0, si_int=0, si_ptr=NULL\} ---
 > /lib64/libc.so.6(__sigsuspend+0x15) [0x3dca5]
 > /usr/bin/timeout(main+0x663) [0x2ae3]
 > /lib64/libc.so.6(__libc_start_call_main+0x7b) [0x27c8b]
 > /lib64/libc.so.6(__libc_start_main@@GLIBC_2.34+0x84) [0x27d44]
 > /usr/bin/timeout(_start+0x20) [0x2d40]
--- SIGTERM \{si_signo=SIGTERM, si_code=SI_USER, si_pid=1234567, si_uid=1000\} ---
 > /lib64/libc.so.6(kill+0x7) [0x3dc37]
 > /usr/bin/timeout(cleanup+0x15c) [0x5bcc]
 > /lib64/libc.so.6(__restore_rt+0x0) [0x3da20]
 > /lib64/libc.so.6(__sigsuspend+0x15) [0x3dca5]
 > /usr/bin/timeout(main+0x663) [0x2ae3]
 > /lib64/libc.so.6(__libc_start_call_main+0x7b) [0x27c8b]
 > /lib64/libc.so.6(__libc_start_main@@GLIBC_2.34+0x84) [0x27d44]
 > /usr/bin/timeout(_start+0x20) [0x2d40]
\end{alltt}
\end{block}
\end{frame}

%%%%%%%
\begin{frame}{Decode various information associated with file descriptors \hfill [\insertframenumber/\inserttotalframenumber]}
\large
\begin{block}{\large \textbf{-{}-decode-fds}=\textit{set}, where \textit{set} can include the following elements}
\begin{description}
	\item[path]: print file path associated with file descriptors and AT\_FDCWD
	\item[socket]: print protocol-specific information associated with sockets
	\item[dev]: print character/block device numbers
	\item[pidfd]: print PIDs associated with pidfd file descriptors
	\item[signalfd]: print signal masks associated with signalfd file descriptors
\end{description}
\end{block}
\begin{block}{\large Defaults and aliases}
\begin{itemize}
\item the default is \textbf{-{}-decode-fds}=\textbf{none}
\item \textbf{-y} is an alias for \textbf{-{}-decode-fds}=\textbf{path}
\item \textbf{-yy} is an alias for \textbf{-{}-decode-fds}=\textbf{all}
\end{itemize}
\end{block}
\end{frame}

%%%%%%%
\begin{frame}[fragile]{Decode information associated with devices and sockets \hfill [\insertframenumber/\inserttotalframenumber]}
\begin{block}{\large strace \textbf{-yy} -e \%desc,\%network nc 127.0.0.1 22 < /dev/null}
\scriptsize
\begin{alltt}
\ldots
socket(AF_INET, SOCK_STREAM, IPPROTO_TCP) = 3\textcolor{darkred}{<TCP:[518663]>}
connect(3\textcolor{darkred}{<TCP:[518663]>}, {sa_family=AF_INET, sin_port=htons(22), \symlinebreak
\symlinecont sin_addr=inet_addr("127.0.0.1")}, 16) = 0
poll([{fd=3\textcolor{darkred}{<TCP:[127.0.0.1:45678->127.0.0.1:22]>}, events=POLLIN}, \symlinebreak
\symlinecont {fd=0\textcolor{darkgreen}{</dev/null<char 1:3>>}, events=POLLIN}], 2, -1) = 1 ([{fd=0, revents=POLLIN}])
read(0\textcolor{darkgreen}{</dev/null<char 1:3>>}, "", 2048)  = 0
shutdown(3\textcolor{darkred}{<TCP:[127.0.0.1:45678->127.0.0.1:22]>}, SHUT_WR) = 0
poll([{fd=3\textcolor{darkred}{<TCP:[127.0.0.1:45678->127.0.0.1:22]>}, events=POLLIN}, {fd=-1}], 2, -1) \symlinebreak
\symlinecont = 1 ([{fd=3, revents=POLLIN}])
read(3\textcolor{darkred}{<TCP:[127.0.0.1:45678->127.0.0.1:22]>}, "SSH-2.0-OpenSSH_9.4{\textbackslash}r{\textbackslash}n", 2048) = 21
write(1\textcolor{darkgreen}{</dev/pts/1<char 136:1>>}, "SSH-2.0-OpenSSH_9.4{\textbackslash}r{\textbackslash}n", 21) = 21
poll([{fd=3\textcolor{darkred}{<TCP:[127.0.0.1:45678->127.0.0.1:22]>}, events=POLLIN}, {fd=-1}], 2, -1) \symlinebreak
\symlinecont = 1 ([{fd=3, revents=POLLIN|POLLHUP}])
read(3\textcolor{darkred}{<TCP:[127.0.0.1:45678->127.0.0.1:22]>}, "", 2048) = 0
shutdown(3\textcolor{darkred}{<TCP:[127.0.0.1:45678->127.0.0.1:22]>}, SHUT_RD) \symlinebreak
\symlinecont = -1 ENOTCONN (Transport endpoint is not connected)
close(3\textcolor{darkred}{<TCP:[127.0.0.1:45678->127.0.0.1:22]>}) = 0
+++ exited with 0 +++
\end{alltt}
\end{block}
\end{frame}

%%%%%%%
\begin{frame}[fragile]{Decoding signalfd signal masks: \textbf{-{}-decode-fds}=\textbf{signalfd} \hfill [\insertframenumber/\inserttotalframenumber]}
\large
\begin{block}{\large Introduced in v6.3 (May 2023)}
\begin{itemize}
\item Enables decoding of signal masks associated with signalfd file descriptors
\item \textbf{-{}-decode-fds}=\textbf{signalfd} is a subset of \textbf{-{}-decode-fds}=\textbf{all}
\end{itemize}
\end{block}
\begin{block}{\large strace -q \textbf{-yy} -{}-trace=/signalfd,/epoll\_ctl -{}-syscall-limit=3 \textbackslash \\ /usr/lib/systemd/systemd-portabled}
\begin{alltt}
\small
signalfd4(-1, [INT], 8, SFD_CLOEXEC|SFD_NONBLOCK) = 4\textcolor{darkred}{<signalfd:[INT]>}
epoll_ctl(3\textcolor{darkgreen}{<anon_inode:[eventpoll]>}, EPOLL_CTL_ADD, 4\textcolor{darkred}{<signalfd:[INT]>}, \symlinebreak
\symlinecont {events=EPOLLIN, data={u32=112224560, u64=94214514829616}}) = 0
signalfd4(4\textcolor{darkred}{<signalfd:[INT]>}, [INT TERM], 8, SFD_CLOEXEC|SFD_NONBLOCK) \symlinebreak
\symlinecont = 4\textcolor{darkred}{<signalfd:[INT TERM]>}
\end{alltt}
\end{block}
\end{frame}

%%%%%%%
\begin{frame}{Decode various information associated with process IDs \hfill [\insertframenumber/\inserttotalframenumber]}
\large
Introduced in v5.15 (December 2021)
\begin{block}{\textbf{-{}-decode-pids}=\textit{set}, where \textit{set} can include the following elements}
\begin{description}
	\item[comm]: print command names associated with thread or process IDs
	\item[pidns]: print thread, process, process group, and session IDs
		in strace's PID namespace if the tracee is in a different PID namespace
\end{description}
\end{block}

\begin{block}{Defaults and aliases}
\begin{itemize}
\item the default is \textbf{-{}-decode-pids}=\textbf{none}
\item \textbf{-Y} is an alias for \textbf{-{}-decode-pids}=\textbf{comm}
\end{itemize}
\end{block}
\end{frame}

%%%%%%%
\begin{frame}[fragile]{Display of command names for PIDs: \textbf{-{}-decode-pids}=\textbf{comm}, \textbf{-Y} \hfill [\insertframenumber/\inserttotalframenumber]}
\small
\begin{block}{strace -f \textbf{-Y} -e\%process timeout 1 sleep 2}
\scriptsize
\begin{alltt}
execve("/usr/bin/timeout", ["timeout", "1", "sleep", "2"], 0x7ffcfb9bb348 /* 17 vars */) = 0
clone(child_stack=NULL, flags=CLONE_CHILD_CLEARTID|CLONE_CHILD_SETTID|SIGCHLD, \symlinebreak
\symlinecont child_tidptr=0x7f116942da10) = 123452\textcolor{darkgreen}{<timeout>}
strace: Process 123452 attached
[pid 123451\textcolor{darkgreen}{<timeout>}] wait4(123452\textcolor{darkgreen}{<timeout>}, 0x7ffc132b1a28, WNOHANG, NULL) = 0
[pid 123452\textcolor{darkgreen}{<timeout>}] execve("/bin/sleep", ["sleep", "2"], 0x7ffc132b1ce0 /* 17 vars */) = 0
[pid 123451\textcolor{darkgreen}{<timeout>}] --- SIGALRM \{si_signo=SIGALRM, si_code=SI_TIMER, \symlinebreak
\symlinecont si_timerid=0, si_overrun=0, si_int=0, si_ptr=NULL\} ---
[pid 123451\textcolor{darkgreen}{<timeout>}] kill(123452\textcolor{darkred}{<sleep>}, SIGTERM) = 0
[pid 123452\textcolor{darkred}{<sleep>}] --- SIGTERM \{si_signo=SIGTERM, si_code=SI_USER, \symlinebreak
\symlinecont si_pid=123451\textcolor{darkgreen}{<timeout>}, si_uid=1000\} ---
[pid 123451\textcolor{darkgreen}{<timeout>}] kill(0, SIGTERM) = 0
[pid 123451\textcolor{darkgreen}{<timeout>}] --- SIGTERM \{si_signo=SIGTERM, si_code=SI_USER, \symlinebreak
\symlinecont si_pid=123451\textcolor{darkgreen}{<timeout>}, si_uid=1000\} ---
[pid 123451\textcolor{darkgreen}{<timeout>}] kill(123452\textcolor{darkred}{<sleep>}, SIGCONT) = 0
[pid 123452\textcolor{darkred}{<sleep>}] +++ killed by SIGTERM +++
--- SIGCHLD \{si_signo=SIGCHLD, si_code=CLD_KILLED, si_pid=123452\textcolor{darkred}{<sleep>}, \symlinebreak
\symlinecont si_uid=1000, si_status=SIGTERM, si_utime=0, si_stime=0\} ---
kill(0, SIGCONT)                        = 0
--- SIGCONT \{si_signo=SIGCONT, si_code=SI_USER, si_pid=123451\textcolor{darkgreen}{<timeout>}, si_uid=1000\} ---
wait4(123452\textcolor{darkred}{<sleep>}, [{WIFSIGNALED(s) && WTERMSIG(s) == SIGTERM}], WNOHANG, NULL) = 123452
exit_group(124)                         = ?
+++ exited with 124 +++
\end{alltt}
\end{block}
\end{frame}

%%%%%%%
\begin{frame}[fragile]{Display of command names and PID namespace translation \hfill [\insertframenumber/\inserttotalframenumber]}
\large
\textbf{-{}-decode-pids}=\textbf{all} is an alias for \textbf{-{}-decode-pids}=\textbf{comm},\textbf{pidns}
\begin{block}{strace \textbf{-{}-decode-pids=all} -q -f -{}-trace=clone,kill \textbackslash \\ unshare -Urpf sh -c 'sleep 2 \& sleep 1 \&\& kill \$!'}
\scriptsize
\begin{alltt}
clone(child_stack=NULL, flags=CLONE_CHILD_CLEARTID|CLONE_CHILD_SETTID|SIGCHLD, \symlinebreak
\symlinecont child_tidptr=0x7ff30b283a10) = 123456\textcolor{darkgreen}{<unshare>}
[pid 123456\textcolor{darkgreen}{<sh>}] clone(child_stack=NULL, flags=CLONE_CHILD_CLEARTID|CLONE_CHILD_SETTID|SIGCHLD, \symlinebreak
\symlinecont child_tidptr=0x7f6dccd89a10) = 2\textcolor{darkgreen}{<sh>} \textcolor{darkred}{/* 123457 in strace's PID NS */}
[pid 123456\textcolor{darkgreen}{<sh>}] clone(child_stack=NULL, flags=CLONE_CHILD_CLEARTID|CLONE_CHILD_SETTID|SIGCHLD, \symlinebreak
\symlinecont child_tidptr=0x7f6dccd89a10) = 3\textcolor{darkgreen}{<sh>} \textcolor{darkred}{/* 123458 in strace's PID NS */}
[pid 123458\textcolor{darkgreen}{<sleep>}] +++ exited with 0 +++
[pid 123456\textcolor{darkgreen}{<sh>}] --- SIGCHLD \{si_signo=SIGCHLD, si_code=CLD_EXITED, \symlinebreak
\symlinecont si_pid=3, si_uid=0, si_status=0, si_utime=0, si_stime=0\} ---
[pid 123456\textcolor{darkgreen}{<sh>}] kill(2\textcolor{darkgreen}{<sleep>} \textcolor{darkred}{/* 123457 in strace's PID NS */}, SIGTERM) = 0
[pid 123457\textcolor{darkgreen}{<sleep>}] --- SIGTERM \{si_signo=SIGTERM, si_code=SI_USER, \symlinebreak
\symlinecont si_pid=1\textcolor{darkgreen}{<sh>} \textcolor{darkred}{/* 123456 in strace's PID NS */}, si_uid=0\} ---
[pid 123457\textcolor{darkgreen}{<sleep>}] +++ killed by SIGTERM +++
[pid 123456\textcolor{darkgreen}{<sh>}] +++ exited with 0 +++
--- SIGCHLD \{si_signo=SIGCHLD, si_code=CLD_EXITED, si_pid=123456, \symlinebreak
\symlinecont si_uid=0, si_status=0, si_utime=0, si_stime=0\} ---
+++ exited with 0 +++
\end{alltt}
\end{block}
\end{frame}

%%%%%%%
\begin{frame}[fragile]{Display of SELinux contexts: \textbf{-{}-secontext} \hfill [\insertframenumber/\inserttotalframenumber]}
\large
Introduced in v5.12 (April 2021)
\begin{block}{strace -P /dev/null cat /dev/null}
\scriptsize
\begin{alltt}
openat(AT_FDCWD, "/dev/null", O_RDONLY) = 3
newfstatat(3, "", \{st_mode=S_IFCHR|0666, st_rdev=makedev(0x1, 0x3), ...\}, AT_EMPTY_PATH) = 0
fadvise64(3, 0, 0, POSIX_FADV_SEQUENTIAL) = 0
read(3, "", 131072)                     = 0
close(3)                                = 0
+++ exited with 0 +++
\end{alltt}
\end{block}
\large
\begin{block}{strace \textbf{-{}-secontext} -P /dev/null cat /dev/null}
\scriptsize
\begin{alltt}
[\textcolor{darkgreen}{unconfined_t}] openat(AT_FDCWD, "/dev/null" [\textcolor{darkred}{null_device_t}], O_RDONLY) = 3 [\textcolor{darkred}{null_device_t}]
[\textcolor{darkgreen}{unconfined_t}] newfstatat(3 [\textcolor{darkred}{null_device_t}], "", \{st_mode=S_IFCHR|0666, \symlinebreak
\symlinecont st_rdev=makedev(0x1, 0x3), ...\}, AT_EMPTY_PATH) = 0
[\textcolor{darkgreen}{unconfined_t}] fadvise64(3 [\textcolor{darkred}{null_device_t}], 0, 0, POSIX_FADV_SEQUENTIAL) = 0
[\textcolor{darkgreen}{unconfined_t}] read(3 [\textcolor{darkred}{null_device_t}], "", 131072) = 0
[\textcolor{darkgreen}{unconfined_t}] close(3 [\textcolor{darkred}{null_device_t}]) = 0
+++ exited with 0 +++
\end{alltt}
\end{block}
\end{frame}

%%%%%%%
\begin{frame}[fragile]{Display of SELinux contexts: \textbf{-{}-secontext}=\textbf{mismatch} \hfill [\insertframenumber/\inserttotalframenumber]}
\large
\begin{block}{Sample SELinux context mismatch}
\scriptsize
\begin{alltt}
\$ matchpathcon $PWD/config.h
/home/me/strace/src/config.h	unconfined_u:object_r:user_home_t:s0
\$ ls -Z $PWD/config.h
system_u:object_r:user_home_t:s0 /home/me/strace/src/config.h
\end{alltt}
\end{block}
\large
\begin{block}{strace \textbf{-{}-secontext}=\textbf{full},\textbf{mismatch} \hfill since v5.16 (January 2022)}
\scriptsize
\begin{alltt}
\$ strace --secontext=full,mismatch --trace=\%file stat config.h
\ldots
[unconfined_u:unconfined_r:unconfined_t:s0-s0:c0.c1023] statx(AT_FDCWD, "config.h" \symlinebreak
\symlinecont [\textcolor{darkred}{system_u:object_r:user_home_t:s0!!unconfined_u:object_r:user_home_t:s0}], \symlinebreak
\symlinecont AT_STATX_SYNC_AS_STAT|AT_SYMLINK_NOFOLLOW|AT_NO_AUTOMOUNT, \symlinebreak
\symlinecont STATX_ALL, {stx_mask=STATX_ALL|STATX_MNT_ID, stx_attributes=0, stx_mode=S_IFREG|0644, \symlinebreak
\symlinecont stx_size=50008, ...}) = 0
\end{alltt}
\end{block}
\large
Hint: combine \textbf{-{}-secontext}=\textbf{full},\textbf{mismatch} with \textbf{-Z}
\end{frame}

%%%%%%%
\begin{frame}{System call return status filtering: \textbf{-e status}=\textit{set} \hfill [\insertframenumber/\inserttotalframenumber]}
\Large
\begin{block}{Introduced in v5.2 (July 2019)}
\large
Print only syscalls with the specified return status. \\
\textit{set} can include the following elements:
\begin{description}
	\item[successful]: returned without an error code, alias to \textbf{-z}
	\item[failed]: returned with an error code, alias to \textbf{-Z}
	\item[unfinished]: did not return
	\item[detached]: detached before return
	\item[unavailable]: returned but failed to fetch the error status
\end{description}
\end{block}
The default is \textbf{-e status}=\textbf{all}.
\end{frame}

%%%%%%%
\begin{frame}[fragile]{System call return status filtering: \textbf{-Z}, \textbf{-z} \hfill [\insertframenumber/\inserttotalframenumber]}
\scriptsize
\begin{block}{env -i LD\_LIBRARY\_PATH=/lib64 strace \textbf{-Z} -e\%file /bin/cat </dev/null}
\begin{alltt}
access("/etc/ld.so.preload", R_OK) = -1 ENOENT (No such file or directory)
openat(AT_FDCWD, "/lib64/tls/x86_64/x86_64/libc.so.6", O_RDONLY|O_CLOEXEC) = -1 ENOENT (No suc
stat("/lib64/tls/x86_64/x86_64", 0x7ffdcc6a0c20) = -1 ENOENT (No such file or directory)
openat(AT_FDCWD, "/lib64/tls/x86_64/libc.so.6", O_RDONLY|O_CLOEXEC) = -1 ENOENT (No such file
stat("/lib64/tls/x86_64", 0x7ffdcc6a0c20) = -1 ENOENT (No such file or directory)
openat(AT_FDCWD, "/lib64/tls/x86_64/libc.so.6", O_RDONLY|O_CLOEXEC) = -1 ENOENT (No such file
stat("/lib64/tls/x86_64", 0x7ffdcc6a0c20) = -1 ENOENT (No such file or directory)
openat(AT_FDCWD, "/lib64/tls/libc.so.6", O_RDONLY|O_CLOEXEC) = -1 ENOENT (No such file or dire
stat("/lib64/tls", 0x7ffdcc6a0c20) = -1 ENOENT (No such file or directory)
openat(AT_FDCWD, "/lib64/x86_64/x86_64/libc.so.6", O_RDONLY|O_CLOEXEC) = -1 ENOENT (No such fi
stat("/lib64/x86_64/x86_64", 0x7ffdcc6a0c20) = -1 ENOENT (No such file or directory)
openat(AT_FDCWD, "/lib64/x86_64/libc.so.6", O_RDONLY|O_CLOEXEC) = -1 ENOENT (No such file or d
stat("/lib64/x86_64", 0x7ffdcc6a0c20) = -1 ENOENT (No such file or directory)
openat(AT_FDCWD, "/lib64/x86_64/libc.so.6", O_RDONLY|O_CLOEXEC) = -1 ENOENT (No such file or d
stat("/lib64/x86_64", 0x7ffdcc6a0c20) = -1 ENOENT (No such file or directory)
+++ exited with 0 +++
\end{alltt}
\end{block}
\begin{block}{env -i LD\_LIBRARY\_PATH=/lib64 strace \textbf{-z} -e\%file /bin/cat < /dev/null}
\begin{alltt}
execve("/bin/cat", ["/bin/cat"], 0x7ffc2b781ca0 /* 1 var */) = 0
openat(AT_FDCWD, "/lib64/libc.so.6", O_RDONLY|O_CLOEXEC) = 3
+++ exited with 0 +++
\end{alltt}
\end{block}
\end{frame}

%%%%%%%
\begin{frame}[fragile]{Filtering syscalls operating on specific fds: \textbf{-{}-trace-fds}=\textit{set} \hfill [\insertframenumber/\inserttotalframenumber]}
\large
\begin{block}{Introduced in v6.3 (May 2023)}
Trace only those syscalls that operate on the specified set of file descriptors.
\end{block}
\begin{block}{strace \textbf{-{}-trace-fds}=0,2 cat < /dev/null }
\small
\begin{alltt}
fstat(0, \{st_mode=S_IFCHR|0666, st_rdev=makedev(0x1, 0x3), ...\}) = 0
fadvise64(0, 0, 0, POSIX_FADV_SEQUENTIAL) = 0
read(0, "", 131072)                     = 0
close(0)                                = 0
close(2)                                = 0
+++ exited with 0 +++
\end{alltt}
\end{block}
\end{frame}

%%%%%%%
\begin{frame}{System call tampering \hfill [\insertframenumber/\inserttotalframenumber]}
\large
\begin{block}{\textbf{-{}-inject}=\textit{set}[:\textbf{when}=\textit{expr}]:\textit{what}}
where \textit{what} can include the following elements:
\begin{itemize}
\item fault injection: \\
\textbf{error}=\textit{errno}[:\textbf{syscall}=\textit{syscall}]
\item return value injection: \\
\textbf{retval}=\textit{value}[:\textbf{syscall}=\textit{syscall}]
\item signal injection: \\
\textbf{signal}=\textit{set}
\item delay injection: \\
\textbf{delay\_enter}=\textit{usecs} \\
\textbf{delay\_exit}=\textit{usecs}
\item poke injection: \\
\textbf{poke\_enter}=\textit{@argN=DATAN,@argM=DATAM...} \\
\textbf{poke\_exit}=\textit{@argN=DATAN,@argM=DATAM...}
\end{itemize}
\end{block}
\end{frame}

%%%%%%%
\begin{frame}[fragile]{System call tampering: fault injection \hfill [\insertframenumber/\inserttotalframenumber]}
\begin{block}{\large -e \textbf{inject}=\textit{set}:\textbf{error}=\textit{errno}[:\textbf{when}=\textit{expr}][:\textbf{syscall}=\textit{syscall}]}
\textbf{inject}=\textit{set} -- fault injection for the specified set of syscalls \\
\textbf{error}=\textit{errno} -- the error code to fail syscalls with \\
\textbf{when}=\textit{expr} -- when to inject, in \textit{first[..last][+[step]]} form \\
\textbf{syscall}=\textit{syscall} -- inject the specified syscall instead of -1
\end{block}

\begin{block}{\large strace -e /open -e \textbf{inject}=all:\textbf{error}=EACCES:\textbf{when}=3 \textbackslash\\
cat /dev/full /dev/null}
\begin{alltt}
openat(AT_FDCWD, "/etc/ld.so.cache", O_RDONLY|O_CLOEXEC) = 3
openat(AT_FDCWD, "/lib64/libc.so.6", O_RDONLY|O_CLOEXEC) = 3
\textcolor{darkred}{openat(AT_FDCWD, "/dev/full", O_RDONLY)
 = -1 EACCES (Permission denied) (INJECTED)}
cat: /dev/full: Permission denied
openat(AT_FDCWD, "/dev/null", O_RDONLY) = 3
+++ exited with 1 +++
\end{alltt}
\end{block}
\end{frame}

%%%%%%%
\begin{frame}[fragile]{System call tampering: return value injection \hfill [\insertframenumber/\inserttotalframenumber]}
\begin{block}{\large -e \textbf{inject}=\textit{set}:\textbf{retval}=\textit{value}[:\textbf{when}=\textit{expr}][:\textbf{syscall}=\textit{syscall}]}
\textbf{inject}=\textit{set} -- fault injection for the specified set of syscalls \\
\textbf{retval}=\textit{value} -- the return value to return \\
\textbf{when}=\textit{expr} -- when to inject, in \textit{first[..last][+[step]]} form \\
\textbf{syscall}=\textit{syscall} -- inject the specified syscall instead of -1
\end{block}

\begin{block}{\large example: recovery of temporary files}
\begin{alltt}
$ cat script.sh
t=$(mktemp); trap 'rm -f "$t"' 0; echo secret $$ > $t
$ strace -qq -f -e signal=none -e /unlink \textbackslash
  -e inject=all:retval=0 sh script.sh
[pid 347] \textcolor{darkred}{unlinkat(AT_FDCWD, "/tmp/tmp.l1AlwyCYH3", 0) = 0 (INJECTED)}
$ cat /tmp/tmp.l1AlwyCYH3
secret 345
\end{alltt}
\end{block}

\end{frame}

%%%%%%%
\begin{frame}[fragile]{System call tampering: delay injection \hfill [\insertframenumber/\inserttotalframenumber]}
\begin{block}{\large syscall delay injection}
strace -e \textbf{inject}=\textit{set}:\textbf{delay\_enter}=\textit{usecs}[:\textbf{when}=\textit{expr}] \\
strace -e \textbf{inject}=\textit{set}:\textbf{delay\_exit}=\textit{usecs}[:\textbf{when}=\textit{expr}]
\end{block}

\begin{block}{\large dd if=/dev/zero of=/dev/null bs=1M count=10}
\begin{alltt}
10+0 records in
10+0 records out
10485760 bytes (10 MB, 10 MiB) copied, \textcolor{darkred}{0.00211354} s, \textcolor{darkred}{5.0 GB}/s
\end{alltt}
\end{block}

\begin{block}{\large strace -einject=write:delay\_exit=100000 -ewrite -o/dev/null \textbackslash\\
dd if=/dev/zero of=/dev/null bs=1M count=10}
\begin{alltt}
10+0 records in
10+0 records out
10485760 bytes (10 MB, 10 MiB) copied, \textcolor{darkred}{1.10658} s, \textcolor{darkred}{9.5 MB}/s
\end{alltt}
\end{block}
\end{frame}

%%%%%%%
\begin{frame}[fragile]{System call poke injection: \textbf{poke\_enter}=@\textit{argN}=\textit{dataN} \hfill [\insertframenumber/\inserttotalframenumber]}
\large
\begin{block}{Introduced in v5.11 (February 2021)}
Tracee's memory at the location pointed to by syscall argument \textit{argN} \\ is overwritten by \textit{dataN} on entering syscall
\end{block}

\begin{block}{strace -P /etc/shadow cat /etc/shadow}
\begin{alltt}
openat(AT_FDCWD, "/etc/shadow", O_RDONLY) = -1 EACCES (Permission denied)
cat: /etc/shadow: Permission denied
+++ exited with 1 +++
\end{alltt}
\end{block}

\begin{block}{hex=2F6465762F6E756C6C00; \\ strace -P /etc/shadow \textbackslash \\ -{}-inject=openat:poke\_enter=@arg2=\$hex cat /etc/shadow}
\begin{alltt}
openat(AT_FDCWD, "/etc/shadow", O_RDONLY) = 3 (INJECTED: args)
+++ exited with 0 +++
\end{alltt}
\end{block}
\end{frame}

%%%%%%%
\begin{frame}[fragile]{System call poke injection: \textbf{poke\_exit}=@\textit{argN}=\textit{dataN} \hfill [\insertframenumber/\inserttotalframenumber]}
\large
\begin{block}{Introduced in v5.11 (February 2021)}
Tracee's memory at the location pointed to by syscall argument \textit{argN} \\ is overwritten by \textit{dataN} on exiting syscall
\end{block}

\begin{block}{strace -s256 -e readlink readlink /etc/localtime}
\small
\begin{alltt}
readlink("/etc/localtime", "../usr/share/zoneinfo/Asia/Jerusalem", 64) = 36
../usr/share/zoneinfo/Asia/Jerusalem
+++ exited with 0 +++
\end{alltt}
\end{block}

\begin{block}{hex=2E2E2F7573722F73686172652F7A6F6E65696E666F2F4574632F555443; \\ strace -e readlink -{}-inject=readlink:retval=29:poke\_exit=@arg2=\$hex \textbackslash \\ readlink /etc/localtime}
\scriptsize
\begin{alltt}
readlink("/etc/localtime", "../usr/share/zoneinfo/Etc/UTC", 64) = 29 (INJECTED: args, retval)
../usr/share/zoneinfo/Etc/UTC
+++ exited with 0 +++
\end{alltt}
\end{block}
\end{frame}

%%%%%%%
\begin{frame}[fragile]{Show strace tips, tricks, and tweaks: \textbf{-{}-tips} \hfill [\insertframenumber/\inserttotalframenumber]}
\large
\begin{block}{Introduced in v5.18 (June 2022) and v6.3 (May 2023)}
Show strace tips, tricks, and tweaks before exit.
\end{block}
\begin{block}{strace \textbf{-{}-tips}[=[[\textbf{id}:]\textit{id}],[[\textbf{format}:]\textit{format}]]}
\begin{itemize}
\item \textit{id} can be one of the following:
\begin{description}
	\item[\textbf{random}] prints a random tip (the default)
	\item[\textit{integer}] prints the tip, trick, or tweak \\ corresponding to the specified number
\end{description}
\item \textit{format} can be one of the following:
\begin{description}
	\item[\textbf{compact}] prints the tip just big enough to contain all the text \\ (the default)
	\item[\textbf{full}] prints the tip in its full glory
	\item[\textbf{none}] prints no tip
\end{description}
\end{itemize}
\end{block}
\end{frame}

%%%%%%%
\begin{frame}[fragile]{Show strace tips, tricks, and tweaks: \textbf{-{}-tips} \hfill [\insertframenumber/\inserttotalframenumber]}
\large
\begin{block}{strace -{}-tips=id:31}
\begin{alltt}
 ______________________________________________         ____
/                                              \textbackslash       /    \textbackslash
| Medicinal effects of strace can be achieved  |      |-. .-.|
| by invoking it with the following options:   \textbackslash      (_@)(_@)
|                                               \textbackslash     .---_  \textbackslash
|     strace -DDDqqq -enone --signal=none       _\textbackslash   /..   \textbackslash_/
|                                              /     |__.-^ /
|                                              |         \}  |
\textbackslash______________________________________________/        |   [
                                                        [  ]
\end{alltt}
\end{block}

\begin{block}{\url{https://github.com/strace/strace/issues/14}}
Document how to just get the 'medicinal' effects of strace with no overhead.
\end{block}
\end{frame}

%%%%%%%
{
\setbeamertemplate{footline}{\pgfuseimage{left-corner-logo}}
\begin{frame}[noframenumbering]{Questions?}
	\begin{columns}
		\column{9cm}
\begin{block}{\large homepage}
	\url{https://strace.io}
\end{block}
\begin{block}{\large strace.git, strace-talks.git}
	\url{https://github.com/strace/strace.git}

	\url{https://gitlab.com/strace/strace.git}

\smallskip

	\url{https://github.com/strace/strace-talks.git}

	\url{https://gitlab.com/strace/strace-talks.git}
\end{block}
\begin{block}{\large mailing list}
	strace-devel@lists.strace.io
\end{block}
\begin{block}{\large IRC channel}
	\#strace@oftc
\end{block}
		\column{3cm}
			\centerline{\pgfuseimage{strace-logo}}
	\end{columns}
\end{frame}
}

\end{document}
